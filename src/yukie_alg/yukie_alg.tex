\documentclass[a4paper]{ltjsreport}
\usepackage[hiragino-pro]{luatexja-preset}
\usepackage{../hogehoge}
\usepackage{fancyhdr}
\pagestyle{fancy}
\lhead{}
\chead{\rightmark}
\rhead{}
\renewcommand{\paragraphmark}[1]{\markright{#1}}
\renewcommand{\chaptermark}[1]{}
\renewcommand{\sectionmark}[1]{}

\renewcommand{\postpartname}{巻}
\makeatletter
\@addtoreset{chapter}{part}
\makeatother

\begin{document}
\chapter*{代数学(雪江明彦)}
\part{群論入門(第1版第9刷)}
\setcounter{chapter}{3}
\chapter{群の作用とSylowの定理}
\section*{演習問題}
\paragraph{4.6.10}
$G = \braket{ x,y,z | x^2 = y^3 = z^5 = xyz = 1 }$および$H = \braket{z}$とする.
$H \backslash G$における$1_G$の剰余類を$1$と表し,Todd-Coxeterの方法を実行する.

\begin{center}
  \begin{tabular}{c|cccccc|cccccccccc|cccccccc}
    & $y$ && $y$ && $y$ && $z$ && $z$ && $z$ && $z$ && $z$ && $y$ && $z$ && $y$ && $z$ & \\ \hline
    1 && 2 && 3 && 1 && 1 && 1 && 1 && 1 && 1 && 2 && 4 && 5 && 1
  \end{tabular}
\end{center}
$5z = 1z = 1$なので$5 = 1$.$4y = 5 = 1 = 3y$なので$4=3$.
1行目を書き換えて2行目を計算する.

\begin{center}
  \begin{tabular}{c|cccccc|cccccccccc|cccccccc}
    & $y$ && $y$ && $y$ && $z$ && $z$ && $z$ && $z$ && $z$ && $y$ && $z$ && $y$ && $z$ & \\ \hline
    1 && 2 && 3 && 1 && 1 && 1 && 1 && 1 && 1 && 2 && 3 && 1 && 1 \\
    2 && 3 && 1 && 2 && 3 && 4 && 5 && 6 && 2 && 3 && 4 && 7 && 2
  \end{tabular}
\end{center}
$7z = 6z = 2$なので$7 = 6$.
2行目を書き換えて3, 4行目を計算する.

\begin{center}
  \begin{tabular}{c|cccccc|cccccccccc|cccccccc}
    & $y$ && $y$ && $y$ && $z$ && $z$ && $z$ && $z$ && $z$ && $y$ && $z$ && $y$ && $z$ & \\ \hline
    1 && 2 && 3 && 1 && 1 && 1 && 1 && 1 && 1 && 2 && 3 && 1 && 1 \\
    2 && 3 && 1 && 2 && 3 && 4 && 5 && 6 && 2 && 3 && 4 && 6 && 2 \\
    3 && 1 && 2 && 3 && 4 && 5 && 6 && 2 && 3 && 1 && 1 && 2 && 3 \\
    4 && 7 && 8 && 4 && 5 && 6 && 2 && 3 && 4 && 7 && 9 && 10 && 4
  \end{tabular}
\end{center}
$10z = 3z = 4$なので$10 = 3$.$9y = 10 = 3 = 2y$なので$9 = 2$.
$7z = 9 = 2 = 6z$なので$7 = 6$.
$8 \to 7$とする.
4行目を書き換えて5行目を計算する.

\begin{center}
  \begin{tabular}{c|cccccc|cccccccccc|cccccccc}
    & $y$ && $y$ && $y$ && $z$ && $z$ && $z$ && $z$ && $z$ && $y$ && $z$ && $y$ && $z$ & \\ \hline
    1 && 2 && 3 && 1 && 1 && 1 && 1 && 1 && 1 && 2 && 3 && 1 && 1 \\
    2 && 3 && 1 && 2 && 3 && 4 && 5 && 6 && 2 && 3 && 4 && 6 && 2 \\
    3 && 1 && 2 && 3 && 4 && 5 && 6 && 2 && 3 && 1 && 1 && 2 && 3 \\
    4 && 6 && 7 && 4 && 5 && 6 && 2 && 3 && 4 && 6 && 2 && 3 && 4 \\
    5 && 8 && 9 && 5 && 6 && 2 && 3 && 4 && 5 && 8 && 10 && 11 && 5
  \end{tabular}
\end{center}
$11z = 4z = 5$なので$11 = 4$.$10y = 11 = 4 = 7y$なので$10 = 7$.
5行目を書き換えて6行目を計算する.

\begin{center}
  \begin{tabular}{c|cccccc|cccccccccc|cccccccc}
    & $y$ && $y$ && $y$ && $z$ && $z$ && $z$ && $z$ && $z$ && $y$ && $z$ && $y$ && $z$ & \\ \hline
    1 && 2 && 3 && 1 && 1 && 1 && 1 && 1 && 1 && 2 && 3 && 1 && 1 \\
    2 && 3 && 1 && 2 && 3 && 4 && 5 && 6 && 2 && 3 && 4 && 6 && 2 \\
    3 && 1 && 2 && 3 && 4 && 5 && 6 && 2 && 3 && 1 && 1 && 2 && 3 \\
    4 && 6 && 7 && 4 && 5 && 6 && 2 && 3 && 4 && 6 && 2 && 3 && 4 \\
    5 && 8 && 9 && 5 && 6 && 2 && 3 && 4 && 5 && 8 && 7 && 4 && 5 \\
    6 && 7 && 4 && 6 && 2 && 3 && 4 && 5 && 6 && 7 && 10 && 11 && 6
  \end{tabular}
\end{center}
$11z = 5z = 6$なので$11 = 5$.$10y = 11 = 5 = 9y$なので$10 = 9$.
6行目を書き換えて7行目を計算する.

\begin{center}
  \begin{tabular}{c|cccccc|cccccccccc|cccccccc}
    & $y$ && $y$ && $y$ && $z$ && $z$ && $z$ && $z$ && $z$ && $y$ && $z$ && $y$ && $z$ & \\ \hline
    1 && 2 && 3 && 1 && 1 && 1 && 1 && 1 && 1 && 2 && 3 && 1 && 1 \\
    2 && 3 && 1 && 2 && 3 && 4 && 5 && 6 && 2 && 3 && 4 && 6 && 2 \\
    3 && 1 && 2 && 3 && 4 && 5 && 6 && 2 && 3 && 1 && 1 && 2 && 3 \\
    4 && 6 && 7 && 4 && 5 && 6 && 2 && 3 && 4 && 6 && 2 && 3 && 4 \\
    5 && 8 && 9 && 5 && 6 && 2 && 3 && 4 && 5 && 8 && 7 && 4 && 5 \\
    6 && 7 && 4 && 6 && 2 && 3 && 4 && 5 && 6 && 7 && 9 && 5 && 6 \\
    7 && 4 && 6 && 7 && 9 && 10 && 11 && 12 && 7 && 4 && 5 && 8 && 7
  \end{tabular}
\end{center}
$12z = 8z = 7$なので$12 = 8$.
7行目を書き換えて8行目を計算する.

\begin{center}
  \begin{tabular}{c|cccccc|cccccccccc|cccccccc}
    & $y$ && $y$ && $y$ && $z$ && $z$ && $z$ && $z$ && $z$ && $y$ && $z$ && $y$ && $z$ & \\ \hline
    1 && 2 && 3 && 1 && 1 && 1 && 1 && 1 && 1 && 2 && 3 && 1 && 1 \\
    2 && 3 && 1 && 2 && 3 && 4 && 5 && 6 && 2 && 3 && 4 && 6 && 2 \\
    3 && 1 && 2 && 3 && 4 && 5 && 6 && 2 && 3 && 1 && 1 && 2 && 3 \\
    4 && 6 && 7 && 4 && 5 && 6 && 2 && 3 && 4 && 6 && 2 && 3 && 4 \\
    5 && 8 && 9 && 5 && 6 && 2 && 3 && 4 && 5 && 8 && 7 && 4 && 5 \\
    6 && 7 && 4 && 6 && 2 && 3 && 4 && 5 && 6 && 7 && 9 && 5 && 6 \\
    7 && 4 && 6 && 7 && 9 && 10 && 11 && 8 && 7 && 4 && 5 && 8 && 7 \\
    8 && 9 && 5 && 8 && 7 && 9 && 10 && 11 && 8 && 9 && 10 && 12 && 8
  \end{tabular}
\end{center}
$12z = 11z = 8$なので$12 = 11$.
8行目を書き換えて9, 10, 11行目を計算する.

\begin{center}
  \scalebox{0.9}{
  \begin{tabular}{c|cccccc|cccccccccc|cccccccc}
    & $y$ && $y$ && $y$ && $z$ && $z$ && $z$ && $z$ && $z$ && $y$ && $z$ && $y$ && $z$ & \\ \hline
    1 && 2 && 3 && 1 && 1 && 1 && 1 && 1 && 1 && 2 && 3 && 1 && 1 \\
    2 && 3 && 1 && 2 && 3 && 4 && 5 && 6 && 2 && 3 && 4 && 6 && 2 \\
    3 && 1 && 2 && 3 && 4 && 5 && 6 && 2 && 3 && 1 && 1 && 2 && 3 \\
    4 && 6 && 7 && 4 && 5 && 6 && 2 && 3 && 4 && 6 && 2 && 3 && 4 \\
    5 && 8 && 9 && 5 && 6 && 2 && 3 && 4 && 5 && 8 && 7 && 4 && 5 \\
    6 && 7 && 4 && 6 && 2 && 3 && 4 && 5 && 6 && 7 && 9 && 5 && 6 \\
    7 && 4 && 6 && 7 && 9 && 10 && 11 && 8 && 7 && 4 && 5 && 8 && 7 \\
    8 && 9 && 5 && 8 && 7 && 9 && 10 && 11 && 8 && 9 && 10 && 11 && 8 \\
    9 && 5 && 8 && 9 && 10 && 11 && 8 && 7 && 9 && 5 && 6 && 7 && 9 \\
    10 && 11 && 12 && 10 && 11 && 8 && 7 && 9 && 10 && 11 && 8 && 9 && 10 \\
    11 && 12 && 10 && 11 && 8 && 7 && 9 && 10 && 11 && 12 && 13 && 14 && 11
  \end{tabular}
  }
\end{center}
$14z = 10z = 11$なので$14 = 10$.$13y = 14 = 10 = 12y$なので$13 = 12$.
11行目を書き換えて12行目を計算する.

\begin{center}
  \scalebox{0.9}{
  \begin{tabular}{c|cccccc|cccccccccc|cccccccc}
    & $y$ && $y$ && $y$ && $z$ && $z$ && $z$ && $z$ && $z$ && $y$ && $z$ && $y$ && $z$ & \\ \hline
    1 && 2 && 3 && 1 && 1 && 1 && 1 && 1 && 1 && 2 && 3 && 1 && 1 \\
    2 && 3 && 1 && 2 && 3 && 4 && 5 && 6 && 2 && 3 && 4 && 6 && 2 \\
    3 && 1 && 2 && 3 && 4 && 5 && 6 && 2 && 3 && 1 && 1 && 2 && 3 \\
    4 && 6 && 7 && 4 && 5 && 6 && 2 && 3 && 4 && 6 && 2 && 3 && 4 \\
    5 && 8 && 9 && 5 && 6 && 2 && 3 && 4 && 5 && 8 && 7 && 4 && 5 \\
    6 && 7 && 4 && 6 && 2 && 3 && 4 && 5 && 6 && 7 && 9 && 5 && 6 \\
    7 && 4 && 6 && 7 && 9 && 10 && 11 && 8 && 7 && 4 && 5 && 8 && 7 \\
    8 && 9 && 5 && 8 && 7 && 9 && 10 && 11 && 8 && 9 && 10 && 11 && 8 \\
    9 && 5 && 8 && 9 && 10 && 11 && 8 && 7 && 9 && 5 && 6 && 7 && 9 \\
    10 && 11 && 12 && 10 && 11 && 8 && 7 && 9 && 10 && 11 && 8 && 9 && 10 \\
    11 && 12 && 10 && 11 && 8 && 7 && 9 && 10 && 11 && 12 && 12 && 10 && 11 \\
    12 && 10 && 11 && 12 && 12 && 12 && 12 && 12 && 12 && 10 && 11 && 12 && 12
  \end{tabular}
  }
\end{center}
以上から,$H \backslash G$の代表は12個.

\part{環と体とGalois理論(第1版第9刷)}
\setcounter{chapter}{1}
\chapter{環上の加群}
\setcounter{section}{5}
\section{\(\GL_n(\mathbb{Z}/m\mathbb{Z})\)}
\paragraph{定理2.6.19}~
\begin{screen}
  \(NF=G\)
\end{screen}
\begin{proof}
  まず,\(SL_n(K)\)の元による\(F\)の共役が\(U\)を含むことを示す.
  \(\sigma(1)=i\), \(\sigma(2)=j\)となる\(\sigma\in\mathfrak{S}(n)\)を適当に定め\((M)_{\alpha\beta} = \delta_{\alpha\sigma(\beta)} \in \SL_n(K)\)とする.
  \(R_{n,12}(c) \in F\)であるので,
  \begin{align*}
    (MR_{n,12}(c))_{\alpha\gamma} &= (M)_{\alpha\beta} (R_{n,12}(c))_{\beta\gamma}
    = \delta_{\alpha\sigma(\beta)} (\delta_{\beta\gamma} + c\delta_{\beta1}\delta_{\gamma2})
    = \delta_{\sigma^{-1}(\alpha)\beta} (\delta_{\beta\gamma} + c\delta_{\beta1}\delta_{\gamma2}) \\
    &= \delta_{\sigma^{-1}(\alpha)\gamma} + c \delta_{\sigma^{-1}(\alpha)1}\delta_{\gamma2}
    = \delta_{\alpha\sigma(\gamma)} + c \delta_{\alpha i}\delta_{\gamma2}.
  \end{align*}
  さらに
  \begin{align*}
    (R_{n,ij}(c)M)_{\alpha\gamma} &= (R_{n,ij}(c))_{\alpha\beta} (M)_{\beta\gamma}
    = (\delta_{\alpha\beta} + c\delta_{\alpha i}\delta_{\beta j}) \delta_{\beta\sigma(\gamma)}
    = \delta_{\alpha\sigma(\gamma)} + c \delta_{\alpha i}\delta_{j\sigma(\gamma)} \\
    = \delta_{\alpha\sigma(\gamma)} + c \delta_{\alpha i}\delta_{\gamma2}
  \end{align*}
  なので\(MR_{n,12}(c) = R_{n,ij}(c)M\)すなわち\(MR_{n,12}(c)M^{-1} = R_{n,ij}(c)\)となる.
  \(NF\triangleleft NP=G\)なので\(U \leq NF\).命題2.6.12から\(G=NF\)となる.
\end{proof}

\setcounter{section}{11}
\section{単項イデアル整域上の有限生成加群}
\paragraph{定理2.12.1}
構成された同型について.
\(M=\braket{x_1, \ldots, x_m}\)である.全射準同型
\[ \phi \colon R^m \ni \boldsymbol{e}_i \mapsto x_i \in M \]
の核の生成元を\(\set{y_1, \ldots, y_n}\)とする:\(\ker\phi = \braket{y_1, \ldots, y_n} \subset R^m\).さらに
\[ f \colon R^n \ni \boldsymbol{e}_j' \mapsto y_j \in R^m \]
とする.準同型定理から
\[
\Coker(f) = R^m / \Im f = R^m / \ker\phi \simeq \Im\phi = M .
\]
よって,\(x_i \in M\)は\([\boldsymbol{e_i}] \in R^m / \Im f\)に対応する.
さらに\(Im(f) = \set{(e_1r_1, \ldots, e_tr_t, 0, \ldots, 0)}\)となるので,
\[ M \ni x_i \mapsto (\ldots, 0, 1, 0, \ldots) \in R/(e_1) \oplus \cdots \oplus R/(e_t) \oplus R^{m-t} \]
に対応する.

\section{完全系列と局所化}
\paragraph{例2.13.12}
\begin{enumerate}[label=(\arabic*)]
  \item \(u=x+iy\), \(v=x-iy\)とすれば\(\mathbb{C}[x,y]/(x^2+y^2) \simeq \mathbb{C}[u,v]/(uv)\)が分かる.
\end{enumerate}

\chapter{体論の基本}
\setcounter{section}{2}
\section{分離拡大}
\paragraph{命題3.3.5}~
\begin{screen}
  (3)の\(n\)は一意に定まる
\end{screen}
\begin{proof}
  主張を満たす\(n\)が一意に定まらないと仮定する.
  \(f(x) = g(x^{p^m}) = h(x^{p^n})\)を満たす\(n > m > 0\)及び既約分離多項式\(g(x)\), \(h(x)\)が存在する.
  \(g(x) = h(x^{p^{n-m}})\)となるので\(g'(x)=0\).命題3.3.5の主張より\(g(x)\)は重根を持ち,分離性に矛盾する.
\end{proof}

\chapter{Galois理論}
\setcounter{section}{5}
\section{Galois拡大の推進定理}
\paragraph{定理4.6.1}~
\begin{screen}
  \(\sigma(M) \subset \bar{K} \cap L\)
\end{screen}
\begin{proof}
  \(\sigma \in \Gal(L/N)\)なので\(\sigma(M) \subset L\).
  \(x\in M\)とする.\(\sigma(x)\)は\(x\in M\subset L\)の\(N\)上の共役である.
  すなわち\(\sigma(x)\)は\(x\in L\)の\(N\)上最小多項式の根.
  命題3.1.24から\(\sigma(x)\)は\(x\in L\)の\(K\)上最小多項式の根なので\(\sigma(x) \in \bar{K}\).
\end{proof}

\setcounter{section}{10}
\section{正規底}
\paragraph{定理4.11.2}
定理3.6.3より\(f(x_1, \ldots, x_n) \neq 0\)となる\(x\in L\)が存在すれば,\(f(x_1, \ldots, x_n) \neq 0\)となる\(x\in K\)が存在する.

系4.10.3から\(\sum_k \sigma_i(a_k)x_k=\delta_{1i}\)となる\(x_k \in L\)が存在する.
\(\sigma_1 = 1\)としているので
\[ \sum_{k=1}^n \sigma_i{}^{-1} \circ \sigma_i(a_k) x_k = \sum_{k=1}^n a_k x_k = \sum_{k=1}^n \sigma_1(a_k)x_k = \delta_{1i} = 1 . \]
\(\Gal(L/K)\)において\(\sigma_i{}^{-1} \circ \sigma_j = \sigma_{p(i,j)}\)と定める.
\(i\neq j\)なら\(\sigma_{p(i,j)} = \sigma_i{}^{-1} \circ \sigma_j \neq 1 = \sigma_1\)なので\(p(i,j) \neq 1\).
よって
\[ \sum_{k=1}^n \sigma_i{}^{-1} \circ \sigma_j(a_k) x_k = \sum_{k=1}^n \sigma_{p(i,j)}(a_k)x_k = \delta_{i1(i,j)} = 0 \quad (i\neq j) . \]
以上から
\[ \sum_{k=1}^n \sigma_i{}^{-1} \circ \sigma_j(a_k) x_k = \delta_{ij} . \]

\paragraph{定理4.11.4}~
\begin{screen}
  \(x^n-1 = \LCM(p_1(x)^{a_1}, \ldots, p_m(x)^{a_m}) =: L(x)\)
\end{screen}
\begin{proof}
  \(L\)は\(K[x]\)加群として有限生成であるが,その生成元を\(\set{\alpha_1, \ldots, \alpha_m}\)とする.
  単項イデアル整域上の有限生成加群の構造定理2.12.1(と証明における同型の構成)から同型
  \[
  \Phi\colon L \ni \sum_{i=1}^m f_i(\sigma) \alpha_i \mapsto (f_i(x)+(p_i(x)^{a_i}))_i \in \bigoplus_{i=1}^m K[x] / (p_i(x)^{a_i})
  \]
  を得る.\(g(x) \in I\)なら\(0 = g(\sigma) \alpha_i \mapsto 0\)なので\(g(x) \in (p_i(x)^{a_i})\)である.
  これが任意の\(i\)に対して成立するので\(L(x) \mid g(x)\).特に\(L(x) \mid x^n-1\)である.

  任意の\(\alpha \in L\)に対して\(L(\sigma)\alpha=0\)となることを示す.
  \(\alpha = \sum f_i(\sigma) \alpha_i\)となる\(f_i(x) \in K[x]\)が存在する.最小公倍元の定義から
  \[ \Phi(L(\sigma)\alpha) = (L(x)f_i(x)+(p_i(x)^{a_i}))_i = 0 . \]
  \(\Phi\)は単射なので\(L\alpha = 0\).すなわち\(L(x) \in I\).
  従って\(x^n-1 \mid L(x)\).

  以上から\(x^n-1 = L(x)\).
\end{proof}

\section{トレース・ノルム}
\paragraph{命題4.12.6}~
\begin{screen}
  \(\alpha\)が非分離的で\(L=K(\alpha)\)の場合.\(p^m = [L:K]_i\)
\end{screen}
\begin{proof}
  命題3.3.5から分離既約多項式\(g(x)\in K[x]\)によって\(\alpha\in L\)の\(K\)上最小多項式は\(g(x^{p^m})\)となる.
  \(g(x)\)は\(\alpha^{p^m} \in L\)を根に持つ.
  もし\(h(\alpha^{p^m}) = 0\)かつ\(\deg h < \deg g\)となる\(h(x) \in K[x]\)が存在すれば,\(h(x^{p^m})\)も\(\alpha\)を根に持ち,\(g\)の最小性に矛盾する.
  よって\(g(x)\)は\(\alpha^{p^m} \in L\)の\(K\)上最小多項式である.
  従って,\(K(\alpha^{p^m})\)は\(K\)の分離拡大であり,\([K(\alpha^{p^m}):K]=\deg g(x) = n\).
  さらに\([L:K]=\deg g(x^{p^m})=np^m\).\(L/K(\alpha^{p^m})\)が純非分離拡大であることは容易に分かる.

  \(L\)における\(K\)の分離閉包を\(L_s\)とする.体の拡大列\(K \subset K(\alpha^{p^m}) \subset L_s \subset L=K(\alpha)\)を得る.
  命題3.3.27から\(L_s/K\)は分離拡大,\(L/L_s\)は純非分離拡大である.

  \(L_s\subset L\)なので\(L_s/K(\alpha^{p^n})\)も純非分離拡大.
  命題3.3.2から\(L_s/K(\alpha^{p^m})\)は分離拡大でもある.
  命題3.3.14と併せれば\(L_s = K(\alpha^{p^m})\)と分かる.

  以上から\([L:K]_i = [L:K(\alpha^{p^m})] = p^m\).
\end{proof}

\paragraph{命題4.12.13}~
\begin{screen}
  有限体の乗法群は巡回群
\end{screen}
\begin{proof}
  \(\# K^\times = n\)とする.位数\(d\mid n\)の元\(\alpha\in K^\times\)が存在すれば,\(\set{1, \alpha, \ldots, \alpha^{d-1}}\)は全て相異なり,\(x^d=1\)を満たす.
  \(x^d=1\)は高々\(d\)個の解しか持たないので,
  \(x^d=1\)を満たす\(x\in K\)は\(\alpha^i\)という形をしている.
  \(\alpha^i\)の位数が\(d\)となるのは\(i\)が\(d\)と互いに素な場合なので,\(\phi(d)\)個存在する.
  位数が\(d\)の元の集合を\(G_d\)とすれば,\(\# G_d\)は\(0\)か\(\phi(d)\)である.
  \[ n = \# K^\times = \sum{d\mid n} \# G_d \leq \sum{d\mid n} \phi(d) = n  \]
  となるので,全ての\(d\mid n\)に対して\(\# G_d = \phi(d)\)である.
  特に位数\(n\)の元が存在するので\(K^\times\)は巡回群.
\end{proof}

\paragraph{例4.12.14}
定理4.9.7において\(R = \set{2^l3^m (K^\times)^p}\)とすれば\(\Gal(K(\sqrt[p]{2}, \sqrt[p]{3})/K) \simeq R/(K^\times)^p\)である.全射準同型
\[ \phi \colon \mathbb{Z} \times \mathbb{Z} \ni (l, m) \mapsto 2^l3^m (K^\times)^p \in R/(K^\times)^p \]
を考える.\((l,m) \in \ker\phi\)とする.\(2^l3^m = x^p\)となる\(x\in K^\times\)が存在する.
ノルムを考えれば
\[ 2^{l(p-1)}3^{m(p-1)} = \N_{K/\mathbb{Q}}(x)^p \in \mathbb{Q}^p \]
であるので\(p \mid l, m\)である.よって\(\ker\phi = p\mathbb{Z} \times p\mathbb{Z}\)である.
よって準同型定理から
\[ \Gal(K(\sqrt[p]{2}, \sqrt[p]{3})/K) \simeq R/(K^\times)^p \simeq (\mathbb{Z} \times \mathbb{Z}) / (p\mathbb{Z} \times p\mathbb{Z}) \simeq \mathbb{Z}/p\mathbb{Z} \times \mathbb{Z}/p\mathbb{Z} . \]

\setcounter{section}{15}
\section{4次多項式のGalois群}
\paragraph{命題4.16.3}
(1)証明に出てくる\(\phi\)は命題4.4.8で考えた制限写像\(\phi\colon\Gal(L/K)\to\Gal(M/K)\).

(3) \(\Gal(L/K)=\braket{(1234)}\)の場合.
\((1234)\)により\(\tau_1\leftrightarrow\tau_3\)および\(\tau_2\mapsto\tau_2\)であるので,Galois理論の基本定理から\(\tau_2 \in K\)および\(\tau_1, \tau_3 \in L\setminus K\)である.
よって\(g(y)=(y-\tau_1)(y-\tau_2)(y-\tau_3)\)は\(K\)上の一次式\(y-\tau_2\)と\(K\)上既約な二次式\((y-\tau_1)(y-\tau_3)\)の積である.

\(h(z)\)の根\(\beta_1\), \(\beta_2\)は(4.5.2)と同様に構成される:
\[
\beta_1 = \tau_1{}^2 \tau_2 + \tau_2{}^2 \tau_3 + \tau_3{}^2 \tau_1 , \quad
\beta_1 = \tau_1 \tau_2{}^2 + \tau_2 \tau_3{}^2 + \tau_3 \tau_1{}^2 .
\]
\((1234) \in\Gal(L/K)\)により\(\tau_1\leftrightarrow\tau_3\)および\(\tau_2\mapsto\tau_2\)であるので,\(\beta_1 \leftrightarrow \beta_2\).
よって\(\beta_1, \beta_2 \in L\setminus K\)である.
従って\(h(z)=(z-\beta_1)(z-\beta_2)\)は\(K\)上既約な二次式である.

\paragraph{定理 4.16.18}~
\begin{screen}
  \(\ch K = 2\)なら
  \[ \set{x^2+x | x \in K(\tau_1) , x^2+x \in K} = \set{\alpha d_2 d_1{}^{-2} + \beta^2 + \beta | \alpha \in \mathbb{F}_2 , \beta \in K} . \]
\end{screen}
\begin{proof}
  \(\tau_1\)は\(g(y) = y^2 + d_1 y + d_2 = y^2 - d_1 y - d_2 \in K[y]\)の根である.
  \(d_1 \neq 0\)なので\(\tau_1d_1{}^{-1}\)は\(y^2-y-d_2d_1{}^{-2}\)の根となる.

  \(x^2+x \in K\)となる\(x \in K(\tau_1d_1{}^{-1}) = K(\tau_1)\)が存在すれば,補題4.15.2から
  \[ x=\beta+\alpha \tau_1 , \quad x^2+x = \beta^2 + \beta + \alpha d_2d_1{}^{-2} \]
  となる\(\alpha\in\mathbb{F}_2\)と\(\beta\in K\)が存在する.よって
  \[ \set{x^2+x | x \in K(\tau_1) , x^2+x \in K} \subset \set{\alpha d_2 d_1{}^{-2} + \beta^2 + \beta | \alpha \in \mathbb{F}_2 , \beta \in K} . \]

  \(\alpha \in \mathbb{F}_2 , \beta \in K\)とする.\(\tau_1d_1{}^{-1}\)は\(y^2-y-d_2d_1{}^{-2}\)の根なので,
  \begin{align*}
    K \ni \alpha d_2 d_1{}^{-2} + \beta^2 + \beta
    &= \alpha \left[ (\tau_1d_1{}^{-1})^2 + \tau_1d_1{}^{-1} \right] + \beta^2 + \beta \\
    &= \alpha (\tau_1d_1{}^{-1})^2 + \alpha \tau_1d_1{}^{-1} + \beta^2 + \beta \\
    &= \alpha^2 (\tau_1d_1{}^{-1})^2 + \alpha \tau_1d_1{}^{-1} + \beta^2 + \beta \\
    &= ( \alpha \tau_1d_1{}^{-1} + \beta)^2 + (\alpha \tau_1d_1{}^{-1} + \beta) .
  \end{align*}
  \( \alpha \tau_1d_1{}^{-1} + \beta \in K(\tau_1)\)なので
  \[ \set{x^2+x | x \in K(\tau_1) , x^2+x \in K} \supset \set{\alpha d_2 d_1{}^{-2} + \beta^2 + \beta | \alpha \in \mathbb{F}_2 , \beta \in K} . \]
\end{proof}

\end{document}
