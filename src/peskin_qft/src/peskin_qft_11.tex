\section*{Problems}\addcontentsline{toc}{section}{Problems}
\subsection{Problem 16.3: Counterterm relations}
\subsubsection{(b)}
3ボソン頂点3つのダイアグラムを考える.
\begin{center}
  \begin{tikzpicture}
    \begin{feynman}
      \vertex (i1) at (90: 0.8);
      \vertex (i2) at (210: 0.8);
      \vertex (i3) at (330: 0.8);
      \vertex (o1) at (90: 2) {$a\mu$};
      \vertex (o2) at (210: 2) {$b\nu$};
      \vertex (o3) at (330: 2) {$c\rho$};
      \diagram*{
      (i1) -- [boson, half right, looseness=1, momentum'=$p-k$, edge label=$d\sigma$] (i2);
      (i3) -- [boson, half left, looseness=1, momentum={[arrow shorten=0.3]$k$}, edge label'=$e\lambda$] (i2);
      (i1) -- [boson, half left, looseness=1, momentum={[arrow shorten=0.3]$k$}, edge label'=$f\kappa$] (i3);
      (o1) -- [boson, momentum=$p$] (i1);
      (i2) -- [boson, momentum=$p$] (o2);
      (o3) -- [boson] (i3);
      };
    \end{feynman}
  \end{tikzpicture}
\end{center}

分母は$k^4(k-p)^2$なので,$k=\ell+xp$で,$\ell^2$の項のみが発散する.分子は\footnote{\verb|./src/py/NonAbelian_1loop.ipynb|}
\begin{align*}
  (\text{Num}) &=
  2 k^2 g^{\mu\nu} k^{\rho}
  + 4 k^2 g^{\mu\nu} p^{\rho}
  + 2 k^2 g^{\mu\rho} k^{\nu}
  - 3 k^2 g^{\mu\rho} p^{\nu}
  \\ & \quad
  + 2 k^2 g^{\nu\rho} k^{\mu}
  - 3 k^2 g^{\nu\rho} p^{\mu}
  - 8 (k \cdot p) g^{\mu\nu} p^{\rho}
  - 2 (k \cdot p) g^{\mu\rho} k^{\nu}
  \\ & \quad
  + 3 (k \cdot p) g^{\mu\rho} p^{\nu}
  - 2 (k \cdot p) g^{\nu\rho} k^{\mu}
  + 3 (k \cdot p) g^{\nu\rho} p^{\mu}
  + 10 p^2 g^{\mu\nu} k^{\rho}
  \\ & \quad
  - p^2 g^{\mu\rho} k^{\nu}
  - p^2 g^{\nu\rho} k^{\mu}
  + 18 k^{\mu} k^{\nu} k^{\rho}
  - 9 k^{\mu} k^{\rho} p^{\nu}
  \\ & \quad
  + 3 k^{\mu} p^{\nu} p^{\rho}
  - 9 k^{\nu} k^{\rho} p^{\mu}
  + 3 k^{\nu} p^{\mu} p^{\rho}
  - 6 k^{\rho} p^{\mu} p^{\nu}
  \\
  &\sim
   2 k^2 g^{\mu\nu} k^{\rho}
  + 4 k^2 g^{\mu\nu} p^{\rho}
  + 2 k^2 g^{\mu\rho} k^{\nu}
  - 3 k^2 g^{\mu\rho} p^{\nu}
  \\ & \quad
  + 2 k^2 g^{\nu\rho} k^{\mu}
  - 3 k^2 g^{\nu\rho} p^{\mu}
  - 2 (k \cdot p) g^{\mu\rho} k^{\nu}
  \\ & \quad
  - 2 (k \cdot p) g^{\nu\rho} k^{\mu}
  \\ & \quad
  + 18 k^{\mu} k^{\nu} k^{\rho}
  - 9 k^{\mu} k^{\rho} p^{\nu}
  \\ & \quad
  - 9 k^{\nu} k^{\rho} p^{\mu}
  \\
  &\sim
  2 g^{\mu\nu} (\ell+xp)^2 (\ell+xp)^{\rho}
  + 4 g^{\mu\nu} (\ell+xp)^2 p^{\rho}
  + 2 g^{\mu\rho} (\ell+xp)^2 (\ell+xp)^{\nu}
  - 3 g^{\mu\rho} (\ell+xp)^2 p^{\nu}
  \\ & \quad
  + 2 g^{\nu\rho} (\ell+xp)^2 (\ell+xp)^{\mu}
  - 3g^{\nu\rho} (\ell+xp)^2 p^{\mu}
  - 2 g^{\mu\rho} p \cdot (\ell+xp) (\ell+xp)^{\nu}
  \\ & \quad
  - 2 g^{\nu\rho} p \cdot (\ell+xp) (\ell+xp)^{\mu}
  \\ & \quad
  + 18 (\ell+xp)^{\mu} (\ell+xp)^{\nu} (\ell+xp)^{\rho}
  - 9 (\ell+xp)^{\mu} (\ell+xp)^{\rho} p^{\nu}
  \\ & \quad
  - 9 (\ell+xp)^{\nu} (\ell+xp)^{\rho} p^{\mu}
  \\
  &\sim
   2 g^{\mu\nu} [xp^\rho\ell^2 + 2x (p \cdot\ell) \ell^\rho]
  + 4 g^{\mu\nu} p^{\rho} \ell^2
  + 2 g^{\mu\rho} [xp^\nu\ell^2 + 2x (p \cdot\ell) \ell^\nu]
  - 3 g^{\mu\rho} p^{\nu} \ell^2
  \\ & \quad
  + 2 g^{\nu\rho} [xp^\mu\ell^2 + 2x (p \cdot\ell) \ell^\mu]
  - 3g^{\nu\rho} p^{\mu} \ell^2
  - 2 g^{\mu\rho} (p \cdot\ell) \ell^{\nu}
  - 2 g^{\nu\rho} (p \cdot\ell) \ell^{\mu}
  \\ & \quad
  + 18x (p^\mu \ell^\nu\ell^\rho + p^\nu \ell^\rho\ell^\mu + p^\rho \ell^\mu\ell^\nu)
  - 9 p^{\nu} \ell^{\mu} \ell^{\rho}
  - 9 p^{\mu} \ell^{\nu} \ell^{\rho}
  \\
  &\sim
  3x g^{\mu\nu} p^\rho \ell^2
  + 4 g^{\mu\nu} p^{\rho} \ell^2
  + 3x g^{\mu\rho} p^\nu\ell^2
  - 3 g^{\mu\rho} p^{\nu} \ell^2
  \\ & \quad
  + 3x g^{\nu\rho} p^\mu \ell^2
  - 3g^{\nu\rho} p^{\mu} \ell^2
  - \frac{1}{2} g^{\mu\rho} p^\nu \ell^2
  - \frac{1}{2} g^{\nu\rho} p^\mu \ell^2
  \\ & \quad
  + \frac{18x-9}{4} g^{\nu\rho} p^\mu \ell^2 + \frac{18x-9}{4} g^{\mu\rho} p^\nu \ell^2 + \frac{9}{2} x g^{\mu\nu} p^\rho \ell^2
  \\
  &= \frac{15x+8}{2} g^{\mu\nu} p^\rho \ell^2 + \frac{30x-23}{4} (g^{\mu\rho} p^\nu + g^{\nu\rho} p^\mu) \ell^2 .
\end{align*}

\subsubsection{(c)}
4ボソン頂点2つのダイアグラムを考える.
\begin{center}
  \begin{tikzpicture}
    \begin{feynman}
      \vertex (i1) at (90: 0.6);
      \vertex (i2) at (270: 0.6);
      \vertex (o1) at (120: 1.5) {$a\mu$};
      \vertex (o2) at (60: 1.5) {$b\nu$};
      \vertex (o3) at (240: 1.5) {$c\rho$};
      \vertex (o4) at (300: 1.5) {$d\sigma$};
      \diagram*{
      (i1) -- [boson, half right, looseness=1.6, edge label'=$e\lambda$] (i2);
      (i1) -- [boson, half left, looseness=1.6, edge label=$f\kappa$] (i2);
      (o1) -- [boson] (i1) -- [boson] (o2);
      (o3) -- [boson] (i2) -- [boson] (o4);
      };
    \end{feynman}
    \begin{scope}[xshift=5cm]
      \begin{feynman}
        \vertex (i1) at (0: 0.6);
        \vertex (i2) at (180: 0.6);
        \vertex (o1) at (150: 1.5) {$a\mu$};
        \vertex (o2) at (30: 1.5) {$b\nu$};
        \vertex (o3) at (210: 1.5) {$c\rho$};
        \vertex (o4) at (330: 1.5) {$d\sigma$};
        \diagram*{
        (i1) -- [boson, half right, looseness=1.6] (i2);
        (i1) -- [boson, half left, looseness=1.6] (i2);
        (o1) -- [boson] (i2) -- [boson] (o3);
        (o2) -- [boson] (i1) -- [boson] (o4);
        };
      \end{feynman}
    \end{scope}
    \begin{scope}[xshift=10cm]
      \begin{feynman}
        \vertex (i1) at (0: 0.6);
        \vertex (i2) at (180: 0.6);
        \vertex (o1) at (150: 1.5) {$a\mu$};
        \vertex (o2) at (30: 1.5) {$b\nu$};
        \vertex (o3) at (210: 1.5) {$c\rho$};
        \vertex (o4) at (330: 1.5) {$d\sigma$};
        \diagram*{
        (i1) -- [boson, half right, looseness=1.2] (i2);
        (i1) -- [boson, half left, looseness=1.2] (i2);
        (o1) -- [boson, quarter left, looseness=1.2] (i1) -- [boson] (o4);
        (o2) -- [boson, quarter right, looseness=1.2] (i2) -- [boson] (o3);
        };
      \end{feynman}
    \end{scope}
  \end{tikzpicture}
\end{center}

1つ目のダイアグラムは
\begin{align*}
  & \left[ f_{abg} f_{efg} \left(- g_{\mu\kappa} g_{\nu\lambda} + g_{\mu\lambda} g_{\nu\kappa}\right)
  + f_{aeg} f_{bfg} \left(g_{\kappa\lambda} g_{\mu\nu} - g_{\mu\kappa} g_{\nu\lambda}\right)
  + f_{afg} f_{beg} \left(g_{\kappa\lambda} g_{\mu\nu} - g_{\mu\lambda} g_{\nu\kappa}\right) \right] \\
  &\quad\times \left[ f_{cdh} f_{efh} \left(- g_{\rho\kappa} g_{\sigma\lambda} + g_{\rho\lambda} g_{\sigma\kappa}\right)
  + f_{ech} f_{fdh} \left(g_{\kappa\lambda} g_{\rho\sigma} - g_{\rho\kappa} g_{\sigma\lambda}\right)
  + f_{edh} f_{fch} \left(g_{\kappa\lambda} g_{\rho\sigma} - g_{\rho\lambda} g_{\sigma\kappa}\right) \right] \\
  &=
  2 f^{abg} f^{cdh} f^{efg} f^{efh} g^{\mu\rho} g^{\nu\sigma}
  - 2 f^{abg} f^{cdh} f^{efg} f^{efh} g^{\mu\sigma} g^{\nu\rho}
  \\ &\quad
  + f^{abg} f^{ech} f^{efg} f^{fdh} g^{\mu\rho} g^{\nu\sigma}
  - f^{abg} f^{ech} f^{efg} f^{fdh} g^{\mu\sigma} g^{\nu\rho}
  \\ &\quad
  - f^{abg} f^{edh} f^{efg} f^{fch} g^{\mu\rho} g^{\nu\sigma}
  + f^{abg} f^{edh} f^{efg} f^{fch} g^{\mu\sigma} g^{\nu\rho}
  \\ &\quad
  + f^{aeg} f^{bfg} f^{cdh} f^{efh} g^{\mu\rho} g^{\nu\sigma}
  - f^{aeg} f^{bfg} f^{cdh} f^{efh} g^{\mu\sigma} g^{\nu\rho}
  \\ &\quad
  + 2 f^{aeg} f^{bfg} f^{ech} f^{fdh} g^{\mu\nu} g^{\rho\sigma}
  + f^{aeg} f^{bfg} f^{ech} f^{fdh} g^{\mu\rho} g^{\nu\sigma}
  \\ &\quad
  + 2 f^{aeg} f^{bfg} f^{edh} f^{fch} g^{\mu\nu} g^{\rho\sigma}
  + f^{aeg} f^{bfg} f^{edh} f^{fch} g^{\mu\sigma} g^{\nu\rho}
  \\ &\quad
  - f^{afg} f^{beg} f^{cdh} f^{efh} g^{\mu\rho} g^{\nu\sigma}
  + f^{afg} f^{beg} f^{cdh} f^{efh} g^{\mu\sigma} g^{\nu\rho}
  \\ &\quad
  + 2 f^{afg} f^{beg} f^{ech} f^{fdh} g^{\mu\nu} g^{\rho\sigma}
  + f^{afg} f^{beg} f^{ech} f^{fdh} g^{\mu\sigma} g^{\nu\rho}
  \\ &\quad
  + 2 f^{afg} f^{beg} f^{edh} f^{fch} g^{\mu\nu} g^{\rho\sigma}
  + f^{afg} f^{beg} f^{edh} f^{fch} g^{\mu\rho} g^{\nu\sigma}
  \\
  %
  & =
  2 f^{abg} f^{cdh} f^{efg} f^{efh} (g^{\mu\rho} g^{\nu\sigma} - g^{\mu\sigma} g^{\nu\rho})
  \\ &\quad
  + (f^{abg} f^{ech} f^{efg} f^{fdh} - f^{abg} f^{edh} f^{efg} f^{fch})
  (g^{\mu\rho} g^{\nu\sigma} - g^{\mu\sigma} g^{\nu\rho})
  \\ &\quad
  + (f^{aeg} f^{bfg} f^{cdh} f^{efh} - f^{afg} f^{beg} f^{cdh} f^{efh})
  (g^{\mu\rho} g^{\nu\sigma} - g^{\mu\sigma} g^{\nu\rho})
  \\ &\quad
  + (f^{aeg} f^{bfg} f^{ech} f^{fdh} + f^{afg} f^{beg} f^{edh} f^{fch})
  (2 g^{\mu\nu} g^{\rho\sigma} + g^{\mu\rho} g^{\nu\sigma})
  \\ &\quad
  + (f^{aeg} f^{bfg} f^{edh} f^{fch} + f^{afg} f^{beg} f^{ech} f^{fdh})
  (2 g^{\mu\nu} g^{\rho\sigma} + g^{\mu\sigma} g^{\nu\rho})
  \\
  %
  & =
  2 f^{abg} f^{cdh} f^{efg} f^{efh} (g^{\mu\rho} g^{\nu\sigma} - g^{\mu\sigma} g^{\nu\rho})
  \\ &\quad
  + 2 f^{abg} f^{ech} f^{efg} f^{fdh} (g^{\mu\rho} g^{\nu\sigma} - g^{\mu\sigma} g^{\nu\rho})
  \\ &\quad
  + 2 f^{aeg} f^{bfg} f^{cdh} f^{efh} (g^{\mu\rho} g^{\nu\sigma} - g^{\mu\sigma} g^{\nu\rho})
  \\ &\quad
  + 2 f^{aeg} f^{bfg} f^{ech} f^{fdh} (2 g^{\mu\nu} g^{\rho\sigma} + g^{\mu\rho} g^{\nu\sigma})
  \\ &\quad
  + 2 f^{aeg} f^{bfg} f^{edh} f^{fch} (2 g^{\mu\nu} g^{\rho\sigma} + g^{\mu\sigma} g^{\nu\rho})
\end{align*}
1行目は
\[
2 f^{abg} f^{cdh} f^{efg} f^{efh} = 2 f^{abg} f^{cdh} C_2(G) \delta^{gh} = 2 f^{abg} f^{cdg} C_2(G) .
\]
(16.79)から
\begin{align}
  i \Tr(t_G^a t_G^b t_G^c) = f^{ade} f^{ebf} f^{fdc} = - \frac{1}{2} f^{abc} C_2(G)
  \label{prob13_6_3trace_adj}
\end{align}
なので,2行目は
\[
2 f^{abg} f^{ech} f^{efg} f^{fdh} = - 2 f^{abg} f^{ceh} f^{hdf} f^{feg} = f^{abg} f^{cdg} C_2(G) .
\]
同様に3行目は
\[
2 f^{aeg} f^{bfg} f^{cdh} f^{efh} = - 2 f^{cdh} f^{aeg} f^{gbf} f^{feh} = f^{cdh} f^{abh} C_2(G) .
\]
以上から,
\begin{align*}
  &= 4 C_2(G) f^{abg} f^{cdg} (g^{\mu\rho} g^{\nu\sigma} - g^{\mu\sigma} g^{\nu\rho})
  \\ &\quad
  + 2 f^{aeg} f^{bfg} f^{ech} f^{fdh} (2 g^{\mu\nu} g^{\rho\sigma} + g^{\mu\rho} g^{\nu\sigma})
  \\ &\quad
  + 2 f^{aeg} f^{bfg} f^{edh} f^{fch} (2 g^{\mu\nu} g^{\rho\sigma} + g^{\mu\sigma} g^{\nu\rho}) \\
  %
  &= 4 C_2(G) f^{abg} f^{cdg} (g^{\mu\rho} g^{\nu\sigma} - g^{\mu\sigma} g^{\nu\rho})
  \\ &\quad
  + 2 \Tr(t_G^a t_G^b t_G^d t_G^c) (2 g^{\mu\nu} g^{\rho\sigma} + g^{\mu\rho} g^{\nu\sigma})
  \\ &\quad
  + 2 \Tr(t_G^a t_G^b t_G^c t_G^d) (2 g^{\mu\nu} g^{\rho\sigma} + g^{\mu\sigma} g^{\nu\rho}) .
\end{align*}
随伴表現は
\[ (t_G^{b\dagger})_{ac} = (t_G^b)_{ca}^\ast = (if_{cba})^\ast = -i f_{cba} = if_{abc} = (t_G^b)_{ac} \]
なので$t_G^a$はHermite行列.さらに$t_G^a t_G^b t_G^c t_G^d$は実数なので
\begin{align}
  \begin{split}
    \Tr(t_G^a t_G^b t_G^c t_G^d)
    &= (t_G^a)_{ij} (t_G^b)_{jk} (t_G^c)_{kl} (t_G^d)_{li}
    = (t_G^a)_{ij}^\ast (t_G^b)_{jk}^\ast (t_G^c)_{kl}^\ast (t_G^d)_{li}^\ast \\
    &= (t_G^a)_{ji} (t_G^b)_{kj} (t_G^c)_{lk} (t_G^d)_{il} \\
    &= \Tr(t_G^a t_G^d t_G^c t_G^b) .
  \end{split}
  \label{prob13_6_4trace_adj}
\end{align}
\eqref{prob13_6_3trace_adj}と併せて
\begin{align}
  \begin{split}
    C_2(G) f^{abg} f^{cdg}
    &= -2i \Tr(t_G^a t_G^b t_G^g) f^{cdg} = -2 \Tr(t_G^a t_G^b [t_G^c, t_G^d]) \\
    &= -2 \Tr(t_G^a t_G^b t_G^c t_G^d) + 2 \Tr(t_G^a t_G^b t_G^d t_G^c) .
  \end{split}
  \label{prob13_6_C_2_to_4trace}
\end{align}
よって,
\begin{align*}
  &= 4 [-2 \Tr(t_G^a t_G^b t_G^c t_G^d) + 2 \Tr(t_G^a t_G^b t_G^d t_G^c)] (g^{\mu\rho} g^{\nu\sigma} - g^{\mu\sigma} g^{\nu\rho}) \\
  &\quad + 2 \Tr(t_G^a t_G^b t_G^d t_G^c) (2 g^{\mu\nu} g^{\rho\sigma} + g^{\mu\rho} g^{\nu\sigma}) \\
  &\quad + 2 \Tr(t_G^a t_G^b t_G^c t_G^d) (2 g^{\mu\nu} g^{\rho\sigma} + g^{\mu\sigma} g^{\nu\rho}) \\
  %
  &= 2 \Tr(t_G^a t_G^b t_G^c t_G^d) (2g^{\mu\nu} g^{\rho\sigma} - 4g^{\mu\rho} g^{\nu\sigma} + 5g^{\mu\sigma} g^{\nu\rho}) \\
  &\quad + 2 \Tr(t_G^a t_G^b t_G^d t_G^c) (2g^{\mu\nu} g^{\rho\sigma} + 5g^{\mu\rho} g^{\nu\sigma} - 4g^{\mu\sigma} g^{\nu\rho}) .
\end{align*}

残りのダイアグラムは$(\!(b\nu\leftrightarrow c\rho)\!)$,$(\!(b\nu\leftrightarrow d\sigma)\!)$によって得られる.
合計は
\begin{align*}
  &= 2 \Tr(t_G^a t_G^b t_G^c t_G^d) (2g^{\mu\nu} g^{\rho\sigma} - 4g^{\mu\rho} g^{\nu\sigma} + 5g^{\mu\sigma} g^{\nu\rho}) \\
  &\quad + 2 \Tr(t_G^a t_G^c t_G^b t_G^d) (2g^{\mu\rho} g^{\nu\sigma} - 4g^{\mu\nu} g^{\rho\sigma} + 5g^{\mu\sigma} g^{\nu\rho}) \\
  &\qquad + 2 \Tr(t_G^a t_G^b t_G^c t_G^d) (2g^{\mu\sigma} g^{\nu\rho} - 4g^{\mu\rho} g^{\nu\sigma} + 5g^{\mu\nu} g^{\rho\sigma}) \\
  & + 2 \Tr(t_G^a t_G^b t_G^d t_G^c) (2g^{\mu\nu} g^{\rho\sigma} + 5g^{\mu\rho} g^{\nu\sigma} - 4g^{\mu\sigma} g^{\nu\rho}) \\
  &\quad + 2 \Tr(t_G^a t_G^b t_G^d t_G^c) (2g^{\mu\rho} g^{\nu\sigma} + 5g^{\mu\nu} g^{\rho\sigma} - 4g^{\mu\sigma} g^{\nu\rho}) \\
  &\qquad + 2 \Tr(t_G^a t_G^c t_G^b t_G^d) (2g^{\mu\sigma} g^{\nu\rho} + 5g^{\mu\rho} g^{\nu\sigma} - 4g^{\mu\nu} g^{\rho\sigma}) \\
  %
  &= 2\Tr(t_G^a t_G^b t_G^c t_G^d) [7g^{\mu\nu} g^{\rho\sigma} - 8g^{\mu\rho} g^{\nu\sigma} + 7g^{\mu\sigma} g^{\nu\rho}] \\
  &\quad + 2\Tr(t_G^a t_G^b t_G^d t_G^c) [7g^{\mu\nu} g^{\rho\sigma} + 7g^{\mu\rho} g^{\nu\sigma} - 8g^{\mu\sigma} g^{\nu\rho}] \\
  &\qquad + 2\Tr(t_G^a t_G^c t_G^b t_G^d) [-8g^{\mu\nu} g^{\rho\sigma} + 7g^{\mu\rho} g^{\nu\sigma} + 7g^{\mu\sigma} g^{\nu\rho}] .
\end{align*}

4ボソン頂点1つと3ボソン頂点2つのダイアグラムを考える.
\begin{center}
  \begin{tikzpicture}
    \begin{feynman}
      \vertex (i1) at (150: 1);
      \vertex (i2) at (30: 1);
      \vertex (i3) at (270: 1);
      \vertex (o1) at (140: 2) {$a\mu$};
      \vertex (o2) at (40: 2) {$b\nu$};
      \vertex (o3) at (240: 2) {$c\rho$};
      \vertex (o4) at (300: 2) {$d\sigma$};
      \diagram*{
      (i2) -- [boson, quarter right, looseness=1.2, edge label=$g\tau$, momentum'=$k$] (i1);
      (i1) -- [boson, quarter right, looseness=1.2, edge label=$e\lambda$, momentum'=$k$] (i3);
      (i3) -- [boson, quarter right, looseness=1.2, edge label=$f\kappa$, momentum'=$k$] (i2);
      (o1) -- [boson] (i1);
      (o2) -- [boson] (i2);
      (o3) -- [boson] (i3) -- [boson] (o4);
      };
    \end{feynman}
    \begin{scope}[xshift=5cm]
      \begin{feynman}
        \vertex (i1) at (90: 1);
        \vertex (i2) at (210: 1);
        \vertex (i3) at (330: 1);
        \vertex (o1) at (120: 2) {$a\mu$};
        \vertex (o2) at (60: 2) {$b\nu$};
        \vertex (o3) at (220: 2) {$c\rho$};
        \vertex (o4) at (320: 2) {$d\sigma$};
        \diagram*{
        (i1) -- [boson, quarter right, looseness=1.2] (i2);
        (i2) -- [boson, quarter right, looseness=1.2] (i3);
        (i3) -- [boson, quarter right, looseness=1.2] (i1);
        (o3) -- [boson] (i2);
        (o4) -- [boson] (i3);
        (o1) -- [boson] (i1) -- [boson] (o2);
        };
      \end{feynman}
    \end{scope}
    \begin{scope}[xshift=10cm]
      \begin{feynman}
        \vertex (i1) at (180: 1);
        \vertex (i2) at (60: 1);
        \vertex (i3) at (300: 1);
        \vertex (o1) at (150: 2) {$a\mu$};
        \vertex (o2) at (50: 2) {$b\nu$};
        \vertex (o3) at (210: 2) {$c\rho$};
        \vertex (o4) at (310: 2) {$d\sigma$};
        \diagram*{
        (i2) -- [boson, quarter right, looseness=1.2] (i1);
        (i3) -- [boson, quarter right, looseness=1.2] (i2);
        (i1) -- [boson, quarter right, looseness=1.2] (i3);
        (o2) -- [boson] (i2);
        (o4) -- [boson] (i3);
        (o1) -- [boson] (i1) -- [boson] (o3);
        };
      \end{feynman}
    \end{scope}
  \end{tikzpicture}
  %
  \begin{tikzpicture}
    \begin{feynman}
      \vertex (i1) at (120: 1);
      \vertex (i2) at (0: 1);
      \vertex (i3) at (240: 1);
      \vertex (o1) at (130: 2) {$a\mu$};
      \vertex (o2) at (30: 2) {$b\nu$};
      \vertex (o3) at (230: 2) {$c\rho$};
      \vertex (o4) at (330: 2) {$d\sigma$};
      \diagram*{
      (i2) -- [boson, quarter right, looseness=1.2] (i1);
      (i1) -- [boson, quarter right, looseness=1.2] (i3);
      (i3) -- [boson, quarter right, looseness=1.2] (i2);
      (o1) -- [boson] (i1);
      (o3) -- [boson] (i3);
      (o2) -- [boson] (i2) -- [boson] (o4);
      };
    \end{feynman}
    \begin{scope}[xshift=5cm]
      \begin{feynman}
        \vertex (i1) at (180: 1);
        \vertex (i2) at (60: 1);
        \vertex (i3) at (300: 1);
        \vertex (o1) at (150: 2) {$a\mu$};
        \vertex (o2) at (50: 2) {$b\nu$};
        \vertex (o3) at (210: 2) {$c\rho$};
        \vertex (o4) at (310: 2) {$d\sigma$};
        \diagram*{
        (i2) -- [boson, quarter right, looseness=1.2] (i1);
        (i3) -- [boson, quarter right, looseness=1.2] (i2);
        (i1) -- [boson, quarter right, looseness=1.2] (i3);
        (o1) -- [boson, quarter left, looseness=1.2] (i2);
        (o2) -- [boson, quarter right, looseness=1.2] (i1);
        (o3) -- [boson] (i1);
        (o4) -- [boson] (i3);
        };
      \end{feynman}
    \end{scope}
    \begin{scope}[xshift=10cm]
      \begin{feynman}
        \vertex (i1) at (120: 1);
        \vertex (i2) at (0: 1);
        \vertex (i3) at (240: 1);
        \vertex (o1) at (130: 2) {$a\mu$};
        \vertex (o2) at (30: 2) {$b\nu$};
        \vertex (o3) at (230: 2) {$c\rho$};
        \vertex (o4) at (330: 2) {$d\sigma$};
        \diagram*{
        (i2) -- [boson, quarter right, looseness=1.2] (i1);
        (i1) -- [boson, quarter right, looseness=1.2] (i3);
        (i3) -- [boson, quarter right, looseness=1.2] (i2);
        (o1) -- [boson, quarter left, looseness=1.2] (i2);
        (o2) -- [boson, quarter right, looseness=1.2] (i1);
        (o3) -- [boson] (i3);
        (o4) -- [boson] (i2);
        };
      \end{feynman}
    \end{scope}
  \end{tikzpicture}
\end{center}

1つ目のダイアグラムは$f^{aeg} f^{bgf} \times (efcd) \times$
\begin{align*}
  &\left( - 2 g_{\lambda\tau} k_{\mu} + g_{\mu\lambda} k_{\tau} + g_{\mu\tau} k_{\lambda} \right)
  \left( - 2 g_{\kappa\tau} k_{\nu} + g_{\nu\kappa} k_{\tau} + g_{\nu\tau} k_{\kappa} \right) \\
  %
  &=g_{\mu\lambda} g_{\nu\kappa} k^2
  + 4 g_{\kappa\lambda} k_{\mu} k_{\nu} - 2 g_{\mu\kappa} k_{\lambda} k_{\nu}
  - g_{\mu\lambda} k_{\kappa} k_{\nu} + g_{\mu\nu} k_{\kappa} k_{\lambda}
  - g_{\nu\kappa} k_{\lambda} k_{\mu} - 2 g_{\nu\lambda} k_{\kappa} k_{\mu} \\
  %
  &\sim g_{\mu\lambda} g_{\nu\kappa} k^2
  + g_{\kappa\lambda} g_{\mu\nu} k^2 - \frac{1}{2} g_{\mu\kappa} g_{\nu\lambda} k^2
  - \frac{1}{4} g_{\mu\lambda} g_{\nu\kappa} k^2 + \frac{1}{4} g_{\mu\nu} g_{\lambda\kappa} k^2
  - \frac{1}{4} g_{\nu\kappa} g_{\mu\lambda} k^2 - \frac{1}{2} g_{\nu\lambda} g_{\mu\kappa} k^2 \\
  %
  &= \frac{5}{4} g_{\mu\nu} g_{\lambda\kappa} k^2
  + \frac{1}{2} g_{\mu\lambda} g_{\nu\kappa} k^2
  - g_{\mu\kappa} g_{\nu\lambda} k^2 .
\end{align*}
従って,$k^2$の係数は$f^{aeg} f^{bgf}/4 \times$
\begin{align*}
  & \left[
  f^{cdh} f^{efh} \left(- g^{\rho\kappa} g^{\sigma\lambda} + g^{\rho\lambda} g^{\sigma\kappa}\right)
  + f^{ech} f^{fdh} \left(g^{\kappa\lambda} g^{\rho\sigma} - g^{\rho\kappa} g^{\sigma\lambda}\right)
  + f^{edh} f^{fch} \left(g^{\kappa\lambda} g^{\rho\sigma} - g^{\rho\lambda} g^{\sigma\kappa}\right) \right] \\
  &\quad\times \left( 5g^{\mu\nu} g^{\lambda\kappa} + 2g^{\mu\lambda} g^{\nu\kappa} - 4g^{\mu\kappa} g^{\nu\lambda} \right) \\
  %
  &= 6 f^{cdh} f^{efh} g^{\mu\rho} g^{\nu\sigma} - 6 f^{cdh} f^{efh} g^{\mu\sigma} g^{\nu\rho}
  + 13 f^{ech} f^{fdh} g^{\mu\nu} g^{\rho\sigma} + 4 f^{ech} f^{fdh} g^{\mu\rho} g^{\nu\sigma} \\
  &\quad - 2 f^{ech} f^{fdh} g^{\mu\sigma} g^{\nu\rho} + 13 f^{edh} f^{fch} g^{\mu\nu} g^{\rho\sigma}
  - 2 f^{edh} f^{fch} g^{\mu\rho} g^{\nu\sigma} + 4 f^{edh} f^{fch} g^{\mu\sigma} g^{\nu\rho} \\
  %
  &= 6 f^{cdh} f^{efh} (g^{\mu\rho} g^{\nu\sigma} - g^{\mu\sigma} g^{\nu\rho}) \\
  &\quad + f^{ech} f^{fdh} (13g^{\mu\nu} g^{\rho\sigma} + 4g^{\mu\rho} g^{\nu\sigma} - 2g^{\mu\sigma} g^{\nu\rho}) \\
  &\qquad + f^{edh} f^{fch} (13g^{\mu\nu} g^{\rho\sigma} - 2g^{\mu\rho} g^{\nu\sigma} + 4g^{\mu\sigma} g^{\nu\rho}) .
\end{align*}
\eqref{prob13_6_3trace_adj}\eqref{prob13_6_C_2_to_4trace}を使えば
\begin{align*}
  & \frac{3}{2} f^{aeg} f^{bgf} f^{cdh} f^{efh} (g^{\mu\rho} g^{\nu\sigma} - g^{\mu\sigma} g^{\nu\rho}) \\
  &\quad + \frac{1}{4} f^{aeg} f^{bgf} f^{ech} f^{fdh} (13 g^{\mu\nu} g^{\rho\sigma} + 4g^{\mu\rho} g^{\nu\sigma} - 2g^{\mu\sigma} g^{\nu\rho})  \\
  &\quad + \frac{1}{4} f^{aeg} f^{bgf} f^{edh} f^{fch} (8 g^{\mu\nu} g^{\rho\sigma} - 2g^{\mu\rho} g^{\nu\sigma} + 4g^{\mu\sigma} g^{\nu\rho}) \\
  %
  &= - \frac{3}{4} C_2(G) f^{abh} f^{cdh} (g^{\mu\rho} g^{\nu\sigma} - g^{\mu\sigma} g^{\nu\rho}) \\
  &\quad - \frac{1}{4} \Tr(t_G^a t_G^b t_G^d t_G^c) (13 g^{\mu\nu} g^{\rho\sigma} + 4g^{\mu\rho} g^{\nu\sigma} - 2g^{\mu\sigma} g^{\nu\rho})  \\
  &\quad - \frac{1}{4} \Tr(t_G^a t_G^b t_G^c t_G^d) (13 g^{\mu\nu} g^{\rho\sigma} - 2g^{\mu\rho} g^{\nu\sigma} + 4g^{\mu\sigma} g^{\nu\rho}) \\
  %
  &= \frac{3}{2} \left[\Tr(t_G^a t_G^b t_G^c t_G^d) - \Tr(t_G^a t_G^b t_G^d t_G^c)\right] (g^{\mu\rho} g^{\nu\sigma} - g^{\mu\sigma} g^{\nu\rho}) \\
  &\quad - \frac{1}{4} \Tr(t_G^a t_G^b t_G^d t_G^c) (13 g^{\mu\nu} g^{\rho\sigma} + 4g^{\mu\rho} g^{\nu\sigma} - 2g^{\mu\sigma} g^{\nu\rho})  \\
  &\quad - \frac{1}{4} \Tr(t_G^a t_G^b t_G^c t_G^d) (13 g^{\mu\nu} g^{\rho\sigma} - 2g^{\mu\rho} g^{\nu\sigma} + 4g^{\mu\sigma} g^{\nu\rho}) \\
  %
  &= \frac{1}{4} \Tr(t_G^a t_G^b t_G^c t_G^d) (-13 g^{\mu\nu} g^{\rho\sigma} + 8g^{\mu\rho} g^{\nu\sigma} - 10 g^{\mu\sigma} g^{\nu\rho}) \\
  &\quad + \frac{1}{4} \Tr(t_G^a t_G^b t_G^d t_G^c) (-13 g^{\mu\nu} g^{\rho\sigma} - 10 g^{\mu\rho} g^{\nu\sigma} + 8g^{\mu\sigma} g^{\nu\rho}) .
\end{align*}

残りのダイアグラムは
$(\!(a\mu, b\nu\leftrightarrow c\rho, d\sigma)\!)$,$(\!(a\mu\leftrightarrow d\sigma)\!)$,
$(\!(b\nu\leftrightarrow c\rho)\!)$,$(\!(b\nu\leftrightarrow d\sigma)\!)$,
$(\!(a\mu\leftrightarrow c\rho)\!)$の置換によって得られる.
\eqref{prob13_6_4trace_adj}に注意して全て足せば(1, 2個目及び3, 4個目及び5, 6個目はそれぞれ同じ値になる),
\begin{align*}
  & \frac{1}{2} \Tr(t_G^a t_G^b t_G^c t_G^d) (-13 g^{\mu\nu} g^{\rho\sigma} + 8g^{\mu\rho} g^{\nu\sigma} - 10 g^{\mu\sigma} g^{\nu\rho}) \\
  &\quad + \frac{1}{2} \Tr(t_G^a t_G^c t_G^b t_G^d) (-13 g^{\nu\sigma} g^{\mu\rho} + 8g^{\rho\sigma} g^{\nu\mu} - 10 g^{\mu\sigma} g^{\nu\rho}) \\
  &\qquad + \frac{1}{2} \Tr(t_G^a t_G^b t_G^c t_G^d) (-13 g^{\mu\sigma} g^{\nu\rho} + 8g^{\mu\rho} g^{\nu\sigma} - 10 g^{\mu\nu} g^{\rho\sigma}) \\
  & + \frac{1}{2} \Tr(t_G^a t_G^b t_G^d t_G^c) (-13 g^{\mu\nu} g^{\rho\sigma} - 10 g^{\mu\rho} g^{\nu\sigma} + 8g^{\mu\sigma} g^{\nu\rho}) \\
  &\quad + \frac{1}{2} \Tr(t_G^a t_G^b t_G^d t_G^c) (-13 g^{\nu\sigma} g^{\mu\rho} - 10 g^{\rho\sigma} g^{\mu\nu} + 8g^{\mu\sigma} g^{\nu\rho}) \\
  &\qquad + \frac{1}{2} \Tr(t_G^a t_G^c t_G^b t_G^d) (-13 g^{\mu\sigma} g^{\nu\rho} - 10 g^{\mu\rho} g^{\nu\sigma} + 8g^{\mu\nu} g^{\rho\sigma}) \\
  %
  &= \Tr(t_G^a t_G^b t_G^c t_G^d) (-23 g^{\mu\nu} g^{\rho\sigma} + 16 g^{\mu\rho} g^{\nu\sigma} - 23 g^{\mu\sigma} g^{\nu\rho}) \\
  &\quad + \Tr(t_G^a t_G^b t_G^d t_G^c) (- 23 g^{\mu\nu} g^{\rho\sigma} - 23 g^{\mu\rho} g^{\nu\sigma} + 16 g^{\mu\sigma} g^{\nu\rho}) \\
  &\qquad + \Tr(t_G^a t_G^c t_G^b t_G^d) (16 g^{\mu\nu} g^{\rho\sigma} - 23 g^{\mu\rho} g^{\nu\sigma} - 23 g^{\mu\sigma} g^{\nu\rho}) .
\end{align*}

3ボソン頂点4つのダイアグラムを考える.
\begin{center}
  \begin{tikzpicture}
    \begin{feynman}
      \vertex (i1) at (135: 1);
      \vertex (i2) at (45: 1);
      \vertex (i3) at (225: 1);
      \vertex (i4) at (315: 1);
      \vertex (o1) at (135: 2) {$a\mu$};
      \vertex (o2) at (45: 2) {$b\nu$};
      \vertex (o3) at (225: 2) {$c\rho$};
      \vertex (o4) at (315: 2) {$d\sigma$};
      \diagram*{
      (i2) -- [boson, quarter right, looseness=1, momentum'=$k$, edge label=$h\eta$] (i1);
      (i1) -- [boson, quarter right, looseness=1, momentum'=$k$, edge label=$e\lambda$] (i3);
      (i3) -- [boson, quarter right, looseness=1, momentum'=$k$, edge label=$f\kappa$] (i4);
      (i4) -- [boson, quarter right, looseness=1, momentum'=$k$, edge label=$g\xi$] (i2);
      (o1) -- [boson] (i1);
      (o2) -- [boson] (i2);
      (o3) -- [boson] (i3);
      (o4) -- [boson] (i4);
      };
    \end{feynman}
    \begin{scope}[xshift=5cm]
      \begin{feynman}
        \vertex (i1) at (135: 1);
        \vertex (i2) at (45: 1);
        \vertex (i3) at (225: 1);
        \vertex (i4) at (315: 1);
        \vertex (o1) at (135: 2) {$a\mu$};
        \vertex (o2) at (45: 2) {$b\nu$};
        \vertex (o3) at (225: 2) {$c\rho$};
        \vertex (o4) at (315: 2) {$d\sigma$};
        \diagram*{
        (i2) -- [boson, quarter right, looseness=1, edge label=$h\eta$] (i1);
        (i1) -- [boson, quarter right, looseness=1, edge label=$e\lambda$] (i3);
        (i3) -- [boson, quarter right, looseness=1, edge label=$f\kappa$] (i4);
        (i4) -- [boson, quarter right, looseness=1, edge label=$g\xi$] (i2);
        (o1) -- [boson, quarter left, looseness=1] (i2);
        (o2) -- [boson, quarter right, looseness=1] (i1);
        (o3) -- [boson] (i3);
        (o4) -- [boson] (i4);
        };
      \end{feynman}
    \end{scope}
    \begin{scope}[xshift=10cm]
      \begin{feynman}
        \vertex (i1) at (135: 1);
        \vertex (i2) at (45: 1);
        \vertex (i3) at (225: 1);
        \vertex (i4) at (315: 1);
        \vertex (o1) at (135: 2) {$a\mu$};
        \vertex (o2) at (45: 2) {$b\nu$};
        \vertex (o3) at (225: 2) {$c\rho$};
        \vertex (o4) at (315: 2) {$d\sigma$};
        \diagram*{
        (i2) -- [boson, quarter right, looseness=1, edge label=$h\eta$] (i1);
        (i1) -- [boson, quarter right, looseness=1, edge label=$e\lambda$] (i3);
        (i3) -- [boson, quarter right, looseness=1, edge label=$f\kappa$] (i4);
        (i4) -- [boson, quarter right, looseness=1, edge label=$g\xi$] (i2);
        (o1) -- [boson] (i1);
        (o2) -- [boson, quarter left, looseness=1] (i4);
        (o3) -- [boson] (i3);
        (o4) -- [boson, quarter right, looseness=1] (i2);
        };
      \end{feynman}
    \end{scope}
  \end{tikzpicture}
\end{center}

1つ目のダイアグラムは$f^{aeh}f^{ecf}f^{gfd}f^{bhg}\times$
\begin{align*}
  & \left( - 2 g^{\lambda\eta} k^{\mu} + g^{\mu\eta} k^{\lambda} + g^{\mu\lambda} k^{\eta} \right)
  \left( - 2 g^{\kappa\lambda} k^{\rho} + g^{\rho\kappa} k^{\lambda} + g^{\rho\lambda} k^{\kappa} \right) \\
  &\quad\times \left( - 2 g^{\kappa\xi} k^{\sigma} + g^{\sigma\kappa} k^{\xi} + g^{\sigma\xi} k^{\kappa} \right)
  \left( g^{\nu\eta} k^{\xi} + g^{\nu\xi} k^{\eta} - 2 g^{\xi\eta} k^{\nu} \right) \\
  %
  &= k^4 g^{\mu\nu} g^{\rho\sigma} + k^4 g^{\mu\rho} g^{\nu\sigma}
  + 3 k^2 g^{\mu\nu} k^{\rho} k^{\sigma} + 3 k^2 g^{\mu\rho} k^{\nu} k^{\sigma}
  + 3 k^2 g^{\nu\sigma} k^{\mu} k^{\rho} + 3 k^2 g^{\rho\sigma} k^{\mu} k^{\nu}
  + 34 k^{\mu} k^{\nu} k^{\rho} k^{\sigma}
\end{align*}
となる.(A.47)(A.48)を参考にすれば
\[ k^{\mu} k^{\nu} k^{\rho} k^{\sigma} \to \frac{1}{24} (g^{\mu\nu} g^{\rho\sigma} + g^{\mu\rho} g^{\nu\sigma} + g^{\mu\sigma} g^{\nu\rho}) \]
として良いことが分かる.よって,$k^4$の係数は$\Tr(t_G^a t_G^b t_G^d t_G^c) \times$
\begin{align*}
  & g^{\mu\nu} g^{\rho\sigma} + g^{\mu\rho} g^{\nu\sigma}
  + \frac{3}{4} g^{\mu\nu} g^{\rho\sigma} + \frac{3}{4} k^2 g^{\mu\rho} g^{\nu\sigma}
  + \frac{3}{4} g^{\nu\sigma} g^{\mu\rho} + \frac{3}{4}  g^{\rho\sigma} g^{\mu\nu}
  + \frac{17}{12} (g^{\mu\nu} g^{\rho\sigma} + g^{\mu\rho} g^{\nu\sigma} + g^{\mu\sigma} g^{\nu\rho}) \\
  %
  &= \frac{47}{12} g^{\mu\nu} g^{\rho\sigma} + \frac{47}{12} g^{\mu\rho} g^{\nu\sigma}
  + \frac{17}{12} g^{\mu\sigma} g^{\nu\rho} .
\end{align*}
残りのダイアグラムは$(\!(a\mu\leftrightarrow b\nu)\!)$,$(\!(b\nu\leftrightarrow d\sigma)\!)$によって得られる.
