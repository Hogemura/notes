\chapter{Quantum Chromodynamics}
\setcounter{section}{3}
\section{Hard-Scattering Processes in Hadron Collisions}
\subsection{(17.58)}
Jacobianを計算するコード.

\begin{lstlisting}[language=python]
  import sympy
  p = sympy.Symbol('p')
  y_3 = sympy.Symbol('y_3')
  y_4 = sympy.Symbol('y_4')
  y = (y_3-y_4)/2
  Y = (y_3+y_4)/2

  x_1 = 2 * p * sympy.cosh(y) * sympy.exp(Y)
  x_2 = 2 * p * sympy.cosh(y) * sympy.exp(-Y)
  t = -2 * p**2 * sympy.cosh(y) * sympy.exp(-y)

  O = sympy.Matrix([x_1, x_2, t])
  N = sympy.Matrix([y_3, y_4, p])
  J = sympy.det(O.jacobian(N))
  print(sympy.simplify(J))
\end{lstlisting}

\subsection{(17.67)}
$e_L^- e_R^+ \to e_R^- e^+_L$の過程を考える.

\begin{center}
  \begin{tikzpicture}
    \begin{feynman}
      \vertex (iv) at (0, 0);
      \vertex (ov) at (0, 1);
      \vertex (i1) at (-1.5, -1) {$e_L^-$};
      \vertex (i2) at (1.5, -1) {$e_R^+$};
      \vertex (o1) at (-1.5, 2) {$e_R^-$};
      \vertex (o2) at (1.5, 2) {$e_L^+$};
      \diagram* {
      (i1) -- [fermion, edge label=$p$] (iv) -- [fermion, reversed momentum=$p'$] (i2);
      (o1) -- [anti fermion, edge label=$k$] (ov) -- [anti fermion, momentum=$k'$] (o2);
      (iv) -- [photon] (ov);
      };
    \end{feynman}
  \end{tikzpicture}
\end{center}

Section 5.2と同様に,射影演算子$(1\pm\gamma^5)/2$を使えば
\begin{align*}
  i\mathcal{M} &= ie^2 \bar{u}(k) \gamma^\mu \frac{1-\gamma^5}{2} v(k') \frac{1}{(p+p')^2}
  \bar{v}(p') \gamma_\mu \frac{1-\gamma^5}{2} u(p) \\
  &= \frac{ie^2}{s} \bar{u}(k) \gamma^\mu \frac{1-\gamma^5}{2} v(k') \bar{v}(p') \gamma_\mu \frac{1-\gamma^5}{2} u(p)
\end{align*}
なので
\begin{align*}
  \lvert\mathcal{M}\rvert^2
  &= \frac{e^4}{s^2} \bar{u}(k) \gamma^\mu \frac{1-\gamma^5}{2} v(k') \bar{v}(k') \gamma^\nu \frac{1-\gamma^5}{2} u(k) \\
  &\qquad\times \bar{v}(p') \gamma_\mu \frac{1-\gamma^5}{2} u(p) \bar{u}(p) \gamma_\nu \frac{1-\gamma^5}{2} v(p') .
\end{align*}
入射電子・陽電子に関してはスピンを平均,散乱電子・陽電子に関してはスピンを合計して,
\begin{align*}
  \frac{1}{4} \sum_\text{spins} \lvert\mathcal{M}\rvert^2
  &= \frac{e^4}{4s^2} \Tr\left( \slashed{k} \gamma^\mu \frac{1-\gamma^5}{2} \slashed{k'} \gamma^\nu \frac{1-\gamma^5}{2} \right)
  \Tr\left( \slashed{p'} \gamma_\mu \frac{1-\gamma^5}{2} \slashed{p} \gamma_\nu \frac{1-\gamma^5}{2} \right) \\
  %
  &= \frac{e^4}{4s^2} \Tr\left( \slashed{k} \gamma^\mu \slashed{k'} \gamma^\nu \frac{1-\gamma^5}{2} \right)
  \Tr\left( \slashed{p'} \gamma_\mu \slashed{p} \gamma_\nu \frac{1-\gamma^5}{2} \right) \\
  %
  &= \frac{e^4}{4s^2} E^2 (1-\cos\theta)^2 \\
  &= \frac{e^4}{4} \left(\frac{t}{s}\right)^2 .
\end{align*}
トレースの計算は(5.23)の結果を利用した.(4.85)から
\[
\frac{d\sigma}{d\cos\theta} = \frac{\lvert\mathcal{M}\rvert^2}{32\pi s}
= \frac{e^2}{128\pi s} \left(\frac{t}{s}\right)^2 = \frac{\pi\alpha^2}{8s} \left(\frac{t}{s}\right)^2 .
\]
さらに
\[ t = - E^2 (1-\cos\theta)^2 - E^2\sin^2\theta = E^2 (2\cos\theta-2) = \frac{s}{2} (\cos\theta-1) \]
なので
\[ \frac{d\sigma}{dt} = \frac{2}{s} \frac{d\sigma}{d\cos\theta} = \frac{\pi\alpha^2}{4s^2} \left(\frac{t}{s}\right)^2 . \]

\section*{Problems}\addcontentsline{toc}{section}{Problems}
\subsection{Problem 17.4: The gluon splitting function}
$P_{g\leftarrow g}(z,)$の規格化条件を求める.
(17.39)と同様に
\[
\int_0^1 dz \, z \left[ \sum_f \left\{ f_f(z, Q) + f_{\bar{f}}(z, Q) \right\} + f_g(z, Q) \right] = 1 .
\]
(17.128)を積分して
\begin{align*}
  & \frac{d}{d\log Q} \int_0^1 dx \, x f_g(x, Q) \\
  &= \frac{\alpha_s}{\pi} \int_0^1 dx \, x \int_x^1 \frac{dz}{z}
  \left[ P_{gq}(z) \sum_f \left\{ f_f(x/z) + f_{\bar{f}}(x/z) \right\} + P_{gg}(z) f_g(x/z) \right] \\
  &= \frac{\alpha_s}{\pi} \int_0^1 ds \, s \int_0^1 dz \, z
  \left[ P_{gq}(z) \sum_f \left\{ f_f(s) + f_{\bar{f}}(s) \right\} + P_{gg}(z) f_g(s) \right]
\end{align*}
となる($s = x/z$とした).他の式も同様にして
\begin{align*}
  & \frac{d}{d\log Q} \int_0^1 dx \, x f_f(x, Q)
  = \frac{\alpha_s}{\pi} \int_0^1 ds \, s \int_0^1 dz \, z \left[ P_{qq}(z) f_f(s) + P_{qg}(z) f_g(s) \right] , \\
  & \frac{d}{d\log Q} \int_0^1 dx \, x f_{\bar{f}}(x, Q)
  = \frac{\alpha_s}{\pi} \int_0^1 ds \, s \int_0^1 dz \, z \left[ P_{qq}(z) f_{\bar{f}}(s) + P_{qg}(z) f_g(s) \right] .
\end{align*}
以上から,
\begin{align*}
  0 &= \frac{d}{d\log Q} \int_0^1 dx \, x \left[ \sum_f \left\{ f_f(x, Q) + f_{\bar{f}}(x, Q) \right\} + f_g(x, Q) \right] \\
  &= \frac{\alpha_s}{\pi} \int_0^1 ds \, s \int_0^1 dz \, z
  \left[ \{P_{gq}(z)+P_{qq}(z)\} \sum_f \left\{ f_f(s) + f_{\bar{f}}(s) \right\} + \{2n_f P_{qg}(z) + P_{gg}(z)\} f_g(s) \right] \\
  &= \frac{\alpha_s}{\pi} \int_0^1 ds \, s \left\{ f_f(s) + f_{\bar{f}}(s) \right\} \int_0^1 dz \, z \{P_{gq}(z)+P_{qq}(z)\} \\
  & \quad+ \frac{\alpha_s}{\pi} \int_0^1 ds \, s f_g(s) \int_0^1 dz \, z \{2n_f P_{qg}(z) + P_{gg}(z)\} \\
  %
  &= - \frac{\alpha_s}{\pi} \int_0^1 ds \, s f_g(s) \int_0^1 dz \, z \{P_{gq}(z)+P_{qq}(z)\}
  + \frac{\alpha_s}{\pi} \int_0^1 ds \, s f_g(s) \int_0^1 dz \, z \{2n_f P_{qg}(z) + P_{gg}(z)\}
\end{align*}
なので,
\begin{align*}
  \int_0^1 dz \, z \{2n_f P_{qg}(z) + P_{gg}(z)\}
  &= \int_0^1 dz \, z \{P_{gq}(z)+P_{qq}(z)\} \\
  &= 3 \int_0^1 dz \, \left[ 1 + (1-z)^2 + \frac{z+z^3}{(1-z)_+} \right] + 2 \\
  &= 3 \int_0^1 dz \, \left[ z^2-2z+2 + \frac{z^3+z-2}{1-z} \right] + 2 \\
  &= 3 \int_0^1 dz \, \left[ z^2-2z+2 - (z^2+z+2) \right] + 2 \\
  &= - 4 \int_0^1 z \, dz + 2 \\
  &= 0 .
\end{align*}

\chapter{Operator Products and Effective Vertices}
\setcounter{section}{4}
\section{Operator Analysis of Deep Inelastic Scattering}
\subsection{(18.136)}
\[ \bar{u}(P) \gamma^\mu u(P) = \Tr[\bar{u}(P) \gamma^\mu u(P)] = \Tr[\gamma^\mu u(P) \bar{u}(P)] . \]
スピンに関して平均を取って
\begin{align*}
  \to \frac{1}{2} \Tr[\gamma^\mu \slashed{P}] = 2 P^\mu .
\end{align*}

\subsection{(18.208)}
(18.207)から
\[
\frac{d}{d\log Q^2} M_n^+ = \frac{d}{d\log Q^2} \int_0^1 dx \, x^{n-1} \sum_f (f_f + f_{\bar{f}})(x) .
\]
(17.128)から
\[
= \frac{\alpha_s}{2\pi} \int_0^1 dx \, x^{n-1} \int_x^1 \frac{dz}{z}
\left[ P_{q\leftarrow q}(z) \sum_f (f_f + f_{\bar{f}})(x/z) + 2P_{q\leftarrow g}(z) f_g(x/z) \right] .
\]
(17.17)(18.199)から
\begin{align*}
  &= \frac{2}{b_0\log(Q^2/\Lambda^2)} \int_0^1 dz \, z^{n-1} P_{q\leftarrow q}(z) \int_0^1 dy \, y^{n-1} \sum_f (f_f + f_{\bar{f}})(y) \\
  &\quad + \frac{2}{b_0\log(Q^2/\Lambda^2)} \int_0^1 dz \, z^{n-1} 2P_{q\leftarrow g}(z) \int_0^1 dy \, y^{n-1} f_g(y) \\
  %
  &= \frac{2}{b_0\log(Q^2/\Lambda^2)}
  \left[ M_n^+ \int_0^1 dz \, z^{n-1} P_{q\leftarrow q}(z) + M_{gn} \int_0^1 dz \, z^{n-1} 2P_{q\leftarrow g}(z) \right] .
\end{align*}
(17.129)(18.181)(18.203)から
\[
= \frac{2}{b_0\log(Q^2/\Lambda^2)}
\left[ \frac{a_{ff}^n}{4} M_n^+ + \frac{a_{fg}^n}{4} M_{gn} \right] .
\]

\section*{Problems}\addcontentsline{toc}{section}{Problems}
\subsection{Problem 18.3: Anomalous dimensions of gluon twsit-2 operators(計算合わない)}
\cite{gross1974asymptotically}を参考にして修正予定.

\subsubsection{$a_{gf}^n$の1ループ補正}
Figure 18.15 (a)のダイアグラムを計算する.

\begin{center}
  \begin{tikzpicture}
    \begin{feynman}[small]
      \vertex[dot] (v1) at (90: 0.5) {};
      \vertex (v2) at (210: 1);
      \vertex (v3) at (330: 1);
      \vertex[below left=of v2] (o2);
      \vertex[below right=of v3] (o3);
      \draw (v2) node [left, xshift=-0.1cm] {$A^a_\rho$};
      \draw (v3) node [right, xshift=0.1cm] {$A^a_\sigma$};
      \diagram*{
      (v2) -- [boson, momentum=$k$] (v1) -- [boson, momentum=$k$] (v3);
      (o2) -- [fermion] (v2) -- [fermion] (v3) -- [fermion] (o3);
      };
    \end{feynman}
  \end{tikzpicture}
\end{center}

(18.174)から
\[
\mathcal{O}_g^{(n)\mu_1\cdots\mu_n} \to - (\partial^{\mu_1}A^\nu - \partial^\nu A^{\mu_1})
(i\partial^{\mu_2}) \cdots (i\partial^{\mu_{n-1}}) (\partial^{\mu_n}A_\nu - \partial_\nu A^{\mu_n}) .
\]
$\partial^{\mu_1}A^\nu$と$A_\sigma$および$\partial^{\mu_n}A_\nu$と$A_\rho$を縮約した場合は,
\[ \mathcal{O} = -(ik^{\mu_1}) (k^{\mu_2}\cdots k^{\mu_{n-1}}) (-ik^{\mu_n}) \]
なので,
\[
- (ig)^2 \int \frac{d^dk}{(2\pi)^d} t^a \gamma^\nu \frac{i}{\slashed{p} - \slashed{k}} t^a \gamma_\nu
\left( \frac{-i}{k^2} \right)^2 k^{\mu_1} \cdots k^{\mu_n} .
\]
(6.42)(A.34)(A.37)から$\ell=k-xp$として
\begin{align*}
  &= 3 ig^2 \int\frac{d^d\ell}{(2\pi)^d} \gamma^\nu \gamma_\alpha \gamma_\nu \int_0^1 dx \, 2(1-x)
  \frac{1}{(\ell^2 - \Delta)^3} [\ell - (1-x)p]^\alpha (\ell+xp)^{\mu_1} \cdots (\ell+xp)^{\mu_n} \\
  %
  &= \frac{8}{3} (2-d) ig^2 \int_0^1 dx \, (1-x) \gamma_\alpha \int\frac{d^d\ell}{(2\pi)^d}
  \frac{1}{(\ell^2 - \Delta)^3} [\ell - (1-x)p]^\alpha (\ell+xp)^{\mu_1} \cdots (\ell+xp)^{\mu_n} .
\end{align*}
このうち(18.161)の$\mathcal{O}_f$の形$\gamma^{\mu_1} p^{\mu_2} \cdots p^{\mu_n}$を持つ項は
\begin{align*}
  &\to \frac{8}{3} (2-d) ig^2 n \int_0^1 dx \, (1-x) \gamma_\alpha \int\frac{d^d\ell}{(2\pi)^d}
  \frac{1}{(\ell^2 - \Delta)^3} \ell^\alpha \ell^{\mu_1} x^{n-1} p^{\mu_2} \cdots p^{\mu_n} \\
  %
  &= \frac{8}{3} \frac{2-d}{d} ig^2 n \int_0^1 dx \, (1-x) x^{n-1} \gamma_\alpha \int\frac{d^d\ell}{(2\pi)^d}
  \frac{\ell^2}{(\ell^2 - \Delta)^3} g^{\alpha\mu_1} p^{\mu_2} \cdots p^{\mu_n} \\
  %
  &= \frac{8}{3} \frac{2-d}{d} ig^2 n \int_0^1 dx \, (1-x) x^{n-1}
  \frac{i}{(4\pi)^{d/2}} \frac{d}{2} \frac{\Gamma(2-d/2)}{\Gamma(3)} \left( \frac{1}{\Delta} \right)^{2-d/2}
  \gamma^{\mu_1} p^{\mu_2} \cdots p^{\mu_n} \\
  %
  &= 3 \frac{g^2}{(4\pi)^2} n \int_0^1 dx \, (1-x) x^{n-1} \gamma^{\mu_1} p^{\mu_2} \cdots p^{\mu_n} \frac{2}{\epsilon} \\
  %
  &= 3 \frac{g^2}{(4\pi)^2} \frac{1}{n+1} \gamma^{\mu_1} p^{\mu_2} \cdots p^{\mu_n} \frac{2}{\epsilon} .
\end{align*}

$-\partial^\nu A^{\mu_1}$と$A_\sigma$および$-\partial_\nu A^{\mu_n}$と$A_\rho$を縮約した場合は,
\begin{align*}
  & -(ig)^2 \int \frac{d^dk}{(2\pi)^d} t^a \gamma^{\mu_1} \frac{i}{\slashed{p} - \slashed{k}} t^a \gamma^{\mu_n}
  \left( \frac{-i}{k^2} \right)^2 k^\nu k^{\mu_2} \cdots k^{\mu_{n-1}} k_\nu \\
  %
  &= 3 ig^2 \int_0^1 dx \int\frac{d^d\ell}{(2\pi)^d} \frac{1}{(\ell^2-\Delta)^2}
  [\ell-(1-x)p]^\alpha (\ell+xp)^{\mu_2} \cdots (\ell+xp)^{\mu_{n-1}} \gamma^{\mu_1} \gamma_\alpha \gamma^{\mu_n} \\
  %
  &\to 3 ig^2 \int_0^1 dx \int\frac{d^d\ell}{(2\pi)^d} \frac{1}{(\ell^2-\Delta)^2}
  [\ell-(1-x)p]^\alpha (\ell+xp)^{\mu_2} \cdots (\ell+xp)^{\mu_{n-1}} \gamma^{\{\mu_1} \gamma_\alpha \gamma^{\mu_n\}} \\
  %
  &\to 3 ig^2 \int_0^1 dx \int\frac{d^d\ell}{(2\pi)^d} \frac{1}{(\ell^2-\Delta)^2}
  [\ell-(1-x)p]^\alpha (\ell+xp)^{\mu_2} \cdots (\ell+xp)^{\mu_{n-1}}
  (\delta^{\mu_n}_\alpha \gamma^{\mu_1} + \delta^{\mu_1}_\alpha \gamma^{\mu_n}) \\
  %
  &\to \frac{8}{3} ig^2 \int_0^1 dx \int\frac{d^d\ell}{(2\pi)^d} \frac{1}{(\ell^2-\Delta)^2}
  \gamma^{\mu_n} [\ell-(1-x)p]^{\mu_1} (\ell+xp)^{\mu_2} \cdots (\ell+xp)^{\mu_{n-1}} \\
  %
  &= -\frac{8}{3} ig^2 \int_0^1 dx \, (1-x)x^{n-2} \int\frac{d^d\ell}{(2\pi)^d} \frac{1}{(\ell^2-\Delta)^2}
  \gamma^{\mu_n} p^{\mu_1} \cdots p^{\mu_{n-1}} \\
  %
  &= \frac{8}{3} \frac{g^2}{(4\pi)^2} \frac{1}{n(n-1)} p^{\mu_1} \cdots p^{\mu_{n-1}} \gamma^{\mu_n} \frac{2}{\epsilon} .
\end{align*}

$\partial^{\mu_1}A^\nu$と$A_\sigma$および$-\partial_\nu A^{\mu_n}$と$A_\rho$を縮約した場合は,
\begin{align*}
  & (ig)^2 \int \frac{d^dk}{(2\pi)^d} t^a \gamma^\nu \frac{i}{\slashed{p} - \slashed{k}} t^a \gamma^{\mu_n}
  \left( \frac{-i}{k^2} \right)^2 k^{\mu_1} \cdots k^{\mu_{n-1}} k_\nu \\
  %
  &= - 3 ig^2 \int\frac{d^dk}{(2\pi)^d} \left( \frac{1}{k^2} \right)^2 \frac{1}{(k-p)^2}
  (k-p)^\alpha k^{\mu_1} \cdots k^{\mu_{n-1}} k_\nu \gamma^\nu \gamma_\alpha \gamma^{\mu_n} .
\end{align*}

$-\partial^\nu A^{\mu_1}$と$A_\sigma$および$\partial^{\mu_n}A_\nu$と$A_\rho$を縮約した場合は,
\begin{align*}
  & (ig)^2 \int \frac{d^dk}{(2\pi)^d} t^a \gamma^{\mu_1} \frac{i}{\slashed{p} - \slashed{k}} t^a \gamma_\nu
  \left( \frac{-i}{k^2} \right)^2 k^{\mu_2} \cdots k^{\mu_n} k^\nu \\
  %
  &\to - 3 ig^2 \int\frac{d^dk}{(2\pi)^d} \left( \frac{1}{k^2} \right)^2 \frac{1}{(k-p)^2}
  (k-p)^\alpha k^{\mu_1} \cdots k^{\mu_{n-1}} k_\nu \gamma^{\mu_n} \gamma_\alpha \gamma^\nu .
\end{align*}
以上2つを足して,
\begin{align*}
  & - 3 ig^2 \int\frac{d^dk}{(2\pi)^d} \left( \frac{1}{k^2} \right)^2 \frac{1}{(k-p)^2}
  (k-p)^\alpha k^{\mu_1} \cdots k^{\mu_{n-1}} k_\nu (\gamma^{\mu_n} \gamma_\alpha \gamma^\nu + \gamma^\nu \gamma_\alpha \gamma^{\mu_n}) \\
  %%
  &= - \frac{8}{3} ig^2 \int\frac{d^dk}{(2\pi)^d} \left( \frac{1}{k^2} \right)^2 \frac{1}{(k-p)^2}
  (k-p)^\alpha k^{\mu_1} \cdots k^{\mu_{n-1}} k_\nu
  (\delta^\nu_\alpha \gamma^{\mu_n} + \delta^{\mu_n}_\alpha \gamma^\nu - g^{\mu_n\nu} \gamma_\alpha) \\
  %%
  &\to - \frac{8}{3} ig^2 \int\frac{d^dk}{(2\pi)^d} \left( \frac{1}{k^2} \right)^2 \frac{1}{(k-p)^2}
  k^{\mu_1} \cdots k^{\mu_{n-1}} \gamma^{\mu_n} k\cdot(k-p) \\
  %
  &\quad - \frac{8}{3} ig^2 \int\frac{d^dk}{(2\pi)^d} \left( \frac{1}{k^2} \right)^2 \frac{1}{(k-p)^2}
  k^{\mu_1} \cdots k^{\mu_{n-1}} (k^{\mu_n}-p^{\mu_n}) \slashed{k} \\
  %
  &\quad + \frac{8}{3} ig^2 \int\frac{d^dk}{(2\pi)^d} \left( \frac{1}{k^2} \right)^2 \frac{1}{(k-p)^2}
  k^{\mu_1} \cdots k^{\mu_n} (\slashed{k}-\slashed{p}) \\
  %%
  &= - \frac{8}{3} ig^2 \int\frac{d^dk}{(2\pi)^d} \left( \frac{1}{k^2} \right)^2 \frac{1}{(k-p)^2}
  k^{\mu_1} \cdots k^{\mu_{n-1}} \gamma^{\mu_n} k\cdot(k-p) \\
  %
  &\quad + \frac{8}{3} ig^2 \int\frac{d^dk}{(2\pi)^d} \left( \frac{1}{k^2} \right)^2 \frac{1}{(k-p)^2}
  k^{\mu_1} \cdots k^{\mu_{n-1}} p^{\mu_n} \slashed{k} \\
  %%
  &= - \frac{8}{3} ig^2 \int\frac{d^dk}{(2\pi)^d} \frac{1}{k^2} \frac{1}{(k-p)^2}
  k^{\mu_1} \cdots k^{\mu_{n-1}} \gamma^{\mu_n} \\
  %
  &= + \frac{8}{3} ig^2 \int\frac{d^dk}{(2\pi)^d} \left( \frac{1}{k^2} \right)^2 \frac{1}{(k-p)^2}
  k^{\mu_1} \cdots k^{\mu_{n-1}} \gamma^{\mu_n} k\cdot p \\
  %
  &\quad + \frac{8}{3} ig^2 \int\frac{d^dk}{(2\pi)^d} \left( \frac{1}{k^2} \right)^2 \frac{1}{(k-p)^2}
  k^{\mu_1} \cdots k^{\mu_{n-1}} p^{\mu_n} \slashed{k} \\
  %%
  &\to - \frac{8}{3} ig^2 \int dx \, x^{n-1} \int\frac{d^d\ell}{(2\pi)^d} \frac{1}{(\ell^2-\Delta)^2}
  p^{\mu_1} \cdots p^{\mu_{n-1}} \gamma^{\mu_n} \\
  %
  &\quad + \frac{16}{3} \frac{n-1}{d} ig^2 \int dx \, (1-x) x^{n-2} \int\frac{d^d\ell}{(2\pi)^d} \frac{\ell^2}{(\ell^2-\Delta)^3}
  p^{\mu_1} \cdots p^{\mu_{n-1}} \gamma^{\mu_n} \\
  %
  &\quad + \frac{16}{3} \frac{n-1}{d} ig^2 \int dx \, (1-x) x^{n-2} \int\frac{d^d\ell}{(2\pi)^d} \frac{\ell^2}{(\ell^2-\Delta)^3}
  p^{\mu_1} \cdots p^{\mu_{n-1}} \gamma^{\mu_n} \\
  % %%
  &= \frac{8}{3} \frac{g^2}{(4\pi)^2} \int dx \, x^{n-1} p^{\mu_1} \cdots p^{\mu_{n-1}} \gamma^{\mu_n} \frac{2}{\epsilon} \\
  &\quad - \frac{8}{3} (n-1) \frac{g^2}{(4\pi)^2} \int dx \, (1-x) x^{n-2} p^{\mu_1} \cdots p^{\mu_{n-1}} \gamma^{\mu_n} \frac{2}{\epsilon} \\
  %%
  &= 0 .
\end{align*}

縮約を逆にしたものも含めて,1ループの補正は
\begin{align*}
  & \frac{8}{3} \frac{g^2}{(4\pi)^2} \frac{1}{n+1} \gamma^{\mu_1} p^{\mu_2} \cdots p^{\mu_n} \frac{2}{\epsilon}
  + \frac{16}{3} \frac{g^2}{(4\pi)^2} \frac{1}{n(n-1)} p^{\mu_1} \cdots p^{\mu_{n-1}} \gamma^{\mu_n} \frac{2}{\epsilon} \\
  %
  &= \frac{8}{3} \frac{g^2}{(4\pi)^2} \frac{n^2+n+2}{n(n^2-1)} \gamma^{\mu_1} p^{\mu_2} \cdots p^{\mu_n} \frac{2}{\epsilon} \\
  %
  &\to \frac{8}{3} \frac{g^2}{(4\pi)^2} \frac{n^2+n+2}{n(n^2-1)} \mathcal{O}_f^{(n)\mu_1 \cdots \mu_n}
  \left( \frac{2}{\epsilon} - \log M^2 + \cdots \right) .
\end{align*}

\subsubsection{$a_{gg}^n$の1ループ補正}
$\mathcal{O}_g$のゲージ場への作用
  \begin{align*}
  \bra\Omega A_b^\sigma \mathcal{O}_g^{(n)} A_a^\rho \ket\Omega
  &= \bra{0} A_b^\sigma \exp\left( i \int d^4x \, \mathcal{L}\right) \mathcal{O}_g^{(n)} A_a^\rho \ket{0} \\
  &= \bra{0} A_b^\sigma \mathcal{O}_g^{(n)} A_a^\rho \ket{0} + \bra{0} A_b^\sigma \left(i \int d^4x \, \mathcal{L}\right) \mathcal{O}_g^{(n)} A_a^\rho \ket{0} + \cdots
\end{align*}
を$g^2$のオーダーで考える.

(18.167)の後の議論と同様に,随伴表現を使えば
\[
(iD^{\mu_j})_{cd} = i\partial_{\mu_j} \delta_{cd} - g A_e^{\mu_j} (t_G^e)_{cd}
= i\partial_{\mu_j} \delta_{cd} + i g A_e^{\mu_j} f^{cde}
\]
である.さらに(16.2)から
\[ F_{\mu\nu}^a = \partial_\mu A^a_\nu - \partial_\nu A^a_\mu + g f^{abc} A^b_\mu A^c_\nu . \]
(18.174)は
  \begin{align*}
    \mathcal{O}_g^{(n)\mu_1\cdots\mu_n}
    &= - (\partial^{\mu_1}A^\nu_c - \partial^\nu A^{\mu_1}_c)
    (i\partial^{\mu_2}) \cdots (i\partial^{\mu_{n-1}}) (\partial^{\mu_n}A_\nu^c - \partial_\nu A^{\mu_n}_c) \\
    &\quad - \sum_{j=2}^{n-1} (\partial^{\mu_1}A^\nu_c - \partial^\nu A^{\mu_1}_c)
    (i\partial^{\mu_2}) \cdots (i g A_e^{\mu_j} f^{cde}) \cdots (i\partial^{\mu_{n-1}}) (\partial^{\mu_n}A_\nu^d - \partial_\nu A^{\mu_n}_d) \\
    &\quad - (\partial^{\mu_1}A^\nu_c - \partial^\nu A^{\mu_1}_c)
    (i\partial^{\mu_2}) \cdots (i\partial^{\mu_{n-1}}) g f^{cde} A^{\mu_n}_d A_\nu^e \\
    &\quad - \cdots
\end{align*}
となる.第1項は2つのゲージ場と縮約でき,第2項は3つのゲージ場と縮約できる.

$\mathcal{O}_g$の第1項は$g^0$オーダーなので,$e^{i\mathcal{L}}$の展開のうち$g^2$のオーダーのものを考えれば良い.
これはfermion頂点2つ,3-boson頂点2つ,もしくは4-boson頂点1つである.
fermion頂点2つの場合は
\begin{center}
  \begin{tikzpicture}
    \begin{feynman}[small]
      \vertex (ol) at (-2, 0);
      \vertex (il) at (-0.6, 0);
      \vertex (ir) at (0.6, 0);
      \vertex[dot] (or) at (1.3, 0) {};
      \vertex (orr) at (2, 0);
      \draw (ol) node [left] {$A^\rho_a$};
      \draw (orr) node [right] {$A^\sigma_b$};
      \diagram*{
      (il) -- [anti fermion, half right] (ir);
      (ol) -- [boson] (il) -- [fermion, half left] (ir) -- [boson] (or) -- [boson] (orr);
      };
    \end{feynman}
  \end{tikzpicture}
\end{center}
となるが,これはexternal leg correction $\delta_3$なので,$\delta_\mathcal{O}$には含めない.
4-boson頂点は$\mathcal{O}_g$の形を持たないので,計算には含めない.
結局,$\delta_\mathcal{O}$の計算に含めるのは3-boson頂点2つからなるFigure 18.15 (b) 1つ目のdiagramである.
\begin{center}
  \begin{tikzpicture}
    \begin{feynman}[small]
      \vertex (ol) at (-2, 0);
      \vertex (il) at (-1, 0);
      \vertex[dot] (ce) at (0, 0) {};
      \vertex (ir) at (1, 0);
      \vertex (or) at (2, 0);
      \draw (ol) node [left] {$A^\rho_a$};
      \draw (or) node [right] {$A^\sigma_b$};
      \diagram*{
        (ol) -- [boson, momentum=$p$] (il) -- [boson, edge label=$c\kappa$, momentum'=$k$] (ce) -- [boson, edge label=$c\kappa'$, momentum'=$k$] (ir) -- [boson, momentum=$p$] (or);
        (il) -- [boson, half left, momentum=$p-k$, edge label'=${d\lambda}$] (ir);
      };
    \end{feynman}
  \end{tikzpicture}
\end{center}
頂点のゲージ場$A_c^\kappa$, $A_c^{\kappa'}$と
\[
\mathcal{O}_g^{(n)\mu_1\cdots\mu_n} \to (i\partial^{\mu_1}A^\nu_c - i\partial^\nu A^{\mu_1}_c)
(i\partial^{\mu_2}) \cdots (i\partial^{\mu_{n-1}}) (i\partial^{\mu_n}A_\nu^c - i\partial_\nu A^{\mu_n}_c)
\]
のゲージ場との縮約を計算する.

$\partial^{\mu_1} \wick{\c1{A}^\nu_c \c1{A}_c^{\kappa'}}$, $\partial^{\mu_n} \wick{\c1{A}_\nu^c \c1{A}_c^\kappa}$の項は
\begin{align*}
  & g^2 f^{adc} f^{bcd} \int \frac{d^dk}{(2\pi)^d} \left( \frac{-i}{k^2} \right)^2 \frac{-i}{(k-p)^2} \\
  &\quad\times [2k^2 g^{\rho\sigma} - 2 (k\cdot p) g^{\rho\sigma} + 5 p^2 g^{\rho\sigma} + 10 k^\rho k^\sigma - 5 k^\rho p^\sigma - 5 k^\sigma p^\rho - 2 p^\rho p^\sigma]
  k^{\mu_1} \cdots k^{\mu_n} \\
  %
  &\to -3 ig^2 \delta^{ab} \int_0^1 dx \, 2(1-x) \int\frac{d^d\ell}{(2\pi)^d} \frac{\ell^2}{(\ell^2 - \Delta)^3}
  \left[ 2 x^n - \frac{2n}{d} x^{n-1} + \frac{10}{d} x^n \right] g^{\rho\sigma} p^{\mu_1} \cdots p^{\mu_n} \\
  %
  &= -3 ig^2 \delta^{ab} \int_0^1 dx \, (1-x) (9x^n - nx^{n-1}) \frac{i}{(4\pi)^2} g^{\rho\sigma} p^{\mu_1} \cdots p^{\mu_n} \frac{2}{\epsilon} \\
  &= 3 \delta^{ab} \frac{g^2}{(4\pi)^2} \left(-\frac{9}{n+2} + \frac{8}{n+1}\right) g^{\rho\sigma} p^{\mu_1} \cdots p^{\mu_n} \frac{2}{\epsilon} .
\end{align*}

$-\partial^\nu \wick{\c1{A}^{\mu_1}_c \c1{A}_c^{\kappa'}}$, $-\partial_\nu \wick{\c1{A}^{\mu_n}_c \c1{A}_c^\kappa}$の項は
\begin{align*}
  & g^2 f^{adc} f^{bcd}
  \int \frac{d^dk}{(2\pi)^d} \left( \frac{-i}{k^2} \right)^2 \frac{-i}{(k-p)^2}
  N^{\rho\sigma\mu_1\mu_n} k^\nu k^{\mu_2} \cdots k^{\mu_{n-1}} k_\nu \\
  %
  &= -3 ig^2 \delta^{ab} \int \frac{d^dk}{(2\pi)^d} \frac{1}{k^2(k-p)^2} N^{\rho\sigma\mu_1\mu_n} k^{\mu_2} \cdots k^{\mu_{n-1}} .
\end{align*}
$N^{\rho\sigma\mu_1\mu_n}$のうち$g^{\rho\sigma}p^{\mu_1}p^{\mu_n}$の形の項は
\begin{align*}
  N^{\rho\sigma\mu_1\mu_n}
  &= k^2 g^{\rho\mu_1} g^{\sigma\mu_n} + 2 (k\cdot p) g^{\rho\mu_1} g^{\sigma\mu_n} + p^2 g^{\rho\mu_1} g^{\sigma\mu_n} \\
  &\quad + g^{\rho\sigma} k^{\mu_n} k^{\mu_1} - 2 g^{\rho\sigma} k^{\mu_n} p^{\mu_1}
  - 2 g^{\rho\sigma} k^{\mu_1} p^{\mu_n} + 4 g^{\rho\sigma} p^{\mu_n} p^{\mu_1} \\
  &\quad - 2 g^{\rho\mu_n} k^{\sigma} k^{\mu_1} + 4 g^{\rho\mu_n} k^{\sigma} p^{\mu_1}
  + g^{\rho\mu_n} k^{\mu_1} p^{\sigma} - 2 g^{\rho\mu_n} p^{\sigma} p^{\mu_1} \\
  &\quad - g^{\rho\mu_1} k^{\sigma} k^{\mu_n} - 4 g^{\rho\mu_1} k^{\sigma} p^{\mu_n}
  + 2 g^{\rho\mu_1} k^{\mu_n} p^{\sigma} - g^{\rho\mu_1} p^{\sigma} p^{\mu_n} \\
  &\quad - g^{\sigma\mu_n} k^{\rho} k^{\mu_1} - 4 g^{\sigma\mu_n} k^{\rho} p^{\mu_1}
   + 2 g^{\sigma\mu_n} k^{\mu_1} p^{\rho} - g^{\sigma\mu_n} p^{\rho} p^{\mu_1} \\
  &\quad - 2 g^{\sigma\mu_1} k^{\rho} k^{\mu_n} + 4 g^{\sigma\mu_1} k^{\rho} p^{\mu_n}
  + g^{\sigma\mu_1} k^{\mu_n} p^{\rho} - 2 g^{\sigma\mu_1} p^{\rho} p^{\mu_n} \\
  &\quad + 4 g^{\mu_1\mu_n} k^{\rho} k^{\sigma} - 2 g^{\mu_1\mu_n} k^{\rho} p^{\sigma}
   - 2 g^{\mu_1\mu_n} k^{\sigma} p^{\rho} + g^{\mu_1\mu_n} p^{\rho} p^{\sigma} \\
  %
  &\to g^{\rho\sigma} k^{\mu_n} k^{\mu_1} - 2 g^{\rho\sigma} k^{\mu_n} p^{\mu_1}
  - 2 g^{\rho\sigma} k^{\mu_1} p^{\mu_n} + 4 g^{\rho\sigma} p^{\mu_n} p^{\mu_1} \\
  %
  &\to (x^2-4x+4) g^{\rho\sigma} p^{\mu_1} p^{\mu_n}
\end{align*}
なので,
\begin{align*}
  &= -3 ig^2 \delta^{ab} \int_0^1 dx \, x^{n-2} (x^2-4x+4) \int \frac{d^d\ell}{(2\pi)^d} \frac{1}{(\ell^2-\Delta)^2} g^{\rho\sigma} p^{\mu_1} \cdots p^{\mu_n} \\
  &= 3 \delta^{ab} \frac{g^2}{(4\pi)^2} \left(\frac{1}{n+1} - \frac{4}{n} + \frac{4}{n-1}\right) g^{\rho\sigma} p^{\mu_1} \cdots p^{\mu_n} \frac{2}{\epsilon} .
\end{align*}

$\partial^{\mu_1} \wick{\c1{A}^\nu_c \c1{A}_c^{\kappa'}}$, $-\partial_\nu \wick{\c1{A}^{\mu_n}_c \c1{A}_c^\kappa}$の項は
\begin{align*}
  & -g^2 f^{adc} f^{bcd}
  \int \frac{d^dk}{(2\pi)^d} \left( \frac{-i}{k^2} \right)^2 \frac{-i}{(k-p)^2}
  N^{\rho\sigma\nu\mu_n} k^{\mu_1} \cdots k^{\mu_{n-1}} k_\nu \\
  %
  &= 3 ig^2 \delta^{ab} \int \frac{d^dk}{(2\pi)^d} \frac{1}{k^4(k-p)^2} N^{\rho\sigma\nu\mu_n} k^{\mu_1} \cdots k^{\mu_{n-1}} k_\nu .
\end{align*}
$-\partial^\nu \wick{\c1{A}^{\mu_1}_c \c1{A}_c^{\kappa'}}$, $\partial^{\mu_n} \wick{\c1{A}_\nu^c \c1{A}_c^\kappa}$の項は
\begin{align*}
  & -g^2 f^{adc} f^{bcd}
  \int \frac{d^dk}{(2\pi)^d} \left( \frac{-i}{k^2} \right)^2 \frac{-i}{(k-p)^2}
  N^{\rho\sigma\mu_1}{}_\nu k^{\mu_2} \cdots k^{\mu_n} k^\nu \\
  %
  &\to 3 ig^2 \delta^{ab} \int \frac{d^dk}{(2\pi)^d} \frac{1}{k^4(k-p)^2} N^{\rho\sigma\mu_n\nu} k^{\mu_1} \cdots k^{\mu_{n-1}} k_\nu .
\end{align*}
$N^{\rho\sigma(\nu\mu_n)}$のうち$g^{\rho\sigma}p^{\mu_1}p^{\mu_n}$の形の項は
\begin{align*}
  &N^{\rho\sigma\nu\mu_n} + N^{\rho\sigma\mu_n\nu} \\
  %%
  &= k^2 g^{\rho\mu_n} g^{\sigma\nu} + k^2 g^{\rho\nu} g^{\sigma\mu_n}
  + 2 (k\cdot p) g^{\rho\mu_n} g^{\sigma\nu} + 2 (k\cdot p) g^{\rho\nu} g^{\sigma\mu_n} + p^2 g^{\rho\mu_n} g^{\sigma\nu} + p^2 g^{\rho\nu} g^{\sigma\mu_n} \\
  &\quad + 2 g^{\rho\sigma} k^{\mu_n} k^{\nu} - 4 g^{\rho\sigma} k^{\mu_n} p^{\nu} - 4 g^{\rho\sigma} k^{\nu} p^{\mu_n} + 8 g^{\rho\sigma} p^{\mu_n} p^{\nu} \\
  &\quad - 3 g^{\rho\mu_n} k^{\sigma} k^{\nu} + 3 g^{\rho\mu_n} k^{\nu} p^{\sigma} - 3 g^{\rho\mu_n} p^{\sigma} p^{\nu} \\
  &\quad - 3 g^{\rho\nu} k^{\sigma} k^{\mu_n} + 3 g^{\rho\nu} k^{\mu_n} p^{\sigma} - 3 g^{\rho\nu} p^{\sigma} p^{\mu_n} \\
  &\quad - 3 g^{\sigma\mu_n} k^{\rho} k^{\nu} + 3 g^{\sigma\mu_n} k^{\nu} p^{\rho} - 3 g^{\sigma\mu_n} p^{\rho} p^{\nu} \\
  &\quad - 3 g^{\sigma\nu} k^{\rho} k^{\mu_n} + 3 g^{\sigma\nu} k^{\mu_n} p^{\rho} - 3 g^{\sigma\nu} p^{\rho} p^{\mu_n} \\
  &\quad + 8 g^{\mu_n\nu} k^{\rho} k^{\sigma} - 4 g^{\mu_n\nu} k^{\rho} p^{\sigma} - 4 g^{\mu_n\nu} k^{\sigma} p^{\rho} + 2 g^{\mu_n\nu} p^{\rho} p^{\sigma} \\
  %
  &\to 2 g^{\rho\sigma} k^{\mu_n} k^{\nu} - 4 g^{\rho\sigma} k^{\mu_n} p^{\nu} - 4 g^{\rho\sigma} k^{\nu} p^{\mu_n} + 8 g^{\rho\sigma} p^{\mu_n} p^{\nu}
  - 3 g^{\rho\nu} k^{\sigma} k^{\mu_n} - 3 g^{\sigma\nu} k^{\rho} k^{\mu_n}
  + 8 g^{\mu_n\nu} k^{\rho} k^{\sigma} .
\end{align*}
従って,
\begin{align*}
  & (N^{\rho\sigma\nu\mu_n} + N^{\rho\sigma\mu_n\nu}) k^{\mu_1} \cdots k^{\mu_{n-1}} k_\nu \\
  %
  &\to 2k^2 g^{\rho\sigma} k^{\mu_1} \cdots k^{\mu_n} - 4(k\cdot p) g^{\rho\sigma} k^{\mu_1} \cdots k^{\mu_n} \\
  &\quad - 4k^2 g^{\rho\sigma} k^{\mu_1} \cdots k^{\mu_{n-1}} p^{\mu_n} + 8(k\cdot p) g^{\rho\sigma} k^{\mu_1} \cdots k^{\mu_{n-1}} p^{\mu_n} \\
  &\quad + 2k^\rho k^\sigma k^{\mu_1} \cdots k^{\mu_n} \\
  %
  &= \left[ 2x^n - \frac{4n}{d}x^{n-1} - 4x^{n-1} + \frac{8(n-1)}{d}x^{n-2} + \frac{2}{d}x^n \right] \ell^2 g^{\rho\sigma} p^{\mu_1} \cdots p^{\mu_n} \\
  &= \left[ \frac{5}{2}x^n - (n+4)x^{n-1} + 2(n-1)x^{n-2} \right] \ell^2 g^{\rho\sigma} p^{\mu_1} \cdots p^{\mu_n} .
\end{align*}
以上から
\begin{align*}
  &= 3 ig^2 \delta^{ab} \int_0^1 dx\, (1-x) [5x^n - 2(n+4)x^{n-1} + 4(n-1)x^{n-2}] \int \frac{d^d\ell}{(2\pi)^d}
  \frac{\ell^2}{(\ell^2-\Delta)^3} g^{\rho\sigma} p^{\mu_1} \cdots p^{\mu_n} \\
  %
  &= -3 \delta^{ab} \frac{g^2}{(4\pi)^2} \left( -\frac{5}{n+2} + \frac{11}{n+1} - \frac{4}{n} \right) g^{\rho\sigma} p^{\mu_1} \cdots p^{\mu_n} \frac{2}{\epsilon} .
\end{align*}

これらを全て足せば,(縮約を逆にしたものも含めて$2$倍,Taylor展開の係数で$1/2$倍)
\begin{align}
    3 \delta^{ab}\frac{g^2}{(4\pi)^2} \left[ - \frac{4}{n+2} - \frac{2}{n+1} + \frac{4}{n-1} \right]
    g^{\rho\sigma} p^{\mu_1} \cdots p^{\mu_n} \frac{2}{\epsilon} .
    \label{problem_18_3_a_gg_AAOg}
\end{align}

$\mathcal{O}_g$の第2項は$g^1$オーダーなので,$e^{i\mathcal{L}}$の展開のうち$g^1$のオーダーのものを考えれば良い.
これは3-boson頂点1つであり,次のdiagramで表現される.

\begin{center}
  \begin{tikzpicture}
    \begin{feynman}[small]
      \vertex (ol) at (-2, 0);
      \vertex (il) at (-0.6, 0);
      \vertex[dot] (ir) at (0.6, 0) {};
      \vertex (or) at (2, 0);
      \draw (ol) node [left] {$A^\rho_a$};
      \draw (or) node [right] {$A^\sigma_b$};
      \diagram*{
      (il) -- [boson, half right, momentum'=$p-k$] (ir);
      (ol) -- [boson, momentum=$p$] (il) -- [boson, half left, momentum=$k$] (ir) -- [boson] (or);
      };
    \end{feynman}
  \end{tikzpicture}
  \begin{tikzpicture}
    \begin{feynman}[small]
      \vertex (ol) at (-2, 0);
      \vertex[dot] (il) at (-0.6, 0) {};
      \vertex (ir) at (0.6, 0);
      \vertex (or) at (2, 0);
      \draw (ol) node [left] {$A^\rho_a$};
      \draw (or) node [right] {$A^\sigma_b$};
      \diagram*{
      (il) -- [boson, half right, momentum'=$p-k$] (ir);
      (ol) -- [boson, momentum=$p$] (il) -- [boson, half left, momentum=$k$] (ir) -- [boson] (or);
      };
    \end{feynman}
  \end{tikzpicture}
\end{center}
$\mathcal{O}_g$の形から,2つのdiagramは同じ振幅を与えるので,左側のみ考える.

\[
  \mathcal{O}_g \to
 (i\partial^{\mu_1}A^\nu_c - i\partial^\nu A^{\mu_1}_c)
  (i\partial^{\mu_2}) \cdots (i g A_e^{\mu_j} f^{cde}) \cdots (i\partial^{\mu_{n-1}}) (i\partial^{\mu_n}A_\nu^d - i\partial_\nu A^{\mu_n}_d)
\]
のゲージ場のうち,$A_b^\sigma$と縮約した際に,$g^{\rho\sigma}$を与えるのは$\partial^{\mu_1}A^\nu_c$(もしくは$\partial^{\mu_n}A_\nu^d$)である.

% 3-boson頂点のゲージ場(のうち$A_a^\rho$と縮約しなかったもの)と$\mathcal{O}_g$の残りのゲージ場($A_e^{\mu_j}$, $\partial^{\mu_n}A_\nu^d - \partial_\nu A^{\mu_n}_d$)の縮約によって2通りのdiagramが得られる.

\begin{center}
  \begin{tikzpicture}
    \begin{feynman}[small]
      \vertex (ol) at (-2, 0);
      \vertex (il) at (-1, 0);
      \vertex[dot] (ir) at (1, 0) {};
      \vertex (or) at (2, 0);
      \draw (ol) node [left] {$A^\rho_a$};
      \draw (or) node [right] {$A^\sigma_b$};
      \draw (ir) node [above right] {$\nu c$};
      \diagram*{
        (il) -- [boson, half right, momentum'=$p-k$, edge label=${\mu_j e}$] (ir);
        (ol) -- [boson, momentum=$p$] (il) -- [boson, half left, momentum=$k$, edge label'=${\nu/\mu_n d}$] (ir) -- [boson] (or);
        };
      \end{feynman}
    \end{tikzpicture}
  %   \begin{tikzpicture}
  %     \begin{feynman}[small]
  %       \vertex (ol) at (-2, 0);
  %       \vertex (il) at (-1, 0);
  %       \vertex[dot] (ir) at (1, 0) {};
  %       \vertex (or) at (2, 0);
  %       \draw (ol) node [left] {$A^\rho_a$};
  %       \draw (or) node [right] {$A^\sigma_b$};
  %       \draw (ir) node [above right] {$\nu c$};
  %     \diagram*{
  %     (il) -- [boson, half right, momentum'=$p-k$, edge label=${\nu/\mu_n d}$] (ir);
  %     (ol) -- [boson, momentum=$p$] (il) -- [boson, half left, momentum=$k$, edge label'=${\mu_j e}$] (ir) -- [boson] (or);
  %     };
  %   \end{feynman}
  % \end{tikzpicture}
\end{center}

% $k \to p-k$とすれば他方のdiagramが得られるので,左側のみ計算する.

% 左側のdiagramで
$\partial^{\mu_n}A_\nu^d$を縮約した項は
\begin{align*}
  & igf^{cde} \delta^{bc} \int \frac{d^dk}{(2\pi)^d} gf^{ade} [\delta^\rho_\nu (p+k)^{\mu_j} + \delta^{\mu_j}_\nu (p-2k)^\rho + g^{\mu_j\rho} (k-2p)_\nu] g^{\nu\sigma} \\
  &\quad \times \frac{-i}{k^2} \frac{-i}{(p-k)^2} p^{\mu_1} \cdots p^{\mu_{j-1}} k^{\mu_{j+1}} \cdots k^{\mu_n} \\
  %
  &\to -3 ig^2 g^{\rho\sigma} \delta^{ab} \int \frac{d^dk}{(2\pi)^d} \frac{1}{k^2(k-p)^2}
  p^{\mu_1} \cdots p^{\mu_{j-1}} (k+p)^{\mu_j} k^{\mu_{j+1}} \cdots k^{\mu_n} \\
  %
  &= -3ig^2 g^{\rho\sigma} \delta^{ab} \int_0^1 dx \, (1+x) x^{n-j} \int \frac{d^d\ell}{(2\pi)^d} \frac{1}{(\ell^2-\Delta)^2} p^{\mu_1} \cdots p^{\mu_n} \\
  %
  &= 3 \delta^{ab} \frac{g^2}{(4\pi)^2} \left( \frac{1}{n-j+1} + \frac{1}{n-j+2} \right) g^{\rho\sigma} p^{\mu_1} \cdots p^{\mu_n} \frac{2}{\epsilon} .
\end{align*}

% 左側のdiagramで
$- \partial_\nu A^{\mu_n}_d$を縮約した項は$0$.

% 右側のdiagramで$\partial^{\mu_n}A_\nu^d$を縮約した項は
% \begin{align*}
%   & -igf^{cde} \delta^{bc} \int \frac{d^dk}{(2\pi)^d} gf^{aed} [g^{\mu_j\rho} (k+p)_\nu + \delta^{\mu_j}_\nu (p-2k)^\rho + \delta^\rho_\nu (k-2p)^{\mu_j}] g^{\nu\sigma} \\
%   &\quad \times \frac{-i}{k^2} \frac{-i}{(p-k)^2} p^{\mu_1} \cdots p^{\mu_{j-1}} (p-k)^{\mu_{j+1}} \cdots (p-k)^{\mu_n} \\
%   %
%   &\to -3 ig^2 g^{\rho\sigma} \delta^{ab} \int \frac{d^dk}{(2\pi)^d} \frac{1}{k^2(k-p)^2}
%   p^{\mu_1} \cdots p^{\mu_{j-1}} (k-2p)^{\mu_j} (p-k)^{\mu_{j+1}} \cdots (p-k)^{\mu_n} \\
%   % %
%   &= - 3 ig^2 g^{\rho\sigma} \delta^{ab} \int_0^1 dx \, (x-2) (1-x)^{n-j} \int \frac{d^d\ell}{(2\pi)^d} \frac{1}{(\ell^2-\Delta)^2} p^{\mu_1} \cdots p^{\mu_n} \\
%   % %
%   &= 3 \delta^{ab} \frac{g^2}{(4\pi)^2} \left( - \frac{1}{n-j+1} - \frac{1}{n-j+2} \right) g^{\rho\sigma} p^{\mu_1} \cdots p^{\mu_n} \frac{2}{\epsilon} .
% \end{align*}
$F^{\mu_1\nu}$, $F^{\mu_n}{}_\nu$からゲージ場を2つ取り出した項はそれぞれ$j=1, n$を与える.
さらにdiagramの対称性は$S=2$なので,
\begin{align}
  \begin{split}
    &- 3 \delta^{ab} \frac{g^2}{(4\pi)^2} 2 \cdot 2 \cdot \frac{1}{2} \sum_{j=1}^n \left( \frac{1}{n-j+1} + \frac{1}{n-j+2} \right) g^{\rho\sigma} p^{\mu_1} \cdots p^{\mu_n} \frac{2}{\epsilon} \\
    &= - 3 \delta^{ab} \frac{g^2}{(4\pi)^2} \left( 4\sum_{j=2}^{n} \frac{1}{j} + \frac{2}{n+1} + 2 \right) g^{\rho\sigma} p^{\mu_1} \cdots p^{\mu_n} \frac{2}{\epsilon} .
  \end{split}
  \label{problem_18_3_a_gg_AAAOg}
\end{align}

\eqref{problem_18_3_a_gg_AAOg}\eqref{problem_18_3_a_gg_AAAOg}を足して,
\[
- \delta_\mathcal{O} =
3 \frac{g^2}{(4\pi)^2} \left[ -2 - 4\sum_{j=2}^{n} \frac{1}{j} - \frac{4}{n+2} - \frac{4}{n+1} + \frac{4}{n-1} \right]
\left(\frac{2}{\epsilon} - \log M^2\right) .
\]
(18.23)(16.74)から
\begin{align*}
  \gamma_\mathcal{O} &= M\frac{\partial}{\partial M} (-\delta_\mathcal{O}+\delta_3) \\
  &= \frac{6g^2}{(4\pi)^2} \left(\frac{1}{3} + \frac{2}{9}n_f + 4\sum_{j=2}^n \frac{1}{j} + \frac{4}{n+2} + \frac{4}{n+1} - \frac{4}{n-1}\right) .
\end{align*}
