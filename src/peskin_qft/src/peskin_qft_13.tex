\chapter{Perturbation Theory Anomalies}
\section{The Axial Current in Two Dimensions}
\subsection{(19.15)}
$\psi$は反交換することに注意して,(4.31)から
\begin{align*}
  \braket{j^\mu(q)} &= \int d^2x \, e^{iq\cdot x} \braket{j^\mu(x)} \\
  &= \int d^2x \, e^{iq\cdot x} \bra\Omega \bar\psi(x) \gamma^\mu \psi(x) \ket\Omega \\
  &\sim i \int d^2x \, e^{iq\cdot x} \int d^2y \, \bra{0} \bar\psi(x) \gamma^\mu \psi(x) \mathcal{L}(y) \ket{0} \\
  %
  &= -ie \int d^2x \int d^2y \, e^{iq\cdot x}
  \bra{0} \bar\psi(x) \gamma^\mu \psi(x) A_\nu(y) \bar\psi(y) \gamma^\nu \psi(y) \ket{0} \\
  %
  &= -ie \int d^2x \int d^2y \, e^{iq\cdot x}
  \bra{0} \wick[offset=4mm]{\c1{\bar\psi}_\alpha(x) (\gamma^\mu)_{\alpha\beta} \c2{\psi}_\beta(x) A_\nu(y)
   \c2{\bar\psi}_\gamma(y) (\gamma^\nu)_{\gamma\delta} \c1{\psi}_\delta(y) \ket{0}} \\
  %
  &= ie \int d^2x \int d^2y \, e^{iq\cdot x}
  \int \frac{d^2k}{(2\pi)^2} \left( \frac{1}{\slashed{k}} \right)_{\delta\alpha} e^{-ik\cdot(y-x)}
  \int \frac{d^2k'}{(2\pi)^2} \left( \frac{1}{\slashed{k'}} \right)_{\beta\gamma} e^{-ik'\cdot(x-y)} \\
  %
  &\qquad\times (\gamma^\mu)^{\alpha\beta} (\gamma^\nu)^{\gamma\delta}
  \int \frac{d^2p}{(2\pi)^2} A_\nu(p) e^{-p\cdot y} \\
  %
  &= ie \int \frac{d^2k}{(2\pi)^2} \Tr\left[ \frac{1}{\slashed{k}} \gamma^\mu \frac{1}{\slashed{k}+\slashed{q}} \gamma^\nu \right] A_\nu(q) .
\end{align*}
これを(7.71)と比べれば良い.

\setcounter{section}{2}
\section{Goldstone Bosons and Chiral Symmetries in QCD}
\subsection{(19.84)}
Lagrangianは
\[
\mathcal{L} = u_L^\dagger i \bar\sigma^\mu D_\mu u_L + d_L^\dagger i \bar\sigma^\mu D_\mu d_L + (R) .
\]
$U(1)$微小変換は
\[
\begin{pmatrix}
  u_L \\ d_L
\end{pmatrix}
\to e^{-i\epsilon}
\begin{pmatrix}
  u_L \\ d_L
\end{pmatrix}
=
\begin{pmatrix}
  e^{-i\epsilon} u_L \\ e^{-i\epsilon} d_L
\end{pmatrix}
\approx
\begin{pmatrix}
  (1-i\epsilon) u_L \\ (1-i\epsilon) d_L
\end{pmatrix}
\]
で与えられる.この変換で$\mathcal{L}$は不変なので,不随するカレントは
\begin{align*}
  \epsilon j^\mu = \frac{\partial\mathcal{L}}{\partial \partial_\mu u_L} \Delta u_L
  + \frac{\partial\mathcal{L}}{\partial \partial_\mu d_L} \Delta d_L
  = u_L^\dagger i \bar\sigma^\mu (-i\epsilon u_L) + d_L^\dagger i \bar\sigma^\mu (-i\epsilon d_L) .
\end{align*}
一方で
\[
Q_L = \left( \frac{1-\gamma^5}{2} \right)
\begin{pmatrix}
  u \\ d
\end{pmatrix}
= \begin{pmatrix}
   \frac{1-\gamma^5}{2} u \\ \frac{1-\gamma^5}{2} d
\end{pmatrix}
\]
に対し
\begin{align*}
  \bar{Q}_L \gamma^\mu Q_L &=
  \begin{pmatrix}
    \bar{u} \frac{1+\gamma^5}{2} & \bar{d} \frac{1+\gamma^5}{2}
  \end{pmatrix}
  \begin{pmatrix}
    \gamma^\mu \frac{1-\gamma^5}{2} u \\ \gamma^\mu \frac{1-\gamma^5}{2} d
  \end{pmatrix}
  \\
  &= \bar{u} \frac{1+\gamma^5}{2} \gamma^\mu \frac{1-\gamma^5}{2} u + (d) \\
  &=
  \begin{pmatrix}
    u_R^\dagger & u_L^\dagger
  \end{pmatrix}
  \begin{pmatrix}
    0 & 0 \\ 0 & 1
  \end{pmatrix}
  \begin{pmatrix}
    0 & \sigma^\mu \\ \bar\sigma^\mu & 0
  \end{pmatrix}
  \begin{pmatrix}
    1 & 0 \\ 0 & 0
  \end{pmatrix}
  \begin{pmatrix}
    u_L \\ u_R
  \end{pmatrix}
  + (d) \\
  &= u_L^\dagger \bar\sigma^\mu u_L + (d)
\end{align*}
なので$j^\mu = \bar{Q}_L \gamma^\mu Q_L$.

$SU(2)$微小変換は
\[
\begin{pmatrix}
  u_L \\ d_L
\end{pmatrix}
\to e^{-i\epsilon_a \tau^a}
\begin{pmatrix}
  u_L \\ d_L
\end{pmatrix}
=
\begin{pmatrix}
  e^{-i\epsilon_a \tau^a} u_L \\ e^{-i\epsilon_a \tau^a} d_L
\end{pmatrix}
\approx
\begin{pmatrix}
  (1-i\epsilon_a \tau^a) u_L \\ (1-i\epsilon_a \tau^a) d_L
\end{pmatrix}
\]
で与えられる.後は同様.

% 怪しいので削除
% \subsection{(19.88)}
% \(\pi\)中間子について,isospinのtripletは
% \[ \pi^b(x) = \bar{Q}(x) \gamma^5 \tau^b Q(x) \]
% で与えられる(\(A=1, 2\)が\(u\), \(d\)に対応)ので,そのFourier変換は
% \begin{align*}
%     \pi^b(p) &= \int d^4x \, e^{ip\cdot x} \pi^b(x) \\
%     &= \int d^4x \, e^{ip\cdot x} \bar{Q}(x) \gamma^5 \tau^b Q(x) \\
%     &= \frac{1}{\sqrt{2}}\int d^4x \, e^{ip\cdot x} \int \frac{d^4k}{(2\pi)^4} e^{-ik\cdot x} \bar{Q}(k)
%     \gamma^5 \tau^b \int \frac{d^4q}{(2\pi)^4} e^{iq\cdot x} Q(q) \\
%     &= \int \frac{d^4q}{(2\pi)^4} \bar{Q}(p+q) \gamma^5 \tau^b Q(q)
% \end{align*}
% となる.縮約は
% \[
% \wick{\c1{Q}_A(p)\c1{\bar{Q}}_B(q)} = (2\pi)^4 \delta^{(4)}(p-q) \delta_{AB} \frac{i}{\slashed{p}-m_A} .
% \]
% \(j^{\mu5a}\)をFourier変換で表せば
% \[
% j^{\mu5a}(x) = \int\frac{d^4k}{(2\pi)^4} \int\frac{d^4k'}{(2\pi)^4} e^{i(k-k')\cdot x}
% \bar{Q}(k) \gamma^\mu \gamma^5 \tau^a Q(k')
% \]
% なので,
% \begin{align*}
%     \bra{0} j^{\mu5a}(x) \pi^b(p) \ket{0}
%     &= \int \frac{d^4q}{(2\pi)^4} \int\frac{d^4k}{(2\pi)^4} \int\frac{d^4k'}{(2\pi)^4} e^{i(k-k')\cdot x} \\
%     &\qquad\times \bra{0} \bar{Q}_A(k) \gamma^\mu \gamma^5 (\tau^a)_{AB} Q_B(k')
%     \bar{Q}_C(p+q) \gamma^5 (\tau^b)_{CD} Q_D(q) \ket{0} \\
%     %
%     &= \int \frac{d^4q}{(2\pi)^4} \int\frac{d^4k}{(2\pi)^4} \int\frac{d^4k'}{(2\pi)^4} e^{i(k-k')\cdot x} \\
%     &\qquad\times \bra{0} \wick{\c1{\bar{Q}}_A(k) \gamma^\mu \gamma^5 (\tau^a)_{AB} \c2{Q}_B(k') \c2{\bar{Q}}_C(p+q) \gamma^5 (\tau^b)_{CD} \c1{Q}_D(q)} \ket{0} + \cdots \\
%     %
%     &= \int \frac{d^4q}{(2\pi)^4} e^{-ip\cdot x}
%     \Tr\left[ \frac{i}{\slashed{q} - m} \gamma^\mu \gamma^5 \frac{i}{\slashed{p} + \slashed{q} - m} \gamma^5 \right] \Tr(\tau^a \tau^b) .
% \end{align*}
% これは(19.88)の右辺を与える.

\chapter{Gauge Theories with Spontaneous Symmetry Breaking}
\section{The Higgs Mechanism}
\subsection{(20.27)}
\(\mathbb{C}^2\)スピノルを\(\mathbb{R}^4\)ベクトルとして表す.
(21.38)から
\begin{align}
  \mathbb{C}^2 \ni
  \frac{1}{\sqrt{2}}
  \begin{pmatrix}
    -\phi^2 - i \phi^1 \\
    h + i\phi^3
  \end{pmatrix}
  \mapsto
  \begin{pmatrix}
    \phi^1 \\ \phi^2 \\ \phi^3 \\ h
  \end{pmatrix}
  \in \mathbb{R}^4
  \label{20_27_C2toR4}
\end{align}
とする.この対応により\(T^a\)は\(4\times4\)行列となる.
(20.14)から
\[ T^a = -i t^a = - \frac{i}{2} \sigma^a . \]
\(a=1\)なら
\begin{align*}
  T^1
  \begin{pmatrix}
    \phi^1 \\ \phi^2 \\ \phi^3 \\ h
  \end{pmatrix}
  &= - \frac{i}{2} \sigma^1
  \frac{1}{\sqrt{2}}
  \begin{pmatrix}
    -\phi^2 - i \phi^1 \\
    h + i\phi^3
  \end{pmatrix}
  = \frac{1}{2\sqrt{2}}
  \begin{pmatrix}
    & -i \\ -i &
  \end{pmatrix}
  \begin{pmatrix}
    -\phi^2 - i \phi^1 \\
    h + i\phi^3
  \end{pmatrix}
  \\
  &= \frac{1}{2\sqrt{2}}
  \begin{pmatrix}
    \phi^3 - ih \\
    -\phi^1 + i \phi^2
  \end{pmatrix}
  = \frac{1}{2}
  \begin{pmatrix}
    h \\ -\phi^3 \\ \phi^2 \\ -\phi^1
  \end{pmatrix}
\end{align*}
なので
\[
T^1 = \frac{1}{2}
\begin{pmatrix}
  & & & 1 \\ & & -1 & \\ & 1 & & \\ -1 & & &
\end{pmatrix}
.
\]

\(a=2\)なら
\begin{align*}
  T^2
  \begin{pmatrix}
    \phi^1 \\ \phi^2 \\ \phi^3 \\ h
  \end{pmatrix}
  &= - \frac{i}{2} \sigma^2
  \frac{1}{\sqrt{2}}
  \begin{pmatrix}
    -\phi^2 - i \phi^1 \\
    h + i\phi^3
  \end{pmatrix}
  = \frac{1}{2\sqrt{2}}
  \begin{pmatrix}
    & -1 \\ 1 &
  \end{pmatrix}
  \begin{pmatrix}
    -\phi^2 - i \phi^1 \\
    h + i\phi^3
  \end{pmatrix}
  \\
  &= \frac{1}{2\sqrt{2}}
  \begin{pmatrix}
    -h - i\phi^3 \\
    -\phi^2 - i \phi^1
  \end{pmatrix}
  = \frac{1}{2}
  \begin{pmatrix}
    \phi^3 \\ h \\ -\phi^1 \\ -\phi^2
  \end{pmatrix}
\end{align*}
なので
\[
T^2 = \frac{1}{2}
\begin{pmatrix}
  & & 1 & \\ & & & 1 \\ -1 & & & \\ & -1 & &
\end{pmatrix}
.
\]

\(a=3\)なら
\begin{align*}
  T^3
  \begin{pmatrix}
    \phi^1 \\ \phi^2 \\ \phi^3 \\ h
  \end{pmatrix}
  &= - \frac{i}{2} \sigma^3
  \frac{1}{\sqrt{2}}
  \begin{pmatrix}
    -\phi^2 - i \phi^1 \\
    h + i\phi^3
  \end{pmatrix}
  = \frac{1}{2\sqrt{2}}
  \begin{pmatrix}
    -i & \\ & i
  \end{pmatrix}
  \begin{pmatrix}
    -\phi^2 - i \phi^1 \\
    h + i\phi^3
  \end{pmatrix}
  \\
  &= \frac{1}{2\sqrt{2}}
  \begin{pmatrix}
    -\phi^1 + i \phi^2 \\
    -\phi^3 + ih
  \end{pmatrix}
  = \frac{1}{2}
  \begin{pmatrix}
    -\phi^2 \\ \phi^1 \\ h \\ -\phi^3
  \end{pmatrix}
\end{align*}
なので
\[
T^3 = \frac{1}{2}
\begin{pmatrix}
  & -1 & & \\ 1 & & & \\ & & & 1 \\ & & -1 &
\end{pmatrix}
.
\]

以上から
\begin{align}
  \begin{split}
    T^1 &= \frac{1}{2}
    \begin{pmatrix}
      & & & 1 \\ & & -1 & \\ & 1 & & \\ -1 & & &
    \end{pmatrix}
    , \quad
    T^2 = \frac{1}{2}
    \begin{pmatrix}
      & & 1 & \\ & & & 1 \\ -1 & & & \\ & -1 & &
    \end{pmatrix}
    , \quad
    T^3 = \frac{1}{2}
    \begin{pmatrix}
      & -1 & & \\ 1 & & & \\ & & & 1 \\ & & -1 &
    \end{pmatrix}
    .
  \end{split}
  \label{20_27_TR4}
\end{align}
これらを
\[
\begin{pmatrix}
  \phi^1 \\ \phi^2 \\ \phi^3 \\ 0
\end{pmatrix}
\in \mathbb{R}^3
\]
に制限すれば
\begin{align}
  T^1 =
  \begin{pmatrix}
    & & \\ & & -1 \\ & 1 &
  \end{pmatrix}
  , \quad
  T^2 =
  \begin{pmatrix}
    & & 1\\ & &  \\ -1 & &
  \end{pmatrix}
  , \quad
  T^3 =
  \begin{pmatrix}
    & -1 & \\ 1 & & \\ & &
  \end{pmatrix}
  .
\label{20_27_TR3}
\end{align}
特に,\(a=1, 2, 3\)に対して
\[ (T^a)_{bc} = \epsilon^{bac} .  \]

(20.27)右辺は\(\phi^c \in \mathbb{R}^3\)に変換したベクトル.
\(\phi \in \mathbb{C}^2\)は\(SU(2)\)の基本表現に属するが,\(\mathbb{R}^3\)ベクトルへの変換によって(見かけ上)随伴表現に属している.

\section{The Glashow-Weinberg-Salam Theory of Weak Interactions}
\subsection{(20.80)}
(20.63)(20.64)から
\[
W_\mu^\pm = \frac{1}{\sqrt{2}} (A_\mu^1 \mp iA_\mu^2) , \quad
Z_\mu^0 = \frac{1}{\sqrt{g^2+g'^2}} (gA_\mu^3 - g'B_\mu) , \quad
A_\mu = \frac{1}{\sqrt{g^2+g'^2}} (g'A_\mu^3 + gB_\mu) .
\]
(20.68)(20.70)から
\[
e = \frac{gg'}{\sqrt{g^2+g'^2}} , \quad
\cos\theta_w = \frac{g}{\sqrt{g^2+g'^2}} , \quad
\sin\theta_w = \frac{g'}{\sqrt{g^2+g'^2}} .
\]
さらに
\[
A_\mu^3 = A_\mu \sin\theta_w + Z_\mu^0 \cos\theta_w , \quad
B_\mu = A_\mu \cos\theta_w - Z_\mu^0 \sin\theta_w , \quad
\]

$E_L$の項は
\begin{align*}
  &
  \begin{pmatrix}
    \bar{\nu}_L & \bar{e}_L
  \end{pmatrix}
  i\gamma^\mu \left( -igA_\mu^a \tau^a + i \frac{1}{2}g'B_\mu \right)
  \begin{pmatrix}
    \nu_L \\ e_L
  \end{pmatrix} \\
  %
  &=
  \begin{pmatrix}
    \bar{\nu}_L & \bar{e}_L
  \end{pmatrix}
  i\gamma^\mu \left[ -i\frac{g}{\sqrt{2}}
  \begin{pmatrix}
    & W_\mu^+ \\ W_\mu^- &
  \end{pmatrix}
  - i \frac{g}{2} A_\mu^3
  \begin{pmatrix}
    1 & \\ & -1
  \end{pmatrix}
  + i \frac{g'}{2} B_\mu
  \begin{pmatrix}
    1 & \\ & 1
  \end{pmatrix}
  \right]
  \begin{pmatrix}
    \nu_L \\ e_L
  \end{pmatrix} \\
  %
  &= \frac{g}{\sqrt{2}} W_\mu^+ (\bar{\nu}_L \gamma^\mu e_L)
  + \frac{g}{\sqrt{2}} W_\mu^- (\bar{e}_L \gamma^\mu \nu_L) \\
  &\quad + \frac{1}{2} \bar{\nu}_L \gamma^\mu (gA_\mu^3-g'B_\mu) \nu_L
  - \frac{1}{2} \bar{e}_L \gamma^\mu (gA_\mu^3+g'B_\mu) e_L .
\end{align*}
第1, 2項は$J_W^{\mu+}$, $J_W^{\mu-}$の第1項を与える.第3項は
\[
\frac{1}{2} \bar{\nu}_L \gamma^\mu (gA_\mu^3-g'B_\mu) \nu_L
= \frac{\sqrt{g^2+g'^2}}{2} Z_\mu^0 \bar{\nu}_L \gamma^\mu \nu_L
= \frac{g}{2} \frac{Z_\mu^0}{\cos\theta_w} (\bar{\nu}_L \gamma^\mu \nu_L)
\]
なので$J_Z^\mu$の第1項を与える.第4項は
\begin{align*}
  & -\frac{1}{2} \bar{e}_L \gamma^\mu (gA_\mu^3+g'B_\mu) e_L \\
  &= -\frac{\sqrt{g^2+g'^2}}{2} (A_\mu^3\cos\theta_w+B_\mu\sin\theta_w) (\bar{e}_L \gamma^\mu e_L) \\
  &= -\frac{\sqrt{g^2+g'^2}}{2} \left[ 2A_\mu \frac{e}{\sqrt{g^2+g'^2}} + Z_\mu^0(1-2\sin^2\theta_w) \right] (\bar{e}_L \gamma^\mu e_L) \\
  &= - eA_\mu (\bar{e}_L \gamma^\mu e_L)
  + g \frac{Z_\mu^0}{\cos\theta_w} \left( -\frac{1}{2}+\sin^2\theta \right) (\bar{e}_L \gamma^\mu e_L)
\end{align*}
なので$J_\text{EM}^\mu$の第1項の左成分と$J_Z^\mu$の第2成分を与える.

$e_R$の項は
\begin{align*}
  \bar{e}_R i\gamma^\mu (ig'B_\mu) e_R
  &= g'(Z_\mu^0\sin\theta_w - A_\mu\cos\theta_w) (\bar{e}_R \gamma^\mu e_R) \\
  &= g \frac{\sin^2\theta_w}{\cos\theta_w} Z_\mu^0 (\bar{e}_R \gamma^\mu e_R)
  - e A_\mu (\bar{e}_R \gamma^\mu e_R)
\end{align*}
なので$J_Z^\mu$の第3項と$J_\text{EM}^\mu$の第1項の右成分を与える.
$u_R$は$Y=2/3$,$d_R$は$Y=-1/3$とすれば良い.

$Q_L$の項は
\begin{align*}
  &
  \begin{pmatrix}
    \bar{u}_L & \bar{d}_L
  \end{pmatrix}
  i\gamma^\mu \left( -igA_\mu^a \tau^a - i \frac{1}{6}g'B_\mu \right)
  \begin{pmatrix}
    u_L \\ d_L
  \end{pmatrix} \\
  %
  &=
  \begin{pmatrix}
    \bar{u}_L & \bar{d}_L
  \end{pmatrix}
  i\gamma^\mu \left[ -i\frac{g}{\sqrt{2}}
  \begin{pmatrix}
    & W_\mu^+ \\ W_\mu^- &
  \end{pmatrix}
  - i \frac{g}{2} A_\mu^3
  \begin{pmatrix}
    1 & \\ & -1
  \end{pmatrix}
  - i \frac{g'}{6} B_\mu
  \begin{pmatrix}
    1 & \\ & 1
  \end{pmatrix}
  \right]
  \begin{pmatrix}
    u_L \\ d_L
  \end{pmatrix} \\
  %
  &= \frac{g}{\sqrt{2}} W_\mu^+ (\bar{u}_L \gamma^\mu d_L)
  + \frac{g}{\sqrt{2}} W_\mu^- (\bar{d}_L \gamma^\mu u_L) \\
  &\quad + \frac{1}{2} \bar{u}_L \gamma^\mu \left(gA_\mu^3+\frac{1}{3}g'B_\mu\right) u_L
  - \frac{1}{2} \bar{d}_L \gamma^\mu \left(gA_\mu^3-\frac{1}{3}g'B_\mu\right) d_L .
\end{align*}
第1, 2項は$J_W^{\mu+}$, $J_W^{\mu-}$の第2項を与える.第3項は
\begin{align*}
  & \frac{1}{2} \bar{u}_L \gamma^\mu \left(gA_\mu^3+\frac{1}{3}g'B_\mu\right) u_L \\
  &= \frac{\sqrt{g^2+g'^2}}{2} \left(A_\mu^3\cos\theta_w+\frac{1}{3}B_\mu\sin\theta_w\right) (\bar{u}_L \gamma^\mu u_L) \\
  &= \frac{g}{2\cos\theta_w} \left[ \frac{4}{3} \cos\theta_w\sin\theta_w A_\mu + Z_\mu^0 \left(\cos^2\theta_w-\frac{\sin^2\theta_w}{3}\right) \right] (\bar{u}_L \gamma^\mu u_L)\\
  &= \frac{2}{3} e A_\mu (\bar{u}_L \gamma^\mu u_L)
  + \frac{g}{2} \frac{Z_\mu^0}{\cos\theta_w} \left(1-\frac{4}{3}\sin^2\theta_w\right) (\bar{u}_L \gamma^\mu u_L)
\end{align*}
なので$J_\text{EM}^\mu$の第2項と$J_Z^\mu$の第4項を与える.第4項は
\begin{align*}
  & - \frac{1}{2} \bar{d}_L \gamma^\mu \left(gA_\mu^3-\frac{1}{3}g'B_\mu\right) d_L \\
  &= \frac{\sqrt{g^2+g'^2}}{2} \left(-A_\mu^3\cos\theta_w+\frac{1}{3}B_\mu\sin\theta_w\right) (\bar{d}_L \gamma^\mu d_L) \\
  &= \frac{g}{2\cos\theta_w} \left[ -\frac{2}{3} \cos\theta_w\sin\theta_w A_\mu - Z_\mu^0 \left(\cos^2\theta_w+\frac{\sin^2\theta_w}{3}\right) \right] (\bar{d}_L \gamma^\mu d_L)\\
  &= -\frac{1}{3} e A_\mu (\bar{u}_L \gamma^\mu u_L)
  + \frac{g}{2} \frac{Z_\mu^0}{\cos\theta_w} \left(-1+\frac{2}{3}\sin^2\theta_w\right) (\bar{u}_L \gamma^\mu u_L)
\end{align*}
なので$J_\text{EM}^\mu$の第3項と$J_Z^\mu$の第5項を与える.

\section*{Problems}\addcontentsline{toc}{section}{Problems}
\subsection{Problem 20.4: Neutral-current deep inelastic scattering}
partonレベルの散乱は次のようになる.

\begin{center}
  \begin{tikzpicture}[>=stealth]
    \draw[->] (-1, 0) node [left] {\(\nu (k)\)} -- (-0.2, 0);
    \draw[->] (1, 0) node [right] {\(q_f (p)\)} -- (0.2, 0);
    \draw[->] (30: 0.2) -- (30: 1) node [right] {\(\nu (k')\)};
    \draw[->] (210: 0.2) -- (210: 1) node [left] {\(q_f (p')\)};
    %
    \coordinate (O) at (0, 0);
    \coordinate (pb) at (1, 0);
    \coordinate (ka) at (30: 1);
    \pic [draw, "$\theta$", angle radius=0.5cm, angle eccentricity=1.4] {angle=pb--O--ka};
  \end{tikzpicture}
\end{center}

Mandelstam variableは
\[
p\cdot k = p'\cdot k' = \frac{\hat{s}}{2} , \quad
p\cdot p' = k\cdot k' = - \frac{\hat{t}}{2} , \quad
p\cdot k' = p'\cdot k = - \frac{\hat{u}}{2} .
\]

(17.31)(20.80)からeffective Lagrangianは
\[
\varDelta\mathcal{L} = \frac{g^2}{\cos^2\theta_w} \frac{1}{m_Z{}^2}
\left[\bar\nu \gamma_\mu \frac{1}{2} \frac{1-\gamma^5}{2} \nu \right]
\left[\bar{q} \gamma^\mu (T^3-\sin^2\theta_wQ) \frac{1\mp\gamma^5}{2} q \right] .
\]

\subsubsection{\(\nu + u_L \to \nu + u_L\)}
不変振幅は
\[
i\mathcal{M} = \frac{g^2}{m_Z{}^2\cos^2\theta_w} \frac{\frac{1}{2} - \frac{2}{3}\sin^2\theta_w}{2}
\left[\bar\nu(k') \gamma_\mu \frac{1-\gamma^5}{2} \nu(k) \right]
\left[\bar{u}(p') \gamma^\mu \frac{1-\gamma^5}{2} u(p) \right]
\]
なので(A.27)から
\begin{align*}
  \lvert\mathcal{M}\rvert^2
  &= \sum_\text{spins} \lvert\mathcal{M}\rvert^2 \\
  %
  &= \frac{1}{4} \left(\frac{g^2}{m_Z{}^2\cos^2\theta_w}\right)^2
  \left(\frac{1}{2} - \frac{2}{3}\sin^2\theta_w\right)^2
  \Tr \left[ \slashed{k}' \gamma_\mu \frac{1-\gamma^5}{2} \slashed{k} \gamma_\nu \right]
  \Tr \left[ \slashed{p}' \gamma^\mu \frac{1-\gamma^5}{2} \slashed{p} \gamma^\nu \right] \\
  %
  &= \left(\frac{g^2}{m_Z{}^2\cos^2\theta_w}\right)^2
  \left(\frac{1}{2} - \frac{2}{3}\sin^2\theta_w\right)^2
  \left[ k^\mu k'^\nu + k'^\mu k^\nu - (k\cdot k') g^{\mu\nu} + ik'_\alpha k_\beta \epsilon^{\alpha\mu\beta\nu} \right] \\
  &\quad\times \left[ p_\mu p'_\nu + p'_\mu p_\nu - (p\cdot p') g_{\mu\nu} + ip'^\gamma p^\delta \epsilon_{\gamma\mu\delta\nu} \right] .
\end{align*}
運動量の積は(A.30)から
\begin{align*}
  & [k^\mu k'^\nu + k'^\mu k^\nu - (k\cdot k') g^{\mu\nu}]
  [p_\mu p'_\nu + p'_\mu p_\nu - (p\cdot p') g_{\mu\nu}]
  - k'_\alpha k_\beta p'^\gamma p^\delta \epsilon^{\alpha\mu\beta\nu}\epsilon_{\gamma\mu\delta\nu} \\
  %
  &= 2 (p\cdot k)(p'\cdot k') + 2 (p\cdot k')(p'\cdot k) + 2 (p\cdot k)(p'\cdot k') - 2 (p\cdot k')(p'\cdot k) \\
  &= 4 (p\cdot k)(p'\cdot k')
\end{align*}
なので,
\[
\lvert\mathcal{M}(\nu u_L)\rvert^2
= 4 \left(\frac{g^2}{m_Z{}^2\cos^2\theta_w}\right)^2
\left(\frac{1}{2} - \frac{2}{3}\sin^2\theta_w\right)^2
(p\cdot k)(p'\cdot k') .
\]

\subsubsection{\(\nu + u_R \to \nu + u_R\)}
不変振幅は
\[
i\mathcal{M} = \frac{g^2}{m_Z{}^2\cos^2\theta_w} \frac{ - \frac{2}{3}\sin^2\theta_w}{2}
\left[\bar\nu(k') \gamma_\mu \frac{1-\gamma^5}{2} \nu(k) \right]
\left[\bar{u}(p') \gamma^\mu \frac{1+\gamma^5}{2} u(p) \right]
\]
なので(A.27)から
\begin{align*}
  \lvert\mathcal{M}\rvert^2
  %
  &= \frac{1}{4} \left(\frac{g^2}{m_Z{}^2\cos^2\theta_w}\right)^2
  \left( - \frac{2}{3}\sin^2\theta_w\right)^2
  \Tr \left[ \slashed{k}' \gamma_\mu \frac{1-\gamma^5}{2} \slashed{k} \gamma_\nu \right]
  \Tr \left[ \slashed{p}' \gamma^\mu \frac{1+\gamma^5}{2} \slashed{p} \gamma^\nu \right] \\
  %
  &= \left(\frac{g^2}{m_Z{}^2\cos^2\theta_w}\right)^2
  \left( - \frac{2}{3}\sin^2\theta_w\right)^2
  \left[ k^\mu k'^\nu + k'^\mu k^\nu - (k\cdot k') g^{\mu\nu} + ik'_\alpha k_\beta \epsilon^{\alpha\mu\beta\nu} \right] \\
  &\quad\times \left[ p_\mu p'_\nu + p'_\mu p_\nu - (p\cdot p') g_{\mu\nu} - ip'^\gamma p^\delta \epsilon_{\gamma\mu\delta\nu} \right] .
\end{align*}
運動量の積は(A.30)から
\begin{align*}
  & [k^\mu k'^\nu + k'^\mu k^\nu - (k\cdot k') g^{\mu\nu}]
  [p_\mu p'_\nu + p'_\mu p_\nu - (p\cdot p') g_{\mu\nu}]
  + k'_\alpha k_\beta p'^\gamma p^\delta \epsilon^{\alpha\mu\beta\nu}\epsilon_{\gamma\mu\delta\nu} \\
  %
  &= 2 (p\cdot k)(p'\cdot k') + 2 (p\cdot k')(p'\cdot k) - 2 (p\cdot k)(p'\cdot k') + 2 (p\cdot k')(p'\cdot k) \\
  &= 4 (p\cdot k')(p'\cdot k)
\end{align*}
なので,
\[
\lvert\mathcal{M}(\nu u_R)\rvert^2
= 4 \left(\frac{g^2}{m_Z{}^2\cos^2\theta_w}\right)^2
\left( - \frac{2}{3}\sin^2\theta_w\right)^2
(p\cdot k')(p'\cdot k) .
\]

同様にして
\begin{align*}
  \lvert\mathcal{M}(\nu d_L)\rvert^2
  &= 4 \left(\frac{g^2}{m_Z{}^2\cos^2\theta_w}\right)^2
  \left(-\frac{1}{2} + \frac{1}{3}\sin^2\theta_w\right)^2
  (p\cdot k)(p'\cdot k') , \\
  %
  \lvert\mathcal{M}(\nu d_R)\rvert^2
  &= 4 \left(\frac{g^2}{m_Z{}^2\cos^2\theta_w}\right)^2
  \left(\frac{1}{3}\sin^2\theta_w\right)^2
  (p\cdot k')(p'\cdot k) .
\end{align*}

\subsubsection{antiquark}
\(p \leftrightarrow p'\)とすればよいので,
\begin{align*}
  \lvert\mathcal{M}(\nu\bar{u}_L)\rvert^2
  &= 4 \left(\frac{g^2}{m_Z{}^2\cos^2\theta_w}\right)^2
  \left(\frac{1}{2} - \frac{2}{3}\sin^2\theta_w\right)^2
  (p'\cdot k)(p\cdot k') , \\
  %
  \lvert\mathcal{M}(\nu\bar{u}_R)\rvert^2
  &= 4 \left(\frac{g^2}{m_Z{}^2\cos^2\theta_w}\right)^2
  \left( - \frac{2}{3}\sin^2\theta_w\right)^2
  (p'\cdot k')(p\cdot k) , \\
  %
  \lvert\mathcal{M}(\nu\bar{d}_L)\rvert^2
  &= 4 \left(\frac{g^2}{m_Z{}^2\cos^2\theta_w}\right)^2
  \left(-\frac{1}{2} + \frac{1}{3}\sin^2\theta_w\right)^2
  (p'\cdot k)(p\cdot k') , \\
  %
  \lvert\mathcal{M}(\nu\bar{d}_R)\rvert^2
  &= 4 \left(\frac{g^2}{m_Z{}^2\cos^2\theta_w}\right)^2
  \left(\frac{1}{3}\sin^2\theta_w\right)^2
  (p'\cdot k')(p\cdot k) .
\end{align*}

\subsubsection{cross section}
(17.27)から
\[ y = \frac{\hat{s}+\hat{u}}{\hat{s}} = - \frac{\hat{t}}{\hat{s}} . \]
(14.1)(14.2)(14.3)と同様に
\[
\frac{d\sigma}{dy} = - \hat{s} \frac{d\sigma}{d\hat{t}} = - 2 \frac{d\sigma}{d\cos\theta}
= - \frac{1}{16\pi \hat{s}} \lvert\mathcal{M}\rvert^2 .
\]

up quarkの散乱断面積は
\begin{align*}
  \frac{d\sigma(\nu u)}{dy}
  &= - \frac{1}{16\pi \hat{s}} \frac{\lvert\mathcal{M}(\nu u_L)\rvert^2 + \lvert\mathcal{M}(\nu u_R)\rvert^2}{2} \\
  %
  &= - \frac{1}{8\pi \hat{s}} \left(\frac{g^2}{m_Z{}^2\cos^2\theta_w}\right)^2
  \left[ \left(\frac{1}{2} - \frac{2}{3}\sin^2\theta_w\right)^2 \frac{\hat{s}^2}{4}
  + \left( - \frac{2}{3}\sin^2\theta_w\right)^2 \frac{\hat{u}^2}{4} \right] \\
  %
  &= - \frac{\hat{s}}{32\pi} \left(\frac{g^2}{m_Z{}^2\cos^2\theta_w}\right)^2
  \left[ \left(\frac{1}{2} - \frac{2}{3}\sin^2\theta_w\right)^2
  + \left( - \frac{2}{3}\sin^2\theta_w\right)^2 (1-y)^2 \right] \\
  %
  &= - \frac{sx}{32\pi} \left(\frac{g^2}{m_Z{}^2\cos^2\theta_w}\right)^2
  \left[ \left(\frac{1}{2} - \frac{2}{3}\sin^2\theta_w\right)^2
  + \left( - \frac{2}{3}\sin^2\theta_w\right)^2 (1-y)^2 \right] .
\end{align*}

down quarkの散乱断面積は
\begin{align*}
  \frac{d\sigma(\nu d)}{dy}
  &= - \frac{1}{16\pi \hat{s}} \frac{\lvert\mathcal{M}(\nu d_L)\rvert^2 + \lvert\mathcal{M}(\nu d_R)\rvert^2}{2} \\
  %
  &= - \frac{1}{8\pi \hat{s}} \left(\frac{g^2}{m_Z{}^2\cos^2\theta_w}\right)^2
  \left[ \left(-\frac{1}{2} + \frac{1}{3}\sin^2\theta_w\right)^2 \frac{\hat{s}^2}{4}
  + \left( \frac{1}{3}\sin^2\theta_w\right)^2 \frac{\hat{u}^2}{4} \right] \\
  %
  &= - \frac{sx}{32\pi} \left(\frac{g^2}{m_Z{}^2\cos^2\theta_w}\right)^2
  \left[ \left( -\frac{1}{2} + \frac{1}{3}\sin^2\theta_w\right)^2
  + \left( \frac{1}{3}\sin^2\theta_w\right)^2 (1-y)^2 \right] .
\end{align*}

up antiquarkの散乱断面積は
\begin{align*}
  \frac{d\sigma(\nu\bar{u})}{dy}
  &= - \frac{1}{16\pi \hat{s}} \frac{\lvert\mathcal{M}(\nu\bar{u}_L)\rvert^2 + \lvert\mathcal{M}(\nu\bar{u}_R)\rvert^2}{2} \\
  %
  &= - \frac{1}{8\pi \hat{s}} \left(\frac{g^2}{m_Z{}^2\cos^2\theta_w}\right)^2
  \left[ \left(\frac{1}{2} - \frac{2}{3}\sin^2\theta_w\right)^2 \frac{\hat{u}^2}{4}
  + \left( - \frac{2}{3}\sin^2\theta_w\right)^2 \frac{\hat{s}^2}{4} \right] \\
  %
  &= - \frac{sx}{32\pi} \left(\frac{g^2}{m_Z{}^2\cos^2\theta_w}\right)^2
  \left[ \left(\frac{1}{2} - \frac{2}{3}\sin^2\theta_w\right)^2 (1-y)^2
  + \left( - \frac{2}{3}\sin^2\theta_w\right)^2 \right] .
\end{align*}

down antiquarkの散乱断面積は
\begin{align*}
  \frac{d\sigma(\nu\bar{d})}{dy}
  &= - \frac{1}{16\pi \hat{s}} \frac{\lvert\mathcal{M}(\nu\bar{d}_L)\rvert^2 + \lvert\mathcal{M}(\nu\bar{d}_R)\rvert^2}{2} \\
  %
  &= - \frac{1}{8\pi \hat{s}} \left(\frac{g^2}{m_Z{}^2\cos^2\theta_w}\right)^2
  \left[ \left(-\frac{1}{2} + \frac{1}{3}\sin^2\theta_w\right)^2 \frac{\hat{u}^2}{4}
  + \left( \frac{1}{3}\sin^2\theta_w\right)^2 \frac{\hat{s}^2}{4} \right] \\
  %
  &= - \frac{sx}{32\pi} \left(\frac{g^2}{m_Z{}^2\cos^2\theta_w}\right)^2
  \left[ \left( -\frac{1}{2} + \frac{1}{3}\sin^2\theta_w\right)^2 (1-y)^2
  + \left( \frac{1}{3}\sin^2\theta_w\right)^2 \right] .
\end{align*}

\(\bar\nu + q \to \bar\nu + q\)の場合は\(k \leftrightarrow k'\)となるので,\(\hat{s}^2\)と\(\hat{u}^2\)を入れ替えればよい.
従って,
\begin{align*}
  \frac{d\sigma(\bar\nu u)}{dy}
  &= - \frac{sx}{32\pi} \left(\frac{g^2}{m_Z{}^2\cos^2\theta_w}\right)^2
  \left[ \left(\frac{1}{2} - \frac{2}{3}\sin^2\theta_w\right)^2 (1-y)^2
  + \left( - \frac{2}{3}\sin^2\theta_w\right)^2 \right] , \\
  %
  \frac{d\sigma(\bar\nu d)}{dy}
  &= - \frac{sx}{32\pi} \left(\frac{g^2}{m_Z{}^2\cos^2\theta_w}\right)^2
  \left[ \left( -\frac{1}{2} + \frac{1}{3}\sin^2\theta_w\right)^2 (1-y)^2
  + \left( \frac{1}{3}\sin^2\theta_w\right)^2 \right] , \\
  %
  \frac{d\sigma(\bar\nu\bar{u})}{dy}
  &= - \frac{sx}{32\pi} \left(\frac{g^2}{m_Z{}^2\cos^2\theta_w}\right)^2
  \left[ \left(\frac{1}{2} - \frac{2}{3}\sin^2\theta_w\right)^2
  + \left( - \frac{2}{3}\sin^2\theta_w\right)^2 (1-y)^2 \right] , \\
  %
  \frac{d\sigma(\bar\nu\bar{d})}{dy}
  &= - \frac{sx}{32\pi} \left(\frac{g^2}{m_Z{}^2\cos^2\theta_w}\right)^2
  \left[ \left( -\frac{1}{2} + \frac{1}{3}\sin^2\theta_w\right)^2
  + \left( \frac{1}{3}\sin^2\theta_w\right)^2 (1-y)^2 \right] .
\end{align*}

なお(20.73)(20.91)から
\[
\frac{1}{32\pi} \left(\frac{g^2}{m_Z{}^2\cos^2\theta_w}\right)^2 = \frac{G_F{}^2}{\pi} .
\]
