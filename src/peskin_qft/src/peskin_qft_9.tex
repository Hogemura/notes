\chapter{Critical Exponents and Scalar Field Theory}
\setcounter{section}{2}
\section{The Nonlinear Sigma Model}
\subsection{Figure 13.1}
非線形シグマ模型(13.73):
\[
\mathcal{L} = \frac{1}{2g^2} \left\lvert \partial_\mu \vec\pi \right\rvert
+ \frac{1}{2g^2} \left( \vec\pi \cdot \partial_\mu \vec\pi \right)^2
\]
の4点相関函数を求める.伝播函数は
\begin{align*}
  \wick{\c1{\pi}^i(x) \c1{\pi}^j(y)} &= \int \frac{d^2k}{(2\pi)^2} \frac{ig^2}{p^2} \delta^{ij} e^{-ik \cdot (x-y)},  \\
  \wick{\c1{\pi}^i(p) \c1{\pi}^j(q)} &= (2\pi)^2 \mathop{\delta^{(2)}}(p+q) \frac{ig^2}{p^2} \delta^{ij} .
\end{align*}
(4.31)から
\begin{align*}
  &\bra\Omega T \left\{ \pi^i(x_1) \pi^j(x_2) \pi^k(x_3) \pi^l(x_4) \right\} \ket\Omega \\
  &\approx \frac{i}{2g^2} \sum_{rs} \int d^2x
  \bra{0} T \left\{ \pi^i(x_1) \pi^j(x_2) \pi^k(x_3) \pi^l(x_4) \pi^r(x) \partial_\mu \pi^r(x) \pi^s(x) \partial^\mu \pi^s(x) \right\} \ket{0} \\
  %
  &= \frac{i}{2g^2} \left( \prod_{i=1}^4 \frac{d^2p_i}{(2\pi)^2} e^{-i p_i \cdot x_i} \right)
  \left( \prod_{i=1}^4 \frac{d^2k_i}{(2\pi)^2} \right)
  \sum_{rs} \int d^2x e^{-i (k_1 + k_2 + k_3 + k_4) \cdot x} \\
  & \quad\times \bra{0} T \left\{ \pi^i(p_1) \pi^j(p_2) \pi^k(p_3) \pi^l(p_4)
  \pi^r(k_1) (-ik_{2\mu}) \pi^r(k_2) \pi^s(k_3) (-ik_4^\mu) \pi^s(k_4) \right\} \ket{0} \\
  %
  &= - \frac{i}{2g^2} \left( \prod_{i=1}^4 \frac{d^2p_i}{(2\pi)^2} e^{-i p_i \cdot x_i} \right)
  \left( \prod_{i=1}^4 \frac{d^2k_i}{(2\pi)^2} \right) (k_2 \cdot k_4)
  \sum_{rs} (2\pi)^2 \mathop{\delta^{(2)}}(k_1 + k_2 + k_3 + k_4) \\
  & \quad\times \bra{0} T \left\{ \pi^i(p_1) \pi^j(p_2) \pi^k(p_3) \pi^l(p_4)
  \pi^r(k_1) \pi^r(k_2) \pi^s(k_3) \pi^s(k_4) \right\} \ket{0} .
\end{align*}
縮約は全部で24通りある.
例えば,$\delta^{ij} \delta^{kl}$を与える項は$(p_1, p_2)$と$(k_1, k_2)$を縮約する(4通り)か,$(p_1, p_2)$と$(k_3, k_4)$を縮約する(4通り),合計8通り.伝播函数を代入すれば
\[
- \frac{i}{g^2} \left( \prod_{i=1}^4 \frac{d^2p_i}{(2\pi)^2} e^{-i p_i \cdot x_i} \frac{ig^2}{p_i{}^2} \right)
(2\pi)^2 \mathop{\delta^{(2)}}(p_1 + p_2 + p_3 + p_4) \delta^{ij} \delta^{kl} (p_1 + p_2) (p_3 + p_4)
\]
を得る.$\delta^{ik} \delta^{jl}$, $\delta^{il} \delta^{jk}$についても同様に計算すれば
\begin{align*}
  &
  \vcenter{\hbox{
  \begin{tikzpicture}
    \begin{feynman}
      \vertex (o) at (0, 0);
      \vertex (a) at (45: 1.5) {$l$};
      \vertex (b) at (135: 1.5) {$k$};
      \vertex (c) at (225: 1.5) {$i$};
      \vertex (d) at (315: 1.5) {$j$};
      \diagram*{
      (a) -- [fermion, edge label=$p_4$] (o);
      (b) -- [fermion, edge label=$p_3$] (o);
      (c) -- [fermion, edge label=$p_1$] (o);
      (d) -- [fermion, edge label=$p_2$] (o);
      };
    \end{feynman}
  \end{tikzpicture}
  }}\\
  &\quad = - \frac{i}{g^2}
  \left[ (p_1+p_2)(p_3+p_4) \delta^{ij} \delta^{kl}
  + (p_1+p_3)(p_2+p_4) \delta^{ik} \delta^{jl}
  + (p_1+p_4)(p_2+p_3) \delta^{il} \delta^{jk} \right] .
\end{align*}

\subsection{(13.96)}
(7.81)を使えば
\begin{align*}
  \langle \phi_a(0) \phi_b(0) \rangle
  &= \int\displaylimits_{\substack{b\Lambda \leq \lvert p \rvert < \Lambda \\ b\Lambda \leq \lvert q \rvert < \Lambda}}
  \frac{d^2p}{(2\pi)^2} \frac{d^2q}{(2\pi)^2} \langle \phi_a(p) \phi_b(q) \rangle
  %
  = \int\displaylimits_{\substack{b\Lambda \leq \lvert p \rvert < \Lambda \\ b\Lambda \leq \lvert q \rvert < \Lambda}}
  \frac{d^2p}{(2\pi)^2} \frac{d^2q}{(2\pi)^2} (2\pi)^2 \mathop{\delta^{(2)}}(p+q) \frac{g^2}{p^2} \delta_{ab} \\
  %
  &= \int\displaylimits_{b\Lambda \leq \lvert p \rvert < \Lambda} \frac{d^2p}{(2\pi)^2} \frac{g^2}{p^2} \delta_{ab}
  %
  = \int\displaylimits_{b\Lambda \leq \lvert p \rvert < \Lambda} \frac{d\lvert p\rvert}{2\pi} \frac{g^2}{\lvert p\rvert} \delta_{ab}
  %
  = \delta_{ab} \frac{g^2}{2\pi} \log \frac{1}{b} .
\end{align*}

\subsection{(13.109)}
$\beta(T)$を$T \sim T_\ast$で展開して
\[ \beta(T) \approx \left[ \left. \frac{d\beta}{dT} \right|_{T=T_\ast} \right] (T - T_\ast) . \]
(12.73)から
\[
\frac{\partial\bar{T}}{\partial\log p/M} = \beta(\bar{T})
\approx \left[ \left. \frac{d\beta}{dT} \right|_{T=T_\ast} \right] (\bar{T} - T_\ast)
= \left[ \left. \frac{d\beta}{dT} \right|_{T=T_\ast} \right] \bar\rho_T .
\]

\subsection{(13.114)}
(13.113)で部分積分を実行して
\begin{align*}
  & \exp \left[ - \frac{1}{2g_0{}^2} \int d^dx \left\{ (\partial_\mu n)^2 + i \alpha (n^2 - 1) \right\} \right] \\
  &= \exp \left[ - \frac{1}{2g_0{}^2} \int d^dx \left\{ - \vec{n} \cdot (\partial^2 \vec{n}) + i \alpha (n^2 - 1) \right\} \right] \\
  &= \exp \left[ - \frac{1}{2g_0{}^2} \int d^dx \left\{ \vec{n} \cdot (-\partial^2 + i \alpha) \vec{n} - i \alpha \right\} \right] \\
  &= \exp \left[ - \frac{1}{2g_0{}^2} \int d^dx \, \vec{n} \cdot (-\partial^2 + i \alpha) \vec{n} \right]
  \exp \left[ \frac{i}{2g_0{}^2} \int d^dx \, \alpha \right] .
\end{align*}
ここで,$\vec{n}$を適当に変換して
\[
Z = \int \mathcal{D}\alpha \mathcal{D}\vec{n} \exp \left[ - \int d^dx \, \vec{n} \cdot (-\partial^2 + i \alpha) \vec{n} \right]
\exp \left[ \frac{i}{2g_0{}^2} \int d^dx \, \alpha \right]
\]
としてよい.(9.24)から
\begin{align*}
  & \int \mathcal{D}\vec{n} \exp \left[ - \int d^dx \, \vec{n} \cdot (-\partial^2+ i \alpha) \vec{n} \right] \\
  &= \int \mathcal{D}n^1 \cdots \mathcal{D}n^N \exp \left[ - \int d^dx \, n^1 (-\partial^2 + i \alpha) n^1 \right] \times \cdots \times
  \exp \left[ - \int d^dx \, n^N (-\partial^2 + i \alpha) n^N \right] \\
  &= [\det(-\partial^2 + i \alpha)]^{-N/2}
\end{align*}
なので,
\[
Z = \int \mathcal{D}\alpha \, [\det(-\partial^2 + i \alpha)]^{-N/2}
\exp \left[ \frac{i}{2g_0{}^2} \int d^dx \, \alpha \right] .
\]
2つ目の表式は(9.77)を使えば得られる.

\subsection{(13.115)}
$(-\partial^2 + i\alpha)$の固有値は固有ベクトル$\ket{k}$に対し$k^2 + i\alpha$で与えられるので,
\[ \Tr \{ \log (-\partial^2 + i\alpha) \} = \frac{1}{V} \sum_k \log (k^2 + i\alpha) . \]
ここで,(9.22)と同様に離散Fourier変換を行った.(13.114)の指数関数の引数が$\alpha(x)$に関し極小なので,
\begin{align*}
  0 &= \frac{\delta}{\delta\alpha(x)} \left[ - \frac{N}{2} \Tr \{ \log (-\partial^2 + i\alpha) \}
  + \frac{i}{2g_0{}^2} \int d^dy \, \alpha(y) \right] \\
  &= \frac{\delta}{\delta\alpha(x)} \left[ - \frac{N}{2} \frac{1}{V} \sum_k \log (k^2 + i\alpha) \right]
  + \frac{i}{2g_0{}^2} \int d^dy \, \delta(x-y) \\
  &= - \frac{N}{2V} \sum_k \frac{1}{k^2 + i\alpha} \frac{\delta}{\delta\alpha(x)} (k^2 + i\alpha) + \frac{i}{2g_0{}^2} \\
  &= - \frac{iN}{2V} \sum_k \frac{1}{k^2 + i\alpha} + \frac{i}{2g_0{}^2} \\
  &\to - \frac{iN}{2} \int \frac{d^dk}{(2\pi)^d} \frac{1}{k^2 + i\alpha} + \frac{i}{2g_0{}^2} .
\end{align*}

\section*{Problems}\addcontentsline{toc}{section}{Problems}
\subsection{Problem 13.3: The $\mathbb{C}P^N$ model}
\subsubsection{(a)}
$\mathbb{C}P^1$モデルのスカラーを
\[
\boldsymbol{z} = \begin{pmatrix}
z_1 \\ z_2
\end{pmatrix}
\]
とする.
$\vec{n} = \boldsymbol{z}^\dagger \vec\sigma \boldsymbol{z}$を計算すれば
\begin{align*}
  n_1 &= (z_1^*, z_2^*) \sigma^1
  \begin{pmatrix}
    z_1 \\ z_2
  \end{pmatrix}
  = (z_1^*, z_2^*)
  \begin{pmatrix}
    & 1 \\
    1 &
  \end{pmatrix}
  \begin{pmatrix}
    z_1 \\ z_2
  \end{pmatrix}
  = z_1 z_2^* + z_1^* z_2 , \\
  %
  n_2 &= (z_1^*, z_2^*) \sigma^2
  \begin{pmatrix}
    z_1 \\ z_2
  \end{pmatrix}
  = (z_1^*, z_2^*)
  \begin{pmatrix}
    & -i \\
    i &
  \end{pmatrix}
  \begin{pmatrix}
    z_1 \\ z_2
  \end{pmatrix}
  = i (z_1 z_2^* - z_1^* z_2) , \\
  %
  n_3 &= (z_1^*, z_2^*) \sigma^3
  \begin{pmatrix}
    z_1 \\ z_2
  \end{pmatrix}
  = (z_1^*, z_2^*)
  \begin{pmatrix}
    1 & \\
    & -1
  \end{pmatrix}
  \begin{pmatrix}
    z_1 \\ z_2
  \end{pmatrix}
  = \lvert z_1 \rvert^2 - \lvert z_2 \rvert^2 .
\end{align*}
この結果を使えば
\[ n_1{}^2 + n_2{}^2 + n_3{}^2 = (\lvert z_1 \rvert^2 + \lvert z_2 \rvert^2)^2 \]
なので,
規格化条件$n_1{}^2 + n_2{}^2 + n_3{}^2 = 1$は$\lvert z_1 \rvert^2 + \lvert z_2 \rvert^2 = 1$と等しい.
非線形シグマ模型のラグランジアン(13.67)に代入して
\begin{align*}
  & (\partial_\mu n^1) (\partial^\mu n^1) + (\partial_\mu n^2) (\partial^\mu n^2) + (\partial_\mu n^3) (\partial^\mu n^3) \\
  %
  &= [ \partial_\mu(z_1 z_2^*) + \partial_\mu(z_1^* z_2) ] [ \partial^\mu(z_1 z_2^*) + \partial^\mu(z_1^* z_2) ] \\
  & \quad - [ \partial_\mu(z_1 z_2^*) - \partial_\mu(z_1^* z_2) ] [ \partial^\mu(z_1 z_2^*) - \partial^\mu(z_1^* z_2) ] \\
  & \quad + [ \partial_\mu(z_1 z_1^*) - \partial_\mu(z_2 z_2^*) ] [ \partial^\mu(z_1 z_1^*) - \partial^\mu(z_2 z_2^*) ] \\
  %
  &= 4 \partial_\mu(z_1 z_2^*) \partial^\mu(z_1^* z_2)
  + \partial_\mu(z_1 z_1^*) \partial^\mu(z_1 z_1^*)
  + \partial_\mu(z_2 z_2^*) \partial^\mu(z_2 z_2^*)
  - 2\partial_\mu(z_1 z_1^*) \partial^\mu(z_2 z_2^*) \\
  %
  &= 4 \lvert z_1 \rvert^2 (\partial_\mu z_2) (\partial^\mu z_2^*)
  + 4 \lvert z_2 \rvert^2 (\partial_\mu z_1) (\partial^\mu z_1^*)
  + 4 z_1 z_2 (\partial_\mu z_1^*) (\partial^\mu z_2^*)
  + 4 z_1^* z_2^* (\partial_\mu z_1) (\partial^\mu z_2) \\
  %
  & \quad + 2 \lvert z_1 \rvert^2 (\partial_\mu z_1) (\partial^\mu z_1^*)
  + z_1{}^2 (\partial_\mu z_1^*) (\partial^\mu z_1^*)
  + z_1^*{}^2 (\partial_\mu z_1) (\partial^\mu z_1) \\
  %
  & \quad + 2 \lvert z_2 \rvert^2 (\partial_\mu z_2) (\partial^\mu z_2^*)
  + z_2{}^2 (\partial_\mu z_2^*) (\partial^\mu z_2^*)
  + z_2^*{}^2 (\partial_\mu z_2) (\partial^\mu z_2) \\
  %
  & \quad -2 z_1 z_2 (\partial_\mu z_1^*) (\partial^\mu z_2^*)
  -2 z_1^* z_2 (\partial_\mu z_1) (\partial^\mu z_2^*)
  -2 z_1 z_2^* (\partial_\mu z_1^*) (\partial^\mu z_2)
  -2 z_1^* z_2^* (\partial_\mu z_1) (\partial^\mu z_2) \\
  %
  &= 4 \lvert z_1 \rvert^2 (\partial_\mu z_2) (\partial^\mu z_2^*)
  + 4 \lvert z_2 \rvert^2 (\partial_\mu z_1) (\partial^\mu z_1^*)
  + 2 \lvert z_1 \rvert^2 (\partial_\mu z_1) (\partial^\mu z_1^*)
  + 2 \lvert z_2 \rvert^2 (\partial_\mu z_2) (\partial^\mu z_2^*) \\
  %
  & \quad + z_1{}^2 (\partial_\mu z_1^*) (\partial^\mu z_1^*)
  + z_1^*{}^2 (\partial_\mu z_1) (\partial^\mu z_1)
  + z_2{}^2 (\partial_\mu z_2^*) (\partial^\mu z_2^*)
  + z_2^*{}^2 (\partial_\mu z_2) (\partial^\mu z_2) \\
  %
  & \quad
  + 2 z_1 z_2 (\partial_\mu z_1^*) (\partial^\mu z_2^*)
  + 2 z_1^* z_2^* (\partial_\mu z_1) (\partial^\mu z_2)
  -2 z_1^* z_2 (\partial_\mu z_1) (\partial^\mu z_2^*)
  -2 z_1 z_2^* (\partial_\mu z_1^*) (\partial^\mu z_2) \\
  %
  &= 4 (\partial_\mu z_1) (\partial^\mu z_1^*)
  + 4 (\partial_\mu z_2) (\partial^\mu z_2^*)
  - 2 \lvert z_1 \rvert^2 (\partial_\mu z_1) (\partial^\mu z_1^*)
  - 2 \lvert z_2 \rvert^2 (\partial_\mu z_2) (\partial^\mu z_2^*) \\
  %
  & \quad -2 z_1^* z_2 (\partial_\mu z_1) (\partial^\mu z_2^*)
  -2 z_1 z_2^* (\partial_\mu z_1^*) (\partial^\mu z_2) \\
  %
  & \quad + (z_1^* \partial_\mu z_1 + z_2^* \partial_\mu z_2)(z_1^* \partial^\mu z_1 + z_2^* \partial^\mu z_2)
  + (z_1 \partial_\mu z_1^* + z_2 \partial_\mu z_2^*)(z_1 \partial^\mu z_1^* + z_2 \partial^\mu z_2^*) \\
  %
  &= 4 (\partial_\mu z_1) (\partial^\mu z_1^*) + 4 (\partial_\mu z_2) (\partial^\mu z_2^*)
  - 2 (z_1^* \partial_\mu z_1 + z_2^* \partial_\mu z_2)(z_1 \partial^\mu z_1^* + z_2 \partial^\mu z_2^*) \\
  %
  & \quad + (z_1^* \partial_\mu z_1 + z_2^* \partial_\mu z_2)(z_1^* \partial^\mu z_1 + z_2^* \partial^\mu z_2)
  + (z_1 \partial_\mu z_1^* + z_2 \partial_\mu z_2^*)(z_1 \partial^\mu z_1^* + z_2 \partial^\mu z_2^*) .
\end{align*}
$\lvert z_1 \rvert^2 + \lvert z_2 \rvert^2 = 1$を微分して
\[
z_1 \partial_\mu z_1^* + z_1^* \partial_\mu z_1
+ z_2 \partial_\mu z_2^* + z_2^* \partial_\mu z_2 = 0
\]
なので,
\begin{align*}
  &= 4 (\partial_\mu z_1) (\partial^\mu z_1^*) + 4 (\partial_\mu z_2) (\partial^\mu z_2^*)
  - 2 (z_1^* \partial_\mu z_1 + z_2^* \partial_\mu z_2)(z_1 \partial^\mu z_1^* + z_2 \partial^\mu z_2^*) \\
  %
  & \quad + 2 (z_1^* \partial_\mu z_1 + z_2^* \partial_\mu z_2)(z_1^* \partial^\mu z_1 + z_2^* \partial^\mu z_2) \\
  %
  &= 4 (\partial_\mu z_1) (\partial^\mu z_1^*) + 4 (\partial_\mu z_2) (\partial^\mu z_2^*) \\
  & \quad - 2 (z_1^* \partial_\mu z_1 + z_2^* \partial_\mu z_2)
  (z_1 \partial^\mu z_1^* + z_2 \partial^\mu z_2^* - z_1^* \partial^\mu z_1 - z_2^* \partial^\mu z_2) \\
  %
  &= 4 (\partial_\mu z_1) (\partial^\mu z_1^*) + 4 (\partial_\mu z_2) (\partial^\mu z_2^*)
  - 4 (z_1^* \partial_\mu z_1 + z_2^* \partial_\mu z_2)(z_1 \partial^\mu z_1^* + z_2 \partial^\mu z_2^*) .
\end{align*}
以上から
\[
\left\lvert \partial_\mu \vec{n} \right\rvert^2
= 4 \left[ (\partial_\mu\boldsymbol{z}) (\partial^\mu\boldsymbol{z}^*)
+ (\boldsymbol{z}^*\cdot\partial_\mu\boldsymbol{z}) (\boldsymbol{z}\cdot\partial^\mu\boldsymbol{z}^*) \right] .
\]

\renewcommand\theequation{\Roman{part}.\arabic{equation}}
\chapter*{Final Project II: The Coleman-Weinberg Potential}\addcontentsline{toc}{chapter}{Final Project II: The Coleman-Weinberg Potential}
\subsection{(a)}
Coleman-Weinberg模型:
\[
\mathcal{L} = - \frac{1}{4} (F_{\mu\nu})^2 + (D_\mu\phi)^\dagger (D_\mu\phi)
+ \mu^2 \phi^\dagger \phi - \frac{\lambda}{6} (\phi^\dagger \phi)^2 .
\]
ポテンシャル項は
\[ \phi_0 = \mu\sqrt{\frac{3}{\lambda}} \]
で最小値を取る.
\[ \phi(x) = \phi_0 + \frac{1}{\sqrt{2}} [\sigma(x) + i \pi(x)] \]
とする.
\begin{align*}
  D_\mu \phi(x) &= (\partial_\mu + ieA_\mu) \left[ \phi_0 + \frac{1}{\sqrt{2}} [\sigma(x) + i \pi(x)] \right] \\
  &= \frac{1}{\sqrt{2}} \partial_\mu \sigma + \frac{i}{\sqrt{2}} \partial_\mu \pi
  + ieA_\mu \phi_0 + \frac{i}{\sqrt{2}}eA_\mu \sigma(x) - \frac{1}{\sqrt{2}} eA_\mu \pi(x) \\
  &= \left[ \frac{1}{\sqrt{2}} \partial_\mu \sigma - \frac{1}{\sqrt{2}} eA_\mu \pi(x) \right]
  + \frac{i}{\sqrt{2}} \partial_\mu \pi + ieA_\mu \phi_0 + \frac{i}{\sqrt{2}}eA_\mu \sigma(x)
\end{align*}
なので,
\begin{align*}
  (D_\mu\phi)^\dagger (D_\mu\phi)
  &= \left[ \frac{1}{\sqrt{2}} \partial_\mu \sigma - \frac{1}{\sqrt{2}} eA_\mu \pi(x) \right]^2
  + \left[ \frac{1}{\sqrt{2}} \partial_\mu \pi + eA_\mu \phi_0 + \frac{1}{\sqrt{2}}eA_\mu \sigma(x) \right]^2 \\
  %
  &= \frac{1}{2} (\partial_\mu\sigma)^2 + \frac{1}{2} e^2 A_\mu A^\mu \pi^2 - e \pi A^\mu \partial_\mu\sigma \\
  &\quad + \frac{1}{2} (\partial_\mu\pi)^2 + \phi_0{}^2 e^2 A_\mu A^\mu + \frac{1}{2} e^2 A_\mu A^\mu \sigma^2 \\
  &\quad + \sqrt{2} \phi_0 e A^\mu \partial_\mu\pi + \sqrt{2} \phi_0 e^2 A^\mu A_\mu \sigma + e \sigma A^\mu \partial_\mu\pi \\
  %
  &= \frac{1}{2} (\sqrt{2}\phi_0)^2 e^2 A_\mu A^\mu + \frac{1}{2} (\partial_\mu\sigma)^2 + \frac{1}{2} (\partial_\mu\pi)^2
  + \frac{1}{2} e^2 A_\mu A^\mu \pi^2 + \frac{1}{2} e^2 A_\mu A^\mu \sigma^2 \\
  &\quad + e \sigma A^\mu \partial_\mu\pi - e \pi A^\mu \partial_\mu\sigma
  + \sqrt{2} \phi_0 e A^\mu \partial_\mu\pi + \sqrt{2} \phi_0 e^2 A^\mu A_\mu \sigma .
\end{align*}
さらに
\[
\phi^\dagger \phi
= \left( \phi_0 + \frac{\sigma}{\sqrt{2}} \right)^2 + \left( \frac{\pi}{\sqrt{2}} \right)^2
= \phi_0{}^2 + \sqrt{2} \phi_0 \sigma + \frac{\sigma^2}{2} + \frac{\pi^2}{2}
\]
および
\begin{align*}
  (\phi^\dagger \phi)^2 &= \left[ \left( \phi_0 + \frac{\sigma}{\sqrt{2}} \right)^2 + \left( \frac{\pi}{\sqrt{2}} \right)^2 \right]^2 \\
  &= \left( \phi_0 + \frac{\sigma}{\sqrt{2}} \right)^4 + \pi^2 \left( \phi_0 + \frac{\sigma}{\sqrt{2}} \right)^2 + \frac{\pi^4}{4} \\
  &= \phi_0{}^4 + (\sqrt{2}\phi_0)^3 \sigma + 3 \phi_0{}^2 \sigma^2 + \sqrt{2} \phi_0 \sigma^3 + \frac{\sigma^4}{4}
  + \pi^2 \left( \phi_0{}^2 + \sqrt{2} \phi_0 \sigma + \frac{\sigma^2}{2} \right) + \frac{\pi^4}{4} .
\end{align*}
以上から
\begin{align*}
  \mathcal{L}
  &\to - \frac{1}{4} (F_{\mu\nu})^2 + \frac{1}{2} (\sqrt{2}\phi_0)^2 e^2 A_\mu A^\mu \\
  &\quad + \frac{1}{2} (\partial_\mu\sigma)^2 + \frac{1}{2} (\partial_\mu\pi)^2
  + \frac{1}{2} e^2 A_\mu A^\mu \pi^2 + \frac{1}{2} e^2 A_\mu A^\mu \sigma^2 \\
  &\quad + e \sigma A^\mu \partial_\mu\pi - e \pi A^\mu \partial_\mu\sigma
  + \sqrt{2} \phi_0 e A^\mu \partial_\mu\pi + \sqrt{2} \phi_0 e^2 A^\mu A_\mu \sigma \\
  &\quad + \mu^2 \left( \sqrt{2} \phi_0 \sigma + \frac{\sigma^2}{2} + \frac{\pi^2}{2} \right) \\
  &\quad - \frac{\lambda}{6} \left[ (\sqrt{2}\phi_0)^3 \sigma + 3 \phi_0{}^2 \sigma^2 + \sqrt{2} \phi_0 \sigma^3 + \frac{\sigma^4}{4}
  + \pi^2 \left( \phi_0{}^2 + \sqrt{2} \phi_0 \sigma + \frac{\sigma^2}{2} \right) + \frac{\pi^4}{4} \right] \\
  %
  &= - \frac{1}{4} (F_{\mu\nu})^2 + \frac{1}{2} \left( \frac{6\mu^2e^2}{\lambda} \right) A_\mu A^\mu
  + \frac{1}{2} (\partial_\mu\sigma)^2 - \frac{1}{2} (2\mu^2) \sigma^2 + \frac{1}{2} (\partial_\mu\pi)^2  \\
  &\quad - \frac{\lambda}{6} \left[ \frac{\sigma^4}{4} + \mu \sqrt{\frac{6}{\lambda}} \sigma^3
  + \frac{\pi^2\sigma^2}{2} + \mu \sqrt{\frac{6}{\lambda}} \pi^2 \sigma + \frac{\pi^4}{4} \right] \\
  &\quad + \frac{1}{2} e^2 A_\mu A^\mu \pi^2 + \frac{1}{2} e^2 A_\mu A^\mu \sigma^2 \\
  &\quad + e \sigma A^\mu \partial_\mu\pi - e \pi A^\mu \partial_\mu\sigma
  + \mu \sqrt{\frac{6}{\lambda}} e A^\mu \partial_\mu\pi + \mu \sqrt{\frac{6}{\lambda}} e^2 A^\mu A_\mu \sigma .
\end{align*}
