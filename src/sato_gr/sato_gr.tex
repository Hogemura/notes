\documentclass[a4paper]{ltjsreport}
\usepackage[hiragino-pro]{luatexja-preset}
\usepackage{../hogehoge}
\begin{document}
\chapter*{相対性理論(佐藤)}
使用しているのは第17刷.

\setcounter{chapter}{3}
\chapter{Riemann幾何学}
\setcounter{section}{4}
\section{曲率}
\paragraph{(4.100)}
Riemannの曲率テンソルを$4$階共変で表すと,
\begin{align}
  R_{mnij} &= g_{mk}{R^k}_{nij}\notag\\
  &= g_{mk}\left(\partial_i{\Gamma^k}_{nj} - \partial_j{\Gamma^k}_{ni} + {\Gamma^a}_{nj}{\Gamma^k}_{ai} - {\Gamma^a}_{ni}{\Gamma^k}_{aj}\right)\notag\\
  &= \partial_i\left(g_{mk}{\Gamma^k}_{nj}\right) - {\Gamma^k}_{nj}\partial_ig_{mk}  -
  \partial_j\left(g_{mk}{\Gamma^k}_{ni}\right) + {\Gamma^k}_{ni}\partial_jg_{mk} + g_{mk}{\Gamma^a}_{nj}{\Gamma^k}_{ai} - g_{mk}{\Gamma^a}_{ni}{\Gamma^k}_{aj}\notag\\
  &= \partial_i\Gamma_{m, nj} - {\Gamma^a}_{nj}\partial_ig_{ma}  -
  \partial_j\Gamma_{m, ni} + {\Gamma^a}_{ni}\partial_jg_{ma} + g_{mk}{\Gamma^a}_{nj}{\Gamma^k}_{ai} - g_{mk}{\Gamma^a}_{ni}{\Gamma^k}_{aj}\notag\\
  &= \partial_i\Gamma_{m, nj} - \partial_j\Gamma_{m, ni}\label{R1}\\
  &\quad + {\Gamma^a}_{nj}\left(g_{mk}{\Gamma^k}_{ai} - \partial_ig_{ma}\right)\label{R2}\\
  &\qquad - {\Gamma^a}_{ni}\left(g_{mk}{\Gamma^k}_{aj} - \partial_jg_{ma}\right)\label{R3}
\end{align}
となる.ここで,Christoffel記号の定義
\begin{align*}
  {\Gamma^k}_{ij}=\dfrac{1}{2}g^{kl}\left(\partial_ig_{jl} + \partial_jg_{li} - \partial_lg_{ij}\right)
\end{align*}
から,
\begin{align}
  \Gamma_{m, ij} &= g_{mk}{\Gamma^k}_{ij}\notag\\
  &= \dfrac{1}{2}g_{mk}g^{kl}\left(\partial_ig_{jl} + \partial_jg_{li} - \partial_lg_{ij}\right)\notag\\
  &= {\delta^l}_m\left(\partial_ig_{jl} + \partial_jg_{li} - \partial_lg_{ij}\right)\notag\\
  &= \dfrac{1}{2}\left(\partial_ig_{jm} + \partial_jg_{mi} - \partial_mg_{ij}\right)\label{3cov}
\end{align}
となる.よって,
\begin{align}
  \eqref{R1} &= \partial_i\Gamma_{m, nj} - \partial_j\Gamma_{m, ni}\notag\\
  &= \dfrac{1}{2}\partial_i\left(\partial_ng_{jm} + \partial_jg_{mn} - \partial_mg_{nj}\right) - \dfrac{1}{2}\partial_j\left(\partial_ng_{im} + \partial_ig_{mn} - \partial_mg_{ni}\right)\notag\\
  &= \dfrac{1}{2}\left(\partial_n\partial_ig_{mj} + \partial_m\partial_jg_{ni} - \partial_m\partial_ig_{nj} - \partial_n\partial_jg_{mi}\right).\label{R1cov}
\end{align}
次に,\eqref{R2}は,
\begin{align}
  {\Gamma^a}_{nj}\left(g_{mk}{\Gamma^k}_{ai} - \partial_ig_{ma}\right) &= {\Gamma^a}_{nj}\left(\Gamma_{m, ai} - \partial_ig_{ma}\right)\notag\\
  &= {\Gamma^a}_{nj}\left[\dfrac{1}{2}\left(\partial_ag_{im} + \partial_ig_{ma} - \partial_mg_{ai}\right) - \partial_ig_{ma}\right]\notag\\
  &= -  \dfrac{1}{2}{\Gamma^a}_{nj}\left(\partial_mg_{ai} + \partial_ig_{ma} - \partial_ag_{im}\right)\notag\\
  &= -  {\Gamma^a}_{nj}\Gamma_{a, im}\notag\\
  &= -  g^{rs}\Gamma_{r, mi}\Gamma_{s, nj}\label{R2cov}
\end{align}
と書くことができる.\eqref{R3}はこの計算で$i$と$j$を入れ替えて,
\begin{align}
  {\Gamma^a}_{ni}\left(g_{mk}{\Gamma^k}_{aj} - \partial_jg_{ma}\right) = -  g^{rs}\Gamma_{r, mj}\Gamma_{s, ni}\label{R3cov}
\end{align}
となる.\eqref{R1cov}\eqref{R2cov}\eqref{R3cov}から,
\begin{align*}
  R_{mnij} &= \dfrac{1}{2}\left(\partial_n\partial_ig_{mj} + \partial_m\partial_jg_{ni} - \partial_m\partial_ig_{nj} - \partial_n\partial_jg_{mi}\right)\notag\\
  &\qquad + g^{rs}\left(\Gamma_{r, mj}\Gamma_{s, ni} - \Gamma_{r, mi}\Gamma_{s, nj}\right)
\end{align*}
となる.

\chapter{一般相対論}
\setcounter{section}{1}
\section{電磁場の共変形式}
まずは特殊相対論が扱うEuclid空間での電磁場の共変形式を考える.この時,場のラグランジアン密度は,
\begin{align}
  \mathcal{L} &= \mathcal{L}_\text{int} + \mathcal{L}_\text{em}\notag\\
  &=  - \dfrac{1}{4\mu_0}f^{ij}f_{ij} + A_ij^i\label{spc_L}
\end{align}
で与えられる.この時,作用$S$は,
\begin{align}
  S &= \dfrac{1}{c}\int\left( - \dfrac{1}{4\mu_0}f^{ij}f_{ij} + A_ij^i\right)\,d^4x\notag\\
  &= \dfrac{1}{c}\int\left( - \dfrac{1}{4\mu_0}\eta^{ik}\eta^{jl}f_{kl}f_{ij} + A_ij^i\right)\,d^4x\notag\\
  &= \dfrac{1}{c}\int\left[ - \dfrac{1}{4\mu_0}\eta^{ik}\eta^{jl}\left(\partial_kA_l - \partial_lA_k\right)\left(\partial_iA_j - \partial_jA_i\right) + A_ij^i\right]\,d^4x
\end{align}
である.計量は$( - , + , + , + )$としておこう.
$A_i$を力学変数として作用$S$を極小にする変分を考えると,
\begin{align}
  \dfrac{\partial{\mathcal{L}}}{\partial{}A_i} - \partial_j\dfrac{\partial{\mathcal{L}}}{\partial\partial_jA_i}=0
\end{align}
が得られる.これを計算すると,
\begin{align}
  \partial_jf^{ij}=\mu_0j^i \label{spc_Maxwell}
\end{align}
もしくは,Maxwell方程式
\[ \partial^k\partial_kA^i = \partial_j\left(\eta^{jk}\partial_kA^i\right) =  - \mu_0j^i \]
となる.

次にRiemann空間での共変形式に拡張する.まず,電磁テンソルは,
\begin{align}
  f_{ij}=A_{j;\,i} - A_{i;\,j}
\end{align}
になる.ちなみに,Christoffel記号の下添字は可換なのでこれは特殊相対論での表式と変わらない.2階反変テンソルは,
\begin{align*}
  f^{ij} &= g^{ik}g^{jl}f_{kl}\\
  &= g^{ik}g^{jl}\left(A_{l;\,k} - A_{k;\,l}\right)\\
  &= g^{ik}{A^j}_{;\,k} - g^{jl}{A^i}_{;\,l}\\
  &= A^{j;\,i} - A^{i;\,j}
\end{align*}
となる.最後の式は反変微分を使った:
\[\,A^{;\,i}\equiv{g}^{ij}A_{;\,j}.\]
さて,\eqref{spc_Maxwell}を一般相対論に拡張すると,
\begin{align}
  \label{gen_Maxwell}
  \begin{split}
     - \mu_0j^i &=  - {f^{ij}}_{;\,j}\\
    &= {A^{i;\,j}}_{;\,j} - {A^{j;\,i}}_{;\,j}\\
    &= {A^{i;\,j}}_{;\,j} - {A^{j;\,i}}_{;\,j} + {{A^j}_{;\,j}}^{;\,i} - {{A^j}_{;\,j}}^{;\,i}\\
    &= {A^{i;\,j}}_{;\,j} - g^{ik}\left({A^j}_{;k\,;\,j} - {A^j}_{;j\,;\,k}\right) - {{A^j}_{;\,j}}^{;\,i}\\
    &= {A^{i;\,j}}_{;\,j} - g^{ik}{A^j}_{\left[;k\,;\,j\right]} - {{A^j}_{;\,j}}^{;\,i}\\
    &= {A^{i;\,j}}_{;\,j} - {{A^j}_{;\,j}}^{;\,i} + g^{ik}{R^j}_{lkj}A^l\\
    &= {A^{i;\,j}}_{;\,j} - {{A^j}_{;\,j}}^{;\,i} - g^{ik}{R^j}_{ljk}A^l\\
    &= {A^{i;\,j}}_{;\,j} - {{A^j}_{;\,j}}^{;\,i} - g^{ik}{R}_{lk}A^l
  \end{split}
\end{align}
になると予想される.

一般相対論での作用$S$を求めて,この変分から\eqref{gen_Maxwell}を証明しよう.
一般相対論では,体積要素が一般座標変換に対し不変である必要がある.
計量$g_{ij}(x)$のRiemann空間から$\eta_{ij}(x')$のEuclid空間に移ることを考えると,
\[
\eta_{ij}(x')=\partial'_i{x^k}\partial'_j{x^l}g_{kl}(x)
\]
なので,これの行列式を取って,
\[ - 1=\left|\dfrac{\partial(x)}{\partial(x')}\right|^2g\]
となる.ただし,$g=\det{g_{kl}}$である.よって,体積要素について,
\begin{align*}
  dx'^4 &= \left|\dfrac{\partial(x')}{\partial(x)}\right|d^4x\\
  &= \sqrt{ - g}\,d^4x
\end{align*}
となる.このように微小体積を選べば作用は一般座標変換に対し不変となる.従って,作用は
\begin{align}
  S = \dfrac{1}{c}\int{}\mathcal{L}\sqrt{ - g}\,d^4x
\end{align}
となる.ここでも同様に力学変数$A_i$の変分を取って($g$は座標系に固有の値なので変分を取らない),
\begin{align}
  \dfrac{\partial{\sqrt{ - g}\mathcal{L}}}{\partial{}A_i} - \partial_j\dfrac{\partial{\sqrt{ - g}\mathcal{L}}}{\partial\partial_jA_i}=0
\end{align}
を解けばよい.ラグランジアン密度は\eqref{spc_L}と同じである.この式の左辺は,
\begin{align*}
  \sqrt{ - g}\dfrac{\partial{\mathcal{L}}}{\partial{}A_i} - \partial_j\dfrac{\sqrt{ - g}\partial{\mathcal{L}}}{\partial\partial_jA_i}
  &= \sqrt{ - g}\dfrac{\partial{\mathcal{L}}}{\partial{}A_i} - \sqrt{ - g}\partial_j\dfrac{\partial{\mathcal{L}}}{\partial\partial_jA_i}
   - \partial_j\sqrt{ - g}\dfrac{\partial{\mathcal{L}}}{\partial\partial_jA_i}\\
  &= \sqrt{ - g}\mu_0j^i - \sqrt{ - g}\partial_jf^{ij} - \partial_j\sqrt{ - g}\,f^{ij}
\end{align*}
なので,
\begin{align}
   - \mu_0j^i= - \partial_jf^{ij} - \dfrac{1}{\sqrt{ - g}}\partial_j\sqrt{ - g}\,f^{ij}\label{Sgen_Maxwell}
\end{align}
となる.電磁テンソルの発散は,
\begin{align*}
  {f^{ij}}_{;\,j} &= \partial_jf^{ij} + {\Gamma^i}_{rj}f^{rj} + {\Gamma^j}_{rj}f^{ir}\\
  &= \partial_jf^{ij} + {\Gamma^i}_{rj}f^{rj} + \dfrac{1}{\sqrt{ - g}}\partial_j\sqrt{ - g}\,f^{ij}
\end{align*}
となる.ここで,$f^{ij}$は反対称テンソル,${\Gamma^i}_{rj}$は$r$,$j$について対称なので,縮約をあらわに書けば,
\begin{align*}
  \sum_{r,j}{\Gamma^i}_{rj}f^{rj} &= \sum_{r>j}\left({\Gamma^i}_{rj}f^{rj} + {\Gamma^i}_{jr}f^{jr}\right) + \sum_{r}{\Gamma^i}_{rr}f^{rr}\\
  &= \sum_{r>j}{\Gamma^i}_{rj}\left(f^{rj} + f^{jr}\right)\\
  &= \sum_{r>j}{\Gamma^i}_{rj}\left(f^{rj} - f^{rj}\right)\\
  &= 0
\end{align*}
である.よって,${f^{ij}}_{;\,j}$の第2項は$0$なので,
\[\partial_jf^{ij} + \dfrac{1}{\sqrt{ - g}}\partial_j\sqrt{ - g}\,f^{ij}={f^{ij}}_{;\,j}\]
となる.\eqref{Sgen_Maxwell}に代入して,
\[  - \mu_0j^i =  - {f^{ij}}_{;\,j} \]
となり,\eqref{gen_Maxwell}が得られた.

\section{場の運動方程式,エネルギー運動量テンソル}
\paragraph{特殊相対論でのエネルギー運動量テンソル}
ラグランジアン密度が${\mathcal{L}}$のとき,作用は
\begin{align}
  S=\iint{\mathcal{L}}\left(q,\partial_iq\right)\,dx^3d\tau=\frac{1}{c}\int{}\mathcal{L}\,d^4x
\end{align}
で与えられるのであった.ただし,$q$は固有時間$\tau$と座標$x^{1,2,3}$を引数にとる力学変数である.
この時,最小作用の原理から,
\[\delta{S}=0\]
を解けば,
\begin{align}
  \frac{\partial{\mathcal{L}}}{\partial{q}} - \partial_i\frac{\partial{\mathcal{L}}}{\partial\partial_iq}=0\label{EL_eom}
\end{align}
が得られる.これは運動方程式に相当するものである(解析力学で言うところのEuler-Lagrangeの運動方程式).これより,
\begin{align*}
  \partial_iL &= \frac{\partial{\mathcal{L}}}{\partial{q}}\partial_iq + \frac{\partial{\mathcal{L}}}{\partial\partial_k{q}}\partial_i\partial_kq\\
  &= \partial_k\frac{\partial{\mathcal{L}}}{\partial\partial_kq}\,\partial_iq + \frac{\partial{\mathcal{L}}}{\partial\partial_k{q}}\partial_k\partial_iq\\
  &= \partial_k\left(\partial_iq\frac{\partial{\mathcal{L}}}{\partial\partial_kq}\right)
\end{align*}
となる.さらに,$\partial_i={\delta^k}_i\partial_k$によって左辺を書き直すと
\[
\partial_k\left(\partial_iq\frac{\partial{\mathcal{L}}}{\partial\partial_kq} - {\delta^k}_i\mathcal{L}\right)=0
\]
となるので,
\begin{align}
  {T^k}_i={\delta^k}_i\mathcal{L} - \partial_iq\frac{\partial{\mathcal{L}}}{\partial\partial_kq}\label{spe_T_mixed}
\end{align}
を定義すれば,力学場の運動方程式は,
\begin{align}
  \partial_k{T^k}_i=0\label{spe_T}
\end{align}
と表すことができる.
$q$がいくつかあるときは,$\partial_i\mathcal{L}\left(q^{(r)},\partial_kq^{(r)}\right)$の展開で全ての$q^{(t)}$及び$\partial_kq^{(r)}$についての微分の和に書き直せばよいので,
\[{T^k}_i=\partial_iq^{(r)}\frac{\partial{\mathcal{L}}}{\partial\partial_kq^{(r)}} - {\delta^k}_i\mathcal{L}\]
となる.

次に,テンソル${T^k}_i$が表すものについて考えよう.ここで,$3$次元でのGaussの定理の拡張として,
\begin{align}
  \int\partial_k{T^k}_i\,d^4x=\int{}{T^k}_i\,dS_k
\end{align}
を使う.これから,\eqref{spe_T}はあるベクトル
\[p^i=\int{T^{ki}}\,dS_k=\int\eta^{li}{{T^k}_l}\,dS_k\]
が保存することを表している(積分範囲は任意の閉超曲面).
ここで,我々は力学場を扱いたいので,$p^i$が$4$元運動量と同じ表式になるようにしよう.
考えやすいように,積分範囲は$x^0=$(一定)とする.$i=0$では,
\begin{align}
  p^0=\int{T^{k0}}\,dS_k=\int{T^{00}}\,dV
\end{align}
となる.2つ目の式変形については,3次元空間で$z=$(一定)で面積分すればベクトルの$z$成分と面素$dxdy$のみを考えればよいことからの類推で分かるだろう.
ところで,\eqref{spe_T_mixed}から,
\[T^{00}={T^0}_l\eta^{l0}=\dot{q}\frac{\partial{\mathcal{L}}}{\partial\dot{q}} - \mathcal{L}\]
となる.これはエネルギー(ハミルトニアン)密度である.
$p^0=E/c$なので,改めて
\begin{align}
  p^i=\frac{1}{c}\int{T^{ki}}\,dS_k\label{spe_p}
\end{align}
とすれば$p^i$は4元運動量となる.

\eqref{spe_T_mixed}で定義された${T^k}_i$もしくは$T^{ik}$は対称でない.
ところで,\eqref{spe_T}を満たすものは\eqref{spe_T_mixed}以外にも無数にある.
$\partial_kT^{ki}=0$をある$T^{ki}$がみたすのであれば,
\begin{align}
  T^{ki} + \partial_l\varphi^{lki},\qquad\varphi^{lki}= - \varphi^{kli}\label{T_transform}
\end{align}
もこの方程式を満たす.実際に代入すれば,
\begin{align*}
  \partial_k\left(T^{ki} + \partial_l\varphi^{lki}\right) &= \partial_k\partial_l\varphi^{lki}\\
  &=  - \partial_l\partial_k\varphi^{kli}
\end{align*}
となる.添字$k$,$l$を入れ替えても同じ値になる:
\[\partial_l\left(T^{li} + \partial_k\varphi^{kli}\right)=\partial_l\partial_k\varphi^{kli}.\]
この2式を比べて,$\partial_k\partial_l\varphi^{lki}=0$である.よって,証明された.
$\varphi^{lki}$を加えて$T^{ki}$を対称にすることを考えるが,これによって運動量の表式が変わってはいけない.
\eqref{spe_p}について,
\begin{align*}
  \int\partial_l\varphi^{lki}\,dS_k &= \frac{1}{2}\int\left(\partial_l\varphi^{lki}\,dS_k + \partial_l\varphi^{lki}\,dS_k\right)\\
  &= \frac{1}{2}\int\left(\partial_l\varphi^{lki}\,dS_k - \partial_k\varphi^{lki}\,dS_l\right)\\
  &= \frac{1}{2}\int\varphi^{lki}df_{kl}
\end{align*}
ただし,$df_{kl}$は超曲面を囲む通常の2次元曲面の積分要素である.
この積分を無限遠で行えば,無限遠では粒子もポテンシャルも存在しないため,この積分は0である.
よって,\eqref{T_transform}を満たせば運動量に関する要請は自動的に満たされる.
よってこれ以外に$T^{ki}$への要請が必要である.

ここで,要請としてテンソル$T^{ki}$から角運動量テンソルの密度\footnote{場古典とか}を求められることを採用しよう.具体的には,\eqref{spe_p}と比べて,
\begin{align}
  M^{ki}=\sum\left(x^kp^i - x^ip^k\right)=\frac{1}{c}\int\left(x^kT^{ri} - x^iT^{rk}\right)\,dS_r
\end{align}
とするのがよいだろう.さて,\eqref{spe_T}以降の運動量に関する議論と同様にして,角運動量$M^{ki}$が保存される条件は,
\begin{align}
  \partial_r\left(x^kT^{ri} - x^iT^{rk}\right)=0
\end{align}
である.$\partial_ix^k={\delta^k}_i$,\eqref{spe_T}から,この式は
\[{\delta^k}_rT^{ri} - {\delta^i}_rT^{rk}=0\]
となり,結局
\begin{align}
  T^{ik}=T^{ki}
\end{align}
となることが分かる.結局,我々は$T^{ki}$が対称テンソルになるように$\varphi^{lki}$を選べばよいのだ.

ここで$T^{ki}$から求まる量について再び記しておく:
\begin{align}
  p^i=\frac{1}{c}\int{T^{ki}}\,dS_k,\quad{}M^{ki}=\frac{1}{c}\int\left(x^kT^{ri} - x^iT^{rk}\right)\,dS_r.
\end{align}
これらの量が保存することから$T^{ki}$に課せられる条件は,
\begin{align}
  \partial_k{T^{ki}}=0,\quad{}T^{ik}=T^{ki}.
\end{align}
$p$について,$k=0$の超曲面で積分すれば,$i=0$及び$i=1,2,3$について,
\[\frac{E}{c}=\frac{1}{c}\int{T^{00}}\,dV,\quad{}p^1=\frac{1}{c}\int{T^{01}}\,dV,\ldots\]
となるので,
$T^{00}=W$はエネルギー密度,$T^{01}/c=T^{10}/c$,$T^{02}/c=T^{20}/c$,$T^{02}/c=T^{20}/c$は運動量密度である.
ここで,流量について考えるために,$T^{ki}$の発散の式で時間成分と空間成分を分けると,
\[\frac{1}{c}\frac{\partial{T^{00}}}{\partial{t}} + \frac{\partial{T^{0a}}}{\partial{x^a}}=0,\quad\frac{1}{c}\frac{\partial{T^{b0}}}{\partial{t}} + \frac{\partial{T^{ba}}}{\partial{x^a}}=0\]
が得られる.第1の式を積分すると,
\[\frac{1}{c}\frac{\partial}{\partial{t}}\int_{\Omega}{T^{00}}\,dV + \int_{\Omega}\frac{\partial{T^{0a}}}{\partial{x^a}}\,dV=0\]
左辺第2項にGaussの定理を適用して,
\[\frac{\partial}{\partial{t}}\int_{\Omega}{T^{00}}\,dV + \int_{\partial\Omega}cT^{0a}\,df_a=0\]
となる.ただし,$\partial\Omega$は積分領域$\Omega$の表面であり,$df_\alpha$は$\partial\Omega$の面素の各軸成分である.
この式はエネルギーが単位時間当たり単位面積当たり$a$の向きに$cT^{0a}=S^a$だけ流出していることを表す.
先程の結果と合わせれば,運動量とはエネルギーの流れであることが分かる.第2の式についても同様にすると,
\[\frac{\partial}{\partial{t}}\int_{\Omega}\frac{T^{b0}}{c}{}\,dV + \int_{\partial\Omega}T^{ba}\,df_a=0\]
となる.$T^{b0}/c=T^{0b}/c=S^b/c^2$で,これを体積分すれば運動量になるので,この式は,
\[\int_{\partial\Omega}T^{ba}\,df_a= - \frac{\partial{}p^b}{\partial{t}}= - F^{b}\]
となる.これはいわゆる応力テンソル$\sigma^{ba}$であり,$x^a$に垂直な面から$x^b$の向きに流れ出る運動量を表す.よって,$T^{ij}$は,
\begin{align}
  T^{ij}=
  \begin{pmatrix}
    W & \frac{S^x}{c} & \frac{S^y}{c} & \frac{S^z}{c}\\
    \frac{S^x}{c} &  - \sigma^{xx} &  - \sigma^{xy} &  - \sigma^{xz}\\
    \frac{S^y}{c} &  - \sigma^{yx} &  - \sigma^{yy} &  - \sigma^{yz}\\
    \frac{S^z}{c} &  - \sigma^{zx} &  - \sigma^{zy} &  - \sigma^{zz}
  \end{pmatrix}
\end{align}
となる.これをエネルギー運動量テンソルと呼ぶ.

\paragraph{一般相対論でのエネルギー運動量テンソル}
先程とは別の考え方によって一般相対論でのエネルギー運動量テンソルを求めることにする.
電磁場で調べたように,作用は
\begin{align}
  S = \frac{1}{c}\int{}\mathcal{L}\sqrt{ - g}\,d^4x
\end{align}
となる.ここでも同様に変分を取るが,物質場の変数$A_i$だけでなく,計量$g_{ij}$についても変分をとる必要がある.つまり,
\[\delta{S}_{\text{tot}}(A_i,g_{ij})=0\]
を解けば,$A_i$,$g_{ij}$に関する条件が分かる\footnote{この先の大まかな流れは次のような感じ:
先ほどの全体の作用変分は$A$を固定した時の作用変分と$g$を固定した時の作用変分の和に等しい.
更に,運動方程式から$g$を固定させた時の作用変分は0,最小作用の原理から全体の作用変分は0である.
よって,$g$を変数とした時の作用変分も0となる.これによって,時空の変化に対する方程式も立てることができる.
この時,この方程式の中にある2階のテンソルをエネルギー運動量テンソルと定義する.(詳しくは本文(5.54)など)}.

結局,
\begin{align}
  T_{ij}=\frac{2}{\sqrt{ - g}}\left[\partial_k\left(\frac{\partial{}(\mathcal{L}\sqrt{ - g})}{\partial(\partial_k{g^{ij}})}\right) - \frac{\partial(\mathcal{L}\sqrt{ - g})}{\partial{g^{ij}}}\right]
  \label{em_ep_tensor}
\end{align}
となる.ところで,テンソル$g^{ij}$による$G$(計量と適当な対称テンソル$X_{ij}$の積)の偏微分は,
\begin{align}
  dG=\frac{\partial{G}}{\partial{g^{ij}}}dg^{ij}
\end{align}
を意味する.よって,計量の対称性から,実際に$i\neq{j}$のような成分を求める際場合は表式通りの値を2倍する必要がある.
何故ならば,$G=g^{ij}X_{ij}$のうち,例えば,$g^{12}$に対応する部分は
\[g^{12}X_{12} + g^{21}X_{21}=2g^{12}X_{12}\]
なので,
\[dG=2X_{12}dg^{12}\]
となる.これは単純な計算結果
\[\frac{\partial{}G}{\partial{g^{ij}}}=X_{ij}\]
の2倍である.

一般相対論での電磁場のエネルギー運動量テンソルを求めよう.電磁場のラグランジアン密度は
\[\mathcal{L}= - \frac{1}{4\mu_0}f^{ij}f_{ij}\]
である.これを\eqref{em_ep_tensor}に代入して,
\begin{align*}
  T_{ij} &=  - \frac{2}{\sqrt{ - g}}\left[\frac{\partial\mathcal{L}}{\partial{g^{ij}}}\sqrt{ - g} + \frac{\partial{\sqrt{ - g}}}{\partial{g^{ij}}}\mathcal{L}\right]\\
  &=  - \frac{2}{\sqrt{ - g}}\left[ - \frac{\sqrt{ - g}}{4\mu_0}\frac{\partial\left(g^{ak}g^{bl}f_{ab}f_{kl}\right)}{\partial{g^{ij}}} + \frac{\partial{\sqrt{ - g}}}{\partial{g^{ij}}}\mathcal{L}\right]\\
  &=  - \frac{2}{\sqrt{ - g}}\left[ - \frac{\sqrt{ - g}}{4\mu_0}\left(g^{bl}f_{ib}f_{jl} + g^{ak}f_{ai}f_{kj}\right) + \frac{\partial{\sqrt{ - g}}}{\partial{g^{ij}}}\mathcal{L}\right]\\
  &= \frac{2}{\sqrt{ - g}}\left[\frac{\sqrt{ - g}}{2\mu_0}g^{bl}f_{ib}f_{jl} - \frac{\partial{\sqrt{ - g}}}{\partial{g^{ij}}}\mathcal{L}\right]\\
  &= \frac{1}{\mu_0}g^{kl}f_{ik}f_{jl} - \frac{2}{\sqrt{ - g}}\frac{\partial{\sqrt{ - g}}}{\partial{g^{ij}}}\mathcal{L}\\
  &= \frac{1}{\mu_0}g^{kl}f_{ik}f_{jl} - \frac{2}{\sqrt{ - g}}\frac{\partial\sqrt{ - g}}{\partial{g}}\frac{\partial{g}}{\partial{g^{ij}}}\mathcal{L}\\
  &= \frac{1}{\mu_0}g^{kl}f_{ik}f_{jl} + \frac{1}{g}\frac{\partial{g}}{\partial{g^{ij}}}\mathcal{L}
\end{align*}
となる.
% ここで,$g=\det{g_{ij}}$であるので,
% \[dg=\frac{\partial{g}}{\partial{g^{ij}}}dg^{ij}\]
% である.
$g_{ij}$の余因子を$\Delta_{ij}$などと表すと,
\[g=\sum_jg^{ij}\Delta_{ji}\]
のように表すことができる.$g^{ij}$で偏微分し,
\begin{align}
  \frac{\partial{g}}{\partial{g^{ij}}}=\Delta_{ji}\label{par1}
\end{align}
が得られる.
ところで,行列式について$i$についても和を取るようにすれば,
\[g=\frac{1}{4}g^{ij}\Delta_{ij}\]
となる.更に,
\[{\delta^i}_ig=g^{ij}\Delta_{ij}\]
と変形し,
\[g^{ij}g_{ij}g=g^{ij}\Delta_{ij}\]
となる.よって,
\[gg_{ij}=\Delta_{ij}\]
となる.これを\eqref{par1}に代入すれば,
\[\frac{\partial{g}}{\partial{g^{ij}}}=gg^{ij}\]
となる.よって,電磁場のエネルギー運動量テンソルは,
\begin{align*}
  T_{ij} &= \frac{1}{\mu_0}g^{kl}f_{ik}f_{jl} - \frac{1}{4\mu_0}\frac{1}{g}gg^{ij}f^{kl}f_{kl}\\
  &= \frac{1}{\mu_0}\left(g^{kl}f_{ik}f_{jl} - \frac{1}{4}g^{ij}f^{kl}f_{kl}\right)
\end{align*}
と求まる.

\chapter{球対称な重力場の真空解と粒子の運動}
\section*{演習問題6.1}
\paragraph{Schwarzschild解}
原点にのみ質点が存在し,周りは球対称で静的な真空空間の計量を求めよう.計量$g_{ij}$を,
\[ds^2= - e^{\nu(r)}(cdt)^2 + e^{\lambda(r)}dr^2 + r^2(d\theta^2 + \sin^2\theta{}d\phi^2)\]
とする.このとき,Christoffel記号は,
\[{\Gamma^k}_{ij}=\frac{1}{2}g^{kl}\left(\partial_ig_{jl} + \partial_jg_{li} - \partial_lg_{ij}\right)\]
で与えられる.Christoffel記号のうち,0でない条件について考えよう.まず,直ちに,()の$g^{kl}$から,$k=l$であることが分かろう.
さらに,()の中が0でないためには,$j=k(=l), i=k(=l), i=j$のいずれかが成立する必要がある.
最後の場合については,$\partial_l=\partial_k$が含まれるので,$i=j=0, 1, 2$の場合は$k=1$,$i=j=3$の場合は$k=1, 2$が可能である.よって,
\begin{align}
  \begin{split}
    {\Gamma^0}_{10}={\Gamma^0}_{01}=\frac{\nu'}{2}, \quad{}{\Gamma^1}_{00}=\frac{\nu'}{2}e^{\nu - \lambda}\\
    {\Gamma^1}_{11}=\frac{\lambda'}{2}, \quad{}{\Gamma^1}_{22}= - re^{ - \lambda}, \quad{}{\Gamma^1}_{33}= - r\sin^2\theta{}e^{ - \lambda}\\
    {\Gamma^2}_{12}={\Gamma^2}_{21}=\frac{1}{r}, \quad{\Gamma^2}_{33}= - \sin\theta\cos\theta\\
    {\Gamma^3}_{13}={\Gamma^3}_{31}=\frac{1}{r}, \quad{\Gamma^3}_{23}={\Gamma^3}_{32}=\cot\theta
  \end{split}
\end{align}
である.$'$は$r$による微分を表す.

先程の計量をEinstein方程式
\[ {R^i}_j - \frac{1}{2}{\delta^i}_j=\frac{8\pi{}G}{c^4}{T^i}_j \]
に代入して,各関数$\nu, \lambda$を求めよう.このために,Ricciテンソル
\[R_{mj}={R^i}_{mij}=\partial_i{\Gamma^i}_{mj} - \partial_j{\Gamma^i}_{mi} + {\Gamma^a}_{mj}{\Gamma^i}_{ai} - {\Gamma^a}_{mi}{\Gamma^i}_{aj}\]
及びRicciスカラー
\[R={R^i}_{i}\]
を求める
(この時点で上の方程式の縮約を取って$R$を$T$に変えると手間が少なくなるが,折角なので$R$も明らかな形にしておこう).
\begin{align*}
  {R^0}_0 &= g^{0i}R_{i0}\\
  &= g^{00}R_{00}\\
  &= -  e^{ - \nu}\left[\partial_i{\Gamma^i}_{00} - \partial_0{\Gamma^i}_{0i} + {\Gamma^a}_{00}{\Gamma^i}_{ai} - {\Gamma^a}_{0i}{\Gamma^i}_{a0}\right]\\
  &= -  e^{ - \nu}\left[\partial_1{\Gamma^1}_{00} + {\Gamma^1}_{00}\left({\Gamma^0}_{10} + {\Gamma^1}_{11} + {\Gamma^2}_{12} + {\Gamma^3}_{13}\right) - \left({\Gamma^0}_{01}{\Gamma^1}_{00} + {\Gamma^1}_{00}{\Gamma^0}_{10}\right)\right]\\
  &= -  \frac{1}{2}\nu''e^{ - \lambda} - \frac{1}{4}\nu'(\nu' - \lambda)e^{ - \lambda} - \frac{\nu'}{r}e^{ - \lambda}
\end{align*}
\begin{align*}
  {R^1}_1 &= g^{1i}R_{i1}\\
  &= g^{11}R_{11}\\
  &= e^{ - \lambda}\left[\partial_i{\Gamma^i}_{11} - \partial_1{\Gamma^i}_{1i} + {\Gamma^a}_{11}{\Gamma^i}_{ai} - {\Gamma^a}_{1i}{\Gamma^i}_{a1}\right]\\
  &= e^{ - \lambda}\left[\partial_1{\Gamma^1}_{11} - \partial_1{\Gamma^i}_{1i} + {\Gamma^1}_{11}{\Gamma^i}_{1i} - \left({\Gamma^0}_{10}{\Gamma^0}_{01} + {\Gamma^1}_{11}{\Gamma^1}_{11} + {\Gamma^2}_{12}{\Gamma^2}_{21} + {\Gamma^3}_{13}{\Gamma^3}_{31}\right)\right]\\
  &= -  \frac{1}{2}\nu''e^{ - \lambda} - \frac{1}{4}\nu'(\nu' - \lambda)e^{ - \lambda} + \frac{\lambda'}{r}e^{ - \lambda}
\end{align*}
\begin{align*}
  {R^2}_2 &= g^{2i}R_{i2}\\
  &= g^{22}R_{22}\\
  &= \frac{1}{r^2}\left[\partial_i{\Gamma^i}_{22} - \partial_2{\Gamma^i}_{2i} + {\Gamma^a}_{22}{\Gamma^i}_{ai} - {\Gamma^a}_{2i}{\Gamma^i}_{a2}\right]\\
  &= \frac{1}{r^2}\left[\partial_1{\Gamma^1}_{22} - \partial_2{\Gamma^3}_{23} + {\Gamma^1}_{22}{\Gamma^i}_{1i} - \left({\Gamma^1}_{22}{\Gamma^2}_{12} + {\Gamma^2}_{21}{\Gamma^1}_{22} - {\Gamma^3}_{23}{\Gamma^3}_{23}\right)\right]\\
  &= \frac{1}{r^2}\left[1 - e^{ - \lambda} - re^{ - \lambda}\left(\frac{\nu'}{2} - \frac{\lambda'}{2}\right)\right]
\end{align*}
\begin{align*}
  {R^3}_3 &= g^{3i}R_{i3}\\
  &= g^{33}R_{33}\\
  &= \frac{1}{r^2\sin^2\theta}\left[\partial_i{\Gamma^i}_{33} - \partial_3{\Gamma^i}_{3i} + {\Gamma^a}_{33}{\Gamma^i}_{ai} - {\Gamma^a}_{3i}{\Gamma^i}_{a3}\right]\\
  &= \frac{1}{r^2\sin^2\theta}\left[\left(\partial_1{\Gamma^1}_{33} + \partial_2{\Gamma^2}_{33}\right) + \left({\Gamma^1}_{33}{\Gamma^i}_{1i} + {\Gamma^2}_{33}{\Gamma^i}_{2i}\right) \right.\\
  & \qquad \qquad \qquad \left. - \left({\Gamma^1}_{33}{\Gamma^3}_{13} + {\Gamma^2}_{33}{\Gamma^3}_{23} + {\Gamma^3}_{13}{\Gamma^1}_{33} + {\Gamma^3}_{23}{\Gamma^2}_{33}\right)\right]\\
  &= \frac{1}{r^2}\left[1 - e^{ - \lambda} - re^{ - \lambda}\left(\frac{\nu'}{2} - \frac{\lambda'}{2}\right)\right]
\end{align*}

さて,空間は真空なので,エネルギー運動量テンソルが0であるので,重力場の方程式は,
\[{R^i}_j - \frac{1}{2}R{\delta^i}_j=0\]
となる.$(0, 0)$成分は,
\[0=\frac{1}{2}\left({R^0}_0 - {R^1}_1 - {R^2}_2 - {R^3}_3\right)=e^{ - \lambda}\left(\frac{1}{r^2} - \frac{\lambda'}{r}\right) - \frac{1}{r^2}\]
となる.$(1, 1)$成分は,
\[0=\frac{1}{2}\left( - {R^0}_0 + {R^1}_1 - {R^2}_2 - {R^3}_3\right)=e^{ - \lambda}\left(\frac{1}{r^2} + \frac{\nu'}{r}\right) - \frac{1}{r^2}\]
となる.$(2, 2)$及び$(3, 3)$成分は,
\[0=\frac{1}{2}\left( - {R^0}_0 - {R^1}_1 + {R^2}_2 - {R^3}_3\right)=e^{ - \lambda}\left(\nu'' + \frac{\nu'^2}{2} + \frac{\nu' - \lambda'}{r} - \frac{\nu'\lambda'}{2}\right)\]
となる.上の式を,$r\to\infty$で
\[g_{00}= - 1 - \frac{2}{c^2}\phi= - 1 + \frac{2GM}{rc^2}\]
に一致するという静的な弱い重力場に対する境界条件の下で解くと,
\[ds^2= - \left(1 - \frac{2GM}{rc^2}\right)(cdt)^2 + \frac{1}{1 - \frac{2GM}{rc^2}}dr^2 + r^2(d\theta^2 + \sin^2\theta{}d\phi^2)\]
が得られる.この厳密解を,Schwarzschild解といい,計量をSchwarzschild計量と呼ぶ.

\paragraph{Birkhoffの定理}
先程は空間が静的という条件を課したが,実はこれを課さなくても,同じ答えが得られる.これをBirkhoffの定理と言う.
空間が静的でなくなったので,$\partial_0$の項が無視できなくなっている.その結果増えるChristoffel記号は,
\[{\Gamma^0}_{00}=\frac{\dot{\nu}}{2c}, \quad{\Gamma^0}_{11}=\frac{\dot{\lambda}}{2c}e^{\lambda - \nu}, \quad{\Gamma^1}_{01}={\Gamma^1}_{10}=\frac{\dot{\lambda}}{2c}\]
である.$\dot{}$は時間微分を表す.この結果,Ricciテンソルには,次の項が加わる.
\begin{align*}
  {R^0}_0 &\colon -  e^{ - \nu}\left[\partial_0{\Gamma^0}_{00} - (\partial_0{\Gamma^0}_{00} + \partial_0{\Gamma^1}_{01})\right]= - e^{ - \nu}\left[ - \frac{\ddot{\lambda}}{2c^2} + \frac{\dot{\lambda}\dot{\nu}}{4c^2} - \frac{\dot{\lambda}^2}{4c^2}\right]\\
  {R^1}_1 &\colon -  e^{ - \nu}\left[\partial_0{\Gamma^0}_{11} + {\Gamma^0}_{11}{\Gamma^i}_{0i} - \left({\Gamma^0}_{11}{\Gamma^1}_{01} + {\Gamma^1}_{10}{\Gamma^0}_{11}\right)\right]= - e^{ - \nu}\left[ - \frac{\ddot{\lambda}}{2c^2} + \frac{\dot{\lambda}\dot{\nu}}{4c^2} - \frac{\dot{\lambda}^2}{4c^2}\right]\\
\end{align*}
${R^2}_2, {R^3}_3$は変わらない.よって,先程の式は,
\[0=e^{ - \lambda}\left(\frac{1}{r^2} - \frac{\lambda'}{r}\right) - \frac{1}{r^2}\]
\[0=e^{ - \lambda}\left(\frac{1}{r^2} + \frac{\nu'}{r}\right) - \frac{1}{r^2}\]
\[0=e^{ - \lambda}\left(\nu'' + \frac{\nu'^2}{2} + \frac{\nu' - \lambda'}{r} - \frac{\nu'\lambda'}{2}\right) + e^{ - \nu}\left[ - \frac{\ddot{\lambda}}{c^2} + \frac{\dot{\lambda}\dot{\nu}}{2c^2} - \frac{\dot{\lambda}^2}{2c^2}\right]\]
となる.さらに,${R^1}_0$について,
\begin{align*}
  0 &= {R^1}_0\\
  &= g^{1i}R_{i0}\\
  &= g^{11}R_{10}\\
  &= e^{ - \lambda}\left[\partial_i{\Gamma^i}_{10} - \partial_0{\Gamma^i}_{1i} + {\Gamma^a}_{10}{\Gamma^i}_{ai} - {\Gamma^a}_{1i}{\Gamma^i}_{a0}\right]\\
  &= e^{ - \lambda}\left[\left(\partial_0{\Gamma^0}_{10} + \partial_1{\Gamma^1}_{10}\right) - \partial_0\left({\Gamma^0}_{10} + {\Gamma^1}_{11}\right) + \left({\Gamma^0}_{10}{\Gamma^i}_{0i} + {\Gamma^1}_{10}{\Gamma^i}_{1i}\right)\right.\\
  &\qquad\left. - \left({\Gamma^0}_{10}{\Gamma^0}_{00} + {\Gamma^0}_{11}{\Gamma^1}_{00} + {\Gamma^1}_{10}{\Gamma^0}_{10} - {\Gamma^1}_{11}{\Gamma^1}_{10}\right)\right]\\
  &= e^{ - \lambda}\frac{\dot{\lambda}}{r}
\end{align*}
なので,
\[\dot{\lambda}=0.\]
さらに,新しい$(1, 1)$成分の式を時間で微分すれば,
\[\dot{\nu}=0.\]
以上から,静的空間でなくても,球対称な真空に対する厳密解はSchwarzschild解のみであると分かる.

\chapter{超高密度天体とブラックホール}
\setcounter{section}{2}
\section{Schwarzschildブラックホール}
\paragraph{Kruskal座標系とPenrose図}
Schwarzschild計量はSchwarzschild半径
\[r=r_{\text{g}}=\frac{2GM}{c^2}\]
で発散する.さらに,その内外で時間の項に対する計量の符号が変化し,内外を統一的に扱うことが難しい.よって,
\begin{align*}
  \left(\frac{r}{r_\text{g}} - 1\right)\exp\left(\frac{r}{r_\text{g}}\right)=u^2 - v^2,\quad
  \tanh\left(\frac{ct}{2r_\text{g}}\right)=\left\{
  \begin{array}{l}
    \dfrac{v}{u}\quad(r\geq{}r_\text{g})\\
    \dfrac{u}{v}\quad(r<{}r_\text{g})
  \end{array}
  \right.
\end{align*}
と座標変換を行う.

$r\geq{}r_\text{g}$の場合のSchwarzschild計量の表式を調べよう.この時,$u,v$は次のように表すことができる:
\begin{align*}
  u^2 &= \cosh^2\left(\frac{ct}{2r_\text{g}}\right)\left(\frac{r}{r_\text{g}} - 1\right)\exp\left(\frac{r}{r_\text{g}}\right)\\
  v^2 &= \sinh^2\left(\frac{ct}{2r_\text{g}}\right)\left(\frac{r}{r_\text{g}} - 1\right)\exp\left(\frac{r}{r_\text{g}}\right).
\end{align*}
両辺微分すれば,
\begin{align*}
  2u\frac{\partial{u}}{\partial{r}} &= \frac{r}{{r_\text{g}}^2}\exp\left(\frac{r}{r_\text{g}}\right)\cosh^2\left(\frac{ct}{2r_\text{g}}\right)\\
  2u\frac{\partial{u}}{\partial{t}} &= \frac{c}{r_\text{g}}\left(\frac{r}{r_\text{g}} - 1\right)\exp\left(\frac{r}{r_\text{g}}\right)\cosh\left(\frac{ct}{2r_\text{g}}\right)\sinh\left(\frac{ct}{2r_\text{g}}\right)\\
  2v\frac{\partial{v}}{\partial{r}} &= \frac{r}{{r_\text{g}}^2}\exp\left(\frac{r}{r_\text{g}}\right)\sinh^2\left(\frac{ct}{2r_\text{g}}\right)\\
  2v\frac{\partial{v}}{\partial{t}} &= \frac{c}{r_\text{g}}\left(\frac{r}{r_\text{g}} - 1\right)\exp\left(\frac{r}{r_\text{g}}\right)\cosh\left(\frac{ct}{2r_\text{g}}\right)\sinh\left(\frac{ct}{2r_\text{g}}\right)\\
\end{align*}
となる.よって,Schwarzschild計量の立体角$r^2(d\theta^2 + \sin^2\theta{d\phi^2})$を含まない部分を次のように表す:
\begin{align*}
  Adu^2 - Bdv^2 &= A\left(\frac{\partial{u}}{\partial{t}}dt + \frac{\partial{u}}{\partial{r}}dr\right)^2 - B\left(\frac{\partial{v}}{\partial{t}}dt + \frac{\partial{v}}{\partial{r}}dr\right)^2\\
  &= \frac{r - r_\text{g}}{4{r_\text{g}}^3}\exp\left(\frac{r}{r_\text{g}}\right)\left[A\sinh^2\left(\frac{ct}{2r_\text{g}}\right) - B\cosh^2\left(\frac{ct}{2r_\text{g}}\right)\right](cdt)^2\\
  &\qquad + \frac{r^2}{4{r_\text{g}}^3(r - r_\text{g})}\exp\left(\frac{r}{r_\text{g}}\right)\left[A\cosh^2\left(\frac{ct}{2r_\text{g}}\right) - B\sinh^2\left(\frac{ct}{2r_\text{g}}\right)\right]dr^2\\
  &\qquad + \frac{rc}{4{r_\text{g}}^3}(A - B)\exp\left(\frac{r}{r_\text{g}}\right)\cosh\left(\frac{ct}{2r_\text{g}}\right)\sinh\left(\frac{ct}{2r_\text{g}}\right)drdt.
\end{align*}
これをSchwarzschild計量の該当部分
\[ - \left(1 - \frac{r_\text{g}}{r}\right)(cdt)^2 + \left(1 - \frac{r_\text{g}}{r}\right)^{ - 1}dr^2\]
と比較すると,
\[A=B=f^2=\frac{4{r_\text{g}}^3}{r}\exp\left( - \frac{r}{r_\text{g}}\right)>0\]
である.よって,$(u,v)$座標系でのSchwarzschild計量は,
\[ds^2=f^2( - dv^2 + du^2) + r^2(d\theta^2 + \sin^2\theta{d\phi^2})\]
で表される.$r<{r_\text{g}}$の場合も同様にすれば,同じ形になることがわかる.この座標系をKruskal座標系と呼ぶ(本文図7-5).

極限について記すと,
\begin{align*}
  \text{I}^ + \,(r\colon\text{finite},t\to + \infty) &\colon u\to\pm\infty,u\sim{}v\\
  \text{I}^ - \,(r\colon\text{finite},t\to - \infty) &\colon u\to\pm\infty,u\sim{} - v\\
  \text{I}^0\,(t\colon\text{finite},r\to + \infty) &\colon |v|<{}|u|\to\infty\\
  \mathscr{I}^ + (ct - r\colon\text{finite},ct + r\to\infty) &\colon u\to\pm\infty,|v|<{}|u|,u/v\sim1\\
  \mathscr{I}^ - (ct + r\colon\text{finite},ct - r\to\infty) &\colon u\to\pm\infty,|v|<{}|u|,u/v\sim - 1
\end{align*}

Kruskal座標系をさらに,
\[u + v=\tan\left(\frac{\phi + \xi}{2}\right),\quad{}u - v= - \tan\left(\frac{\phi - \xi}{2}\right)\]
によって変換する.各領域の式は,
\begin{align*}
  \text{I} &\colon u>0, - v<{}u<{}v\\
  \text{I}\hspace{ - .1em}\text{I} &\colon u<{}v<\sqrt{1 + u^2}\quad(u>0),\qquad - u<{}v<\sqrt{1 + u^2}\quad(u<{}0)\\
  \text{I}\hspace{ - .1em}\text{I}\hspace{ - .1em}\text{I} &\colon u<{}0,v<{}u<{} - v\\
  \text{I}\hspace{ - .1em}\text{V} &\colon  - \sqrt{1 + u^2}<{}v<{} - u\quad(u>0),\qquad - \sqrt{1 + u^2}<{}v<{}u\quad(u<{}0)
\end{align*}
である.
これによってPenrose図(本文図7-7)が得られる.

\chapter{宇宙論}
\section{完全流体近似}
相対論的宇宙論では宇宙を完全流体として扱うことが多い.
完全流体とは,圧力が等方的で,粘性・熱伝導がない流体である.
粘性がないので応力は必ず断面に垂直な向きであり,熱伝導がないので,エネルギーは粒子の流れによって運ばれる.
これらの条件から,静止している(流体が静止して見える座標系から見た時の)完全流体は,
\begin{align}
  \frac{\partial\rho{c^2}}{\partial{t}}=0,\quad\frac{\partial{p}}{\partial{x^{\mu}}}=0
\end{align}
を満たしているはずである.前者がエネルギー保存,後者が運動量保存則を表している.ここで,エネルギー運動量テンソルとして,
\begin{align}
  T'^{ij}=
  \begin{pmatrix}
    \rho{c^2} & 0 & 0 & 0\\
    0 & p & 0 & 0\\
    0 & 0 & p & 0\\
    0 & 0 & 0 & p
  \end{pmatrix}
\end{align}
を採用すれば,先程の条件は,
\[\partial_{j}T'^{ij}=0\]
と言うように,今までと同様の形式になる(例えば,力学場でのテンソルなど).

このテンソルは他の座標系ではどのように表現されるだろうか?流体が4元速度$u^i$で動いて見える座標系を考える.
簡単のため,流体は$x$軸方向に運動しているとする.考えている座標系と流体に固定された座標系のLorentz変換は,
\[
\begin{pmatrix}
  ct\\
  x\\
  y\\
  z
\end{pmatrix}
=
\begin{pmatrix}
  \dfrac{1}{\sqrt{1 - (v/c)^2}} & \dfrac{v/c}{\sqrt{1 - (v/c)^2}} & 0 & 0\\
  \dfrac{v/c}{\sqrt{1 - (v/c)^2}} & \dfrac{1}{\sqrt{1 - (v/c)^2}} & 0 & 0\\
  0 & 0 & 1 & 0\\
  0 & 0 & 0 & 1
\end{pmatrix}
\begin{pmatrix}
  c\tau\\
  x'\\
  y'\\
  z'
\end{pmatrix}
\]
で与えられる.$x=vt$を代入すれば,
\[dt=\frac{d\tau}{\sqrt{1 - (v/c)^2}}\]
が分かる.これは固有時間についての式である.これを使って4元速度を具体的に求めれば,
\begin{align}
  u^0 &= \frac{cdt}{d\tau} = \frac{c}{\sqrt{1 - (v/c)^2}}\\
  u^1 &= \frac{dx}{d\tau} = \frac{v}{\sqrt{1 - (v/c)^2}} = \sqrt{\left(u^0\right)^2 - c^2}
\end{align}
が得られる.よって,Lorentz変換は4元速度を使えば,
\begin{align}
  \frac{\partial{x^i}}{\partial{x'^j}}=
  \begin{pmatrix}
    {u^0}/{c} & {u^1}/{c} & 0 & 0\\
    {u^1}/{c} & {u^0}/{c} & 0 & 0\\
    0 & 0 & 1 & 0\\
    0 & 0 & 0 & 1
  \end{pmatrix}
\end{align}
と表すことができる.我々が知りたいエネルギー運動量テンソルは,
\[T^{ij}=\frac{\partial{x^i}}{\partial{x'^a}}\frac{\partial{x^j}}{\partial{x'^b}}T'^{ab}\]
と変換されるので,先程の式を代入すれば,
\begin{align}
  T^{00} &= \left(\frac{u^0}{c}\right)^2 T'^{00} + \left(\frac{u^1}{c}\right)^2 T'^{11}\notag\\
  &= \rho\left(u^0\right)^2 + \left(\frac{u^0}{c}\right)^2p - p
\end{align}
となる.他の成分についても同様に計算すればエネルギー運動量テンソルが求まる.

これを一般相対論に拡張してみよう.
流体に固定した座標系の計量は$\eta^{ab}$(瞬間的に流体と速度が一致している慣性系を考えている)で,我々の座標系の計量は$g^{ij}$なので,
\[g^{ij}=\frac{\partial{x^i}}{\partial{x'^a}}\frac{\partial{x^j}}{\partial{x'^b}}\eta^{ab}\]
と変換される.時間成分と空間成分に分ければ,
\begin{align*}
  g^{ij} &=  - \frac{1}{c^2}\frac{\partial{x^i}}{\partial{\tau}}\frac{\partial{x^j}}{\partial{\tau}} + \sum_{a=1}^{3}\frac{\partial{x^i}}{\partial{x'^a}}\frac{\partial{x^j}}{\partial{x'^a}}\\
  &=  - \frac{1}{c^2}u^iu^j + \sum_{a=1}^{3}\frac{\partial{x^i}}{\partial{x'^a}}\frac{\partial{x^j}}{\partial{x'^a}}
\end{align*}
となる.さらに,エネルギー運動量テンソルの変換も全く同様に,
\[T^{ij}= - \rho{}u^iu^j + p\sum_{a=1}^{3}\frac{\partial{x^i}}{\partial{x'^a}}\frac{\partial{x^j}}{\partial{x'^a}}\]
と表すことができる.以上2式から,
\begin{align}
  T^{ij} &= \left(\rho + \frac{p}{c^2}\right)u^iu^j + pg^{ij}
\end{align}
と求まる.もしも我々の座標系が慣性系なら計量を$g^{ij}$から$\eta^{ij}$にしてやればよい.

\section{Friedmannモデル}
\paragraph{(8.13), (8.14)}
一様で等方な空間に時間を加えて作った時空の計量
\begin{align*}
  ds^2 &= -  c^2dt^2 + \frac{1}{1 - \frac{k}{a^2}r^2}dr^2 + r^2(d\theta^2 + \sin\theta^2d\phi^2)\quad(k=1, 0, -  1)\\
  &= -  c^2dt^2 + a^2(t)[d\chi^2 + \left.
  \begin{cases}
    \sin^2\chi\\
    \chi^2\\
    \sinh^2\chi
  \end{cases}
  \right\}(d\theta^2 + \sin\theta^2d\phi^2)]
\end{align*}
をRobertson-Walker計量という.$a(t)$を宇宙のスケール項と呼ぶ.

さらに,宇宙を満たす物体は静止している完全流体とする.
そのエネルギー運動量テンソルは,
\[
{T^i}_j=T^{ik}g_{kj}=
\begin{pmatrix}
  - \rho{c^2} & 0 & 0 & 0\\
  0 & p & 0 & 0\\
  0 & 0 & p & 0\\
  0 & 0 & 0 & p
\end{pmatrix}
\]
である.

まずは,$k=1$の時のChristoffel記号を求めよう.
\[{\Gamma^k}_{ij}=\frac{1}{2}g^{kl}\left(\partial_ig_{jl} + \partial_jg_{li} - \partial_lg_{ij}\right)\]
にRobertson - Walker計量を代入すれば,
\begin{align*}
  {\Gamma^0}_{11}=\frac{a\dot{a}}{c}, \quad{\Gamma^0}_{22}=\frac{a\dot{a}}{c}\sin^2\chi, \quad{\Gamma^0}_{33}=\frac{a\dot{a}}{c}\sin^2\chi\sin^2\theta\\
  {\Gamma^1}_{01}=\frac{1}{c}\frac{\dot{a}}{a}, \quad{\Gamma^1}_{22}= - \sin\chi\cos\chi, \quad{\Gamma^1}_{33}= - \sin\chi\cos\chi\sin^2\theta\\
  {\Gamma^2}_{02}=\frac{1}{c}\frac{\dot{a}}{a}, \quad{\Gamma^2}_{12}=\cot\chi, \quad{\Gamma^2}_{33}= - \sin\theta\cos\theta\\
  {\Gamma^3}_{03}=\frac{1}{c}\frac{\dot{a}}{a}, \quad{\Gamma^3}_{13}=\cot\chi, \quad{\Gamma^3}_{23}=\cot\theta
\end{align*}
となる(下添字を入れ替えたものについては省略した).この時,Ricciテンソルは,
\begin{align*}
  R_{00} &= -  \partial_0{\Gamma^i}_{0i} - {\Gamma^a}_{0i}{\Gamma^i}_{a0}\\
  &= -  \frac{3}{c^2}\frac{\ddot{a}}{a}\\
  R_{11} &= \partial_i{\Gamma^i}_{11} - \partial_1{\Gamma^i}_{1i} + {\Gamma^a}_{11}{\Gamma^i}_{ai} - {\Gamma^a}_{1i}{\Gamma^i}_{a1}\\
  &= \frac{a\ddot{a}}{c^2} + 2\frac{\dot{a}^2}{c^2} + 2
\end{align*}
である.よって,混合形式にすれば,
\[{R^0}_0=\frac{3}{c^2}\frac{\ddot{a}}{a}, \quad{R^1}_1=\frac{1}{c^2}\frac{\ddot{a}}{a} + \frac{2}{c^2}\left(\frac{\dot{a}}{a}\right)^2 + \frac{2}{a^2}\]
となる.

Einstein方程式は,
\[{R^i}_j - \frac{1}{2}R{\delta^i}_j + \Lambda{\delta^i}_j=\frac{8\pi{G}}{c^4}{T^i}_j\]
である.両辺縮約すれば,
\[R=4\Lambda - \frac{8\pi{G}}{c^4}T\]
が得られる.$T$はエネルギー運動量テンソルのトレース${T^i}_i$である.上の2式から$R$を消去すれば,
\[{R^i}_j - \Lambda{\delta^i}_j + \frac{4\pi{G}}{c^4}T{\delta^i}_j=\frac{8\pi{G}}{c^4}{T^i}_j\]
となり,完全流体であれば$T=3p - \rho{c^2}$なので,
\begin{align}
  {R^i}_j - \Lambda{\delta^i}_j + \frac{4\pi{G}}{c^4}\left(3p - \rho{c^2}\right){\delta^i}_j=\frac{8\pi{G}}{c^4}{T^i}_j
  \label{ein_field_eq}
\end{align}
となる.$k=1$でのRicciテンソルを代入すれば,$(0, 0)$及び$(1, 1)$成分について,それぞれ,
\[
\begin{cases}
  \dfrac{\ddot{a}}{a} = -  \dfrac{4\pi{G}}{3}\rho - \dfrac{4\pi{G}}{c^2}p + \dfrac{\Lambda}{3}c^2 \\[10pt]
  \dfrac{\ddot{a}}{a} + 2\left(\dfrac{\dot{a}}{a}\right)^2 + 2\dfrac{c^2}{a^2}=4\pi{G}\rho - \dfrac{4\pi{G}}{c^2}p + {\Lambda}c^2
\end{cases}
\]
が得られる.

$k=0$では,
\[{\Gamma^2}_{12}\to\frac{1}{\chi}, \quad{\Gamma^3}_{13}\to\frac{1}{\chi}\]
なので,
\[R_{11}\to\frac{\ddot{a}a}{c^2} + 2\frac{\dot{a}^2}{c^2}.\]

$k= - 1$では,
\[{\Gamma^2}_{12}\to\coth\chi, \quad{\Gamma^3}_{13}\to\coth{\chi}\]
なので,
\[R_{11}\to\frac{\ddot{a}a}{c^2} + 2\frac{\dot{a}^2}{c^2} - 2.\]

以上から,全ての場合をまとめれば,
\begin{align*}
  \left(\frac{\dot{a}}{a}\right)^2 + \frac{k}{a^2}c^2=\frac{8\pi{G}}{3}\rho + \frac{\Lambda}{3}c^2\\
  2\frac{\ddot{a}}{a} + \left[\left(\frac{\dot{a}}{a}\right)^2 + \frac{k}{a^2}c^2\right]= - 8\pi{G}\frac{p}{c^2} + \Lambda{c^2}
\end{align*}
が得られる.

\chapter{重力波}
\setcounter{section}{2}
\section{重力波のエネルギー}
\paragraph{エネルギー運動量擬テンソル}
一般相対論において,無限小変換を考えれば,エネルギー・運動量が保存される条件は次のようになる:
\begin{align}
  {{T^j}_i}_{;j}=0.\label{T;=0}
\end{align}
さて,この条件式をそのまま共変微分すると,左辺は
\begin{align*}
  {{T^j}_i}_{;j} &= \partial_j{T^j}_i + {\Gamma^j}_{rj}{T^r}_i - {\Gamma^r}_{ij}{T^j}_r\\
  &= \partial_j{T^j}_i + \frac{1}{\sqrt{ - g}}\partial_r\sqrt{ - g}\cdot{T^r}_i - {\Gamma^r}_{ij}{T^j}_r\\
  &= \frac{1}{\sqrt{ - g}}\partial_j\left(\sqrt{ - g}\cdot{T^j}_i\right) - \frac{1}{2}g^{rl}\left(\partial_ig_{jl} + \partial_jg_{li} - \partial_lg_{ij}\right){T^j}_r\\
  &= \frac{1}{\sqrt{ - g}}\partial_j\left(\sqrt{ - g}\cdot{T^j}_i\right) - \frac{1}{2}\partial_ig_{jl}\cdot{}T^{jl} - \frac{1}{2}\left(T^{jl}\partial_jg_{li} - T^{jl}\partial_lg_{ij}\right)
\end{align*}
となる.ところで,最後の括弧は2項目の$j, l$を入れ替えると0になることが分かる.よって,条件式\eqref{T;=0}は
\begin{align}
  \frac{1}{\sqrt{ - g}}\partial_j\left(\sqrt{ - g}\cdot{T^j}_i\right) - \frac{1}{2}\partial_ig_{jl}\cdot{}T^{jl}=0\label{T;=0??}
\end{align}
と表される.ところで,力学場でのエネルギー運動量テンソルは,
\begin{align}
  T_{ij}=\frac{2}{\sqrt{ - g}}\left[\partial_k\left(\frac{\partial{}(\mathcal{L}\sqrt{ - g})}{\partial(\partial_k{g^{ij}})}\right) - \frac{\partial(\mathcal{L}\sqrt{ - g})}{\partial{g^{ij}}}\right]
  \label{ep_tensor}
\end{align}
のように表され,このエネルギー・運動量が保存される条件は
\begin{align}
  \partial_j\left(\sqrt{ - g}\cdot{T^j}_i\right)=0\label{div_trueT}
\end{align}
である.これは明らかに\eqref{T;=0??}と矛盾する.
よって,\eqref{T;=0??}ないしは\eqref{T;=0}の${T^j}_i$は力学場\eqref{ep_tensor}以外のものを含んでいると考えるのがよさそうだ.
そして,これを${t^j}_i$と表すことにしよう.つまり,エネルギー・運動量が保存される条件は,\eqref{div_trueT}改め
\begin{align}
  \partial_k\left[\sqrt{ - g}({T^j}_i + {t^j}_i)\right]=0
\end{align}
である.これは\eqref{T;=0}と両立可能なので,${t^j}_i$には次の条件が課せられる:
\begin{align}
  \frac{1}{\sqrt{ - g}}\partial_j\left(\sqrt{ - g}\cdot{t^j}_i\right) + \frac{1}{2}\partial_ig_{jl}\cdot{}T^{jl}=0.\label{pseudo_Eptenosr}
\end{align}
この${t^j}_i$をエネルギー運動量擬テンソルと呼ぶ.

\paragraph{(9.48)}
使う式について簡単にまとめておく.
\begin{align*}
  g_{ij}=\eta_{ij} + h_{ij}, \quad{}g^{ij}=\eta^{ij} - h^{ij}, \quad{}|h_{ij}|\ll1, \quad{}\sqrt{ - g}\sim1\\
  \phi_{ij}=h_{ij} - \frac{1}{2}hg_{ij}\sim{}h_{ij} - \frac{1}{2}h\eta_{ij}=h_{ij} + \frac{1}{2}\phi\eta_{ij}\\
  \Box\phi_{ij}= - \frac{16\pi{G}}{c^4}T_{ij}\colon(\text{Einstein's Equation})\\
  \partial_j{\phi^j}_i=0\colon(\text{gauge condition})
\end{align*}
これらを\eqref{pseudo_Eptenosr}に代入すれば,
\begin{align*}
  \partial_j{t^j}_i &= \frac{c^4}{32\pi{G}}\partial_i\left(\phi_{kl} - \frac{\eta_{kl}}{2}\phi\right)\eta^{st}\partial_s\partial_t\phi^{kl}\\
  &= \frac{c^4}{32\pi{G}}\partial_i\phi_{kl}\cdot{}\eta^{st}\partial_s\partial_t\phi^{kl} - \frac{c^4}{64\pi{G}}\eta_{kl}\eta^{st}\partial_i\phi\cdot\partial_s\partial_t\phi^{kl}\\
  &= \frac{c^4}{32\pi{G}}\partial_i\phi_{kl}\cdot{}\eta^{st}\partial_s\partial_t\phi^{kl} - \frac{c^4}{64\pi{G}}\eta^{st}\partial_i\phi\cdot\partial_s\partial_t\phi\\
  &= \frac{c^4}{32\pi{G}}\left[\partial_s(\partial_i\phi_{kl}\cdot{}\eta^{st}\partial_t\phi^{kl}) - \partial_s\partial_i\phi_{kl}\cdot{}\eta^{st}\partial_t\phi^{kl}\right]\\
  &\qquad - \frac{c^4}{64\pi{G}}\left[\partial_s\left(\eta^{st}\partial_i\phi\cdot\partial_t\phi\right) - \eta^{st}\partial_s\partial_i\phi\cdot\partial_t\phi\right]
\end{align*}
が求まる.[]内の2項目について,
\begin{align*}
  \partial_s\partial_i\phi_{kl}\cdot{}\eta^{st}\partial_t\phi^{kl}
  &= \frac{1}{2}\partial_s\partial_i\phi_{kl}\cdot{}\eta^{st}\partial_t\phi^{kl} + \frac{1}{2}\partial_s\partial_i\phi_{kl}\cdot{}\eta^{st}\partial_t\phi^{kl}\\
  &= \frac{1}{2}\partial_s\partial_i\phi_{kl}\cdot{}\eta^{st}\partial_t\phi^{kl} + \frac{1}{2}\partial_t\partial_i\phi_{kl}\cdot{}\eta^{ts}\partial_s\phi^{kl}\\
  &= \frac{1}{2}\left(\partial_i\partial_s\phi_{kl}\cdot{}\eta^{st}\partial_t\phi^{kl} + \partial_s\phi^{kl}\cdot{}\partial_i\eta^{st}\partial_t\phi_{kl}\right)\\
  &= \frac{1}{2}\left(\partial_i\partial_s\phi_{kl}\cdot{}\eta^{st}\partial_t\phi^{kl} + \partial_s\phi_{kl}\cdot{}\partial_i\eta^{st}\partial_t\phi^{kl}\right)\\
  &= \frac{1}{2}\partial_i\left(\partial_s\phi_{kl}\cdot{}\eta^{st}\partial_t\phi^{kl}\right)\\
  &= \partial_j\left[\frac{1}{2}{\delta^j}_{i}\left(\partial_s\phi_{kl}\cdot{}\eta^{st}\partial_t\phi^{kl}\right)\right]
\end{align*}
などとなるので,先程の計算結果に代入し,
\begin{align*}
  \partial_j{t^j}_i = \frac{c^4}{64\pi{G}}\partial_j\left[2\partial_i\phi_{kl}\cdot{}\eta^{jt}\partial_t\phi^{kl} - \eta^{jt}\partial_i\phi\cdot\partial_t\phi + {\delta^j}_i\left(\frac{1}{2}\eta^{st}\partial_s\phi\cdot\partial_t\phi - \partial_s\phi_{kl}\cdot{}\eta^{st}\partial_t\phi^{kl}\right)\right]
\end{align*}
となる.よって,
\[{t^j}_i=\frac{c^4}{64\pi{G}}\left[2\partial_i\phi_{kl}\cdot{}\eta^{jt}\partial_t\phi^{kl} - \eta^{jt}\partial_i\phi\cdot\partial_t\phi + {\delta^j}_i\left(\frac{1}{2}\eta^{st}\partial_s\phi\cdot\partial_t\phi - \partial_s\phi_{kl}\cdot{}\eta^{st}\partial_t\phi^{kl}\right)\right]\]
と求まる.

\end{document}
