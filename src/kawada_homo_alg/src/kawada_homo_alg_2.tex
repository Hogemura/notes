\chapter{複体とホモロジー}
\setcounter{section}{1}
\section{2重複体}
\paragraph{例2.8}~
\begin{screen}
  \textbf{平坦加群の定義}.
  左$R$加群$L$に対し,以下の2つは同値である:
  \begin{enumerate}[label=(\roman*)]
    \item $L$は平坦加群である(定義1.5,p.46)
    \item 任意の右$R$加群の完全列$ \cdots \longrightarrow M_1 \longrightarrow M_2 \longrightarrow \cdots$に対し$ \cdots \longrightarrow M_1 \otimes_R L \longrightarrow M_2 \otimes_R L \longrightarrow \cdots$が完全である.
  \end{enumerate}
\end{screen}
\begin{proof}
  (ii)が成立すれば(i)が成立することは明らか($0$からはじまる適当な完全列を考えればよい).

  (i)が成立しているときに(ii)が成立することを示すには,(i)のもとで右$R$加群の完全系列
  \[
  \begin{CD}
    M_1 @>{f}>> M_2 @>{g}>> M_3
  \end{CD}
  \]
  に対し
  \[
  \begin{CD}
    M_1 \otimes_R L @>{f \otimes 1}>> M_2 \otimes_R L @>{g \otimes 1}>> M_3 \otimes_R L
  \end{CD}
  \]
  が完全となること,すなわち$\Im(f \otimes 1) = \ker (g \otimes 1)$を証明すればよい.

  $g \colon M_2 \to M_3$に対し,$R$加群の準同型
  \[ g' \colon M_2 \ni m_2 \mapsto g(m_2) \in \Im g \]
  と包含写像$\imath \colon \Im g \hookrightarrow M_3$を考えれば,明らかに$g = \imath \circ g'$となる.
  $g'$は全射なので
  \[
  \begin{CD}
    M_1 @>{f}>> M_2 @>{g'}>> \Im g' @>>> 0
  \end{CD}
  \]
  は完全列で,テンソル積は右完全なので
  \[
  \begin{CD}
    M_1 \otimes_R L @>{f \otimes 1}>> M_2 \otimes_R L @>{g' \otimes 1}>> \Im g' \otimes_R L @>>> 0
  \end{CD}
  \]
  も完全である:$\Im(f \otimes 1) = \ker (g' \otimes 1)$.

  $\imath \colon \Im g \hookrightarrow M_3$は単射であるので,(i)から$\imath \otimes 1 \colon \Im g' \otimes_R L \to M_3 \otimes_R L$も単射である:
  % \[
  % \begin{CD}
  %   M_1 \otimes_R L @>{f \otimes 1}>> M_2 \otimes_R L @>{g' \otimes 1}>> \Im g' \otimes_R L @>>> 0 \\
  %   @. @V{g \otimes 1}VV  @VV{\imath \otimes 1}V \\
  %   @. M_3 \otimes_R L @= M_3 \otimes_R L @.
  % \end{CD}
  % \]
  \[
  \begin{tikzcd}
    M_1 \otimes_R L \arrow[r, "f \otimes 1"] & M_2 \otimes_R L \arrow[rr, "g' \otimes 1"]\arrow[rd, "g \otimes 1" description] &[-20pt]&[-20pt] \Im g' \otimes_R L \arrow[r]\arrow[ld, "\imath \otimes 1" description] & 0 \\[8pt]
    & & M_3 \otimes_R L
  \end{tikzcd}
  \]

  $\ker (g' \otimes 1) = \ker (g \otimes 1)$を示せば証明は完了する.
  $\sum x_i \otimes y_i \in \ker (g'\otimes 1)$とすれば,$(g \otimes 1)(\sum x_i \otimes y_i) = (\imath \otimes 1) \circ (g' \otimes 1)(\sum x_i \otimes y_i) = 0$なので,$\sum x_i \otimes y_i \in \ker(g \otimes 1)$である.
  従って$\ker (g' \otimes 1) \subset \ker (g \otimes 1)$.
  $\sum x_i \otimes y_i \in \ker (g\otimes 1)$とすれば,$0 = (g \otimes 1)(\sum x_i \otimes y_i) = (\imath \otimes 1) \circ (g' \otimes 1)(\sum x_i \otimes y_i)$である.
  $\imath \otimes 1$は単射なので,$(g' \otimes 1)(\sum x_i \otimes y_i) = 0$すなわち$\sum x_i \otimes y_i \in \ker(g' \otimes 1)$であり,$\ker (g \otimes 1) \subset \ker (g' \otimes 1)$.
\end{proof}

\paragraph{例2.10}~
\begin{screen}
  \textbf{単射的加群の定義}.
  左$R$加群$I$に対し,以下の2つは同値である:
  \begin{enumerate}[label=(\roman*)]
    \item $I$は単射的加群である(定義1.2,p.40)
    \item 任意の左$R$加群の完全列$ \cdots \longrightarrow M_1 \longrightarrow M_2 \longrightarrow \cdots$に対し$ \cdots \longrightarrow \Hom_R(M_2, I) \longrightarrow \Hom_R(M_1, I) \longrightarrow \cdots$が完全である.
  \end{enumerate}
\end{screen}
\begin{proof}
  (ii)が成立すれば(i)が成立することは定理1.7(p.49)から明らか.

  (i)が成立しているときに(ii)が成立することを示すには,(i)のもとで左$R$加群の完全系列
  \[
  \begin{CD}
    M_1 @>{f}>> M_2 @>{g}>> M_3
  \end{CD}
  \]
  に対し
  \[
  \begin{CD}
    \Hom_R(M_3, I) @>{^\#g}>> \Hom_R(M_2, I) @>{^\#f}>> \Hom_R(M_1, I)
  \end{CD}
  \]
  が完全となること,すなわち$\Im(^\#g) = \ker (^\#f)$を証明すればよい.
  $g \colon M_2 \to M_3$に対し,$R$加群の準同型
  \[ g' \colon M_2 \ni m_2 \mapsto g(m_2) \in \Im g \]
  と包含写像$\imath \colon \Im g \hookrightarrow M_3$を考えれば,明らかに$g = \imath \circ g'$となる.
  $g'$は全射なので
  \[
  \begin{CD}
    M_1 @>{f}>> M_2 @>{g'}>> \Im g @>>> 0
  \end{CD}
  \]
  は完全列で,$\Hom_R$は左完全なので
  \[
  \begin{CD}
    0 @>>> \Hom_R(\Im g, I) @>{^\#g'}>> \Hom_R(M_2, I) @>{^\#f}>> \Hom_R(M_1, I)
  \end{CD}
  \]
  も完全である:$\Im(^\#g') = \ker (^\#f)$.
  $\imath \colon \Im g \hookrightarrow M_3$は単射なので,(i)の仮定から$\varphi \in \Hom_R(\Im g, I)$に対し,$\varphi = \tilde{\varphi} \circ \imath = {}^\#\imath (\tilde\varphi)$となる$\tilde{\varphi} \in \Hom_R(M_3, I)$が存在する:
  % \[
  % \begin{CD}
  %   0 @>>> \Hom_R(\Im g, I) @>{^\#g'}>> \Hom_R(M_2, I) @>{^\#f}>> \Hom_R(M_1, I) \\
  %   @. @A{^\#\imath}AA  @AA{^\# g}A \\
  %   @. \Hom_R(M_3, I) @= \Hom_R(M_3, I) @.
  % \end{CD}
  % \]
  \[
  \begin{tikzcd}
    0 \arrow[r] & \Hom_R(\Im g, I) \arrow[rr, "^\#g'"] &[-30pt]&[-30pt] \Hom_R(M_2, I) \arrow[r, "^\#f"] & \Hom_R(M_1, I) \\[8pt]
    & & \Hom_R(M_3, I)\arrow[ul, "^\#\imath" description]\arrow[ur, "^\# g" description]
  \end{tikzcd}
  \]

  $\Im(^\# g') = \Im(^\# g)$を示せば証明は完了する.
  $\Im(^\# g')$の元は$\varphi \in \Hom_R(\Im g, I)$によって$^\# g'(\varphi)$と表すことができ,
  \[^\# g'(\varphi) = \varphi \circ g' = \tilde{\varphi} \circ \imath \circ g' = \tilde\varphi \circ g ={} ^\# g (\varphi) \in \Im(^\# g)\]
  となるので,$\Im(^\# g') \subset \Im(^\# g)$.
  $\Im(^\# g)$の元は$\tilde\varphi \in \Hom_R(M_3, I)$によって$^\# g(\tilde\varphi)$と表すことができ,
  \[^\# g(\tilde\varphi) = \tilde\varphi \circ g = \tilde{\varphi} \circ \imath \circ g' = \varphi \circ g' ={} ^\# g' (\varphi) \in \Im(^\# g')\]
  となるので,$\Im(^\# g) \subset \Im(^\# g')$.
\end{proof}

\begin{screen}
  \textbf{射影的加群の定義}.
  左$R$加群$P$に対し,以下の2つは同値である:
  \begin{enumerate}[label=(\roman*)]
    \item $P$は射影的加群である(定義1.1,p.36)
    \item 任意の左$R$加群の完全列$ \cdots \longrightarrow M_1 \longrightarrow M_2 \longrightarrow \cdots$に対し$ \cdots \longrightarrow \Hom_R(P, M_1) \longrightarrow \Hom_R(P, M_2) \longrightarrow \cdots$が完全である.
  \end{enumerate}
\end{screen}
\begin{proof}
  (ii)が成立すれば(i)が成立することは定理1.3(p.37)から明らか.

  (i)が成立しているときに(ii)が成立することを示すには,(i)のもとで左$R$加群の完全系列
  \[
  \begin{CD}
    M_1 @>{f}>> M_2 @>{g}>> M_3
  \end{CD}
  \]
  に対し
  \[
  \begin{CD}
    \Hom_R(P, M_1) @>{f^\#}>> \Hom_R(P, M_2) @>{g^\#}>> \Hom_R(P, M_3)
  \end{CD}
  \]
  が完全となること,すなわち$\Im f^\# = \ker g^\#$を証明すればよい.
  $f \colon M_1 \to M_2$に対し,$R$加群の準同型
  \[ f' \colon \Coim f = M_1 / \ker f \ni m_1 + \ker f \mapsto f(m_1) \in M_2 \]
  と射影$\pi \colon M_1 \to \Coim f$を考えれば,明らかに$f = f' \circ \pi$となる.
  $f'$は単射なので
  \[
  \begin{CD}
    0 @>>> \Coim f @>{f'}>> M_2 @>{g}>> M_3
  \end{CD}
  \]
  は完全列で,$\Hom_R$は左完全なので
  \[
  \begin{CD}
    0 @>>> \Hom_R(P, \Coim f) @>{f'^\#}>> \Hom_R(P, M_2) @>{g^\#}>> \Hom_R(P, M_3)
  \end{CD}
  \]
  も完全である:$\Im f'^\# = \ker g^\#$.
  $\pi \colon M_1 \to \Coim f$は全射なので,(i)の仮定から$\varphi \in \Hom_R(P, \Coim f)$に対し,$\varphi = \pi \circ \tilde{\varphi} = \pi^\# (\tilde\varphi)$となる$\tilde{\varphi} \in \Hom_R(P, M_1)$が存在する:
  % \[
  % \begin{CD}
  %   0 @>>> \Hom_R(P, \Coim f) @>{f'^\#}>> \Hom_R(P, M_2) @>{g^\#}>> \Hom_R(P, M_3) \\
  %   @. @A{\pi^\#}AA  @AA{f^\#}A \\
  %   @. \Hom_R(P, M_1) @= \Hom_R(P, M_1) @.
  % \end{CD}
  % \]
  \[
  \begin{tikzcd}
    0 \arrow[r] & \Hom_R(P, \Coim f) \arrow[rr, "f'^\#"] &[-30pt] &[-30pt] \Hom_R(P, M_2) \arrow[r, "g^\#"] & \Hom_R(P, M_3) \\[8pt]
    & & \Hom_R(P, M_1) \arrow[ul, "\pi^\#"]\arrow[ur, "f^\#"']
  \end{tikzcd}
  \]

  $\Im(f'^\#) = \Im(f^\#)$を示せば証明は完了する.
  $\Im(f'^\#)$の元は$\varphi \in \Hom_R(P, \Coim f)$によって$f'^\# (\varphi)$と表すことができ,
  \[f'^\# (\varphi) = f' \circ \varphi = f' \circ \pi \circ \tilde{\varphi} = f \circ \tilde\varphi = f^\# (\tilde\varphi) \in \Im(f^\#)\]
  となるので,$\Im(f'^\#) \subset \Im(f^\#)$.
  $\Im(f^\#)$の元は$\tilde\varphi \in \Hom_R(P, M_1)$によって$f^\#(\tilde\varphi)$と表すことができ,
  \[f^\#(\tilde\varphi) = f \circ \tilde\varphi = f' \circ \pi \circ \tilde{\varphi} = f' \circ \varphi = f'^\# (\varphi) \in \Im(f'^\#)\]
  となるので,$\Im(f^\#) \subset \Im(f'^\#)$.
\end{proof}

\begin{screen}
  $\boldsymbol{H}(\Hom_R(A, \boldsymbol{Y})) \simeq \boldsymbol{H}(\Hom_R(\boldsymbol{X}, \boldsymbol{Y}))$
\end{screen}
\[
\begin{CD}
  @. @. 0 @. 0 @. 0 \\
  @. @. @VVV @VVV @VVV \\
  @. @. \Hom_R(X_0, B) @>{^\#(\partial_1')}>> \Hom_R(X_1, B) @>{^\#(\partial_2')}>> \Hom_R(X_2, B) \\
  @. @. @V{(\varepsilon'')^\#}VV @V{(\varepsilon'')^\#}VV @V{(\varepsilon'')^\#}VV \\
  0 @>>> \Hom_R(A, Y^0) @>{^\#(\varepsilon')}>> \Hom_R(X_0, Y^0) @>{^\#(\partial_1')}>> \Hom_R(X_1, Y^0) @>{^\#(\partial_2')}>> \Hom_R(X_2, Y^0) \\
  @. @V{-(d''^0)^\#}VV @V{-(d''^0)^\#}VV @V{-(d''^0)^\#}VV @V{-(d''^0)^\#}VV \\
  0 @>>> \Hom_R(A, Y^1) @>{^\#(\varepsilon')}>> \Hom_R(X_0, Y^1) @>{^\#(\partial_1')}>> \Hom_R(X_1, Y^1) @>{^\#(\partial_2')}>> \Hom_R(X_2, Y^1) \\
  @. @V{(d''^1)^\#}VV @V{(d''^1)^\#}VV @V{(d''^1)^\#}VV @V{(d''^1)^\#}VV \\
  0 @>>> \Hom_R(A, Y^2) @>{^\#(\varepsilon')}>> \Hom_R(X_0, Y^2) @>{^\#(\partial_1')}>> \Hom_R(X_1, Y^2) @>{^\#(\partial_2')}>> \Hom_R(X_2, Y^2)
\end{CD}
\]
\begin{proof}
  ふちどり双対鎖複体,全双対鎖複体は次のように定義される:
  \begin{align*}
    \boldsymbol{A} & = \Set{ \Hom_R(A, Y^q) \mid (-1)^{q+1}(d''^q)^\# }, \\
    \boldsymbol{B} & = \Set{ \Hom_R(X_p, B) \mid {}^\#(\partial_{p+1}') }, \\
    \boldsymbol{W} & = \Set{\left. \bigoplus_{p+q=n} \Hom_R(X_p, Y^q) ~\middle|~ \sum_{p+q=n} {}^\#(\partial_{p+1}') + (-1)^{n+1}(d''^q)^\# \right.} .
  \end{align*}
  ここで,$W^n \to W^{n+1}$の$n$次双対境界作用素は次のようになる($1 \leq k \leq n$):
  \[ \tilde{f}_W^n \colon W^n \ni
  \begin{pmatrix}
    f^{n, 0} \\
    f^{n-1, 1} \\
    \vdots \\
    f^{0, n}
  \end{pmatrix}
  \mapsto
  \begin{pmatrix}
    f^{n, 0} \circ \partial_{n+1}' \\[10pt]
    f^{n-1, 1} \circ \partial_n + (-1)^{n+1} d''^0 \circ f^{n, 0}) \\
    \vdots \\
    f^{{n-k}, k}\circ \partial_{n-k+1}' + (-1)^{n+1}d''^{k-1} \circ f^{n-k+1, k-1} \\
    \vdots \\
    (-1)^{n+1} d''^n \circ f^{0, n}
  \end{pmatrix}
  \in W^{n+1}.
  \]
  準同型$\boldsymbol\varphi\colon \boldsymbol A \to \boldsymbol W$を
  \[
  \begin{CD}
    \varphi^n \colon \Hom_R(A, Y^n) @>{^\#(\varepsilon')}>> \Hom_R(X_0, Y^n) @>{i_n}>> W_n = \bigoplus_{p+q=n}\Hom_R(X_p, Y^q)
  \end{CD}
  \]
  によって定める.

  $\boldsymbol\varphi$が鎖準同型であることを証明する.そのためには次の図式が可換であることを示せばよい.
  \[
  \begin{CD}
    \Hom_R(A, Y^n) @>{i_n \circ{} ^\#(\varepsilon')}>> \bigoplus_{p+q=n}\Hom_R(X_p, Y^q) \\
    @V{(-1)^{n+1}(d''^n)^\#}VV @VV{\tilde{f}_W^n}V \\
    \Hom_R(A, Y^{n+1}) @>>{i_{n+1} \circ{} ^\#(\varepsilon')}> \bigoplus_{p+q=n+1}\Hom_R(X_p, Y^q)
  \end{CD}
  \]
  $f^n \in \Hom_R(A, Y^n)$とすれば,
  \begin{align*}
    \tilde{f}_W^n \circ i_n \circ{} ^\#(\varepsilon') (f^n) & = \tilde{f}_W^n ((0, \ldots, 0, f^n \circ \varepsilon')) = (0, \ldots, 0, (-1)^{n+1}(d''^n)^\#(f^n \circ \varepsilon')) \\
    & = (0, \ldots, 0, (-1)^{n+1} d''^n \circ f^n \circ \varepsilon') \\
    & = i_{n+1} \circ{} ^\#(\varepsilon') \circ (-1)^{n+1}(d''^n)^\# (f^n)
  \end{align*}
  となるので,$\tilde{f}_W^n \circ i_n \circ{} ^\#(\varepsilon') = i_{n+1} \circ{} ^\#(\varepsilon') \circ (-1)^{n+1}(d''^n)^\#$.従って,鎖準同型
  \[ \varphi^{n*} \colon H^n(\boldsymbol{A}) = \ker (d''^n)^\# / \Im (d''^{n-1})^\# \to H^n(\boldsymbol{W}) = \ker \tilde{f}_W^n / \Im \tilde{f}_W^{n-1}  \]
  が引き起こされる.

  \begin{lem}\label{lem1.1}
    $f^{p, q} \in \Hom_R(X_p, Y^q)~(p+q=n)$に対し,$\sum_{p+q=n} f^{p, q} \in \ker \tilde{f}_W^n \subset W^n$とする:
    \begin{align}
      \begin{split}
        & f^{n, 0} \circ \partial_{n+1}' = 0, \\
        & f^{n-1, 1} \circ \partial_n' + (-1)^{n+1} d''^0 \circ f^{n, 0} = 0, \\
        & \qquad\qquad\qquad \vdots \\
        & f^{{n-k}, k} \circ \partial_{n-k+1}' + (-1)^{n+1} d''^{k-1} \circ f^{n-k+1, k-1} = 0, \\
        & \qquad\qquad\qquad \vdots \\
        & (-1)^{n+1} d''^n f^{0, n} = 0.
      \end{split}
      \label{exm2_10_eq_1}
    \end{align}
    このとき,$f^{p, q} \in \Hom_R(X_p, Y^q)~(p+q=n-1)$が存在し,$\tilde{f}_W^{n-1}(f^{n-1, 0} + \cdots + f^{0, n-1})$のはじめの$n-1$成分を$(f^{n, 0}, \ldots, f^{1, n-1})$と等しくできる:
    \begin{align}
      \begin{split}
        & f^{n-1, 0} \circ \partial_{n}' = f^{n, 0}, \\
        & f^{n-2, 1} \circ \partial_{n-1}' + (-1)^{n}d''^0 \circ f^{n-1, 0} = f^{n-1, 1}, \\
        & \qquad\qquad\qquad \vdots \\
        & f^{n-k-1, k} \circ \partial_{n-k}' + (-1)^{n} d''^{k-1} f^{n-k, k-1} = f^{n-k, k}, \\
        & \qquad\qquad\qquad \vdots \\
        & f^{0, n-1} \circ \partial_1' + (-1)^{n} d''^{n-2} f^{1, n-2} = f^{1, n-1} .
      \end{split}
      \label{exm2_10_eq_2}
    \end{align}
  \end{lem}
  \begin{proof}
    まず$f^{n-1, 0}$の存在を証明する.
    \[
    \begin{CD}
      \Hom_R(X_{n-1}, Y^0) @>{^\#(\partial_n')}>> \Hom_R(X_n, Y^0) @>{^\#(\partial_{n+1}')}>> \Hom_R(X_{n+1}, Y^0)
    \end{CD}
    \]
    が完全なので,
    \eqref{exm2_10_eq_1}1番目の式から$f^{n, 0} \in \ker {}^\#(\partial_{n+1}') = \Im {}^\#(\partial_n')$なので,\eqref{exm2_10_eq_2}1番目の式を満たす$f^{n-1, 0}$の存在が分かる.
    \eqref{exm2_10_eq_1}2番目の式と\eqref{exm2_10_eq_2}1番目の式から
    \[ 0 = f^{n-1, 1} \circ \partial_n' + (-1)^{n+1} d''^0 \circ f^{n-1, 0} \circ \partial_{n}' = {}^\#(\partial_n') \left\{ f^{n-1, 1} + (-1)^{n+1} d''^0 \circ f^{n-1, 0} \right\} \]
    なので,
    \[ f^{n-1, 1} + (-1)^{n+1} d''^0 \circ f^{n-1, 0} \in \ker {}^\#(\partial_n') = \Im {}^\#(\partial_{n-1}').  \]
    従って,\eqref{exm2_10_eq_2}2番目の式を満たす$f^{n-1, 1}$の存在が分かる.以下同様にして$f^{n-2, 1}, \ldots, f^{0, n-1}$が構成できる.
  \end{proof}

  \begin{lem}\label{lem1.4}
    上の補題の状況で,
    \begin{align}
      f^{0, n} + (-1)^{n+1} d''^{n-1} \circ f^{0, n-1} = a^n \circ \varepsilon' \label{exm2_10_eq_3}
    \end{align}
    となる$a^n \in \ker (d''^n)^\# \subset \Hom_R(A, X^n)$が存在する.
  \end{lem}
  \begin{proof}
    \eqref{exm2_10_eq_2}の最後の式に$d''^{n-1}$を作用させ($d''^{n-1} \circ d''^{n-2} = 0$)
    \[ d''^{n-1} \circ f^{0, n-1} \circ \partial_1' = d''^{n-1} \circ f^{1, n-1}.\]
    従って,
    \begin{align*}
      ^\#(\partial_1') (f^{0, n} + (-1)^{n+1} d''^{n-1} \circ f^{0, n-1} ) & = f^{0, n} \circ \partial_1' + (-1)^{n+1} d''^{n-1} \circ f^{0, n-1}  \circ \partial_1' \\
      & = f^{0, n} \circ \partial_1' + (-1)^{n+1} d''^{n-1} \circ f^{1, n-1} \\
      & = 0.
    \end{align*}
    最後に\eqref{exm2_10_eq_1}で$k=n$とした式を用いた.従って,
    \[ f^{0, n} + (-1)^{n+1} d''^{n-1} \circ f^{0, n-1} \in \ker {}^\#(\partial_1') = \Im {}^\#(\varepsilon'). \]
    これで\eqref{exm2_10_eq_3}を満足する$a^n \in \Hom_R(A, X^n)$の存在が証明された.あとは$a^n \in \ker (d''^n)^\#$を示せばよい.
    \eqref{exm2_10_eq_1}最後の式から$f^{0, n} \in \ker (d''^n)^\#$なので,\eqref{exm2_10_eq_3}に$d''^n$を作用させて,$d''^n \circ a^n \circ \varepsilon' = 0$.
    すなわち$d''^n \circ a^n \circ \varepsilon'(X_0) = 0$.
    $\varepsilon'\colon X_0 \to A$は全射なので,$d''^n \circ a^n (A) = 0$となり,$a^n \in \ker (d''^n)^\#$.
  \end{proof}

  $\varphi^{n\ast}$が全射であることを証明する.
  $H^n(\boldsymbol{W}) = \ker \tilde{f}_W^n / \Im \tilde{f}_W^{n-1}$の任意の元は$(f^{n, 0}, \ldots, f^{1, n-1}, f^{0,n}) + \Im \tilde{f}_W^{n-1}$と表せる.
  % <a exm2-10-lem-1">補題1</a>
  追加補題\ref{lem1.1}から$f^{n-1, 0}, \ldots, f^{0, n-1}$が存在して,
  \[ \tilde{f}_W^{n-1}(f^{n-1, 0}, \ldots, f^{0, n-1}) = (f^{n, 0}, \ldots, f^{1, n-1}, (-1)^n d''^{n-1} \circ f^{0, n-1}) \in \Im \tilde{f}_W^{n-1}\]
  となる.従って,
  \[ (f^{n, 0}, \ldots, f^{0,n}) + \Im \tilde{f}_W^{n-1} = (0, \ldots, 0, f^{0, n} - (-1)^n d''^{n-1} \circ f^{0, n-1}) + \Im \tilde{f}_W^{n-1}. \]
  さらに,追加補題\ref{lem1.4}から$f^{0, n} - (-1)^n d''^{n-1} \circ f^{0, n-1} = a^n \circ \varepsilon'$ならしめる$a^n \in \ker (d'')^\# \subset \Hom_R(A, Y^n)$が存在するので,
  \[ \varphi^{n\ast}(a^n + \Im (d''^{n-1})^\#) = (0, \ldots, 0, a^n \circ \varepsilon') + \Im \tilde{f}_W^{n-1} = (f^{n, 0}, \ldots, f^{1, n-1}, f^{0,n}) + \Im \tilde{f}_W^{n-1}. \]

  $\varphi^{n\ast}$が単射であることを証明する.
  $a^n \in \Hom_R(A, Y^n)$が$\varphi^{n\ast}(a^n + \Im (d''^{n-1})^\#) = 0$とする.
  これは$\varphi^n(a^n) \in \Im \tilde{f}_W^{n-1}$と同値.すなわち,$f^{p, q} \in \Hom_R(X_p, Y^q)~(p+q=n-1)$が存在し,
  \begin{align}
    \begin{split}
      & 0 = f^{n-1, 0} \circ \partial_n' \\
      & \qquad \vdots \\
      & 0 = f^{n-k-1, k} \circ \partial_{n-k}' + (-1)^n d''^{k-1} \circ f^{n-k, k-1} \\
      & \qquad \vdots \\
      & a^n \circ \varepsilon' = (-1)^n d''^{n-1} \circ f^{0, n-1}
    \end{split}
    \label{exm2_10_eq_4}
  \end{align}
  となる.$a^n + \Im (d''^{n-1})^\# \in \ker \varphi^{n\ast}$と仮定したので,$a^n + \Im (d''^{n-1})^\# = 0$,すなわち$a^n \in \Im (d''^{n-1})^\# \subset \Hom_R(A, Y^n)$を示せばよい.

  \eqref{exm2_10_eq_4}1番目の式から$0 = {}^\# (\partial_n')(f^{n-1, 0})$なので,$a^n \in \ker ^\# (\partial_n') = \Im ^\# (\partial_{n-1}')$.
  従って,$g^{n-2, 0} \in \Hom_R(X_{n-2}, Y^0)$によって$f^{n-1, 0} = {}^\# (\partial_{n-1}')(g^{n-2, 0}) = g^{n-2, 0} \circ \partial_{n-1}'$と表せる.

  \eqref{exm2_10_eq_4}で$k=1$として,
  \begin{align*}
    0 & = f^{n-2, 1} \circ \partial_{n-1}' + (-1)^n d''^{0} \circ f^{n-1, 0} \\
    & = f^{n-2, 1} \circ \partial_{n-1}' + (-1)^n d''^{0} \circ g^{n-2, 0} \circ \partial_{n-1}' \\
    & = {}^\#(\partial_{n-1}') \left( f^{n-2, 1} + (-1)^n d''^{0} \circ g^{n-2, 0} \right).
  \end{align*}
  従って,$f^{n-2, 1} + (-1)^n d''^{0} \circ g^{n-2, 0} \in \ker ^\#(\partial_{n-1}') = \Im ^\#(\partial_{n-2}')$なので,$g^{n-3, 1} \in \Hom_R(X_{n-3}, Y^1)$によって,
  \[ f^{n-2, 1} + (-1)^n d''^{0} \circ g^{n-2, 0} = {}^\#(\partial_{n-2}')(g^{n-3, 1}) = g^{n-3, 1} \circ \partial_{n-2}' \]
  となる.よって,
  \[ f^{n-2, 1} = g^{n-3, 1} \circ \partial_{n-2}' - (-1)^n d''^{0} \circ g^{n-2, 0}. \]

  \eqref{exm2_10_eq_4}で$k=2$として,
  \begin{align*}
    0 & = f^{n-3, 2} \circ \partial_{n-2}' + (-1)^n d''^{1} \circ f^{n-2, 1} \\
    & = f^{n-3, 2} \circ \partial_{n-2}' + (-1)^n d''^{1} \circ g^{n-3, 1} \circ \partial_{n-2}' \\
    & = {}^\#(\partial_{n-2}') \left( f^{n-3, 2} + (-1)^n d''^{1} \circ g^{n-3, 1} \right).
  \end{align*}
  従って,$f^{n-3, 2} + (-1)^n d''^{1} \circ g^{n-3, 1} \in \ker ^\#(\partial_{n-2}') = \Im ^\#(\partial_{n-3}')$なので,$g^{n-4, 2} \in \Hom_R(X_{n-4}, Y^2)$によって,
  \[ f^{n-3, 2} + (-1)^n d''^{1} \circ g^{n-3, 1} = {}^\#(\partial_{n-3}')(g^{n-4, 2}) = g^{n-4, 2} \circ \partial_{n-3}' \]
  となる.よって,
  \[ f^{n-3, 2} = g^{n-4, 2} \circ \partial_{n-3}' - (-1)^n d''^{1} \circ g^{n-3, 1}. \]

  以降同様に繰り返して,$k=n-2$から$f^{1, n-2} = g^{0, n-2} \circ \partial_1' - (-1)^n d''^{n-3} \circ g^{1, n-3}$が得られる.

  \eqref{exm2_10_eq_4}で$k=n-1$として,
  \begin{align*}
    0 & = f^{0, n-1} \circ \partial_{1}' + (-1)^n d''^{n-2} \circ f^{1, n-2} \\
    & = f^{0, n-1} \circ \partial_{1}' + (-1)^n d''^{n-2} \circ g^{0, n-2} \circ \partial_1' \\
    & = {}^\#(\partial_1') \left( f^{0, n-1} + (-1)^n d''^{n-2} \circ g^{0, n-2} \right).
  \end{align*}
  従って,$f^{0, n-1} + (-1)^n d''^{n-2} \circ g^{0, n-2} \in \ker ^\#(\partial_{1}') = \Im ^\#(\varepsilon')$なので,$a^{n-1} \in \Hom_R(A, Y^{n-1})$によって,
  \[ f^{0, n-1} + (-1)^n d''^{n-2} \circ g^{0, n-2} = {}^\#(\varepsilon')(a^{n-1}) = a^{n-1} \circ \varepsilon' \]
  となる.よって,
  \[ f^{0, n-1} = a^{n-1} \circ \varepsilon' - (-1)^n d''^{n-2} \circ g^{0, n-2}. \]

  \eqref{exm2_10_eq_4}最後の式から
  \[ a^n \circ \varepsilon' = (-1)^n d''^{n-1} \circ f^{0, n-1} = (-1)^n d''^{n-1} \circ a^{n-1} \circ \varepsilon' \]
  なので,$0 = {}^\# (\varepsilon') \left( a^n - (-1)^n d''^{n-1} \circ a^{n-1} \right)$である.
  $^\# (\varepsilon') \colon \Hom_R(A, Y^n) \to \Hom_R(X_0, Y^n)$は単射なので,$a^n = (-1)^n d''^{n-1} \circ a^{n-1}$.
  すなわち$a^n \in \Im (d''^{n-1})^\# $.
\end{proof}

\section{係数加群を持つホモロジー}
\begin{screen}
  $A$が射影的$R$加群なら$H^n (\Hom_R(A, \boldsymbol{X})) \simeq \Hom_R{}(A, H^n(\boldsymbol{X}))$(p.94).
\end{screen}
\begin{proof}
  $R$加群の完全系列
  \[
  \begin{CD}
    0 @>>> B^n @>{i}>> Z^n @>{p}>> Z^n/B^n @>>> 0 \\
  \end{CD}
  \]
  に対し,
  \[
  \begin{CD}
    0 @>>> \Hom_R (A, B^n) @>{i^\#}>> \Hom_R (A, Z^n) @>{p^\#}>> \Hom_R (A, Z^n/B^n) @>>> 0
  \end{CD}
  \]
  は完全系列となる(定理1.3, p.37).従って,
  \begin{align}
    \Hom_R (A, Z^n/B^n) \simeq \Hom_R (A, Z^n) / \ker p^\# = \Hom_R (A, Z^n) / i^\# \Hom_R (A, B^n). \label{HnHom_HomHn_eq1}
  \end{align}
  % <!-- well-definedな全射
  % \[ X^n/B^n \ni x^n + B^n \mapsto d^n(x^n) \in B^{n+1}\]
  % によって
  % \[
  % \begin{CD}
  %   0 @>>> Z^n/B^n @>>> X^n/B^n @>>> B^{n+1} @>>> 0
  % \end{CD}
  % \]
  % および
  % \[
  % \begin{CD}
  %   0 @>>> \Hom_R (A, Z^n/B^n) @>>> \Hom_R (A, X^n/B^n) @>>> \Hom_R (A, B^{n+1}) @>>> 0
  % \end{CD}
  % \]
  % は完全系列となる(定理1.3, p.37). -->
  $\Hom_R (A, \boldsymbol{X}) = \{ \Hom_R (A, X^n), (d^n)^\# \}$において,
  \[ \ker (d^n)^\# = \left\{\, f^n \in \Hom_R (A, X^n) \mid d^n \circ f^n = 0 \,\right\} \simeq \Hom_R (A, Z^n)\]
  である.実際,$f^n \in \ker (d^n)^\#$とすれば,$d^n \circ f^n = 0$なので$\Im f^n \subset \ker d^n = Z^n$.
  従って,値域を制限することで$\ker (d^n)^\#$の元から一意に$\Hom_R (A, Z^n)$の元が定まる.
  $\Hom_R (A, Z^n)$の任意の元に対応する$\ker (d^n)^\#$の元が存在するので,$ \ker (d^n)^\# \simeq \Hom_R (A, Z^n)$.

  また,
  \[ \Im (d^{n-1})^\# = \left\{\, d^{n-1} \circ f^{n-1} \mid f^{n-1} \in \Hom_R (A, X^{n-1}) \,\right\} \simeq \Hom_R (A, B^n) \]
  である.実際,次の図式:
  % \[
  % \begin{CD}
  %   A @= A \\
  %   @V{f^{n-1}}VV @VV{f^n}V \\
  %   X^{n-1} @>>{d^{n-1}}> B^n @>>> 0
  % \end{CD}
  % \]
  \[
  \begin{tikzcd}
    & A \arrow[ld, "f^{n-1}"']\arrow[rd, "f^n"] \\[8pt]
    X^{n-1} \arrow[rr, "d^{n-1}"] &[-20pt] & B^n \arrow[r] & 0
  \end{tikzcd}
  \]
  において$A$は射影的であることから従う.
  % \[
  % \begin{CD}
  %   \Im (d^{n-1})^\# @>{\subset}>> \ker (d^n)^\# \\
  %   @V{\simeq}VV @VV{\simeq}V \\
  %   \Hom_R (A, B^n) @>>{i^\#}> \Hom_R (A, Z^n)
  % \end{CD}
  % \]
  \[
  \begin{tikzcd}
    \Im (d^{n-1})^\# \arrow[r, phantom, "\subset"]\arrow[d, "\simeq"] & \ker (d^n)^\# \arrow[d, "\simeq"] \\
    \Hom_R (A, B^n) \arrow[r, "i^\#"] & \Hom_R (A, Z^n)
  \end{tikzcd}
  \]
  \eqref{HnHom_HomHn_eq1}と併せて
  \[ H^n (\Hom_R(A, \boldsymbol{X})) = \ker (d^n)^\# / \Im (d^{n-1})^\# \simeq \Hom_R (A, Z^n) / i^\# \Hom_R (A, B^n) \simeq \Hom_R (A, Z^n/B^n). \]
\end{proof}

\paragraph{例2.21}~
\begin{screen}
  $H_n(\varinjlim \boldsymbol{X}_\nu) = \varinjlim H_n(\boldsymbol{X}_\nu)$
\end{screen}
\begin{proof}
  例1.13(iii)から$\tilde\partial_n = \varinjlim \partial_{\nu, n} \colon \varinjlim X_{\nu, n} \to \varinjlim X_{\nu, n-1}$が存在し,以下の図式が可換となる.
  \[
  \begin{CD}
    X_{\lambda, n} @>{f_{\mu\lambda, n}}>> X_{\mu, n} @>{i_{\mu, n}}>> \bigoplus_{\lambda \in \Lambda} X_{\lambda, n} @>{\tilde{p}_n}>> \varinjlim X_{\nu, n} \\
    @V{\partial_{\lambda, n}}VV  @V{\partial_{\mu, n}}VV  @V{\oplus \partial_{\lambda, n}}VV  @VV{\tilde{\partial}_n}V \\
    X_{\lambda, n-1} @>>{f_{\mu\lambda, n-1}}> X_{\mu, n-1} @>>{i_{\mu, n-1}}> \bigoplus_{\lambda \in \Lambda} X_{\lambda, n-1} @>>{\tilde{p}_{n-1}}> \varinjlim X_{\nu, n-1}
  \end{CD}
  \]
  まず,$\varinjlim \boldsymbol{X}_\nu = \{ \varinjlim X_{\nu, n}, \tilde\partial_n \}$が鎖複体であること,すなわち$\tilde\partial_{n-1} \circ \tilde\partial_n = 0$を証明する.
  \[
  \begin{CD}
    X_{\lambda, n}  @>{\tilde{p}_n \circ i_{\lambda, n}}>> \varinjlim X_{\nu, n} \\
    @V{\partial_{\lambda, n}}VV  @VV{\tilde\partial_n}V \\
    X_{\lambda, n-1}  @>{\tilde{p}_{n-1} \circ i_{\lambda, n-1}}>> \varinjlim X_{\nu, n-1} \\
    @V{\partial_{\lambda, n-1}}VV  @VV{\tilde\partial_{n-1}}V \\
    X_{\lambda, n-2}  @>>{\tilde{p}_{n-2} \circ i_{\lambda, n-2}}> \varinjlim X_{\nu, n-2}
  \end{CD}
  \]
  $\tilde{x}_n \in \varinjlim X_{\nu, n}$に対し,$\tilde{x}_n = \tilde{p}_n \circ i_{\lambda, n}(x_{\lambda, n})$となる$x_{\lambda, n} \in X_{\lambda, n}$が存在する.
  従って,
  \[ \tilde\partial_{n-1} \circ \tilde\partial_n (\tilde{x}_n) = \tilde\partial_{n-1} \circ \tilde\partial_n \circ \tilde{p}_n \circ i_{\lambda, n}(x_{\lambda, n}) = \tilde{p}_{n-2} \circ i_{\lambda, n-2} \circ \partial_{\lambda, n-1} \circ \partial_{\lambda, n}(x_{\lambda, n}) = 0. \]
  例1.13(iv)から
  \begin{align}
    H_n (\varinjlim \boldsymbol{X}_\nu) = \ker \tilde\partial_n / \Im \tilde\partial_{n+1} = \varinjlim (\ker \partial_{\lambda, n}) / \varinjlim (\Im \partial_{\lambda, n+1}). \label{Hnlim_limHn_eq1}
  \end{align}
  さらに,各行が完全な可換図式
  \[
  \begin{CD}
    0 @>>> \Im \partial_{\lambda, n+1} @>>> \ker \partial_{\lambda, n} @>>> H_n (\boldsymbol{X}_\lambda) @>>> 0 \\
    @. @VVV  @VVV  @VVV \\
    0 @>>> \Im \partial_{\mu, n+1} @>>> \ker \partial_{\mu, n} @>>> H_n (\boldsymbol{X}_\mu) @>>> 0
  \end{CD}
  \]
  に対し例1.13(v)を適用して,完全系列
  \[
  \begin{CD}
    0 @>>> \varinjlim (\Im \partial_{\lambda, n+1}) @>>> \varinjlim (\ker \partial_{\lambda, n}) @>>> \varinjlim (H_n(\boldsymbol{X}_\lambda)) @>>> 0
  \end{CD}
  \]
  が得られる.従って\eqref{Hnlim_limHn_eq1}から
  \[ \varinjlim (H_n(\boldsymbol{X}_\lambda)) \simeq \varinjlim (\ker \partial_{\lambda, n}) / \varinjlim (\Im \partial_{\lambda, n+1}) = H_n (\varinjlim \boldsymbol{X}_\nu). \]
\end{proof}
