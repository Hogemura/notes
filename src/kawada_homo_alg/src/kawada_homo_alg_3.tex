\chapter{$\Tor$と$\Ext$}
\setcounter{section}{1}
\section{$\Tor$}
\paragraph{定理3.14}~
\begin{screen}
  (i)の証明
\end{screen}
\begin{proof}
  % <!-- f_n -->
  $B_1$の射影的分解を$(\boldsymbol{Y}, B_1, \varepsilon)$とする.
  \[
  \begin{CD}
    Y_n @>{\partial_n}>> Y_{n-1} @>{\partial_{n-1}}>> \cdots @>{\partial_1}>> Y_0 @>{\varepsilon}>> B_1 @>>> 0
  \end{CD}
  \]
  $f \colon A_1 \to A_2$に対し,$f_n = f \otimes 1 \colon A_1 \otimes_R Y_n \to A_2 \otimes_R Y_n$とする.
  $\boldsymbol{f}$は鎖準同型である.すなわち,次の可換図式が得られる.
  \[
  \begin{CD}
    A_1 \otimes_R Y_n @>{1 \otimes \partial_n}>> A_1 \otimes_R Y_{n-1} \\
    @VV{f_n}V @VV{f_{n-1}}V \\
    A_2 \otimes_R Y_n @>{1 \otimes \partial_n}>> A_2 \otimes_R Y_{n-1}
  \end{CD}
  \]
  定理2.1から次数つきホモロジー加群の準同型$\boldsymbol{f}^\ast$が得られる:
  \begin{align}
    f_n^{\ast} \colon H_n(A_1 \otimes_R \boldsymbol{Y}) \ni a_1 \otimes y_n + \Im (1 \otimes \partial_{n+1}) \mapsto f(a_1) \otimes y_n + \Im (1 \otimes \partial_{n+1}) \in H_n(A_2 \otimes_R \boldsymbol{Y}). \label{thm_3_14_eq_1}
  \end{align}

  % <!-- f_n' -->
  $B_2$の射影的分解を$(\boldsymbol{Y}', B_2, \varepsilon')$とする.
  \[
  \begin{CD}
    Y_n' @>{\partial_n'}>> Y_{n-1}' @>{\partial_{n-1}'}>> \cdots @>{\partial_1'}>> Y_0' @>{\varepsilon'}>> B_2 @>>> 0
  \end{CD}
  \]
  $f \colon A_1 \to A_2$に対し,$f_n' = f \otimes 1 \colon A_1 \otimes_R Y_n' \to A_2 \otimes_R Y_n'$とする.
  $\boldsymbol{f}'$は鎖準同型である.すなわち,次の可換図式が得られる.
  \[
  \begin{CD}
    A_1 \otimes_R Y_n' @>{1 \otimes \partial_n'}>> A_1 \otimes_R Y_{n-1}' \\
    @VV{f_n'}V @VV{f_{n-1}'}V \\
    A_2 \otimes_R Y_n' @>{1 \otimes \partial_n'}>> A_2 \otimes_R Y_{n-1}'
  \end{CD}
  \]
  定理2.1から次数つきホモロジー加群の準同型$\boldsymbol{f}'^\ast$が得られる:
  \begin{align}
    f_n'^{\ast} \colon H_n(A_1 \otimes_R \boldsymbol{Y}') \ni a_1 \otimes y_n' + \Im (1 \otimes \partial_{n+1}') \mapsto f(a_1) \otimes y_n' + \Im (1 \otimes \partial_{n+1}') \in H_n(A_2 \otimes_R \boldsymbol{Y}'). \label{thm_3_14_eq_2}
  \end{align}

  % <!-- phi_n -->
  $\varphi \colon B_1 \to B_2$に対し,定理3.2から$\varphi_n \colon Y_n \to Y_n'$が得られ,次の図式が可換となる.
  \[
  \begin{CD}
    Y_n @>{\partial_n}>> Y_{n-1} \\
    @VV{\varphi_n}V @VV{\varphi_{n-1}}V \\
    Y_n' @>{\partial_n'}>> Y_{n-1}'
  \end{CD}
  \]
  従って,次の図式が可換となる.
  \[
  \begin{CD}
    A_1 \otimes_R Y_n @>{1 \otimes \partial_n}>> A_1 \otimes_R Y_{n-1} \\
    @VV{1 \otimes \varphi_n}V @VV{1 \otimes \varphi_{n-1}}V \\
    A_1 \otimes_R Y_n' @>{1 \otimes \partial_n'}>> A_1 \otimes_R Y_{n-1}'
  \end{CD}
  \]
  定理2.1から次数つきホモロジー加群の準同型$\boldsymbol{\varphi}^\ast$が得られる:
  \begin{align}
    \varphi_n^{\ast} \colon H_n(A_1 \otimes_R \boldsymbol{Y}) \ni a_1 \otimes y_n + \Im (1 \otimes \partial_{n+1}) \mapsto a_1 \otimes \varphi_n(y_n) + \Im (1 \otimes \partial_{n+1}') \in H_n(A_1 \otimes_R \boldsymbol{Y}'). \label{thm_3_14_eq_3}
  \end{align}

  % <!-- phi_n' -->
  同様にして,次の図式が可換となる.
  \[
  \begin{CD}
    A_2 \otimes_R Y_n @>{1 \otimes \partial_n}>> A_2 \otimes_R Y_{n-1} \\
    @VV{1 \otimes \varphi_n}V @VV{1 \otimes \varphi_{n-1}}V \\
    A_2 \otimes_R Y_n' @>{1 \otimes \partial_n'}>> A_2 \otimes_R Y_{n-1}'
  \end{CD}
  \]
  定理2.1から次数つきホモロジー加群の準同型$\boldsymbol{\varphi}'^\ast$が得られる:
  \begin{align}
    \varphi_n'^{\ast} \colon H_n(A_2 \otimes_R \boldsymbol{Y}) \ni a_2 \otimes y_n + \Im (1 \otimes \partial_{n+1}) \mapsto a_2 \otimes \varphi_n(y_n) + \Im (1 \otimes \partial_{n+1}') \in H_n(A_2 \otimes_R \boldsymbol{Y}'). \label{thm_3_14_eq_4}
  \end{align}
  \eqref{thm_3_14_eq_1}\eqref{thm_3_14_eq_2}\eqref{thm_3_14_eq_3}\eqref{thm_3_14_eq_4}から次の可換図式が得られる.
  \[
  \begin{CD}
    H_n(A_1 \otimes_R \boldsymbol{Y}) @>{f_n^\ast}>> H_n(A_2 \otimes_R \boldsymbol{Y}) \\
    @VV{\varphi_n^\ast}V @VV{\varphi_n'^\ast}V \\
    H_n(A_1 \otimes_R \boldsymbol{Y}') @>{f_n'^\ast}>> H_n(A_2 \otimes_R \boldsymbol{Y}')
  \end{CD}
  \]
  すなわち次の可換図式が得られる.
  \[
  \begin{CD}
    \Tor_n{}^R(A_1, B_1) @>{f_n^\ast}>> \Tor_n{}^R(A_2, B_1) \\
    @VV{\varphi_n^\ast}V @VV{\varphi_n'^\ast}V \\
    \Tor_n{}^R(A_1, B_2) @>{f_n'^\ast}>> \Tor_n{}^R(A_2, B_2)
  \end{CD}
  \]
\end{proof}

\begin{screen}
  (ii)の証明
\end{screen}
\begin{proof}
  完全系列
  \[
  \begin{CD}
    0 @>>> B_1 @>{\varphi}>> B_2 @>{\psi}>> B_3 @>>> 0
  \end{CD}
  \]
  に対し,次の可換図式が得られる(定理3.4).
  \[
  \begin{CD}
    0 @. 0 @. @. 0 @. 0 \\
    @VVV @VVV @. @VVV @VVV \\
    Y_n @>{\partial_n}>> Y_{n-1} @>{\partial_{n-1}}>> \cdots @>{\partial_1}>> Y_0 @>{\varepsilon}>> B_1 @>>> 0 \\
    @VV{\varphi_n}V @VV{\varphi_{n-1}}V @. @VV{\varphi_0}V @VV{\varphi}V \\
    Y_n' @>{\partial_n'}>> Y_{n-1}' @>{\partial_{n-1}'}>> \cdots @>{\partial_1'}>> Y_0' @>{\varepsilon'}>> B_2 @>>> 0 \\
    @VV{\psi_n}V @VV{\psi_{n-1}}V @. @VV{\psi_0}V @VV{\psi}V \\
    Y_n @>{\partial_n''}>> Y_{n-1}'' @>{\partial_{n-1}''}>> \cdots @>{\partial_1''}>> Y_0 @>{\varepsilon''}>> B_3 @>>> 0 \\
    @VVV @VVV @. @VVV @VVV \\
    0 @. 0 @. @. 0 @. 0
  \end{CD}
  \]
  上の図式で
  \[
  \begin{CD}
    0 @>>> Y_n @>{\varphi_n}>> Y_n' @>{\psi_n}>> Y_n'' @>>> 0
  \end{CD}
  \]
  は分裂しているので,$f \colon A_1 \to A_2$によって,次の可換図式が得られる.
  \[
  \begin{CD}
    0 @>>> A_1 \otimes_R Y_n @>{1 \otimes \varphi_n}>> A_1 \otimes_R Y_n' @>{1 \otimes \psi_n}>> A_1 \otimes_R Y_n'' @>>> 0 \\
    @. @VV{f \otimes 1}V @VV{f \otimes 1}V @VV{f \otimes 1}V \\
    0 @>>> A_2 \otimes_R Y_n @>{1 \otimes \varphi_n}>> A_2 \otimes_R Y_n' @>{1 \otimes \psi_n}>> A_2 \otimes_R Y_n'' @>>> 0
  \end{CD}
  \]
  上の図式の写像はいずれも鎖準同型である.ここで,
  \begin{align*}
    f_n^{\ast} & \colon H_n(A_1 \otimes_R \boldsymbol{Y}) \ni a_1 \otimes y_n + \Im (1 \otimes \partial_{n+1}) \mapsto f(a_1) \otimes y_n + \Im (1 \otimes \partial_{n+1}) \in H_n(A_2 \otimes_R \boldsymbol{Y}) \\
    f_n'^{\ast} & \colon H_n(A_1 \otimes_R \boldsymbol{Y}') \ni a_1 \otimes y_n' + \Im (1 \otimes \partial_{n+1}') \mapsto f(a_1) \otimes y_n' + \Im (1 \otimes \partial_{n+1}') \in H_n(A_2 \otimes_R \boldsymbol{Y}') \\
    f_n''^{\ast} & \colon H_n(A_1 \otimes_R \boldsymbol{Y}'') \ni a_1 \otimes y_n'' + \Im (1 \otimes \partial_{n+1}'') \mapsto f(a_1) \otimes y_n'' + \Im (1 \otimes \partial_{n+1}'') \in H_n(A_2 \otimes_R \boldsymbol{Y}'')
  \end{align*}
  とすれば,定理2.4から連結準同型$\delta_n \colon H_n(A_1 \otimes_R \boldsymbol{Y}'') \to H_{n-1}(A_1 \otimes_R \boldsymbol{Y})$及び$\delta_n' \colon H_n(A_2 \otimes_R \boldsymbol{Y}'') \to H_{n-1}(A_2 \otimes_R \boldsymbol{Y})$が存在して,次の図式が可換となる.
  \[
  \begin{CD}
    H_n(A_1 \otimes_R \boldsymbol{Y}'') @>{f_n''^{\ast}}>> H_n(A_2 \otimes_R \boldsymbol{Y}'') \\
    @VV{\delta_n}V @VV{\delta_n'}V \\
    H_{n-1}(A_1 \otimes_R \boldsymbol{Y}) @>{f_{n-1}^{\ast}}>> H_{n-1}(A_2 \otimes_R \boldsymbol{Y})
  \end{CD}
  \]
  $\Tor$に換言すれば次のようになる.
  \[
  \begin{CD}
    \Tor_n{}^R (A_1, B_3) @>{f_n''^{\ast}}>> \Tor_n{}^R (A_2, B_3) \\
    @VV{\delta_n}V @VV{\delta_n'}V \\
    \Tor_{n-1}{}^R (A_1, B_1) @>{f_{n-1}^{\ast}}>> \Tor_{n-1}{}^R (A_2, B_1) \\
  \end{CD}
  \]
\end{proof}

\begin{screen}
  (iii)の証明
\end{screen}
\begin{proof}
  $\varphi \colon B_1 \to B_2$に対し,次の可換図式が得られる(定理3.2).
  \[
  \begin{CD}
    Y_n @>{\partial_n}>> Y_{n-1} @>{\partial_{n-1}}>> \cdots @>{\partial_1}>> Y_0 @>{\varepsilon}>> B_1 @>>> 0 \\
    @VV{\varphi_n}V @VV{\varphi_{n-1}}V @. @VV{\varphi_0}V @VV{\varphi}V \\
    Y_n' @>{\partial_n'}>> Y_{n-1}' @>{\partial_{n-1}'}>> \cdots @>{\partial_1'}>> Y_0' @>{\varepsilon'}>> B_2 @>>> 0
  \end{CD}
  \]
  完全系列
  \[
  \begin{CD}
    0 @>>> A_1 @>{f}>> A_2 @>{g}>> A_3 @>>> 0
  \end{CD}
  \]
  に対し,$Y_n$, $Y_n'$は射影的,従って平坦である(補題1.7系)から,次の可換図式が得られる.
  \[
  \begin{CD}
    0 @>>> A_1 \otimes_R Y_n @>{f \otimes 1}>> A_2 \otimes_R Y_n @>{g \otimes 1}>> A_3 \otimes_R Y_n @>>> 0 \\
    @. @VV{1 \otimes \varphi_n}V @VV{1 \otimes \varphi_n}V @VV{1 \otimes \varphi_n}V \\
    0 @>>> A_1 \otimes_R Y_n' @>{f \otimes 1}>> A_2 \otimes_R Y_n' @>{g \otimes 1}>> A_3 \otimes_R Y_n' @>>> 0
  \end{CD}
  \]
  上の図式の写像はいずれも鎖準同型である.ここで,
  \begin{align*}
    \varphi_n^{\ast} & \colon H_n(A_1 \otimes_R \boldsymbol{Y}) \ni a_1 \otimes y_n + \Im (1 \otimes \partial_{n+1}) \mapsto a_1 \otimes \varphi_n(y_n) + \Im (1 \otimes \partial_{n+1}') \in H_n(A_1 \otimes_R \boldsymbol{Y}') \\
    \varphi_n'^{\ast} & \colon H_n(A_2 \otimes_R \boldsymbol{Y}) \ni a_2 \otimes y_n + \Im (1 \otimes \partial_{n+1}) \mapsto a_2 \otimes \varphi_n(y_n) + \Im (1 \otimes \partial_{n+1}') \in H_n(A_2 \otimes_R \boldsymbol{Y}') \\
    \varphi_n''^{\ast} & \colon H_n(A_3 \otimes_R \boldsymbol{Y}) \ni a_3 \otimes y_n + \Im (1 \otimes \partial_{n+1}) \mapsto a_3 \otimes \varphi_n(y_n) + \Im (1 \otimes \partial_{n+1}') \in H_n(A_3 \otimes_R \boldsymbol{Y}') \\
  \end{align*}
  とすれば,定理2.4から連結準同型$\delta_n \colon H_n(A_3 \otimes_R \boldsymbol{Y}) \to H_{n-1}(A_1 \otimes_R \boldsymbol{Y})$及び$\delta_n' \colon H_n(A_3 \otimes_R \boldsymbol{Y}') \to H_{n-1}(A_1 \otimes_R \boldsymbol{Y}')$が存在して,次の図式が可換となる.
  \[
  \begin{CD}
    H_n(A_3 \otimes_R \boldsymbol{Y}) @>{\delta_n}>> H_{n-1}(A_1 \otimes_R \boldsymbol{Y}) \\
    @VV{\varphi_n''^\ast}V @VV{\varphi_{n-1}^\ast}V \\
    H_n(A_3 \otimes_R \boldsymbol{Y}') @>{\delta_n'}>> H_{n-1}(A_1 \otimes_R \boldsymbol{Y}')
  \end{CD}
  \]
  $\Tor$に換言すれば次のようになる.
  \[
  \begin{CD}
    \Tor_n{}^R (A_3, B_1) @>{\delta_n}>> \Tor_{n-1}{}^R (A_1, B_1) \\
    @VV{\varphi_n''^\ast}V @VV{\varphi_{n-1}^\ast}V \\
    \Tor_n{}^R(A_3, B_2) @>{\delta_n'}>> \Tor_{n-1}{}^R (A_1, B_2)
  \end{CD}
  \]
\end{proof}

\begin{screen}
  (iv)の証明
\end{screen}
\begin{proof}
  完全系列
  \[
  \begin{CD}
    0 @>>> A_1 @>{f_1}>> A_2 @>{f_2}>> A_3 @>>> 0
  \end{CD}
  \]
  \[
  \begin{CD}
    0 @>>> B_1 @>{\varphi_1}>> B_2 @>{\varphi_2}>> B_3 @>>> 0
  \end{CD}
  \]
  を考える.定理3.4から次の可換図式が得られる.
  \[
  \begin{CD}
    0 @. 0 @. @. 0 @. 0 \\
    @VVV @VVV @. @VVV @VVV \\
    Y_{1, n} @>{\partial_{1, n}}>> Y_{1, n-1} @>{\partial_{1, n-1}}>> \cdots @>{\partial_{1, 1}}>> Y_{1, 0} @>{\varepsilon_1}>> B_1 @>>> 0 \\
    @VV{\varphi_{1, n}}V @VV{\varphi_{1, n-1}}V @. @VV{\varphi_{1, 0}}V @VV{\varphi_1}V \\
    Y_{2, n} @>{\partial_{2, n}}>> Y_{2, n-1} @>{\partial_{2, n-1}}>> \cdots @>{\partial_{2, 1}}>> Y_{2, 0} @>{\varepsilon_1}>> B_2 @>>> 0 \\
    @VV{\varphi_{2, n}}V @VV{\varphi_{2, n-1}}V @. @VV{\varphi_{2, 0}}V @VV{\varphi_2}V \\
    Y_{3, n} @>{\partial_{3, n}}>> Y_{3, n-1} @>{\partial_{3, n-1}}>> \cdots @>{\partial_{3, 1}}>> Y_{3, 0} @>{\varepsilon_1}>> B_3 @>>> 0 \\
    @VVV @VVV @. @VVV @VVV \\
    0 @. 0 @. @. 0 @. 0
  \end{CD}
  \]
  さらに,完全系列
  \[
  \begin{CD}
    0 @>>> Y_{1, n} @>{\varphi_{1, n}}>> Y_{2, n} @>{\varphi_{2, n}}>> Y_{3, n} @>>> 0
  \end{CD}
  \]
  は分裂している.以上から,次の可換図式が得られる.
  \[
  \begin{CD}
    @. 0 @. 0 @. 0 \\
    @. @VVV @VVV @VVV \\
    0 @>>> A_1 \otimes_R Y_{1, n} @>{1 \otimes \varphi_{1, n}}>> A_1 \otimes_R Y_{2, n} @>{1 \otimes \varphi_{2, n}}>> A_1 \otimes_R Y_{3, n} @>>> 0 \\
    @. @VV{f_1 \otimes 1}V @VV{f_1 \otimes 1}V @VV{f_1 \otimes 1}V \\
    0 @>>> A_2 \otimes_R Y_{1, n} @>{1 \otimes \varphi_{1, n}}>> A_2 \otimes_R Y_{2, n} @>{1 \otimes \varphi_{2, n}}>> A_2 \otimes_R Y_{3, n} @>>> 0 \\
    @. @VV{f_2 \otimes 1}V @VV{f_2 \otimes 1}V @VV{f_2 \otimes 1}V \\
    0 @>>> A_3 \otimes_R Y_{1, n} @>{1 \otimes \varphi_{1, n}}>> A_3 \otimes_R Y_{2, n} @>{1 \otimes \varphi_{2, n}}>> A_3 \otimes_R Y_{3, n} @>>> 0 \\
    @. @VVV @VVV @VVV \\
    @. 0 @. 0 @. 0
  \end{CD}
  \]
  定理1.2,補題1.7系から各行・列は完全である.
  定理2.2の証明から,次の図式を右上から左下に辿れば$\delta_n \circ \delta_{n+1}^\ast \colon H_{n+1}(A_3 \otimes_R \boldsymbol{Y}_3) \to H_{n-1}(A_1 \otimes_R \boldsymbol{Y}_1)$が得られる.
  \[
  \begin{tikzcd}
    & & A_3 \otimes_R Y_{2, n+1} \arrow[r,"1 \otimes \varphi_{2, n+1}"]\arrow[rd, "1 \otimes \partial_{2, n+1}" description] & A_3 \otimes_R Y_{3, n+1} \\[20pt]
    & A_2 \otimes_R Y_{1, n} \arrow[r, "f_2 \otimes 1"]\arrow[rd, "1 \otimes \partial_{1, n}" description] & A_3 \otimes_R Y_{1, n} \arrow[r, "1 \otimes \varphi_{1, n}"] & A_3 \otimes_R Y_{2, n} \\[20pt]
    0 \arrow[r] & A_1 \otimes_R Y_{1, n-1} \arrow[r, "f_1 \otimes 1"] & A_2 \otimes_R Y_{1, n-1}
  \end{tikzcd}
  \]

  同様に,次の図式を右上から左下に辿れば$\delta_{n+1} \circ \delta_n^\ast \colon H_{n+1}(A_3 \otimes_R \boldsymbol{Y}_3) \to H_{n-1}(A_1 \otimes_R \boldsymbol{Y}_1)$が得られる.
  \[
  \begin{tikzcd}
    & & A_3 \otimes_R Y_{2, n+1} \arrow[r,"f_2 \otimes 1"]\arrow[rd, "1 \otimes \partial_{3, n+1}" description] & A_3 \otimes_R Y_{3, n+1} \\[20pt]
    & A_1 \otimes_R Y_{2, n} \arrow[r, "1 \otimes \varphi_{2, n}"]\arrow[rd, "1 \otimes \partial_{2, n}" description] & A_1 \otimes_R Y_{3, n} \arrow[r, "f_1 \otimes 1"] & A_2 \otimes_R Y_{3, n} \\[20pt]
    0 \arrow[r] & A_1 \otimes_R Y_{1, n-1} \arrow[r, "1 \otimes \varphi_{1, n-1}"] & A_1 \otimes_R Y_{2, n-1}
  \end{tikzcd}
  \]

  p.112の可換図式から,次の可換図式が得られる.1行目が$\delta_n \circ \delta_{n+1}^\ast$で3行目が$\delta_{n+1} \circ \delta_n^\ast$である.
  \[
  \begin{CD}
    A_1 \otimes_R Y_{1, n-1} @>{f_1 \otimes 1}>> A_2 \otimes_R Y_{1, n-1} @<{1 \otimes \partial_{1, n}}<< A_2 \otimes_R Y_{1, n} @>{f_2 \otimes \varphi_{1, n}}>> \\
    % A_3 \otimes_R Y_{2, n} @<{1 \otimes \partial_{2, n+1}}<< A_3 \otimes_R Y_{2, n+1} @>{1 \otimes \varphi_{2, n+1}}>> A_3 \otimes_R Y_{3, n+1} \\
    @A{\psi_{1, n-1}}AA @A{\psi_{3, n-1}}AA @A{\psi_{3, n}}AA \\
    % @AA{f_2 \otimes 1}A @AA{f_2 \otimes 1}A @| \\
    A_1 \otimes_R Y_{1, n-1} @>{g_{n, 1}}>> Z_{n-1} @<{\partial_n}<< Z_n  @>{g_{2,n}}>> \\
    % A_2 \otimes_R Y_{2, n} @<{1 \otimes \partial_{2, n}}<< A_2 \otimes_R Y_{2, n+1} @>{f_2 \otimes \varphi_{2, n}}>> A_3 \otimes_R Y_{3, n+1} \\
    @| @VV{\psi_{2, n-1}}V @VV{\psi_{2, n}}V \\
    % @VV{1 \otimes \varphi_{2, n}}V @VV{1 \otimes \varphi_{2, n+1}}V @| \\
    A_1 \otimes_R Y_{1, n-1} @>{1 \otimes \varphi_{1, n-1}}>> A_1 \otimes_R Y_{2, n-1} @<{1 \otimes \partial_{2, n}}<< A_1 \otimes_R Y_{2, n} @>{f_1 \otimes \varphi_{2, n}}>>
    %  A_2 \otimes_R Y_{3, n} @<{1 \otimes \partial_{3, n+1}}<< A_2 \otimes_R Y_{3, n+1} @>{f_2 \otimes 1}>> A_3 \otimes_R Y_{3, n+1}
  \end{CD}
  \]
  (続き)
  \[
  \begin{CD}
    @>{f_2 \otimes \varphi_{1, n}}>>  A_3 \otimes_R Y_{2, n} @<{1 \otimes \partial_{2, n+1}}<< A_3 \otimes_R Y_{2, n+1} @>{1 \otimes \varphi_{2, n+1}}>> A_3 \otimes_R Y_{3, n+1} \\
    @. @AA{f_2 \otimes 1}A @AA{f_2 \otimes 1}A @| \\
    @>{g_{2,n}}>>  A_2 \otimes_R Y_{2, n} @<{1 \otimes \partial_{2, n}}<< A_2 \otimes_R Y_{2, n+1} @>{f_2 \otimes \varphi_{2, n}}>> A_3 \otimes_R Y_{3, n+1} \\
    @. @VV{1 \otimes \varphi_{2, n}}V @VV{1 \otimes \varphi_{2, n+1}}V @| \\
    @>{f_1 \otimes \varphi_{2, n}}>> A_2 \otimes_R Y_{3, n} @<{1 \otimes \partial_{3, n+1}}<< A_2 \otimes_R Y_{3, n+1} @>{f_2 \otimes 1}>> A_3 \otimes_R Y_{3, n+1}
  \end{CD}
  \]
  % 分割していなやつ
  % \[
  % \begin{CD}
  %   A_1 \otimes_R Y_{1, n-1} @>{f_1 \otimes 1}>> A_2 \otimes_R Y_{1, n-1} @<{1 \otimes \partial_{1, n}}<< A_2 \otimes_R Y_{1, n} @>{f_2 \otimes \varphi_{1, n}}>> A_3 \otimes_R Y_{2, n} @<{1 \otimes \partial_{2, n+1}}<< A_3 \otimes_R Y_{2, n+1} @>{1 \otimes \varphi_{2, n+1}}>> A_3 \otimes_R Y_{3, n+1} \\
  %   @A{\psi_{1, n-1}}AA @A{\psi_{3, n-1}}AA @A{\psi_{3, n}}AA @AA{f_2 \otimes 1}A @AA{f_2 \otimes 1}A @| \\
  %   A_1 \otimes_R Y_{1, n-1} @>{g_{n, 1}}>> Z_{n-1} @<{\partial_n}<< Z_n  @>{g_{2,n}}>> A_2 \otimes_R Y_{2, n} @<{1 \otimes \partial_{2, n}}<< A_2 \otimes_R Y_{2, n+1} @>{f_2 \otimes \varphi_{2, n}}>> A_3 \otimes_R Y_{3, n+1} \\
  %   @| @VV{\psi_{2, n-1}}V @VV{\psi_{2, n}}V @VV{1 \otimes \varphi_{2, n}}V @VV{1 \otimes \varphi_{2, n+1}}V @| \\
  %   A_1 \otimes_R Y_{1, n-1} @>{1 \otimes \varphi_{1, n-1}}>> A_1 \otimes_R Y_{2, n-1} @<{1 \otimes \partial_{2, n}}<< A_1 \otimes_R Y_{2, n} @>{f_1 \otimes \varphi_{2, n}}>> A_2 \otimes_R Y_{3, n} @<{1 \otimes \partial_{3, n+1}}<< A_2 \otimes_R Y_{3, n+1} @>{f_2 \otimes 1}>> A_3 \otimes_R Y_{3, n+1} \\
  % \end{CD}
  % \]
  なお,
  \[ Z_n = A_1 \otimes_R Y_{2, n-1} \oplus A_2 \otimes_R Y_{1, n-1},\quad \partial_n = 1 \otimes \partial_{2, n} \oplus 1 \otimes \partial_{1, n}  \]
  である.
  \begin{align*}
    \delta_{n+1} \delta^\ast (a_3 \otimes y_{3, n+1} + \Im (1 \otimes \partial_{3, n+2})) & = \psi_{1, n-1} \delta_n \circ \delta_{n+1}^\ast (a_3 \otimes y_{3, n+1} + \Im (1 \otimes \partial_{3, n+2})) \\
    & = - \delta_n \circ \delta_{n+1}^\ast (a_3 \otimes y_{3, n+1} + \Im (1 \otimes \partial_{3, n+2}))
  \end{align*}
  となるので,$\delta_n \circ \delta_{n+1}^\ast = -\delta_{n+1} \circ \delta_n^\ast$.
\end{proof}

\paragraph{例3.2}~
\begin{screen}
  $T_n(A, B) \simeq \ker (1_{n-1} \otimes 1)$の構成
\end{screen}
\begin{proof}
  $A$の射影的分解$(\boldsymbol{P}, A, p_0)$を以下のように構成する(定理3.1).
  \begin{align}
    &
    \begin{CD}
      0 @>>> R_0 @>{i_0}>> P_0 @>{p_0}>> A @>>> 0
    \end{CD}
    \notag\\
    &
    \begin{CD}
      0 @>>> R_1 @>{i_1}>> P_1 @>{p_1}>> R_0 @>>> 0
    \end{CD}
    \notag\\
    & \qquad\qquad\qquad\vdots\notag\\
    &
    \begin{CD}
      0 @>>> R_{n-1} @>{i_{n-1}}>> P_{n-1} @>{p_{n-1}}>> R_{n-2} @>>> 0
    \end{CD}
    \notag\\
    &
    \begin{CD}
      0 @>>> R_n @>{i_n}>> P_n @>{p_n}>> R_{n-1} @>>> 0
    \end{CD}
    \label{exm_3_2_eq_0}
  \end{align}
  $\partial_k = i_{k-1} \circ p_k \colon P_{k} \to P_{k-1}$である.
  $0 \to R_{n-1} \to P_{n-1} \to R_{n-2} \to 0$に(iv)を適用して,完全系列
  \begin{align}
    \begin{CD}
      T_1 (P_{n-1}, B) @>>> T_1 (R_{n-2}, B) @>{\tilde{\delta}_{1, n-1}}>> R_{n-1} \otimes_R B @>{i_{n-1} \otimes 1}>> P_{n-1} \otimes_R B
    \end{CD}
    \label{exm_3_2_eq_1}
  \end{align}
  を得る.(ii)から$T_1 (P_{n-1}, B) = 0$なので,同型
  \[
  \begin{CD}
    0 @>>> T_1 (R_{n-2}, B) @>{\tilde{\delta}_{1, n-1}}>> \ker (i_{n-1} \otimes 1) @>>> 0
  \end{CD}
  \]
  を得る.同様に,$0 \to R_{n-2} \to P_{n-2} \to R_{n-3} \to 0$に(iv)を適用して,完全系列
  \[
  \begin{CD}
    T_2 (P_{n-2}, B) @>>> T_2 (R_{n-3}, B) @>{\tilde{\delta}_{2, n-2}}>> T_1 (R_{n-2}, B) @>>> T_1 (P_{n-2}, B)
  \end{CD}
  \]
  を得る.(ii)から$T_2 (P_{n-2}, B) = T_1 (P_{n-2}, B) = 0$なので,同型
  \begin{align}
    \begin{CD}
      0 @>>> T_2 (R_{n-3}, B) @>{\tilde{\delta}_{2, n-2}}>> T_1 (R_{n-2}, B) @>>> 0
    \end{CD}
    \label{exm_3_2_eq_2}
  \end{align}
  を得る.以上を続けて,同型
  \[ \tilde{\delta}_{1, n-1} \circ \tilde{\delta}_{2, n-2} \circ \cdots \circ \tilde{\delta}_{n, 0} \colon T_n (A, B) \simeq \ker (i_{n-1} \otimes 1) \]
  を得る.
\end{proof}

\begin{screen}
  $T_n(A, B) \simeq \ker (1_{n-1} \otimes 1)$の可換性.
\end{screen}
\begin{proof}
  準同型$f \colon A_1 \to A_2$を考える.先と同様に$A_1$, $A_2$の射影的分解$(\boldsymbol{P}, A_1, p_0)$, $(\boldsymbol{P}', A_2, p_0')$を構成する.
  $f_0$, $\phi_0$が存在し次の図式が可換となる.
  \[
  \begin{CD}
    0 @>>> R_0 @>{i_0}>> P_0 @>{p_0}>> A_1 @>>> 0 \\
    @. @VV{\phi_0}V @VV{f_0}V @VV{f}V @. \\
    0 @>>> R_0' @>{i_0'}>> P_0' @>{p_0'}>> A_2 @>>> 0
  \end{CD}
  \]
  同様に次の可換図式を得る.
  \[
  \begin{CD}
    0 @>>> R_1 @>{i_1}>> P_1 @>{p_1}>> R_0 @>>> 0 \\
    @. @VV{\phi_1}V @VV{f_1}V @VV{\phi_0}V @. \\
    0 @>>> R_1' @>{i_1'}>> P_1' @>{p_1'}>> R_0' @>>> 0
  \end{CD}
  \]
  これを繰り返し,次の可換図式を得る.
  \begin{align}
    \begin{CD}
      0 @>>> R_{n-1} @>{i_{n-1}}>> P_{n-1} @>{p_{n-1}}>> R_{n-2} @>>> 0 \\
      @. @VV{\phi_{n-1}}V @VV{f_{n-1}}V @VV{\phi_{n-2}}V @. \\
      0 @>>> R_{n-1}' @>{i_{n-1}'}>> P_{n-1}' @>{p_{n-1}'}>> R_{n-2}' @>>> 0
    \end{CD}
    \label{exm_3_2_eq_3}
  \end{align}
  \eqref{exm_3_2_eq_1}から次の可換図式を得る.
  \[
  \begin{CD}
    0 @>>> T_1 (R_{n-2}, B) @>{\tilde{\delta}_{1, n-1}}>> R_{n-1} \otimes_R B @>{i_{n-1} \otimes 1}>> P_{n-1} \otimes_R B \\
    @. @VV{\exists_1\psi_1}V @VV{\phi_{n-1} \otimes 1}V @VV{f_{n-1} \otimes 1}V \\
    0 @>>> T_1 (R_{n-2}', B) @>{\tilde{\delta}_{1, n-1}'}>> R_{n-1}' \otimes_R B @>{i_{n-1}' \otimes 1}>> P_{n-1}' \otimes_R B
  \end{CD}
  \]
  $\psi_1$が一意に存在することは容易に分かる.\eqref{exm_3_2_eq_2}から次の可換図式を得る.
  \[
  \begin{CD}
    0 @>>> T_2 (R_{n-3}, B) @>{\tilde{\delta}_{2, n-2}}>> T_1 (R_{n-2}, B) @>>> 0 \\
    @. @VV{\exists_1\psi_2}V @VV{\psi_1}V \\
    0 @>>> T_2 (R_{n-3}', B) @>{\tilde{\delta}_{2, n-2}'}>> T_1 (R_{n-2}', B) @>>> 0
  \end{CD}
  \]
  以上を続けて,可換図式
  \[
  \begin{CD}
    T_n (A_1, B) @>{\tilde\delta}>>  \ker (i_{n-1} \otimes 1) \\
    @VV{\psi_n}V @VV{\phi_{n-1} \otimes 1}V \\
    T_n (A_2, B) @>{\tilde\delta'}>>  \ker (i_{n-1}' \otimes 1)
  \end{CD}
  \]
  を得る.
\end{proof}

\begin{screen}
  $\ker (1_{n-1} \otimes 1) \simeq \Tor_n{}^R (A, B)$の構成
\end{screen}
\begin{proof}
  \eqref{exm_3_2_eq_0}のように射影的分解を構成する.
  \[ \Tor_n{}^R (A, B) = H_n(\boldsymbol{P} \otimes_R B) = \ker (\partial_n \otimes 1) / \Im (\partial_{n+1} \otimes 1) \]
  である.テンソル積の右完全性から
  \[
  \begin{CD}
    R_n \otimes_R B @>{i_n \otimes 1}>> P_n \otimes_R B @>{p_n \otimes 1}>> R_{n-1} \otimes_R B @>>> 0
  \end{CD}
  \]
  \[
  \begin{CD}
    R_{n+1} \otimes_R B @>{i_{n+1} \otimes 1}>> P_{n+1} \otimes_R B @>{p_{n+1} \otimes 1}>> R_n \otimes_R B @>>> 0
  \end{CD}
  \]
  であり,$\partial_{n+1} = i_n \circ p_{n+1}$なので,
  \[ \Im (\partial_{n+1} \otimes 1) = \Im ((i_n \otimes 1) \circ (p_{n+1} \otimes 1)) = \Im (i_n \otimes 1) = \ker (p_n \otimes 1) . \]
  また,
  \[ \ker (\partial_n \otimes 1) = \ker ((i_{n-1} \otimes 1) \circ (p_n \otimes 1)) = (p_n \otimes 1)^{-1}(\ker (i_{n-1} \otimes 1)) \]
  となるので,全射準同型$p_n \otimes 1 \colon \ker (\partial_n \otimes 1) \to \ker (i_{n-1} \otimes 1)$を得る.準同型定理から$\ker (\partial_n \otimes 1) / \ker (p_n \otimes 1) \simeq \ker (i_{n-1} \otimes 1)$となる.以上から,同型
  \[ \tilde{p}_n \colon \Tor_n{}^R (A, B) \ni a \otimes b + \Im (\partial_{n+1} \otimes 1) \mapsto p_n(a) \otimes b \in \ker (i_{n-1} \otimes 1) \]
  を得る.
\end{proof}

\begin{screen}
  $\ker (1_{n-1} \otimes 1) \simeq \Tor_n{}^R (A, B)$の可換性
\end{screen}
\begin{proof}
  \eqref{exm_3_2_eq_3}から,$\boldsymbol{f} \otimes 1 \colon \boldsymbol{P} \otimes_R B \to \boldsymbol{P}' \otimes_R B$は鎖準同型である.従って,
  \[ (f_n \otimes 1)^\ast \colon H_n (\boldsymbol{P} \otimes_R B) \ni (a \otimes b) + \Im (\partial_{n+1} \otimes 1) \mapsto f_n(a) \otimes b + \Im (\partial_{n+1}' \otimes 1) \in H_n (\boldsymbol{P} \otimes_R B) \]
  は同型である(定理2.1).\eqref{exm_3_2_eq_3}と同様の式から$p'_n \circ f_n = \phi_{n-1} \circ p_n$なので,次の図式が可換となる.
  \[
  \begin{CD}
    \Tor_n{}^R (A_1, B) @>{\tilde{p}_n}>> \ker (i_{n-1} \otimes 1) \\
    @VV{(f_n \otimes 1)^\ast}V @VV{\phi_{n-1} \otimes 1}V \\
    \Tor_n{}^R (A_2, B) @>{\tilde{p}_n'}>> \ker (i_{n-1}' \otimes 1)
  \end{CD}
  \]
\end{proof}

\paragraph{例3.5(問題2)}~
\subparagraph{(イ)}
\[
\begin{tikzcd}
  0 \arrow[r] & B \arrow[r]\arrow[d, equal] & E \arrow[r] & A \arrow[r] & 0 \\
  0 \arrow[r] & B \arrow[r] & E' \arrow[r]\arrow[u] & A' \arrow[r]\arrow[u, "\gamma"'] & 0
\end{tikzcd}
\]
に対して,定理3.18(iii)(ニ)から
\[
\begin{tikzcd}
  \Hom_R(B, B) \arrow[r, "\delta_E"]\arrow[d, equal] & \Ext_R{}^1(A, B) \arrow[d, "{}^\#\gamma^1"] \\
  \Hom_R(B, B) \arrow[r, "\delta_{E'}"] & \Ext_R{}^1(A', B)
\end{tikzcd}
\]
が可換となるので,$\delta_{E'}(1_{B'}) = \ch(E') = {}^\#\gamma^1 (\ch(E)) = {}^\#\gamma^1 (\delta_E(1_B))$.
よって,次の可換図式を得る.
\[
\begin{tikzcd}
  E(A, B) \arrow[r, "\Phi_\gamma"]\arrow[d, "\ch"] & E(A', B)\arrow[d, "\ch"] \\
  \Ext_R{}^1(A, B) \arrow[r, "^\#\gamma"] & \Ext_R{}^1(A', B)
\end{tikzcd}
\]

\subparagraph{(ロ)}
\[
\begin{tikzcd}
  0 \arrow[r] & B' \arrow[r] & E' \arrow[r] & A \arrow[r]\arrow[d, equal] & 0 \\
  0 \arrow[r] & B \arrow[r]\arrow[u, "\alpha"'] & E \arrow[r]\arrow[u] & A \arrow[r] & 0
\end{tikzcd}
\]
に対して,定理3.18(iii)(ニ)から
\[
\begin{tikzcd}
  \Hom_R(B', B') \arrow[r, "\delta_{E'}"]\arrow[d, "^\#\alpha"] & \Ext_R{}^1(A, B') \arrow[d, equal] \\
  \Hom_R(B, B') \arrow[r, "\delta'"] & \Ext_R{}^1(A, B')
\end{tikzcd}
\]
なので,$\alpha \circ \delta_E(1_B) = \delta' \circ \alpha^\# (1_B) = \delta' \circ \alpha$.

さらに,定理3.18(v)中央左の図式から
\[
\begin{tikzcd}
  \Hom_R(B, B) \arrow[r, "\delta_E"]\arrow[d, "\alpha^\#"] & \Ext_R{}^1(A, B) \arrow[d, "\alpha^\#"] \\
  \Hom_R(B, B') \arrow[r, "\delta'"] & \Ext_R{}^1(A, B')
\end{tikzcd}
\]
なので,$\delta_{E'}(1_{B'}) = \delta' \circ {}^\#\alpha (1_{B'}) = \delta' \circ \alpha$.
以上から,$\ch(E') = \alpha \circ \ch(E)$.
よって,次の可換図式を得る.
\[
\begin{tikzcd}
  E(A, B) \arrow[r, "\Psi_\alpha"]\arrow[d, "\ch"] & E(A, B')\arrow[d, "\ch"] \\
  \Ext_R{}^1(A, B) \arrow[r, "\alpha^\#"] & \Ext_R{}^1(A, B')
\end{tikzcd}
\]

\subparagraph{(ハ)}
$A$の射影的分解を
\[
\begin{tikzcd}
  X_n \arrow[r, "\partial_n"] & X_{n-1} \arrow[r] & \cdots \arrow[r] & X_1 \arrow[r, "\partial_1"] & X_0 \arrow[r] & A \arrow[r] & 0
\end{tikzcd}
\]
$A'$の射影的分解を
\[
\begin{tikzcd}
  X_n' \arrow[r, "\partial_n'"] & X_{n-1} \arrow[r] & \cdots \arrow[r] & X_1' \arrow[r, "\partial_1'"] & X_0' \arrow[r] & A' \arrow[r] & 0
\end{tikzcd}
\]
とする.$A \oplus A'$の射影的分解として$X_n \oplus X_n'$が取れる(定理3.4の証明).

\S2.3から
\begin{align*}
  \Ext_R{}^1(A, B) \oplus \Ext_R{}^1(A', B') &= H^1(\Hom_R(\boldsymbol{X}, B)) \oplus H^1(\Hom_R(\boldsymbol{X}', B') \\
  &\simeq H^1(\Hom_R(\boldsymbol{X}, B) \oplus \Hom_R(\boldsymbol{X}', B'))
\end{align*}
および
\[ \Ext_R{}^1(A \oplus A', B \oplus B') \simeq H^1(\Hom_R(\boldsymbol{X} \oplus \boldsymbol{X}', B \oplus B')) \]
と書ける.

\subparagraph{(ニ)}
(イ)(ロ)から次の図式を得る.

\[
\begin{tikzcd}
  0 \arrow[r] & B \arrow[r] & \Psi_{\nabla_B} \circ \Phi_{\Delta_A}(E \oplus E') \arrow[r] & A \arrow[r]\arrow[d, equal] & 0 \\
  0 \arrow[r] & B \oplus B \arrow[r]\arrow[u, "\nabla_B"']\arrow[d, equal] & \Phi_{\Delta_A}(E \oplus E') \arrow[r]\arrow[u]\arrow[d] & A \arrow[r]\arrow[d, "\Delta_A"] & 0 \\
  0 \arrow[r] & B \oplus B \arrow[r] & E \oplus E' \arrow[r] & A \oplus A \arrow[r] & 0
\end{tikzcd}
\]

解答で構成した$\tilde{E}$は$E+E' = \Psi_{\nabla_B} \circ \Phi_{\Delta_A}(E \oplus E')$と同型である.
さらに,
\[ \ch (E + E') = \ch(\Psi_{\nabla_B} \circ \Phi_{\Delta_A}(E \oplus E')) = \nabla_B{}^\# \circ \ch( \Phi_{\Delta_A}(E \oplus E')) = \nabla_B{}^\# \circ {}^\#\Delta_A \ch(E \oplus E') . \]
\begin{align*}
\ch (E + E') &= \ch(\Psi_{\nabla_B} \circ \Phi_{\Delta_A}(E \oplus E')) = \nabla_B{}^\# \circ \ch( \Phi_{\Delta_A}(E \oplus E')) \\
&= \nabla_B{}^\# \circ {}^\#\Delta_A \ch(E \oplus E') & = \nabla_B \circ \ch(E \oplus E') \circ \Delta_A .
\end{align*}

\begin{screen}
  $\ch \colon E(A, B) \to \Ext_R{}^1 (A, B)$は準同型,すなわち$\ch (E+E') = \ch(E) + \ch(E')$.
\end{screen}
\begin{proof}
  定理3.20の証明と同様にして,次の可換図式を得る.
  \[
  \begin{tikzcd}
    0 \arrow[r] & Q \arrow[r, "\imath"]\arrow[d, "\gamma"] & P \arrow[r, "\pi"]\arrow[d, "\phi"] & A \arrow[r]\arrow[d, equal] & 0 \\
    0 \arrow[r] & B \arrow[r, "f"] & E \arrow[r, "g"] & A \arrow[r] & 0 \\[10pt]
    0 \arrow[r] & Q \arrow[r, "\imath"]\arrow[d, "\gamma'"] & P \arrow[r, "\pi"]\arrow[d, "\phi'"] & A \arrow[r]\arrow[d, equal] & 0 \\
    0 \arrow[r] & B \arrow[r, "f'"] & E' \arrow[r, "g"] & A \arrow[r] & 0
  \end{tikzcd}
  \]
  連結準同型を$\delta \colon \Hom_R(Q, B) \to \Ext_R{}^1(A, B)$とする.
  \[
  \begin{tikzcd}
    \Hom_R(B, B) \arrow[r, "\delta_E"]\arrow[d, "{}^\#\gamma"] & \Ext_R{}^1(A, B) \\
    \Hom_R(Q, B) \arrow[ru, "\delta"'] \\[10pt]
    \Hom_R(B, B) \arrow[r, "\delta_{E'}"]\arrow[d, "{}^\#\gamma'"] & \Ext_R{}^1(A, B) \\
    \Hom_R(Q, B) \arrow[ru, "\delta"']
  \end{tikzcd}
  \]

  \begin{align*}
    \ch(E) = \delta_E(1_B) = \delta \circ {}^\#\gamma (1_B) = \delta \circ \gamma & , & \ch(E') = \delta_{E'}(1_B) = \delta \circ {}^\#\gamma' (1_B) = \delta \circ \gamma' .
  \end{align*}

  $\delta \oplus \delta \colon \Hom_R(Q \oplus Q, B \oplus B) \to \Ext_R{}^1(A \oplus A, B \oplus B)$が連結準同型となる(コホモロジーの連結準同型の構成から分かる).
  従って,次の可換図式を得る.
  \[
  \begin{tikzcd}
    \Hom_R(B \oplus B, B \oplus B) \arrow[r, "\delta_{E \oplus E'}"]\arrow[d, "{}^\#(\gamma \oplus \gamma')"'] & \Ext_R{}^1(A \oplus A, B \oplus B) \\
    \Hom_R(Q \oplus Q, B \oplus B) \arrow[ru, "\delta \oplus \delta"']
  \end{tikzcd}
  \]
  従って,
  \[
  \ch (E \oplus E') = \delta_{E \oplus E'}(1_{B \oplus B}) = (\delta \oplus \delta) \circ {}^\#(\gamma \oplus \gamma') (1_{B \oplus B})
  = (\delta \oplus \delta) \circ (\gamma \oplus \gamma') = (\delta \circ \gamma) \oplus (\delta \circ \gamma') .
  \]

  $a \in A$に対し,
  \begin{align*}
      \ch(E + E')(a) &= \nabla_B \circ \ch(E \oplus E') \circ \Delta_A (a) = \nabla_B \circ ((\delta \circ \gamma) \oplus (\delta \circ \gamma')) (a, a) \\
     &= \nabla_B \circ (\delta \circ \gamma (a) \oplus \delta \circ \gamma' (a)) = \delta \circ \gamma (a) + \delta \circ \gamma (a) \\
     &= \ch(E)(a) + \ch(E')(a)
  \end{align*}
  となるので,$\ch(E + E') = \ch(E) + \ch(E')$
\end{proof}

\section{K\"unnethの定理}
\paragraph{定理3.21}~
\begin{screen}
  $\delta_n = \oplus (\jmath_p \otimes 1)^\ast$(ただし$\jmath_p \colon B_p \hookrightarrow Z_p$)
\end{screen}
\begin{proof}
  $\partial_n' \colon X_n \to X_{n-1}$, $Z_n = \ker \partial_n'$, $B_{n-1} = \Im \partial_n'$および$\partial_n" \colon Y_n \to Y_{n-1}$とする.
  $Y_m$は平坦$R$加群なので,鎖複体の完全系列
  \[
  \begin{tikzcd}
    0 \arrow[r] & Z_n \arrow[r, "\imath_n"]\arrow[d, "\partial_n'"] & X_n \arrow[r, "\partial_n'"]\arrow[d, "\partial_n'"] & B_{n-1} \arrow[r]\arrow[d, "\partial_{n-1}'"] & 0 \\
    0 \arrow[r] & Z_{n-1} \arrow[r, "\imath_{n-1}"] & X_{n-1} \arrow[r, "\partial_{n-1}'"] & B_{n-2} \arrow[r] & 0 \\
  \end{tikzcd}
  \]
  から完全系列
  \[
  \begin{tikzcd}
    0 \arrow[r] & Z_n \otimes_R Y_m \arrow[r, "\imath_n \otimes 1"] & X_n \otimes_R Y_m \arrow[r, "\partial_n' \otimes 1"] & B_{n-1}  \otimes_R Y_m \arrow[r] & 0
  \end{tikzcd}
  \]
  が得られる.
  テンソル積鎖複体は
  \[ (\boldsymbol{Z} \otimes_R \boldsymbol{Y})_n = \bigoplus_{p=0}^n Z_p \otimes_R Y_{n-p}, \quad (\boldsymbol{X} \otimes_R \boldsymbol{Y})_n = \bigoplus_{p=0}^n X_p \otimes_R Y_{n-p}, \quad (\boldsymbol{B} \otimes_R \boldsymbol{Y})_n = \bigoplus_{p=1}^n B_{p-1} \otimes_R Y_{n-p} \]
  で与えられ,$n$次境界作用素は
  \begin{align*}
    \partial_n = \sum_{p=0}^n \left( \partial_p' \otimes 1 + (-1)^p 1 \otimes \partial_{n-p}'' \right)
  \end{align*}
  で与えられる.$(\boldsymbol{Z} \otimes_R \boldsymbol{Y})_n$, $(\boldsymbol{B} \otimes_R \boldsymbol{Y})_n$では第1項は$0$である.
  以上から,テンソル積鎖複体の完全系列
  \[
  \begin{tikzcd}
    0 \arrow[r]
    & (\boldsymbol{Z} \otimes_R \boldsymbol{Y})_n \arrow[r, "\oplus(\imath_p \otimes 1)"]\arrow[d, "\partial_n"]
    & (\boldsymbol{X} \otimes_R \boldsymbol{Y})_n \arrow[r, "\oplus(\partial_p' \otimes 1)"]\arrow[d, "\partial_n"]
    & (\boldsymbol{B} \otimes_R \boldsymbol{Y})_n \arrow[r]\arrow[d, "\partial_n"]
    & 0 \\
    0 \arrow[r] & (\boldsymbol{Z} \otimes_R \boldsymbol{Y})_{n-1} \arrow[r, "\oplus(\imath_p \otimes 1)"] & (\boldsymbol{X} \otimes_R \boldsymbol{Y})_{n-1} \arrow[r, "\oplus(\partial_p' \otimes 1)"] & (\boldsymbol{B} \otimes_R \boldsymbol{Y})_{n-1} \arrow[r] & 0
  \end{tikzcd}
  \]
  が得られる.

  定理2.2の証明(p.62)にあるように,$\delta_n \colon H_n(\boldsymbol{B} \otimes_R \boldsymbol{Y}) \to H_{n-1}(\boldsymbol{Z} \otimes_R \boldsymbol{Y})$は
  \[ \delta_n = \{\oplus(\imath_p \otimes 1)^\ast\}^{-1} \circ \partial_n' \circ \{\oplus(\partial_p' \otimes 1)^\ast\}^{-1} \]
  で与えられる.

  $H_n(\boldsymbol{B} \otimes_R \boldsymbol{Y})$の代表元は
  \[ \ker \partial_n = \ker \left(\sum (-1)^p 1 \otimes \partial_{n-p}''\right) \subset (\boldsymbol{B} \otimes_R \boldsymbol{Y})_n = \bigoplus_{p=1}^n B_{p-1} \otimes_R Y_{n-p} \]
  の元なので,$\sum \partial_p'(x_p) \otimes y_{n-p}$(ただし$\partial_{n-p}''(y_{n-p}) = 0$)と表せる.
  これを$\delta_n$で移せば$\sum \partial_p'(x_p) \otimes y_{n-p}$となるので,結局$\delta_n = \oplus (\jmath_p \otimes 1)^\ast$である(ただし$\jmath_p \colon B_p \hookrightarrow Z_p$).
\end{proof}
