\documentclass[a4paper, leqno]{ltjsreport}
\usepackage[hiragino-pro]{luatexja-preset}
\usepackage{../hogehoge}
\renewcommand{\prechaptername}{}
\renewcommand{\postchaptername}{}
\begin{document}
\chapter*{群と位相(横田一郎)}
使用しているのは第17版第17刷.

\chapter{射影空間と古典群の定義}
\paragraph{記号等}
\begin{itemize}
  \item $\mathfrak{J}(n, K)$,$\mathfrak{J}_+(n, K)$.(p.51)
  \item $KP(n-1) = \{\, X \in M(n, K) \mid X^\ast=X, X^2=X, \Tr(X)=1 \,\}$.(p.58)\par
  補題27から$X\in KP(n-1)$は$A\in G(n, K)$によって$X=AE_nA^\ast$と表すことができる.
  これによって,積を導入する.すなわち,$X, Y\in KP(n-1)$を$A, B\in G(n, K)$を用いて$X=AE_nA^\ast$,$Y=BE_nB^\ast$とした時,$X$と$Y$の積を$(AB)E_n(AB)^\ast$とする.
  \item $S_d=\{\, (\xi, x) \in \mathbb{R} \times K \mid \xi(\xi - 1) = \lvert x\rvert^2 \,\}$.(p.58)
\end{itemize}

\chapter{射影空間と古典群の位相}
\paragraph{記号等}
\begin{itemize}
  \item $D_{\boldsymbol{a}}\in M(n, K)~ (\boldsymbol{a} \in S_K{}^{n-1})$:$D_{\boldsymbol{a}}\boldsymbol{a}=-\boldsymbol{a}$;$(\boldsymbol{x}, \boldsymbol{a}) = 0\ (\boldsymbol{x}\in K^n)$なら$D\boldsymbol{x}=\boldsymbol{x}$.(p.123)
\end{itemize}

\chapter{射影空間と古典群の胞体分割}
\paragraph{記号等}
\begin{itemize}
  \item $D_{(\kappa, \boldsymbol{a})} = (\delta_{ij} + (\kappa - 1) a_i \overline{a}_j)_{i,j=1,\ldots,n} \in M(n, \mathbb{C})$;$\boldsymbol{a} \in S_\mathbb{C}^0$,$\boldsymbol{a} \in S_\mathbb{C}^{n-1}$.(p.151)
\end{itemize}

\paragraph{定理30}
$h\colon S_d \cup V_K{}^2 \to KP(2)$は$f\colon S_d \to KP(2)$,$\varphi_2\colon V_K{}^2 \to KP(2)$で構成される.
$f$は$S_d\to KP(1)$の同相写像であり(定理28の証明),$\varphi_2$の制限は同相写像$E_K{}^2\to e^{2d}=KP(2)-KP(1)$と,$S_K{}^1\to KP(1)$(定理29の証明)である.
従って,$h\colon S_d \cup V_K{}^2 \to KP(2)$が全射であることは明らか.
ここで,$p,p'\in S_d \cup V_K{}^2$に対し,次の同値関係を定義する:
\[p \sim p' \Leftrightarrow p = p', \quad f(p) = \varphi_2(p'), \quad f(p') = \varphi_2(p).\]
$(\xi, z) \in S_d$,$(x, y) \in S_K{}^1$として,$f(\xi, z) = \varphi_2(x, y)$とおく.
実際に計算すれば,これは$\nu_K(x, y) = (\xi, z)$であることが分かる.すなわち,上の同値関係は
\[p \sim p' \Leftrightarrow p = p', \quad p = \nu_K(p'), \quad p' = \nu_K(p).\]
この同値関係による等化集合は$S_d \cup_\nu e^{2d}$である.
ところで,これは$S_d$の元を$f$で写したものと$S_K{}^1$の元を$h$で写したものが等しければ,それらの元を同一視しているので,全単射$\tilde{h}\colon S_d \cup_\nu e^{2d} \to KP(2)$が誘導される.

\paragraph{補題112}~
\begin{screen}
  準備のための写像
\end{screen}
\begin{proof}
  $p_k\colon O(k) \ni A \mapsto A\boldsymbol{e}_k \in S^{k-1}$は定理15の証明(p.104)で与えられている.
  $h_k\colon V^{k-1} \ni \boldsymbol{x} \mapsto (-2\boldsymbol{x}\sqrt{1-\lVert\boldsymbol{x}\rVert^2}, 2\lVert\boldsymbol{x}\rVert^2 - 1)$は定理27(p.136)で与えられた$e^{k-1}=S^{k-1}-e^0=S^{k-1}-\boldsymbol{e}_k$の特性写像で,同相写像$h_K\colon E^{k-1}\to e^{k-1}=S^{k-1}-\boldsymbol{e}_k$を誘導する.
  $\varphi_k\colon V^{k-1} \ni \boldsymbol{x} \mapsto (x_i \overline{x}_j)_{1\leq i, j \leq k}$は定理31(p.140)で与えられた$e^{k-1}$の特性写像で,同相写像$E^{k-1}\to e^{k-1}$を誘導する.
\end{proof}

\paragraph{定理32}~
\begin{screen}
  $O(n)$の胞体分割
\end{screen}
\begin{proof}
  $\mathbb{R}P(n-1)$の胞体$e^{k-1} = \mathbb{R}P(k-1) - \mathbb{R}P(k-2)$(定理31,p.140)は$\mathbb{R}P(k-1)$のうち,$(n, n)$成分が$0$で無いもの.
  定理25の単射$f \colon \mathbb{R}P(k-1) \ni X \mapsto E-2X \in O(k)$によって$e^{k-1}$を写せば,$O(n)$のうち$(n, n)$成分が$1$で無いものに含まれる.
  すなわち,$e^{k-1} \subset O(k) - (k-1)$である.
\end{proof}

\paragraph{補題127}~
\begin{screen}
  (3) $\overline{e^3} = e^0 \cup e^3 = S_\mathbb{H}{}^0 = Sp(1)$
\end{screen}
\begin{proof}
  $\psi_1 \colon V^3 \times V_\mathbb{H}{}^0 \ni (q, 0) \mapsto 1 + 2\sqrt{1-\lvert q\rvert^2}(q - \sqrt{1-\lvert q\rvert^2}) \in Sp(1)$によって,$\psi_1(V^3 \times V_\mathbb{H}{}^0) = \overline{e^3} = e^0 \cup e^3 \subset Sp(1) = S_\mathbb{H}{}^0$である(補題127(1)).
  $a + b \in S_\mathbb{H}{}^0 \ (a \in \mathbb{R}, b \in V^3)$とする.
  $a=1$なら$a + b = 1 \in \overline{e^3}$なので,$a \neq 1$とする.
  $q = b / \sqrt{2 - 2a}$とおけば,$a + b = \psi_1(q, 0) \in \overline{e^3}$である.従って,$S_\mathbb{H}{}^0 \subset \overline{e^3}$.
\end{proof}

\begin{screen}
  四元数$\mathbb{H}^3$での外積.
\end{screen}
\begin{proof}
  $\boldsymbol{x}, \boldsymbol{y} \in \mathbb{H}^3$は$\mathbb{H}$上右線型独立であるとする.
  \[a_1x_1 + a_2x_2 + a_3x_3 = 0,\quad a_1y_1 + a_2y_2 + a_3y_3 = 0\]
  を満たす$\boldsymbol{a} = (a_1, a_2, a_3) \neq (0, 0, 0) \in \mathbb{H}^3$を見つけたい.

  $\boldsymbol{x}$の成分2つが$0$のとき.$\boldsymbol{x}=(x_1, 0, 0)$とする.
  $y_2=y_3=0$なら$\boldsymbol{x}$と$\boldsymbol{y}$が線形従属となるので矛盾.
  $y_2=0$なら$\boldsymbol{a} = (0, 1, 0)$とする.
  $y_3=0$なら$\boldsymbol{a} = (0, 0, 1)$とする.
  $y_2, y_3 \neq 0$なら$\boldsymbol{a} = (0, y_2{}^{-1}, -y_3{}^{-1})$とする.

  $\boldsymbol{x}$の成分1つが$0$のとき.$\boldsymbol{x} = (x_1, x_2, 0)$とする.
  $y_3 = 0$なら$\boldsymbol{a} = (0, 0, 1)$とする.
  $y_3 \neq 0$なら$a_1 = x_1{}^{-1}$,$a_2 = -x_2{}^{-1}$,$a_3 = (x_2{}^{-1} y_2 - x_1{}^{-1} y_1) y_3{}^{-1}$とする.

  $\boldsymbol{x}, \boldsymbol{y}$共に成分が非零のとき.
  $x_1 x_3{}^{-1} - y_1 y_3{}^{-1} = 0$なら$\boldsymbol{a} = (x_1{}^{-1}, 0, -x_3{}^{-1})$とする.
  $x_1 x_3{}^{-1} - y_1 y_3{}^{-1} \neq 0$なら$y_3 y_1{}^{-1} - x_3 x_1{}^{-1} \neq 0$となるので,
  \begin{align*}
    a_1 &= a_2 (y_2 {y_3}^{-1} - x_2 {x_3}^{-1}) (x_1 x_3{}^{-1} - y_1 y_3{}^{-1})^{-1},\\
    a_3 &= a_2 (x_2 {x}_1^{-1} - y_2 {y}_1^{-1}) (y_3 y_1{}^{-1} - x_3 x_1{}^{-1})^{-1}
  \end{align*}
  とすればよい.実際,これを変形して
  \begin{align*}
    (a_1x_1 + a_2x_2){x_3}^{-1} &= (a_1y_1 + a_2y_2){y_3}^{-1} \\
    (a_2x_2 + a_3x_3){x}_1^{-1} &= (a_2y_2 + a_3y_3){y}_1^{-1}
  \end{align*}
  となる.
  $a_1x_1 + a_2x_2 + a_3x_3 = s$,$a_1y_1 + a_2y_2 + a_3y_3 = t$とすれば,$s{x_3}^{-1} = t{y_3}^{-1}$,$s{x}_1^{-1} = t{y}_1^{-1}$となる.
  $s \neq 0$とすれば$t \neq 0$となるので,2つ目の両辺の逆元を取って$x_1s^{-1} = y_1t^{-1}$.1つ目の式とかけて,$x_1{x_3}^{-1} = y_1{y_3}^{-1}$となり矛盾.従って$s=t=0$.
\end{proof}

\begin{screen}
  $A \in Sp(k-1)$の構成
\end{screen}
\begin{proof}
  $\boldsymbol{x}, \boldsymbol{y} \in V_\mathbb{H}{}^{k-1}$が(右$\mathbb{H}$加群の基底として)一次独立であるとする.
  $W = \boldsymbol{x}\mathbb{H} + \boldsymbol{y}\mathbb{H}$とすればこれは階数$2$の自由右$\mathbb{H}$加群.
  まず,$A\boldsymbol{x}, A\boldsymbol{y} \in \mathbb{H}^{k-1}$の第$3 \sim k-1$成分が$0$となるような$A \in Sp(k-1)$が存在する.実際,
  \[A_{k-1} = \begin{pmatrix} \boldsymbol{a}_1^\ast \\ \vdots \\ \boldsymbol{a}_{k-1}^\ast \end{pmatrix}\]
  とした際に,$(\boldsymbol{a}_{k-1}, \boldsymbol{x}) = (\boldsymbol{a}_{k-1}, \boldsymbol{y}) = 0$となるような$\boldsymbol{a}_k \in \mathbb{H}$を選ぶことができる.
  $\{\boldsymbol{a}_1, \ldots, \boldsymbol{a}_{k-1}\}$が正規直交系となるように選べば$A_{k-1} \in Sp(k-1)$であり(補題19,p.44),$A_{k-1}\boldsymbol{x}, A_{k-1}\boldsymbol{y}$は第$k-1$成分が$0$である.従って,$A_{k-1}\boldsymbol{x}, A_{k-1}\boldsymbol{y}$は$\mathbb{H}^{k-2}$の元とみなすことができる.上と同様にして,
  \[A_{k-2} = \begin{pmatrix} A_{k-2}' & \\ & 1 \end{pmatrix}\quad (A_{k-2}' \in Sp(k-2))\]
  によって$A_{k-2}A_{k-1}\boldsymbol{x}, A_{k-2}A_{k-1}\boldsymbol{y}$は第$k-2, k-1$成分が$0$となる.これを繰り返せば,
  \[A\boldsymbol{x} = \begin{pmatrix} x_1 \\ x_2 \\ 0 \\ \vdots \\ 0 \end{pmatrix}, \quad A\boldsymbol{y} = \begin{pmatrix} y_1 \\ y_2 \\ 0 \\ \vdots \\ 0 \end{pmatrix}\]
  となる$A \in Sp(k-1)$の存在が証明される.

  $1 \leq r < k-1$を,$\boldsymbol{x}, \boldsymbol{y}$の$s+1 \sim k-1$成分を取ったベクトルが線形従属となるような$s$のうち最大のものとする.
  $\boldsymbol{z} = \boldsymbol{x}a + \boldsymbol{y}b$の$r+1 \sim k-1$が全て$0$であるような$a, b\in\mathbb{H}$が存在する.
  $W_r = \boldsymbol{z}\mathbb{H}$とすればこれは階数$1$の$W$の部分加群となる.あとは,$A\boldsymbol{z}$が第$1$成分のみ非零であるように$A$を再構成すればよい.
  \[A\boldsymbol{z} = \begin{pmatrix} z_1 = x_1a + y_1b \\ z_2 = x_2a + y_2b \\ 0 \\ \vdots \\ 0 \end{pmatrix}\]
  となるが,$z_1 = z_2 = 0$なら$\boldsymbol{z} = 0$となり,$\boldsymbol{x}, \boldsymbol{y}$が線形従属となるので矛盾.
  $z_2 = 0$ならそれでよい.$z_1 = 0$なら
  \[
  \begin{pmatrix}
    & 1 &   &        & \\
    1 &   &   &        & \\
    &   & 1 &        & \\
    &   &   & \ddots & \\
    &   &   &        & 1
  \end{pmatrix}
  A \in Sp(k-1)
  \]
  を改めて$A$とすればよい.
  $z_1, z_2 \neq 0$なら
  \[
  \begin{pmatrix}
    1          &             &   &        & \\
    s{z}_1^{-1} & -s{z}_2^{-1} &   &        & \\
    &             & 1 &        & \\
    &             &   & \ddots & \\
    &             &   &        & 1
  \end{pmatrix}
  A\in Sp(k-1),\quad s = \frac{1}{\sqrt{\lvert {z}_1^{-1} \rvert^2 + \lvert {z}_2^{-1} \rvert^2 }}
  \]
  を改めて$A$とすればよい.

  $A\boldsymbol{x}, A\boldsymbol{y}$が線形従属とすれば,$\boldsymbol{x}, \boldsymbol{y}$が線形従属となるので矛盾.
  従って$A\boldsymbol{x}, A\boldsymbol{y}$は線形独立.
\end{proof}

\chapter{射影空間と古典群の基本群と被覆空間}
\paragraph{補題146}~
\begin{screen}
  $p^{-1}(h(\tau))$が離散空間
\end{screen}
\begin{proof}
  ファイバー空間の定義(p.176)の直後に書かれているように,$x\in X$に対し,$p^{-1}(x)$と$F$は同相である.
\end{proof}

\paragraph{補題149}~
\begin{screen}
  $u(I) \subset X$がコンパクト
\end{screen}
\begin{proof}
  $I$は$\mathbb{R}$の有界閉集合なのでコンパクト.
  連続な全射$u\colon I \twoheadrightarrow u(I)$に対し命題59を適用すればよい.
\end{proof}

\begin{screen}
  $v_i$が全単射
\end{screen}
\begin{proof}
  まず$p^{-1}(x_{i+1}) \cap U(b_{i\lambda})$が唯1点からなることを示す.
  $p$の制限によって$U(b_{i\lambda})$と$U(x_i) \ni x_{i+1}$は同相なので$p^{-1}(x_{i+1}) \cap U(b_{i\lambda}) \neq \varnothing$.
  $b, b' \in p^{-1}(x_{i+1}) \cap U(b_{i\lambda})$とする.
  $p(b) = p(b')$で,$p$は$U(b_{i\lambda})$上で単射なので$b=b'$.
  よって,写像
  \[
    v_i\colon p^{-1}(x_i) \ni b_{i\lambda} \mapsto p^{-1}(x_{i+1}) \cap U(b_{i\lambda}) \in p^{-1}(x_{i+1})
  \]
  が得られる.

  $x_{i+1} \in U(x_i)$なので$p^{-1}(x_{i+1}) \subset p^{-1}(U(x_i)) = \bigcup_{\lambda} U(b_{i\lambda})$.
  したがって,全ての$b\in p^{-1}(x_{i+1})$に対して$b \in U(b_{i\lambda})$となる$\lambda$が存在する.
  よって,$v_i(b) = b$となり,$v_i$は全射.

  $U(b_{i\lambda}) \cap U(b_{i\mu}) = \varnothing$なので$v_i$は単射.
\end{proof}

\paragraph{命題152}~
\begin{screen}
  \((B', \tilde{h}, B)\)は被覆空間の性質(3)を満たす
\end{screen}
\begin{proof}
  \(b \in B\), \(x = p(b) \in X\)とする.

  \(\tilde{h}^{-1}(b) = p'^{-1}(x)\)を示す.
  \(b' \in \tilde{h}^{-1}(b)\)とする.\(b = \tilde{h}(b')\)なので\(x = p(b) = p\circ\tilde{h}(b') = p'(b')\).
  従って\(b' \in p'^{-1}(x)\)なので\(\tilde{h}^{-1}(b) \subset p'^{-1}(x)\).
  逆に\(b' \in p'^{-1}(x)\)とする.補題150より\(b = \tilde{h}(b')\)なので\(b' \in\tilde{h}^{-1}(b)\).
  従って\(p'^{-1}(x) \subset\tilde{h}^{-1}(b)\).

  \(b' \in \tilde{h}^{-1}(b) = p'^{-1}(x)\)とする.標準近傍\(V \ni x\), \(U(b)\ni b\), \(U'(b')\ni b'\)を取る.
  \(\tilde{h}\)の制限により\(U'(b') \simeq U(b)\)となることを示す.
  \[
  \begin{tikzcd}[column sep=small]
    B' \arrow[rrrr, "\tilde{h}", twoheadrightarrow]\arrow[rrddd, "p'", twoheadrightarrow, bend right] &&&& B \arrow[llddd, "p", twoheadrightarrow, bend left] \\
    & U'(b') \arrow[lu, "i'", hookrightarrow] \arrow[rr] \arrow[rd, "\phi'"] && U(b) \arrow[ru, "i", hookrightarrow] \arrow[ld, "\phi"]& \\
    && V \arrow[d, "j", hookrightarrow] && \\
    && X &&
  \end{tikzcd}
  \]
  標準近傍の性質より\(\phi\), \(\phi'\)は同相で,\(p'\circ i' = j\circ\phi'\), \(p\circ i = j\circ\phi\)である.よって
  \[
    p\circ i\circ\phi^{-1}\circ\phi' = j\circ\phi\circ\phi^{-1}\circ\phi' = j\circ\phi' = p'\circ i' = p\circ\tilde{h}\circ i' .
  \]
  \(p\circ i = j\circ\phi\)は単射なので
  \[
    \phi^{-1}\circ\phi' = (p\circ i)^{-1}\circ p\circ i\circ\phi^{-1}\circ\phi' = (p\circ i)^{-1}\circ p\circ\tilde{h}\circ i'
    = i^{-1}\circ p^{-1} \circ p \circ\tilde{h}\circ i' .
  \]
  したがって
  \[ i\circ\phi^{-1}\circ\phi' = p^{-1} \circ p \circ\tilde{h}\circ i' = \tilde{h}\circ i' . \]

  さらに標準近傍の性質から
  \[
    \tilde{h}^{-1}(U(b)) = p'^{-1}(V) = \bigcup_{b_\lambda' \in p'^{-1}(x)} U'(b'_\lambda) = \bigcup_{b_\lambda' \in \tilde{h}^{-1}(b)} U'(b'_\lambda) .
  \]
\end{proof}

\paragraph{命題155}~
\begin{screen}
  \(v_1\), \(v_2\), \(v_3\)の構成
\end{screen}
\begin{proof}
  \(v_1\), \(v_2\)は\(\alpha_\ast\), \(\beta_\ast\)の定義から存在する.
  \(b_0\), \(b_\beta\)を結ぶ道\(v_2\)に対し,\(\alpha_\ast(b_0)\)を始点とし,\(pv_2\)を被覆する道を\(v_3\)とする.
  \(\alpha_\ast\)の定義から\(v_3(1) = \alpha_\ast(b_\beta)\).
  \(pv_2 = pv_3\)なので,\([pv_3]=\beta\).
\end{proof}

\begin{screen}
  \((v_1\cdot v_3)(1) = (\alpha\beta)_\ast(b_0)\)
\end{screen}
\begin{proof}
  \(\alpha=[pv_1]\), \(\beta=[pv_3]\)なので\(\alpha\cdot\beta \in \pi_1(X, x_0)\)の代表元として\(p(v_1\cdot v_3)\)を取れる.
  \( (\alpha\beta)_\ast(b_0)\)は\(v_1\cdot v_3\)の終点なので,\((v_1\cdot v_3)(1) = (\alpha\beta)_\ast(b_0)\).
\end{proof}

\begin{screen}
  \((\alpha\beta)_\ast = \alpha_\ast\circ\beta_\ast\)
\end{screen}
\begin{proof}
  \((\alpha\beta)_\ast(b_0) = \alpha_\ast\circ\beta_\ast(b_0)\)なので,命題152を使うためには\(p\circ(\alpha\beta)_\ast = p\circ\alpha_\ast\circ\beta_\ast\)を示せばよい.
  \[
  \begin{tikzcd}
    B \arrow[rr]\arrow[rd, "p"'] & & B \arrow[ld, "p"] \\
    & X &
  \end{tikzcd}
  \]
  \(\alpha_\ast\)の定義から,
  \[
    p(\alpha_\ast(\beta_\ast(b))) = p(\beta_\ast(b)) = p(b) = p((\alpha\beta)_\ast(b)) .
  \]
\end{proof}

\begin{screen}
  被覆写像\(p\colon B\twoheadrightarrow X\)は開写像
\end{screen}
\begin{proof}
  \(V\)を\(B\)の開集合とする.\(x \in p(V)\)を任意にとる.\(b \in p^{-1}(x) \cap V\)とする.
  標準近傍\(U \ni x\), \(U(b) \ni b\)を取る.
  \(V \cap U(b)\)は\(U(b)\)の開集合であり,\(p\)は\(U(b)\)と\(U\)の同相写像なので,\(p(V\cap U(b))\)は\(U\)の開集合.
  従って,\(p(V\cap U(b))\)は\(X\)の開集合.
  \(x \in p(V\cap U(b)) \subset p(V)\)なので,\(p(V)\)は開集合.
\end{proof}

\chapter*{付録}
\paragraph{命題182}~
\begin{screen}
  \(\ker\varphi = \{\pm E\}\)
\end{screen}
\begin{proof}
  \(A \in \ker\varphi\)とすれば,任意のHermite行列\(X \in \mathfrak{J}(3, K)\)に対し\(AX = XA\)となる.
  ここで,
  \[
  X =
  \begin{pmatrix}
    1 & 0 & 0 \\
    0 & 0 & 0 \\
    0 & 0 & 0
  \end{pmatrix}
  \]
  とすれば,
  \[
  \begin{pmatrix}
    a_{11} & 0 & 0 \\
    a_{21} & 0 & 0 \\
    a_{31} & 0 & 0
  \end{pmatrix}
  =
  \begin{pmatrix}
    a_{11} & a_{12} & a_{13} \\
    0 & 0 & 0 \\
    0 & 0 & 0
  \end{pmatrix}
  .
  \]
  よって\(a_{12} = a_{21} = a_{13} = a_{31} = 0\).
  \[
  X =
  \begin{pmatrix}
    0 & 0 & 0 \\
    0 & 1 & 0 \\
    0 & 0 & 0
  \end{pmatrix}
  \]
  とすれば\(a_{23} = a_{32} = 0\).以上から,\(A\)は対角行列.
  \[
  X =
  \begin{pmatrix}
    0 & 1 & 0 \\
    1 & 0 & 0 \\
    0 & 0 & 0
  \end{pmatrix}
  \]
  とすれば\(a_{11} = a_{22}\).
  \[
  X =
  \begin{pmatrix}
    0 & 0 & 1 \\
    0 & 0 & 0 \\
    1 & 0 & 0
  \end{pmatrix}
  \]
  とすれば\(a_{11} = a_{33}\).
  よって,\(A = a E\) (\(\lvert a\rvert^2 = 1\))と表せる.

  \(K = \mathbb{R}\)なら\(a=\pm 1\).
  \(K = \mathbb{C}\)なら\(a=e^{i\theta}\).

  \(K = \mathbb{H}\)の場合を考える.
  \[
  X =
  \begin{pmatrix}
    0 & i & 0 \\
    -i & 0 & 0 \\
    0 & 0 & 0
  \end{pmatrix}
  \]
  とすれば\(ai = ia\).
  \(a = a_0 + a_1 i + a_2 j + a_3 k\) (\(a_i \in \mathbb{R}\))とすれば\(a_2 = a_3 = 0\).
  \[
  X =
  \begin{pmatrix}
    0 & j & 0 \\
    -j & 0 & 0 \\
    0 & 0 & 0
  \end{pmatrix}
  \]
  とすれば\(a_1 = a_3 = 0\).
  以上から\(a \in \mathbb{R}\)なので\(a = \pm 1\).
\end{proof}

\begin{screen}
  \(\Tr \alpha(E_i) = 1\),\(U E_i U^\ast = \alpha(E_i)\)
\end{screen}
\begin{proof}
  \(\alpha(E_i \circ E_i) = \alpha(E_i) \circ \alpha(E_i)\)なので\(\alpha(E_i)\alpha(E_i) = \alpha(E_i)\).
  \(i \neq j\)ならば\(E_i E_j = 0\)なので\(\alpha(E_i) \alpha(E_j) = 0\).
  \(E_1 + E_2 + E_3 = 1\)なので\(\alpha(E_1) + \alpha(E_2) + \alpha(E_3) = 1\).

  冪等行列\(\alpha(E_i)\)の固有ベクトル\(\boldsymbol{v}_i\)の張る部分空間\(W_i\)を考えれば\footnote{佐武線型代数p.131--132}
  \[ \Tr \alpha(E_i) = \rank \alpha(E_i) = \dim W_i = 1 . \]
  さらに\(\boldsymbol{v}_i\)の固有値は\(1\).

  \(\boldsymbol{v}_2 \not\in W_1\)なので\(\alpha(E_1) \boldsymbol{v}_2 = \boldsymbol{0}\).
  よって\(\boldsymbol{v}_1^\ast \alpha(E_1) \boldsymbol{v}_2 = \boldsymbol{0}\).
  したがって
  \[
  0 = \boldsymbol{v}_2^\ast \alpha(E_1)^\ast \boldsymbol{v}_1
  = \boldsymbol{v}_2^\ast \alpha(E_1) \boldsymbol{v}_1 = \boldsymbol{v}_2^\ast \boldsymbol{v}_1 .
  \]
  同様にして\(\boldsymbol{v}_i^\ast \boldsymbol{v}_j = \delta_{ij}\)となるので
  \[
  U =
  \begin{pmatrix}
    \boldsymbol{v}_1 & \boldsymbol{v}_2 & \boldsymbol{v}_3
  \end{pmatrix}
  \in G(3, K) .
  \]
  したがって,
  \[
  U^\ast \alpha(E_1) U =
  \begin{pmatrix}
    \boldsymbol{v}_1^\ast \\ \boldsymbol{v}_2^\ast \\ \boldsymbol{v}_3^\ast
  \end{pmatrix}
  \begin{pmatrix}
    \boldsymbol{v}_1 & \boldsymbol{0} & \boldsymbol{0}
  \end{pmatrix}
  = E_1 .
  \]
  よって,\(U E_i U^\ast = \alpha(E_i)\).
\end{proof}

\begin{screen}
  \(E_i \circ F_i{}^x = 0\)および\(2E_i \circ F_j{}^x = F_j{}^x\) (\(i\neq j\))を満たす\(F_i{}^x \in \mathfrak{J}(3, K)\)は
  \[
  F_1{}^x =
  \begin{pmatrix}
    0 & 0 & 0 \\
    0 & 0 & x \\
    0 & \bar{x} & 0
  \end{pmatrix}
  , \quad
  F_2{}^x =
  \begin{pmatrix}
    0 & 0 & \bar{x} \\
    0 & 0 & 0 \\
    x & 0 & 0
  \end{pmatrix}
  , \quad
  F_3{}^x =
  \begin{pmatrix}
    0 & x & 0 \\
    \bar{x} & 0 & 0 \\
    0 & 0 & 0
  \end{pmatrix}
  \]
  のみ.
\end{screen}

\end{document}
