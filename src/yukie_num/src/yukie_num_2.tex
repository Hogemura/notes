\chapter{分岐と完備化}
\setcounter{section}{1}
\section{Dedekind環の完備化}
\paragraph{命題1.2.13}~
\begin{screen}
  自然な写像$\phi\colon\mathfrak{p}^n/\mathfrak{p}^m \ni a + \mathfrak{p}^m \mapsto a + \mathfrak{p}^mA_\mathfrak{p} \in \mathfrak{p}^nA_\mathfrak{p}/\mathfrak{p}^mA_\mathfrak{p}$は単射
\end{screen}
\begin{proof}
  $\ker(\phi) = (\mathfrak{p}^mA_\mathfrak{p}\cap\mathfrak{p}^n) + \mathfrak{p}^m$であるが,命題I-6.5.9 (4)から$\mathfrak{p}^mA_\mathfrak{p}\cap\mathfrak{p}^n\subset\mathfrak{p}^mA_\mathfrak{p}\cap A = \mathfrak{p}^m$なので,$\ker(\phi)\subset 0 + \mathfrak{p}^m$.
  $\ker(\phi)\supset 0 + \mathfrak{p}^m$は明らかなので,$\ker(\phi) = 0 + \mathfrak{p}^m$.
\end{proof}

\begin{screen}
  自然な写像$\mathfrak{p}^n/\mathfrak{p}^m\to\mathfrak{p}^nA_\mathfrak{p}/\mathfrak{p}^mA_\mathfrak{p}$は全射
\end{screen}
\begin{proof}
  $s\in A\setminus\mathfrak{p}$とすると,$b + cs = 1$となる$b\in\mathfrak{p}^{m-n}$,$c\in A$が存在する.
  % <!-- COMBAK: ほんとに? -->
  $a\in\mathfrak{p}^n$とすれば,$a + \mathfrak{p}^m = (b + cs)(a + \mathfrak{p}^m) = (ab + acs) + \mathfrak{p}^m$となるが,$ab\in\mathfrak{p}^m$なので$a + \mathfrak{p}^m = acs + \mathfrak{p}^m$.つまり$a-acs\in\mathfrak{p}^m$.従って,$(a/s)-ca\in\mathfrak{p}^mA_\mathfrak{p}$.以上から,自然な写像は
  \[\mathfrak{p}^n/\mathfrak{p}^m\ni ca + \mathfrak{p}^m\mapsto ca + \mathfrak{p}^mA_\mathfrak{p} = a/s + \mathfrak{p}^mA_\mathfrak{p}\in\mathfrak{p}^nA_\mathfrak{p}/\mathfrak{p}^mA_\mathfrak{p}.\]
  と表すことができ,これは全射となる.
\end{proof}

\begin{screen}
  準同型$\phi \colon A/\mathfrak{p}^{m-n} \ni a + \mathfrak{p}^{m-n} \mapsto ax + \mathfrak{p}^{m} \in \mathfrak{p}^{n}/\mathfrak{p}^{m} ~ (x \in \mathfrak{p}^n \setminus \mathfrak{p}^{n + 1})$は単射
\end{screen}
\begin{proof}
  $\ker(\phi) = \set{b + \mathfrak{p}^{m-n}\in A/\mathfrak{p}^{m-n} | b\in A, bx\in\mathfrak{p}^m}$である.
  $\ord_\mathfrak{p}(bx)\geq m$,$\ord_\mathfrak{p}(x) = n$なので,$\ord_\mathfrak{p}(b)\geq m-n$であり,$b\in\mathfrak{p}^{m-n}A_\mathfrak{p}$.
  $b\in A$でもあるので,命題I-6.5.9 (4)から$b\in A\cap\mathfrak{p}^{m-n}A_\mathfrak{p} = \mathfrak{p}^{m-n}$.
  従って,$\ker(\phi) = 0 + \mathfrak{p}^{m-n}$.
\end{proof}

\paragraph{命題1.2.14}~
\begin{screen}
  Dedekind環$A$の全ての素イデアル$\mathfrak{p}$に対して$a\in A_\mathfrak{p}$なら$a\in A$
\end{screen}
\begin{proof}
  命題I-8.3.12で$L$の部分$A$加群として$A$を取れば$A = \bigcap_\mathfrak{p}A_\mathfrak{p}A\supset\bigcap_\mathfrak{p}aA = aA$.よって$a\in A$.
\end{proof}

\section{分岐と完備化}
\paragraph{注1.3.6 (2)}~
\begin{screen}
  $\mathfrak{p}$がDedekind環$A$の素イデアル,Dedekind環$B$が$A$の整閉包,$P_1, \ldots, P_t\in B$を$\mathfrak{p}$の上にある全ての素イデアルとした場合,イデアル$\mathfrak{p}B$の素イデアル分解が,$\mathfrak{p}B = P_1{}^{e_1}\cdots P_t{}^{e_t}\ (e_i>0)$であり,これらが$\mathfrak{p}$の上にある$B$の全ての素イデアルである
\end{screen}
\begin{proof}
  $\mathfrak{p}B$は$B$の零でないイデアルなので,命題I-8.3.12から$B$の適当な極大イデアル達$P'_i$によって$\mathfrak{p}B = \bigcap_i\mathfrak{p}BB_{P'_i}$.
  Dedekind環に関する仮定から,$b\in\mathfrak{p}B$であれば$B/(b)$が有限である.
  よって,($b$を含み,$(b)$と異なるイデアルを作るには$B/(b)$の完全代表系からいくつかの生成元を選ぶことになるので)$b$を含む素イデアルの数は有限である.
  従って,$\mathfrak{p}B$を含む素イデアルの数も有限である.

  $P'_i\subset B$が素イデアルで$\mathfrak{p}B\nsubseteq P'_i$なら$s\in\mathfrak{p}B\setminus P'_i$とすると,$s$は$B_{P'_i}$の単元.
  つまり,有限個の($\mathfrak{p}B\subset P_i'$を満たす)素イデアル$P_i'$に対し,$\mathfrak{p}B_{P_i'} = B_{P_i'}$が成立する.
  $P_1, \ldots, P_t$を$\mathfrak{p}B$を含む全ての素イデアルとする.
  命題I-8.3.15により$\mathfrak{p}BB_{P_i} = P_i{}^{e_i}$なる$e_i\in\mathbb{Z}$が存在.
  $\mathfrak{p}B\in P_i$なので命題I-8.3.12から
  \[\mathfrak{p}B = \bigcap_i(\mathfrak{p}BB_{P_i}\cap B) = \bigcap_i(P_i{}^{e_i}B_{P_i}\cap B) = \bigcap_i P_i{}^{e_i} = P_1{}^{e_1}\cdots P_t{}^{e_t}\]
  が命題I-6.5.9 (4)と中国剰余定理から成立.
  $\mathfrak{p}B\subset P_i$ならば$1\in B$を考え$\mathfrak{p}\subset P_i$.
  逆に,$\mathfrak{p}\subset P_i$ならば両辺に$B$をかけて($P_i$は$B$のイデアルなので)$\mathfrak{p}B\subset P_i$.
  よって,$\mathfrak{p}\subset P_i$なる全ての素イデアル$P_1, \ldots, P_t$によって$\mathfrak{p}B = P_1{}^{e_1}\cdots P_t{}^{e_t} ~ (e_i>0)$.
  補題1.3.3から$P_1, \ldots, P_t\in B$は$\mathfrak{p}$の上にある全ての素イデアルで,$\mathfrak{p}B = P_1{}^{e_1}\cdots P_t{}^{e_t}~ (e_i>0)$である.

  上の結果により,$\mathfrak{p}B$の素イデアル分解に現れる素イデアルだけを分岐の考察対象にすれば良い事が分かる.
\end{proof}

\paragraph{例1.3.8}~
\begin{screen}
  $\mathbb{F}_2[x]/((x + 1)^2)$の極大イデアルを求める
\end{screen}
\begin{proof}
  $((x + 1)^2)\subset(x + 1)\subset\mathbb{F}_2[x]$なので,定理I-6.1.34から
  \[(\mathbb{F}[x]/((x + 1)^2))/((x + 1)/((x + 1)^2))\simeq\mathbb{F}_2[x]/(x + 1)\simeq\mathbb{F}_2.\]
  $\mathbb{F}_2$は体なので,$(x + 1)/((x + 1)^2)$は$\mathbb{F}[x]/((x + 1)^2)$における極大イデアル(命題6.3.4).
\end{proof}

\paragraph{定理1.3.23}~
\begin{screen}
  $B_\mathfrak{p}$が階数$n$の自由$A_\mathfrak{p}$加群なら,$A$加群として$B\otimes_A\widehat{A}_\mathfrak{p}\simeq\widehat{A}_\mathfrak{p}{}^n$である
\end{screen}
\begin{proof}
  $A$加群の同型を構成すればよい:
  \begin{align*}
    B\otimes_A\widehat{A}_\mathfrak{p} \ni b \otimes a &\mapsto b \otimes 1 \otimes a \in B\otimes_A A_\mathfrak{p} \otimes_{A_\mathfrak{p}} \widehat{A}_\mathfrak{p} \\
    &\mapsto b \otimes a \in B_\mathfrak{p} \otimes_{A_\mathfrak{p}} \widehat{A}_\mathfrak{p} \\
    &\mapsto \left(\sum b_i\right) \otimes a \in \left(A_\mathfrak{p}{}^{\oplus n}\right)\otimes_{A_\mathfrak{p}}\widehat{A}_\mathfrak{p} \\
    &\mapsto \sum\left(b_i \otimes a\right) \in \left(A_\mathfrak{p}\otimes_{A_\mathfrak{p}}\widehat{A}_\mathfrak{p}\right)^{\oplus n} \\
    &\mapsto \sum\left(ab_i\right) \in \widehat{A}_\mathfrak{p}{}^{\oplus n} \\
    &\mapsto (ab_1, \ldots, ab_n) \in \widehat{A}_\mathfrak{p}{}^n.
  \end{align*}
\end{proof}

\begin{screen}
  $A$加群として$(B\otimes_A\widehat{A}_\mathfrak{p})/\mathfrak{p}^m(B\otimes_A\widehat{A}_\mathfrak{p}) \simeq \widehat{A}_\mathfrak{p}{}^n/\mathfrak{p}^m\widehat{A}_\mathfrak{p}{}^n \simeq (\widehat{A}_\mathfrak{p}/\mathfrak{p}^m\widehat{A}_\mathfrak{p})^n$である
\end{screen}
\begin{proof}
  1つめの同型は,上で示した同型によって自然に引き起こされる:
  \[B\otimes_A\widehat{A}_\mathfrak{p}/\mathfrak{p}^m(B\otimes_A\widehat{A}_\mathfrak{p}) \ni b\otimes a + \mathfrak{p}^m(B\otimes_A\widehat{A}_\mathfrak{p}) \mapsto  (ab_1, \ldots, ab_n) + \mathfrak{p}^m\widehat{A}_\mathfrak{p}{}^n \in \widehat{A}_\mathfrak{p}{}^n/\mathfrak{p}^m\widehat{A}_\mathfrak{p}{}^n.\]
  準同型$\phi\colon \widehat{A}_\mathfrak{p}{}^n\ni(a_1, \ldots, a_n)\mapsto(a_1 + \mathfrak{p}^mA_\mathfrak{p}, \ldots, a_n + \mathfrak{p}^mA_\mathfrak{p})\in(\widehat{A}_\mathfrak{p}/\mathfrak{p}^mA_\mathfrak{p})^n$の核は$\mathfrak{p}^mA_\mathfrak{p}{}^n$なので,準同型定理から$\widehat{A}_\mathfrak{p}{}^n/\mathfrak{p}^m\widehat{A}_\mathfrak{p}{}^n\simeq(\widehat{A}_\mathfrak{p}/\mathfrak{p}^m\widehat{A}_\mathfrak{p})^n$.
\end{proof}

\begin{screen}
  $K\otimes_AB\simeq L$($A$代数の環同型)
\end{screen}
\begin{proof}
  $S = A\setminus\{0\}$とする.命題I-8.1.13から$S^{-1}B$は$S^{-1}A = K$上整で$K$は体なので,命題I-8.1.12から$K\otimes_AB = S^{-1}B$も体.
  よって,$S^{-1}B$の元$b/a ~ (b\in B, a\in A\setminus\{0\})$に対し$c\in B, d\in A\setminus\{0\}$が存在し$bc/ad = 1$,つまり$bc = ad\in A$となる.
  $b$は任意に選ぶことができるので,$\forall b\in B$に対し$bc\in A$となる$c\in A$が存在することが分かる.
  よって,$L$の元$b_0/b_1\ (b_0, b_1\in B)$として$b_1c_1\in A$となる$c_1\in A$を選べば,$b_0/b_1 = b_0c_1/b_1c_1\in S^{-1}B$なので$L\subset S^{-1}B$.
  $S^{-1}B\subset L$は明らか.よって,$S^{-1}B = L$.全射準同型
  \[\phi\colon K\otimes_AB\ni a\otimes b\mapsto ab\in S^{-1}B = L\]
  を考える.$\ker\phi = \{0\}$なので単射.よって$a\otimes b\mapsto ab$によって$K\otimes_AB\simeq L$.
\end{proof}

\begin{screen}
  \begin{lem}
    \label{field_product_number}
    体の直積は異なる数の体の直積と環同型にはなり得ない
  \end{lem}
\end{screen}
\begin{proof}
  同型$\phi\colon E_1\times\cdots E_m\to F_1\times\cdots\times F_n\ (m>n)$によって
  \begin{align*}
    \phi(1, 0, \ldots, 0) & =  (a_1^{(1)}, \ldots, a_n^{(1)})\\
    \phi(0, 1, \ldots, 0) & =  (a_1^{(2)}, \ldots, a_n^{(2)})\\
                       &\vdots\\
    \phi(0, \ldots, 0, 1) & =  (a_1^{(m)}, \ldots, a_n^{(m)})
  \end{align*}
  に写るとする.右辺は$0$にならないので,$a_1^{(i)} = \cdots = a_n^{(i)} = 0$となることはない.
  $i$番目の式と$j$番目の式をかけて$(0, \ldots, 0) = (a_1^{(i)}a_1^{(j)}, \ldots, a_n^{(i)}a_n^{(j)})$となる.
  よって$a_1^{(i)}, a_1^{(j)}$のうち少なくとも片方は$0$..
  このような式が$m(m-1)/2$個得られ,これらを全て満たすには$a_1^{(1)}, \ldots, a_1^{(m)}$のうち$m-1$個が$0$である必要がある.
  $a_2$などについても同様の議論を行えば,$1\leq i\leq n$に対し$a_i^{(1)}, \ldots, a_i^{(m)}$のうち$0$でないものは高々1個しかない.
  $m>n$なので$a_1^{(j)} = \cdots = a_n^{(j)} = 0$となる$1\leq j\leq m$が必ず存在し,矛盾.
\end{proof}

\begin{screen}
  \begin{lem}
    \label{field_product_iso}
    体の直積が同型なら,構成する体の間で同型となるペアが存在する
  \end{lem}
\end{screen}
\begin{proof}
  同型$\phi\colon E_1\times\cdots E_n\to F_1\times\cdots\times F_n$とする.
  先程と同様の議論から,必要ならば適当に$F_i$を並び替えることによって
  \begin{align*}
    \phi(1, 0, \ldots, 0) & =  (c_1, 0, \ldots, 0)\\
    \phi(0, 1, \ldots, 0) & =  (0, c_2, \ldots, 0)\\
                       &\vdots\\
    \phi(0, \ldots, 0, 1) & =  (0, \ldots, 0, c_n)
  \end{align*}
  に写るとして良い.初めの式を$2$乗すれば
  \[ (c_1{}^2, 0, \ldots, 0) = \phi(1, 0, \ldots, 0)^2 = \phi(1, 0, \ldots, 0) = (c_1, 0, \ldots, 0) \]
  なので,$c_1{}^2 = c_1$.$E_1$は整域なので,$c_1 = 1$である.
  他についても同様に,$c_1 = \ldots = c_n = 0$となる.

  $\phi(a_1, \ldots) = (b_1, \ldots)$とする.
  上の式とかければ$\phi(a_1, 0, \ldots, 0) = (b_1, 0, \ldots, 0)$となる.
  これは写像$\varphi\colon E_1\to F_1$を引き起こす:
  \[\varphi\colon E_1\ni a_1\mapsto \pi_1(\phi(a_1, 0, \ldots, 0))\in F_1.\]
  ただし,$\pi_1$は$F_1\times\cdots F_n$の元の第1成分のみを取り出す射影.
  $\varphi$が同型になることは容易に分かる.他についても同様.
\end{proof}

\begin{screen}
  \begin{lem}
    \label{iso_field_integral_closure}
    体$E$, $F$に対し$\phi \colon E \simeq F$とすれば,$\phi(\mathcal{O}_E) = \mathcal{O}_F$.
  \end{lem}
\end{screen}
\begin{proof}
  $x\in \mathcal{O}_E$とする.
  $x$は$\mathbb{Z}$上整なので,$a_1, \ldots, a_n \in \mathbb{Z}$,$\psi\colon\mathbb{Z}\to E$があり
  \[x^n + \psi(a_1)x^{n-1} + \cdots + \psi(a_n) = 0.\]
  これを$\phi$でうつせば
  \[\phi(x)^n + \phi\circ\psi(a_1)\phi(x)^{n-1} + \cdots + \phi\circ\psi(a_n) = 0\]
  なので$F$上整となり,$\phi(x) \in \mathcal{O}_F$.

  $y \in \mathcal{O}_F$とする.
  $y$は$\mathbb{Z}$上整なので,$a_1, \ldots, a_n \in \mathbb{Z}$,$\psi\colon\mathbb{Z}\to F$があり
  \[y^n + \psi(a_1)y^{n-1} + \cdots + \psi(a_n) = 0.\]
  $y = \phi(x)$とすれば
  \begin{align*}
    0 &= \phi(x^n) + \phi\circ\phi^{-1}\circ\psi(a_1) \phi(x^{n-1}) + \cdots + \phi\circ\phi^{-1}\circ\psi(a_n) \\
      &= \phi \left( x^n + \phi^{-1}\circ\psi(a_1) x^{n-1} + \cdots + \phi^{-1}\circ\psi(a_n) \right)
  \end{align*}
  なので,$x^n + \phi^{-1}\circ\psi(a_1) x^{n-1} + \cdots + \phi^{-1}\circ\psi(a_n) = 0$となり,$x \in \mathcal{O}_E$.
  従って,$y = \phi(x) \in \phi(\mathcal{O}_E)$.
\end{proof}

\begin{screen}
  \begin{thm}
    \label{Thm_1_3_23_2}
    定理1.3.23 (2)の環同型について
  \end{thm}
\end{screen}
p.24の終盤に書かれているように$\phi_i\colon B\hookrightarrow\widehat{B}_i$として,環同型
\[B\otimes_A\widehat{A}_\mathfrak{p}\ni x \otimes y\mapsto(\phi_1(x)y, \ldots, \phi_g(x)y)\in\widehat{B}_1\times\cdots\times\widehat{B}_g\]
によって$B\otimes_A\widehat{A}_\mathfrak{p}\simeq\widehat{B}_1\times\cdots\times\widehat{B}_g$.

もしくは,逆極限によっても構成できる\footnote{\cite{加藤和也2005数論}命題6.50,補題6.69など}.
$\mathfrak{p}^nB = P_1{}^{e_1} \cdots P_g{}^{e_g}$なので,中国式剰余定理から$B/\mathfrak{p}^nB \simeq \prod_{i = 1}^g B/P_i{}^{ne_i}$である.
$A$はNoether環で$B$は有限生成$A$加群なので,
\[ B \otimes_A \varprojlim (A/\mathfrak{p}^n) \simeq \varprojlim(B/\mathfrak{p}^nB) \simeq \prod_{i = 1}^g \varprojlim (B/P_i{}^{ne_i}). \]

$K$代数としての環同型$L\otimes_K\widehat{K}_\mathfrak{p}\simeq\widehat{K}_1\times\cdots\times\widehat{K}_g$はp.26の終盤に書いてるように構成される:
\begin{align*}
  L\otimes_K\widehat{K}_\mathfrak{p}\ni a/b\otimes c &\mapsto a\otimes1/b\otimes c\in B\otimes_AK\otimes_K\widehat{K}_\mathfrak{p}\\
  & =  a\otimes1\otimes c/b\in B\otimes_AK\otimes_K\widehat{K}_\mathfrak{p}\\
  &\mapsto a\otimes c/b\in B\otimes_A\widehat{K}_\mathfrak{p}\\
  &\mapsto a\otimes1\otimes c/b\in B\otimes_A\widehat{A}_\mathfrak{p}\otimes_{\widehat{A}_\mathfrak{p}}\widehat{K}_\mathfrak{p}\\
  &\mapsto (\phi_1(a), \ldots, \phi_g(a))\otimes c/b\in\left(\widehat{B}_1\times\cdots\times\widehat{B}_g\right)\otimes_{\widehat{A}_\mathfrak{p}}\widehat{K}_\mathfrak{p}\\
  &\mapsto (\phi_1(a)c/b, \ldots, \phi_g(a)c/b)\in\widehat{L}_1\times\cdots\times\widehat{L}_g.
\end{align*}
これは$K$加群としての同型であるが,$\widehat{K}_\mathfrak{p}$代数としての環同型にもなる.

$L = K(\alpha)$,$\alpha$の$K$上最小多項式を$f(x)$,$f(x)$の$\widehat{K}_\mathfrak{p}$での因数分解を$f_1(x), \ldots, f_g(x)\in\widehat{K}_\mathfrak{p}[x]$とおく.
$f_i(x)$の根を$\alpha_i$,$\phi_i(\alpha) = \alpha_i$とする.
$K$はPIDで,$K_\mathfrak{p}$はtorsion-freeなので,$\widehat{K}_\mathfrak{p}$は$K$上平坦である.従って,
\begin{align*}
  L\otimes_K\widehat{K}_\mathfrak{p} &\simeq \left(K[x]/(f(x))\right)\otimes_K\widehat{K}_\mathfrak{p} \simeq \left(K[x]\otimes_K\widehat{K}_\mathfrak{p}\right)/(f(x)) \simeq \widehat{K}_\mathfrak{p}[x]/(f(x)) \\
  &\simeq \widehat{K}_\mathfrak{p}[x]/(f_1(x))\times\cdots\times\widehat{K}_\mathfrak{p}[x]/(f_g(x)) \simeq \widehat{K}_\mathfrak{p}(\alpha_1)\times\cdots\times\widehat{K}_\mathfrak{p}(\alpha_g)
\end{align*}
となる.
$L\otimes_K\widehat{K}_\mathfrak{p}\simeq\widehat{L}_1\times\cdots$なので,$\widehat{K}_\mathfrak{p}$代数の同型
\[\widehat{L}_1\simeq\widehat{K}_\mathfrak{p}(\alpha_1), \ldots, \widehat{L}_g\simeq\widehat{K}_\mathfrak{p}(\alpha_g)\]
を得る.

\begin{screen}
  $[L:K] = 1$ならば$L = K$
\end{screen}
\begin{proof}
  $L\supset K$は明らか.$L$の$K$基底として$\{v\}$が取れる.
  $L$の任意の元は$kv\ (k\in K, v\in L)$と表すことができる.
  特に,$1 = kv$なので$v = k^{-1}\in K$.
  よって,$L$の任意の元は$K$の元の積となるので$K$の元:$L\subset K$.
\end{proof}

\begin{screen}
  $\widehat{B}_i$は$\widehat{A}_\mathfrak{p}$上整
\end{screen}
\begin{proof}
  $\widehat{B}_i$は有限生成$\widehat{A}_\mathfrak{p}$加群.
  $\forall x\in\widehat{B}_i$に対し,$\widehat{A}_\mathfrak{p}[x]$はやはり$\widehat{A}_\mathfrak{p}$加群$\widehat{B}_i$の元となる.
  つまり$\widehat{A}_\mathfrak{p}[x]\subset\widehat{B}_i$.
  命題I-8.1.3から$x$は$\widehat{A}_\mathfrak{p}$上整である.
  よって,$\widehat{B}_i$は$\widehat{A}_\mathfrak{p}$上整.
\end{proof}

\paragraph{定理1.3.26}~
\begin{screen}
  \begin{lem}\label{sigmaB}
    $L/K$がGalois拡大なら,$\forall\sigma\in\Gal(L/K)$に対し$\sigma(B) = B$である
  \end{lem}
\end{screen}
\begin{proof}
  $b\in B\subset L$とすれば$b$は$A$上整なので,系I-8.1.11から,$b$の$L$における$K$上の共軛は全て$A$上整となる.
  つまり,$B$の元である.
  $\Gal$は共軛の置換であることに注意して,$\forall\sigma\in\Gal(L/K)$に対し$\sigma(B)\subset B$.
  両辺に$\sigma$をかけて$\sigma^2(B)\subset\sigma(B)\subset B$.
  これを繰り返せば,$B = \sigma^{[L:K]}(B)\subset\cdots\subset B$となる.従って,$\sigma B = B$.
\end{proof}

\section{Hilbertの理論と分岐・不分岐}
\paragraph{命題1.4.4}~
\begin{screen}
  $\N_{L_D/K}(x)$を考えるときに$\sigma\in\Hom_K^\text{al}(L_D, L)$が出てくる理由
\end{screen}
\begin{proof}
  $\Hom_K^\text{al}(L_D, \overline{K})$は$\Hom_K^\text{al}(K, \overline{K})$に延長できるが,$\Hom_K^\text{al}(K, \overline{K}) = \Aut_K^\text{al}L\colon L\to L$なので,$\Hom_K^\text{al}(L_D, \overline{K})$は$L_D\to L$.
  従ってこれを$\Hom_K^\text{al}(L_D, L)$と書くことができる.
\end{proof}

\begin{screen}
  包含写像$\phi\colon \mathbb{F}_K\hookrightarrow\mathbb{F}_D$によって$\mathbb{F}_K\simeq\mathbb{F}_D$である
\end{screen}
\begin{proof}
  $\phi$は包含写像なので単射.証明の前半で示したように,$\forall x\in B_D$に対し,$y\in A$が存在し$x\equiv y\bmod P_D$とできる.
  \[\phi\colon \mathbb{F}_K = A/\mathfrak{p}\ni y + \mathfrak{p}\mapsto y + P_D = x + P_D\in B_D/P_D = \mathbb{F}_D\]
  であるので,$\phi$は全射となる.よって$\phi$により$\mathbb{F}_K\simeq\mathbb{F}_D$.
\end{proof}

\paragraph{定理1.4.5}~
\begin{screen}
  $\phi_L$の構成
\end{screen}
\begin{proof}
  $\sigma\in D_P\subset\Gal(L/K)$に対し,自然な準同型$\phi_L(\sigma)\colon\mathbb{F}_L = B/P\ni b + P\mapsto\sigma(b) + P\in\mathbb{F}_L$が定まる.
  $a + \mathfrak{p}\in\mathbb{F}_K$に対し,$\phi_L(\sigma)(a + \mathfrak{p}) = \sigma(a) + \mathfrak{p} = a + \mathfrak{p}$なので,$\phi_L(\sigma)$は$\mathbb{F}_K$の元に対し恒等的.
  すなわち,$\phi_L(\sigma)\in\Gal(\mathbb{F}_L/\mathbb{F}_K)$であり,$\phi_L\colon\Gal(L/K)\supset D_P\to\Gal(\mathbb{F}_L/\mathbb{F}_K)$となる.
  $\ker(\phi_L)$は$\phi_L(\sigma)$が恒等的になるような$\sigma$,すなわち$\forall b\in B$に対し$\sigma(b) + P = b + P$となるような$D_P$の元である($I_P$の定義).
\end{proof}

\begin{screen}
  Hilbertの理論
\end{screen}
\begin{proof}
  $D_P/I_P\simeq\Gal(\mathbb{F}_L/\mathbb{F}_K)$である.この同型は
  \[\overline{\phi}_L\colon D_P/I_P\ni\sigma \circ I_P\mapsto (b + P\mapsto\sigma(b) + P) \in \Gal(\mathbb{F}_L/\mathbb{F}_K)\]
  で与えられる.
\end{proof}

\paragraph{例1.4.9}~
\begin{screen}
  $\mathcal{O}_K$での素イデアル分解
\end{screen}
\begin{proof}
  \begin{align*}
    \mathcal{O}_K/(2) &\simeq \mathbb{F}_2[x, y]/(x^2-2, y^2-y-1) \\
    &\simeq \mathbb{F}_2[x, y]/(x^2, y^2 + y + 1) \\
    &\simeq\mathbb{F}_4[x]/(x^2)
  \end{align*}
  $\mathbb{F}_4$は$\mathbb{F}_2[\omega] = \{0, 1, \omega, \omega + 1\}$である.
  $\mathbb{F}_4[x]/(x^2)$において,$0, x, \omega x, (1 + \omega)x$以外の元は可逆(6乗すれば$1$になる).
  すなわち,$(x)/(x^2)$以外の元が可逆であるので,命題I-6.5.8から$\mathbb{F}_4[x]/(x^2)$は$(x)/(x^2)$を極大イデアルとする局所環である.
  先程の同型による$(x)/(x^2) = \{0, x, \omega x, \omega x + x\} + (x^2)$の逆像を求める:
  \begin{align*}
    \mathbb{F}_4[x]/(x^2) \supset \{0, x, \omega x, \omega x + x\} + (x^2) &\mapsto \{0, x, xy, xy + x\} + (x^2, y^2 + y + 1) \in \mathbb{F}_2[x, y]/(x^2, y^2 + y + 1) \\
    & =  (x) + (x^2, y^2 + y + 1) \in \mathbb{F}_2[x, y]/(x^2, y^2 + y + 1) \\
    &\mapsto (\sqrt{2}) + (2) \subset \mathcal{O}_K/(2).
  \end{align*}
  従って,$\mathbb{F}_4[x]/(x^2)$の極大イデアルは$(x)/(x^2)$で,これに対応する$\mathcal{O}_K/(2)$の極大イデアルは$(\sqrt{2})/(2)$となる.よって,
  \begin{align*}
    \mathbb{F}_4 &\simeq \mathbb{F}_4[x]/(x) \\
    &\simeq (\mathbb{F}_4[x]/(x^2))/((x)/(x^2)) \\
    &\simeq (\mathcal{O}_K/(2))/(\sqrt{2}/(2)) \\
    &\simeq \mathcal{O}_2/(\sqrt{2}).
  \end{align*}
  従って$(\sqrt{2})$は$2\mathbb{Z}$の上にある$\mathcal{O}_K$の素イデアルで,$2\mathcal{O}_K = (\sqrt{2})^2$であるので,分岐指数は$e((\sqrt{2})/2\mathbb{Z}) = 2$である.
\end{proof}

\begin{screen}
  $\mathbb{F}_{25}(y)/((y-3)^2)$は$(y-3)/((y-3)^2)$を極大イデアルとする局所環
\end{screen}
\begin{proof}
  $\mathbb{F}_{25}(y)/((y-3)^2)$から$(y-3)/((y-3)^2)$を除いた集合の元$a(y-3) + b\ (a, b\in\mathbb{F}_5, b\neq 0)$を考える.
  $(a(y-3) + b)^{120}\equiv b^{120} = (b^{24})^{10} = 1^{10} = 1$なので,命題I-6.5.8から$\mathbb{F}_{25}[x]/((y-3)^2)$は$(y-3)/((y-3)^2)$を極大イデアルとする局所環である.
\end{proof}

\begin{screen}
  $K$を体,$f(x)$を$K$のモニック既約多項式,$f(x)$の根を$\alpha\in\overline{K}\setminus K$として,$K[x]/(f(x))\simeq K[\alpha]$
\end{screen}
\begin{proof}
  系I-6.6.14から$K(x)/(f(x))$は体.準同型$\phi\colon K[x]/(f(x))\ni g(x) + (f(x))\mapsto g(\alpha)\in K[\alpha]$を考える.系I-6.1.28から$\phi$は単射である.全射性は明らか.よって,$K[x]/(f(x))\simeq K$.

  例えば$x^2-x-1$が既約となる$\mathbb{F}_p$(実際には$p\equiv 1, 4\bmod5$)として,$\mathbb{F}_p[x]/(x^2-x-1)\simeq\mathbb{F}_p[\phi]\simeq\mathbb{F}_{p^2}$.
\end{proof}

\begin{screen}
  $K$を体,$f(x)$を$K$の1次モニック多項式として,$K[x]/(f(x))\simeq K$
\end{screen}
\begin{proof}
  $\alpha\in K$を$f(x)$の根として,$\phi\colon K[x]/(f(x))\ni a + (f(x))\mapsto a\in K$.明らかに全単射.
\end{proof}

\begin{screen}
  $K$を体,$f(x)$を$K$の2次モニック可約多項式として,$K[x]/(f(x))\simeq K\times K$
\end{screen}
\begin{proof}
  $\alpha, \beta\in K$を$f(x)$の異なる根として,準同型
    \[\phi\colon K[x]/(f(x))\ni ax + b + (f(x))\mapsto(a\alpha + \beta, a\beta + b)\in K\times K\]
    を考える.
    $a\alpha + b = a\beta + b = 0$であれば$a = 0, b = 0$となるので$\ker(\phi) = 0$となり$\phi$は単射である.
    $\forall c, d\in K\times K$に対し,$a = (c-d)(\alpha-\beta)^{-1}, b = c-\alpha(c-d)(\alpha-\beta)^{-1}$とすれば$\phi(a + bx + (f(x))) = (c, d)$となるので$\phi$は全射.
\end{proof}

\paragraph{命題1.4.11}~
\begin{screen}
  不分岐性はGalois群の作用で不変
\end{screen}
\begin{proof}
  $L/K$をGalois拡大,$M$を中間体とする.
  $M/K$が不分岐拡大であれば,任意の$A$の素イデアル$\mathfrak{p}$と,$\mathfrak{p}$の上にある$B\cap M$の素イデアル$P_1, P_2, \ldots$について,$\mathfrak{p}(B\cap M) = P_1P_2\cdots$が成立する.
  これに$\sigma\in\Gal(L/K)$を作用させれば$\sigma(\mathfrak{p})\sigma(B\cap M) = \sigma(P_1)\cdots$.
  $\mathfrak{p}\subset A\subset K$なので$\sigma(\mathfrak{p}) = \mathfrak{p}$.
  追加補題\ref{sigmaB}(p.\pageref{sigmaB})から,$\sigma(P_i\cap B) = \sigma(P_i)\cap\sigma(B) = \sigma(P_i)\cap B$.
  以上まとめて,$\mathfrak{p}B\cap\sigma(M) = \sigma(P_1)\cdots$.
  ところで,$P_i$は$M\cap B$の素イデアルなので$\sigma^{-1}(a), \sigma^{-1}(b)\in M\cap B$に対し$\sigma^{-1}(a)\sigma^{-1}(b)\in P_i$ならば$\sigma^{-1}(a)\in P_i$もしくは$\sigma^{-1}(b)\in P_i$.
  これに$\sigma$を作用させて$a, b\in\sigma{M}\cap B$に対し$\sigma(\sigma^{-1}(a)\sigma^{-1}(b)) = ab\in\sigma(P_i)$ならば$a\in\sigma{P_i}$もしくは$b\in\sigma{P_i}$.
  よって$\sigma(P_i)$は$B\cap\sigma(M)$の素イデアル.
  よって,$\sigma(P_1), \ldots$は$\mathfrak{p}$の上にある$B\cap\sigma(M)$の全ての素イデアルで,分岐指数は$e(\sigma(P_i)/\mathfrak{p}) = 1$.
  よって$\sigma(M)$は$K$の不分岐拡大.

  上の話で$M = L\ (B\cap M = B)$とすれば$\mathfrak{p}$の上にある$B$の素イデアル$P_1, P_2, \ldots$は$\Gal(L/K)$によって互いに写りあう(このうち$P_i\to P_i$になるのが$P_i$の分解群).
  定理1.3.26から$P_i = \sigma(P_j)$なる$\sigma\in\Gal(L/K)$は存在するので,$\forall{\sigma}\in\Gal(L/K)$によって$P_i$は$P_1, P_2, \ldots$に写る.
  つまり,$\Gal(L/K)$は$\mathfrak{p}$の上にある$B$の素イデアルに推移的に作用する.
\end{proof}

\paragraph{命題1.4.12}~
\begin{screen}
  $L/K$がGalois拡大なら$\widehat{L}_1/\widehat{K}_\mathfrak{p}$もGalois拡大である
\end{screen}
\begin{proof}
  $\sigma\in D_1$は$P_1$進距離を変えないので,$(x_n)$が$L$のCauchy列であれば$(\sigma(x_n))$も$L$のCauchy列.
  さらに$\sigma\in\Gal(L/K)$は$\widehat{K}_\mathfrak{p}$を不変にする.
  よって群準同型$\phi\colon D_1\to\Aut_{\widehat{K}_\mathfrak{p}}^\text{al}(\widehat{L}_1)$を考えることができる.
  ここで,$\sigma$が$\widehat{L}_1$に自明に作用するなら,$\sigma$は$L$にも自明に作用するので,$\ker\phi = 1$,つまり$\phi$は単射である.
  よって,$|D_1|\leq|\Aut_{\widehat{K}_\mathfrak{p}}^\text{al}(\widehat{L}_1)|$.
  $L/K$はGalois拡大であるので,命題1.4.7から$|D_1| = ef$.命題I-7.4.3 (1)から$|\Aut_{\widehat{K}_\mathfrak{p}}^\text{al}(\widehat{L}_1)|\leq[\widehat{L}_1:\widehat{K}_\mathfrak{p}]$.
  さらに定理1.3.23 (4)から$[\widehat{L}_1:\widehat{K}_\mathfrak{p}] = ef$.
  以上から$ef = |D_1|\leq|\Aut_{\widehat{K}_\mathfrak{p}}^\text{al}(\widehat{L}_1)|\leq[\widehat{L}_1:\widehat{K}_\mathfrak{p}] = ef$.
  よって$\phi$は全単射で$|\Aut_{\widehat{K}_\mathfrak{p}}^\text{al}(\widehat{L}_1)| = [\widehat{L}_1:\widehat{K}_\mathfrak{p}]$.
  命題I-7.4.3 (2)から$\widehat{L}_1/\widehat{K}_\mathfrak{p}$はGalois拡大である.
  さらに,$\Gal(\widehat{L}_1/\widehat{K}_\mathfrak{p})$の元を$L$に制限すれば$D_1\simeq\Gal(\widehat{L}_1/\widehat{K}_\mathfrak{p})$.
\end{proof}

\section{局所体}
\paragraph{定理1.5.6}~
\begin{screen}
  $\alpha\in\mathbb{F}_{q^n}$に対し,$h(\alpha_0)\equiv0\bmod P$となる$ \alpha_0\in\mathcal{O}_K$が存在し,異なる$\alpha$に対しては異なる$\alpha_0$が定まる(Henselの補題を使うにはこれを示す必要がある)
\end{screen}
\begin{proof}
  同型を$\phi\colon\mathbb{F}_{q^n}\to\mathcal{O}_{K^n}/P$と表す.
  $(\mathbb{F}^{q^n-1})^\times$は位数$q^n-1$の群なので,$\{\phi(\alpha)\}^{q^n-1} = \phi(\alpha^{q^n-1}) = \phi(1) = 1$.
  従って,$\phi(\alpha)\in\mathcal{O}_{K^n}/P$は$x^{q^n-1}-1$の根である.
  $\phi(\alpha)$の代表元を$\alpha_0$とすれば,これは$(\alpha_0 + P)^{q^n-1} = \alpha_0{}^{q^n-1} + P = 1 + P$を意味する.
  従って,$\alpha_0\in\mathcal{O}_K$は$h(\alpha_0)\equiv0\bmod P$となる.
  $\phi$は単射なので異なる$\alpha$に対しては異なる$\phi(\alpha)$が対応し,その代表元も当然異なる.
\end{proof}

\setcounter{section}{6}
\section{絶対判別式}
\paragraph{系1.7.5}~
\begin{screen}
  代数体$K$の整数環$\mathcal{O}_K$の$\mathbb{Z}$基底$\boldsymbol{v} = \{v_1, \ldots, v_n\}, \boldsymbol{w} = \{w_1, \ldots, w_n\}$について$\boldsymbol{w} = A\boldsymbol{b}$となる$A\in\GL_n(\mathbb{Z})$があり,$\det{A} = \pm 1$である
\end{screen}
\begin{proof}
  系I-8.1.25から$\mathcal{O}_K$は$\mathbb{Z}$加群.
  $w_1, \ldots, w_n\in\mathcal{O}_K$なので,$B\in M_n(\mathbb{Z})$が存在し$\boldsymbol{w} = B\boldsymbol{v}$.
  同様に,$C\in\ M_n(\mathbb{Z})$が存在し$\boldsymbol{v} = C\boldsymbol{w}$.
  よって$B, C$は互いの逆行列になっていて,主張の$A\in\GL_n(\mathbb{Z})$が存在する.
  系I-6.7.9 (1)から$\det A\in\mathbb{Z}^\times = \{\pm 1\}$.
\end{proof}

\begin{screen}
  代数体$K$の$\mathbb{Q}$基底のうち$K$の整数環$\mathcal{O}_K$に含まれるものを$\boldsymbol{v} = \{v_1, \ldots, v_n\}$として,$V$を$\boldsymbol{v}$で生成される$\mathbb{Z}$加群とする.
  $K$の整数環$\mathcal{O}_K$の$\mathbb{Z}$基底を$\boldsymbol{w} = \{w_1, \ldots, w_n\}$として$v_i = \sum_ja_{ij}w_j$, $A = (a_{ij})$とすれば,$|\det A| = |\mathcal{O}_K/V|$である
\end{screen}
\begin{proof}
  $\mathcal{O}_K$は$\boldsymbol{w}$を基底とする$\mathbb{Z}$加群なので,命題I-6.8.26から,$\mathcal{O}_K\simeq\bigoplus_{\boldsymbol{w}}\mathbb{Z}\simeq\mathbb{Z}^n$($\mathbb{Z}$加群としての同型).また,
  \[V = v_1\mathbb{Z} + \cdots + v_n\mathbb{Z} = \sum_ia_{1i}w_i\mathbb{Z} + \cdots + \sum_ia_{ni}w_i\mathbb{Z} = \left(\sum_ia_{i1}\mathbb{Z}\right)w_1 + \cdots + \left(\sum_ia_{in}\mathbb{Z}\right)w_n\]
  であることに注意して,次の準同型を考える:
  \[\phi\colon {}^tA\mathbb{Z}^n \ni \left(\sum_ia_{i1}t_i, \ldots, \sum_ia_{in}t_i\right) \mapsto \left(\sum_ia_{i1}t_i\right)w_1 + \cdots + \left(\sum_ia_{in}t_i\right)w_n \in V.\]
  $\phi$は明らかに全射.
  $V\subset\mathcal{O}_K$で,$\boldsymbol{w}$は$\mathcal{O}_K$の$\mathbb{Z}$基底なので$\mathbb{Z}$上1次独立.
  $v_i\in\mathcal{O}_K$で$\boldsymbol{w}$は$\mathcal{O}_K$の$\mathbb{Z}$基底であるので,$a_{ij}\in\mathbb{Z}$.
  よって,$(\sum_ia_{i1}t_i)w_1 + \cdots + (\sum_ia_{in}t_i)w_n = 0$ならば$\mathbb{Z}\ni\sum_ia_{i1}t_i = \cdots = \sum_ia_{in}t_n = 0$.
  つまり,$\ker\phi = 0$なので$\phi$は単射.以上から,$V\simeq{}^tA\mathbb{Z}^n$.
  よって系1.6.3で$A = \mathbb{Z}, P = {}^tA$とすれば,$|\mathbb{Z}/(\det A)| = |\mathbb{Z}^n/{}^tA\mathbb{Z}^n|$.
  $\phi$の延長によって$\mathbb{Z}^n\simeq\mathcal{O}_K$,$\phi\colon{}^tA\mathbb{Z}^n\simeq V$なので,$|\mathbb{Z}^n/{}^tA\mathbb{Z}^n| = |\mathcal{O}_K/V|$.
  よって,$|\det A| = |\mathcal{O}_K/V|$.
\end{proof}

\section{相対判別式}
\paragraph{補題1.8.3}~
\begin{screen}
  $\det P\in A_\mathfrak{p}^\times$
\end{screen}
\begin{proof}
  $\ord_\mathfrak{p}(\varDelta_{L/K}(v_1, \ldots, v_n))\leq 1$なので,加法的付値の性質(命題I-8.4.5 (1))から,
  \[\ord_\mathfrak{p}((\det P)^2\varDelta_{L/K}(w_1, \ldots, w_n)) = 2\ord_\mathfrak{p}(\det P) + \ord_\mathfrak{p}(\varDelta_{L/K}(w_1, \ldots, w_n))\leq 1.\]
  よって,$\ord_\mathfrak{p}(\det P) = 0$であり,$\det P\in A_\mathfrak{p}^\times$.
\end{proof}

\paragraph{補題1.8.7}~
\begin{screen}
  $s\in\mathfrak{p}^n$をかける写像$I/J\to I/J$が全単射ならば自然な写像$\mathfrak{p}^nI/\mathfrak{p}^nJ\to I/J$が全射
\end{screen}
\begin{proof}
  自然な写像$\mathfrak{p}^nI/\mathfrak{p}^nJ\ni a + \mathfrak{p}^nJ\mapsto a + J\in I/J$はwell-defined.
  $s$をかける写像$I/J\to I/J$が全単射であるので,$a + J\in I/J$に対し$a's + J = a + J$となる$a'\in J$が存在する.
  従って,自然な写像は次のように書ける:
  \[\mathfrak{p}^nI/\mathfrak{p}^nJ\ni a's + \mathfrak{p}^nJ\mapsto a's + J = a + J\in I/J.\]
  従って自然な写像は全射.
\end{proof}

\paragraph{命題1.8.6}~
\begin{screen}
  $I/J\simeq I/\mathfrak{a}_1I\oplus\cdots\oplus I/\mathfrak{a}_tI$
\end{screen}
\begin{proof}
  $\mathfrak{a}_1\cdots\mathfrak{a}_t(I/J) = \{0\}$なので$I/J\simeq (I/J)/((\mathfrak{a}_1\cdots\mathfrak{a}_t)(I/J))$.
  $\mathfrak{a}_i + \mathfrak{a}_j = A\ (i\neq j)$なので命題I-6.8.27から$\simeq (I/J)/(\mathfrak{a}_1I/J)\oplus\cdots\oplus(I/J)/(\mathfrak{a}_tI/J)$.
  命題I-6.8.22から$(I/J)/(\mathfrak{a}_iI/J)\simeq I/\mathfrak{a}_iI$.
\end{proof}

\begin{screen}\label{order_prime_power}
  $|A/\mathfrak{p}_1{}^a| = |A/\mathfrak{p}_1|^a$
\end{screen}
\begin{proof}
  $A\supset\mathfrak{p}_1{}^{a-1}\supset\mathfrak{p}_1{}^a$なので命題I-6.8.22から,$(A/\mathfrak{p}_1{}^a)/(\mathfrak{p}_1{}^{a-1}/\mathfrak{p}_1{}^a)\simeq A/\mathfrak{p}_1{}^{a-1}$.
  命題1.2.13 (1)から$\mathfrak{p}_1{}^{a-1}/\mathfrak{p}_1{}^a\simeq A/\mathfrak{p}_1$なので,$(A/\mathfrak{p}_1{}^a)/(A/\mathfrak{p}_1)\simeq A/\mathfrak{p}_1{}^{a-1}$となり,$((A/\mathfrak{p}_1{^a}):(A/\mathfrak{p}_1)) = |A/\mathfrak{p}_1{}^{a-1}|$.
  Lagrangeの定理から,
  \[|A/\mathfrak{p}_1{}^a| = ((A/\mathfrak{p}_1{^a}):(A/\mathfrak{p}_1))|A/\mathfrak{p}_1| = |A/\mathfrak{p}_1{}^{a-1}||A/\mathfrak{p}_1|.\]
  これを繰り返して,$|A/\mathfrak{p}_1{}^a| = |A/\mathfrak{p}_1|^a$.
\end{proof}

\paragraph{命題1.8.9}~
\begin{screen}
  $I(B_\mathfrak{p}/M\otimes_AA_\mathfrak{p}) = \{0\}$,$B_\mathfrak{p} = M \otimes_A A_\mathfrak{p}$
\end{screen}
\begin{proof}
  1つ目については, $IB_\mathfrak{p} \subset M \otimes_A A_\mathfrak{p}$を示せばよい.
  $B \otimes_A A_\mathfrak{p} = B \otimes_A S^{-1}A \simeq S^{-1}B = B_\mathfrak{p}$なので(補題1.3.22),$IB_\mathfrak{p} \simeq I(B \otimes_A A_\mathfrak{p})$.ここで$I\subset A$なので,$A$上テンソル積の双線形性から右辺は$(IB) \otimes_A A_\mathfrak{p}$に等しい.
  $I$の定義から$IB \subset M$なので,$IB_\mathfrak{p} \simeq (IB) \otimes_A A_\mathfrak{p} \subset M \otimes_A A_\mathfrak{p}$となる.さらに,$I$は$A_\mathfrak{p}$の単元を含むので,$B_\mathfrak{p} = IB_\mathfrak{p} \subset M \otimes_A A_\mathfrak{p}$.
  $M \otimes_A A_\mathfrak{p} \subset B \otimes_A A_\mathfrak{p} \simeq B_\mathfrak{p}$なので,$B_\mathfrak{p} = M \otimes_A A_\mathfrak{p}$となる.
\end{proof}

\begin{screen}
  $s\phi(\pi_1(c)) = \phi(s\pi_1(c)) = \phi(\pi_1(b)) = \pi_2(b/1) = s\pi_2(b/s)$
\end{screen}
\begin{proof}
  自然な写像$\psi\colon B\hookrightarrow B_\mathfrak{p}$, $\psi$が引き起こす準同型$\phi\colon B/(\mathfrak{p}^aB + M)\to B_\mathfrak{p}/(\mathfrak{p}^aB_\mathfrak{p} + S^{-1}M)$,
  自然な写像$\pi_1\colon B\to B/(\mathfrak{p}^aB + M)$,
  $\pi_2\colon B_\mathfrak{p}\to B_\mathfrak{p}/(\mathfrak{p}^aB_\mathfrak{p} + S^{-1}M)$に対し,次のような図式:
  \[
  \begin{tikzcd}
    B \ni b \arrow[r, "\psi", mapsto]\arrow[d, "\pi_1", mapsto] & b/1\in B_\mathfrak{p}\arrow[d, "\pi_2", mapsto] \\
    B/(\mathfrak{p}^aB + M) \ni [b] \arrow[r, "\phi", mapsto] & {[b/1]}\in B_\mathfrak{p}/(\mathfrak{p}^aB_\mathfrak{p} + S^{-1}M)
  \end{tikzcd}
  \]
  が得られる.
  $\pi_1(c) = b' + (\mathfrak{p}^aB + M)$とすれば,
  \begin{align*}
    & s\phi(\pi_1(c)) = s\phi(b' + (\mathfrak{p}^aB + M)) = s[b'/1 + (\mathfrak{p}^aB_\mathfrak{p} + S^{-1}M)] = sb'/1 + (\mathfrak{p}^aB_\mathfrak{p} + S^{-1}M) \\
    & =  \phi(sb' + (\mathfrak{p}^aB + M)) = \phi(s(b' + (\mathfrak{p}^aB + M))) = \phi(s\pi_1(c)).
  \end{align*}

  $s\pi_1(c) = \pi_1(b)$なので$\phi(s\pi_1(c)) = \phi(\pi_1(b)) = \pi_2(\psi(b)) = \pi_2(b/1)$.さらに,
  \[\pi_2(b/1) = b/1 + (\mathfrak{p}^aB_\mathfrak{p} + S^{-1}M) = s[b/s + (\mathfrak{p}^aB_\mathfrak{p} + S^{-1}M)] = s\pi_2(b/s).\]
\end{proof}

\begin{screen}
  $\phi(\pi_1(b)) = 0$であれば$c\in\mathfrak{p}^aB + M$と$s\in S$があり,$b/1\in c/s$
\end{screen}
\begin{proof}
  $\phi(\pi_1(b)) = b/1 + (\mathfrak{p}^aB_\mathfrak{p} + S^{-1}M)$なので,$b/1\in(\mathfrak{p}^aB_\mathfrak{p} + S^{-1}M)$である.
  よって,$b/1 = c/s\in B_\mathfrak{p}$となる$c, s$が存在する.
\end{proof}

\begin{screen}
  $N$は有限生成・自由$A_\mathfrak{p}$加群
\end{screen}
\begin{proof}
  補題I-8.3.3から$A_\mathfrak{p}$はDedekind環(つまりNoether環でもある).
  $A_\mathfrak{p}$の商体は$K$で,命題I-8.1.14から$A_\mathfrak{p}$の$L$における整閉包は$B_\mathfrak{p}$.
  よって命題I-8.1.24から$B_\mathfrak{p}$は有限生成$A_\mathfrak{p}$加群.
  $A_\mathfrak{p}$はNoether環なので,命題I-6.8.36から$N\subset B_\mathfrak{p}$は有限生成$A_\mathfrak{p}$加群.
  $N$は有限生成$A_\mathfrak{p}$加群でねじれがないので,自由$A_\mathfrak{p}$加群である(定理I-6.8.38).
\end{proof}

\begin{screen}
  $(B_\mathfrak{p}:N)<\infty$ならば$N$と$B_\mathfrak{p}$の階数は等しい
\end{screen}
\begin{proof}
  $N$は$B_\mathfrak{p}$の部分加群なので,$N$の$A_\mathfrak{p}$基底として$\{\boldsymbol{x}_1, \ldots, \boldsymbol{x}_n\}$, $B_\mathfrak{p}$の$A_\mathfrak{p}$基底として$\{\boldsymbol{x}_1, \ldots, \boldsymbol{x}_m\}$を取ることができる.
  写像$\phi\colon B_\mathfrak{p}\ni a_1\boldsymbol{x}_1 + \cdots + a_m\boldsymbol{x}_m\mapsto(a_1, \ldots, a_m)\in A_\mathfrak{p}{}^m$によって$B_\mathfrak{p}\simeq A_\mathfrak{p}{}^m$となる.
  $N$の元は$a_{n + 1} = \cdots = a_m = 0$となる$B_\mathfrak{p}$の元と見做すことによって,$\phi$を制限すれば$N\simeq A_\mathfrak{p}{}^n$となる.
  よって$\phi$は同型$B_\mathfrak{p}/N\simeq A_\mathfrak{p}{}^m/A_\mathfrak{p}{}^n$を引き起こす.
  右辺の代表元として$\{(0, \ldots, 0, a_{n + 1}, \ldots, a_m)\mid a_i\in A_\mathfrak{p}\}$を取ることができ,これが有限個となるのは$n = m$の時だけ.
\end{proof}

\begin{screen}
  $(B_\mathfrak{p}:N) = |A_\mathfrak{p}/(\det P)A_\mathfrak{p}|$
\end{screen}
\begin{proof}
  $B_\mathfrak{p}$は自由$A_\mathfrak{p}$加群であるので,自由基底$\{\boldsymbol{x}_1, \ldots, \boldsymbol{x}_n\}$があり,$B_\mathfrak{p} = \boldsymbol{x}_1A_\mathfrak{p} + \cdots + \boldsymbol{x}_nA_\mathfrak{p}$.準同型
  \[\phi\colon B_\mathfrak{p}\ni\boldsymbol{x}_1a_1 + \cdots + \boldsymbol{x}_na_n\mapsto (a_1, \ldots, a_n)\in A_\mathfrak{p}{}^n\]
  は全単射なので,$\phi$によって$B_\mathfrak{p}\simeq A_\mathfrak{p}{}^n$となる.
  $N$が自由基底$\{\boldsymbol{y}_1, \ldots, \boldsymbol{y}_n\}$で生成されるとする.
  $\boldsymbol{y}_i = \sum_{j}\boldsymbol{x}_jp_{ji}$とすれば,
  \begin{align*}
    \phi & \colon N = \boldsymbol{y}_1A_\mathfrak{p} + \cdots + \boldsymbol{y}_nA_\mathfrak{p}\ni\boldsymbol{y}_1a_1 + \cdots + \boldsymbol{y}_na_n = \sum_ip_{1i}a_i\boldsymbol{x}_1 + \cdots + \sum_ip_{ni}a_i\boldsymbol{x}_n \\
    & \mapsto \left(\sum_ip_{1i}a_i, \ldots, \sum_ip_{ni}a_i\right)\in PA_\mathfrak{p}{}^n
  \end{align*}
  によって$N\simeq PA_\mathfrak{p}{}^n$.以上から,$\phi$は写像
  \[B_\mathfrak{p}/N\ni b + N\mapsto \phi(b) + PA_\mathfrak{p}{}^n\in A_\mathfrak{p}{}^n/PA_\mathfrak{p}{}^n\]
  を引き起こし,これによって$B_\mathfrak{p}/N\simeq A_\mathfrak{p}{}^n/PA_\mathfrak{p}{}^n$.よって,系1.6.3から
  \[(B_\mathfrak{p}:N) = (A_\mathfrak{p}{}^n:PA_\mathfrak{p}{}^n) = |A_\mathfrak{p}{}/(\det P)A_\mathfrak{p}|.\]
\end{proof}

\begin{screen}
  $(B_\mathfrak{p}\otimes_{A_\mathfrak{p}}\widehat{A}_\mathfrak{p}:N\otimes_{A_\mathfrak{p}}\widehat{A}_\mathfrak{p}) = |A_\mathfrak{p}/(\det P)A_\mathfrak{p}|$
\end{screen}
\begin{proof}
  先に示した写像$\phi\colon B_\mathfrak{p}\simeq A_\mathfrak{p}{}^n$,例1.3.13,命題1.3.15を使えば,
  \[
  \begin{tikzcd}
    B_\mathfrak{p}\otimes_{A_\mathfrak{p}}\widehat{A}_\mathfrak{p} \arrow[r, "\phi\otimes\id"]\arrow[d, phantom, "\ni" sloped] & A_\mathfrak{p}{}^n\otimes_{A_\mathfrak{p}}\widehat{A}_\mathfrak{p} \arrow[r, "\simeq"]\arrow[d, phantom, "\ni" sloped] & \widehat{A}_\mathfrak{p}{}^n \arrow[d, phantom, "\ni" sloped] \\
    (\boldsymbol{x}_1a_1 + \cdots + \boldsymbol{x}_na_n)\otimes a \arrow[r, mapsto] & (a_1, \ldots, a_n)\otimes a \arrow[r, mapsto] & (a_1a, \ldots, a_na)
  \end{tikzcd}
  \]
  によって同型$B_\mathfrak{p}\otimes_{A_\mathfrak{p}}\widehat{A}_\mathfrak{p}\simeq \widehat{A}_\mathfrak{p}{}^n$が得られる($\widehat{A}_\mathfrak{p}$は平坦$A$加群).これを制限して,
  \[
  \begin{tikzcd}
    N\otimes_{A_\mathfrak{p}}\widehat{A}_\mathfrak{p} \arrow[r, "\phi\otimes\id"]\arrow[d, phantom, "\ni" sloped] & PA_\mathfrak{p}{}^n\otimes_{A_\mathfrak{p}}\widehat{A}_\mathfrak{p} \arrow[r, "\simeq"]\arrow[d, phantom, "\ni" sloped] & P\widehat{A}_\mathfrak{p}{}^n \arrow[d, phantom, "\ni" sloped] \\
    (\boldsymbol{y}_1a_1 + \cdots + \boldsymbol{y}_na_n)\otimes a \arrow[r, mapsto] & \left(\sum_ip_{1i}a_i, \ldots, \sum_ip_{ni}a_i\right)\otimes a \arrow[r, mapsto] & \left(\sum_ip_{1i}a_ia, \ldots, \sum_ip_{ni}a_ia\right)
  \end{tikzcd}
  \]
  となる.よって,2つの同型$B_\mathfrak{p}\otimes_{A_\mathfrak{p}}\widehat{A}_\mathfrak{p}\simeq\widehat{A}_\mathfrak{p}{}^n$及び$N\otimes_{A_\mathfrak{p}}\widehat{A}_\mathfrak{p}\simeq P\widehat{A}_\mathfrak{p}{}^n$は同じ写像で構成される.
  さらに,この写像は同型$B_\mathfrak{p}\otimes_{A_\mathfrak{p}}\widehat{A}_\mathfrak{p}/N\otimes_{A_\mathfrak{p}}\widehat{A}_\mathfrak{p}\simeq\widehat{A}_\mathfrak{p}{}^n/P\widehat{A}_\mathfrak{p}{}^n$を引き起こす.
  系1.6.3から$(B_\mathfrak{p}\otimes_{A_\mathfrak{p}}\widehat{A}_\mathfrak{p}:N\otimes_{A_\mathfrak{p}}\widehat{A}_\mathfrak{p}) = |A_\mathfrak{p}/(\det P)A_\mathfrak{p}|$.
\end{proof}

\paragraph{命題1.8.11}~
\begin{screen}
  $\displaystyle W/V\simeq \bigoplus_{i = 1}^tB_i{}^n/(\mathfrak{p}_i{}^{a_i}B_i{}^n + {}^tPB_i{}^n)$
\end{screen}
\begin{proof}
  準同型$W/JW\ni w + JW\mapsto w + V\in W/V$の$\ker$は$V/JW$なので$(W/JW)/(V/JW)\simeq W/V$.準同型$\phi\colon W = w_1A + \cdots + w_nA\ni w_1a_1 + \cdots + w_na_n\mapsto(a_1, \ldots, a_n)\in A^n$によって$W/JW\simeq A^n/JA^n$.中国式剰余定理から準同型
  \[\varphi\colon A^n/JA^n\ni(a_1, \ldots, a_n) + JA^n\mapsto \left((a_1, \ldots, a_n) + \mathfrak{p}_i{}^{a_i}A^n\right)_{1\leq i\leq t} \in\bigoplus_{i = 1}^tA^n/\mathfrak{p}_i{}^{a_i}A^n\]
  によって$A^n/JA^n\simeq\bigoplus_{i = 1}^tA^n/\mathfrak{p}_i{}^{a_i}A^n$.命題1.2.13から準同型
  \[\psi_i\colon A^n/\mathfrak{p}_i{}^{a_i}A^n\ni (a_1, \ldots, a_n) + \mathfrak{p}_i{}^{a_i}A^n\mapsto(a_1/s, \ldots, a_n/s) + \mathfrak{p}_i{}^{a_i}A_{\mathfrak{p}_i}{}^n\in A_{\mathfrak{p}_i}{}^n/\mathfrak{p}_i{}^{a_i}A_{\mathfrak{p}_i}{}^n\]
  によって$\bigoplus_{i = 1}^t A^n/\mathfrak{p}_i{}^{a_i}A^n\simeq\bigoplus_{i = 1}^tA_{\mathfrak{p}_i}{}^n/\mathfrak{p}_i{}^{a_i}A_{\mathfrak{p}_i}{}^n$.これらを制限して,
  \begin{align*}
    \phi \colon V = v_1A + \cdots + v_nA &\ni v_1a_1 + \cdots + v_na_n = \sum_jp_{1j}w_ja_1 + \cdots + \sum_jp_{nj}w_ja_n \\
    & =  \sum_ip_{i1}a_iw_1 + \cdots\sum_ip_{in}a_iw_n \mapsto \left(\sum_ip_{i1}a_i, \ldots, \sum_ip_{in}a_i\right)\in{}^tPA^n
  \end{align*}
  によって$V/JW\simeq{}^tPA^n/JA^n$.
  \begin{align*}
    \varphi &\colon {}^tPA^n/JA^n\ni\left(\sum_jp_{j1}a_i, \ldots, \sum_jp_{jn}a_i\right) + JA^n \\
    & \mapsto \left( \left(\sum_jp_{j1}a_i, \ldots, \sum_jp_{jn}a_i\right) + \mathfrak{p}_i{}^{a_i}A^n \right)_{1\leq i\leq t} \in\bigoplus_{i = 1}^t{}^tPA^n/\mathfrak{p}_i{}^{a_i}A^n
  \end{align*}
  によって$^tPA^n/JA^n\simeq\bigoplus_{i = 1}^t{}^tPA^n/\mathfrak{p}_i{}^{a_i}A^n$.
  \begin{align*}
    \psi_i &\colon {}^tPA^n/\mathfrak{p}_i{}^{a_i}A^n\ni\left(\sum_jp_{j1}a_i, \ldots, \sum_jp_{jn}a_i\right) + \mathfrak{p}_i{}^{a_i}A^n \\
    &\mapsto \left(\sum_jp_{j1}a_i/s, \ldots, \sum_jp_{jn}a_i/s\right) + \mathfrak{p}_i{}^{a_i}A_{\mathfrak{p}_i}{}^n\in{}^tPA_{\mathfrak{p}_i}{}^n/\mathfrak{p}_i{}^{a_i}A_{\mathfrak{p}_i}{}^n
  \end{align*}
  によって$\bigoplus_{i = 1}^t{}^tPA^n/\mathfrak{p}_i{}^{a_i}A^n\simeq\bigoplus_{i = 1}^t{}^tPA_{\mathfrak{p}_i}{}^n/\mathfrak{p}_i{}^{a_i}A_{\mathfrak{p}_i}{}^n$.以上から,2つの同型
  \[W/JW\simeq\bigoplus_{i = 1}^tA_{\mathfrak{p}_i}{}^n/\mathfrak{p}_i{}^{a_i}A_{\mathfrak{p}_i}{}^n, \quad V/JW\simeq\bigoplus_{i = 1}^t{}^tPA_{\mathfrak{p}_i}{}^n/\mathfrak{p}_i{}^{a_i}A_{\mathfrak{p}_i}{}^n\]
  は同じ写像によって構成される.よって,
  \[W/V\simeq(W/JW)/(V/JW)\simeq\bigoplus_{i = 1}^t(A_{\mathfrak{p}_i}{}^n/\mathfrak{p}_i{}^{a_i}A_{\mathfrak{p}_i}{}^n)/(^tPA_{\mathfrak{p}_i}{}^n/\mathfrak{p}_i{}^{a_i}A_{\mathfrak{p}_i}{}^n).\]
  ところで,自然な準同型
  \[A_{\mathfrak{p}_i}{}^n/\mathfrak{p}_i{}^{a_i}A_{\mathfrak{p}_i}{}^n\to A_{\mathfrak{p}_i}{}^n/(\mathfrak{p}_i{}^{a_i}A_{\mathfrak{p}_i}{}^n + {}^tPA_{\mathfrak{p}_i}{}^n)\]
  の$\ker$は
  \[(\mathfrak{p}_i{}^{a_i}A_{\mathfrak{p}_i}{}^n + {}^tPA_{\mathfrak{p}_i}{}^n)/\mathfrak{p}_i{}^{a_i}A_{\mathfrak{p}_i}{}^n = {}^tPA_{\mathfrak{p}_i}{}^n/\mathfrak{p}_i{}^{a_i}A_{\mathfrak{p}_i}{}^n\]
  なので,準同型定理から
  \[(A_{\mathfrak{p}_i}{}^n/\mathfrak{p}_i{}^{a_i}A_{\mathfrak{p}_i}{}^n)/(^tPA_{\mathfrak{p}_i}{}^n/\mathfrak{p}_i{}^{a_i}A_{\mathfrak{p}_i}{}^n)\simeq A_{\mathfrak{p}_i}{}^n/(\mathfrak{p}_i{}^{a_i}A_{\mathfrak{p}_i}{}^n + {}^tPA_{\mathfrak{p}_i}{}^n).\]
  以上から,$W/V\simeq \bigoplus_{i = 1}^tA_{\mathfrak{p}_i}{}^n/(\mathfrak{p}_i{}^{a_i}A_{\mathfrak{p}_i}{}^n + {}^tPA_{\mathfrak{p}_i}{}^n)$.
\end{proof}

\section{判別式と終結式}
\paragraph{定理1.9.3}~
\begin{screen}
  $(\alpha_i-\beta_j)$は$L[a_0, \alpha_1, \ldots, \beta_m]$の素イデアルである
\end{screen}
\begin{proof}
  準同型
  \begin{align*}
    \phi\colon L[a_0, \alpha_1, \ldots, \beta_m]/(\alpha_i-\beta_j)\ni f(a_0, \alpha_1, \ldots, \alpha_{i-1}, \alpha_{i + 1}, \ldots, \beta_m) + (\alpha_i-\beta_j) \\
    \mapsto f(a_0, \alpha_1, \ldots, \alpha_{i-1}, \alpha_{i + 1}, \ldots, \beta_m)\in L[a_0, \alpha_1, \ldots, \alpha_{i-1}, \alpha_{i + 1}, \ldots, \beta_m)]
  \end{align*}

  を考える.
  $ab\in(\alpha_i-\beta_j)$とする.
  $0 = \phi(ab) = \phi(a)\phi(b)$で$L[a_0, \alpha_1, \ldots, \alpha_{i-1}, \alpha_{i + 1}, \ldots, \beta_m)]$は体上の多項式環なので一意分解環であり(命題I-6.6.23),整域である.よって,$\phi(a) = 0$もしくは$\phi(b) = 0$である.よって,$a\in(\alpha_i-\beta_j)$もしくは$b\in(\alpha_i-\beta_j)$なので$(\alpha_i-\beta_j)$は素イデアルである.
\end{proof}

\section{イデアルの相対ノルム}
\paragraph{補題1.10.6}~
\begin{screen}
    $m(\alpha)$が基底$\{\alpha^iv_j\}$に関して$M_\alpha$が対角に$n/l$個並んだブロック行列となる
\end{screen}
\begin{proof}
  $L$の元に$\alpha\ (\alpha^l+a_1\alpha^{l-1}+\cdots+a_l=0)$をかける写像は次のようになる:
  \begin{align*}
    m(\alpha) \colon l = \sum_{j=1}^{n/l}\sum_{i=0}^{l-1}b_{ji}\alpha^iv_j \mapsto \sum_{j=1}^{n/l}\sum_{i=0}^{l-1}b_{ji}\alpha^{i+1}v_j = \sum_{j=1}^{n/l}\left(\sum_{i=0}^{l-2}b_{ji}\alpha^{i+1}v_j+b_{j,l-1}\alpha^lv_j\right) \\
    = \sum_{j=1}^{n/l}\left(\sum_{i=1}^{l-1}b_{j, i-1}\alpha^iv_j-b_{j, l-1}\sum_{i=0}^{l-1}a_{l-i}\alpha^iv_j\right).
  \end{align*}
  ある$j$に対し,$m(\alpha)(l)$の$\alpha^iv_j$成分は$i=0$に対し$-b_{j,l-1}a_l$であり,$i=1,\ldots,l-i$に対し$-b_{j, l-1}a_{l-i}+b_{j, i-1}$となり,これを行列で表せば
  \[
  \left(
  \begin{array}{cccc}
    & & & -a_l\\
    1& & & -a_{l-1}\\
    & \ddots & & \vdots\\
    & & 1 & -a_1
  \end{array}
  \right)\left(
  \begin{array}{c}
    b_{j,0}\\
    b_{j,1}\\
    \vdots\\
    b_{j, l-1}
  \end{array}
  \right)
  \]
  となる.よって,この行列をあらわに書けば
  \[
  \left(
  \begin{array}{cccc|cccc|c|cccc}
    & & & -a_l & & & & & & & &\\
    1& & & -a_{l-1} & & & & & & & &\\
    & \ddots & & \vdots & & & & & & & &\\
    & & 1 & -a_1 & & & & & & & &\\
    \hline
    & & & & & & & -a_l & & & & &\\
    & & & & 1& & & -a_{l-1} & & & & &\\
    & & & & & \ddots & & \vdots & & & & &\\
    & & & & & & 1 & -a_1 & & & & &\\
    \hline
    & & & & & &  &  & \ddots & & & &\\
    \hline
    & & & & & & & & & & & & -a_l\\
    & & & & & & & & & 1 & & & -a_{l-1}\\
    & & & & & & & & & & \ddots & & \vdots\\
    & & & & & & & & & & & 1 & -a_1
  \end{array}
  \right).
  \]
\end{proof}

\paragraph{命題1.10.7}~
\begin{screen}
    $\pi_B\in B$が$A$上整なら$\pi_B$の$K$上最小多項式$f(x)\in K[x]$は$A[x]$の元である
\end{screen}
\begin{proof}
  $\pi_B$は$A$上整なので,$h(\pi_B)=0$なるモニック$h(x)\in A[x]$が存在し,さらに命題I-7.1.11 (2)から$h(x)=g(x)f(x)$なる$g(x)\in K[x]$が存在する.
  $\pi_B=\alpha_1,\alpha_2,\ldots,\alpha_n\in\overline{K}$を$f(x)$の根とすると,これらは$h(x)$の根なので$A$上整である.
  $f(x)=\sum_{i=1}^na_ix^i$とおくと,係数$a_i$は$\alpha_i$の基本対称式となるので,やはり$A$上整である($\overline{K}$は$A$の拡大環なので命題I-8.1.4 (4)から従う).
  $f(x)\in K[x]$なので,$a_i\in K$.よって,$a_i\in K$は$A$上整となるが,$A$が整閉整域であることから,$a_i\in A$.よって$f(x)\in A[x]$.
\end{proof}

\begin{screen}
    $\pi_B\in P$, $\sigma\in\Hom^\text{al}_{K}(L,\overline{K})$に対し$\sigma(\pi_B)\in\mathcal{P}$
\end{screen}
\begin{proof}
  $\tau\in\Hom^\text{al}_K(\tilde{L},\overline{K})=\Gal(\tilde{L}/K)$について,$\tau C=C$(追加補題\ref{sigmaB}, p.\pageref{sigmaB})となる.
  $\mathcal{P}$は$C$の素イデアルなので,$a,b\in\mathcal{P}$ならば$a+b\in\mathcal{P}$となる.
  $a\in\mathcal{P}, \tau^{-1}(c)\in C\ (c\in C)$に対し$a\tau^{-1}(c)\in\mathcal{P}$.
  よって,$\tau(a),\tau(b)\in\tau(\mathcal{P})$に対し$\tau(a)+\tau(b)=\tau(a+b)\in\tau(\mathcal{P})$, $\tau(a)\in\tau(\mathcal{P}), c\in C$に対し$\tau(a)c=\tau(a)\tau(\tau^{-1}(c))=\tau(a\tau^{-1}(c))\in\tau(\mathcal{P})$となるので,$\tau(\mathcal{P})$は$C$のイデアル.
  $\mathcal{P}$は$C$の素イデアルなので$a,b\in C$について,$\tau^{-1}(a)\tau^{-1}(b)\in\mathcal{P}$ならば$\tau^{-1}(a)\in\mathcal{P}$もしくは$\tau^{-1}(b)\in\mathcal{P}$.
  よって,$\tau(\tau^{-1}(a)\tau^{-1}(b))=ab\in\tau(\mathcal{P})$ならば$a\in\tau(\mathcal{P})$もしくは$b\in\tau(\mathcal{P})$となり,$\tau(\mathcal{P})$は$C$の素イデアル.
  $C$の素イデアルは唯一なので,$\tau(\mathcal{P})=\mathcal{P}$.
  命題I-8.1.4 (2)から$C$は$B$上整.命題I-8.1.15から$P$の上にある$C$の素イデアルが存在し,これは$\mathcal{P}$.
  よって$\pi_B\in P\subset\mathcal{P}$.以上から,$\tau(\pi_B)\in\mathcal{P}$.
  $\tau$を$L$に制限すれば$\sigma\in\Hom^\text{al}_K(L,\overline{K})$になるので,$\sigma(\pi_B)\in\mathcal{P}$.
\end{proof}

\begin{screen}
  $A, B$が離散付値環で$P$の$\mathfrak{p}$上の分岐指数を$e$とすれば.
  $a\in K^\times$に対し,$\ord_P(a)=e\ord_\mathfrak{p}(a)$である
\end{screen}
\begin{proof}
  $\mathfrak{p}=\pi_AA$とすれば,$c,d\in A$によって$a=c/d=(\pi_A{}^ic')/(\pi_A{}^jd')=\pi_A{}^{i-j}c'/d'\ (c',d'\notin A\setminus\mathfrak{p})$なので$a=\pi_A{}^ns$なる$s\in\set{c/d | c,d\in A\setminus\mathfrak{p}}=A_\mathfrak{p}^\times$が存在する.
  $s^{-1}\in A_\mathfrak{p}$なので,$aA_\mathfrak{p}=\pi_A{}^nsA_\mathfrak{p}=\pi_A{}^nA_\mathfrak{p}$となり,$\ord_\mathfrak{p}(a)=n$.
  また,分岐指数が$e$なので,$\pi_AB_P=\pi_B{}^eB_P$,つまり$\ord_P(\pi_A)=e$.
  以上から$\ord_P(a)=\ord_P(\pi_A{}^ns)=n\ord(\pi_A)+\ord_P(s)$.
  ここで,$P$が$\mathfrak{p}$の上にあることに注意して$s \in \set{c/d | c,d\in B\setminus P}$なので$\ord_P(s)=0$.
  よって,$=n\ord(\pi_A)=en$.以上から,$\ord_P(a)=e\ord_\mathfrak{p}(a)$.
\end{proof}

\paragraph{補題1.10.10}~
\begin{screen}
  $\Tr_{L/K}(x)=\sum_i\Tr_{\widehat{L}_i/\widehat{K}_\mathfrak{p}}(x)$
\end{screen}
\begin{proof}
  $L$の$K$基底を$\{y_1, \ldots, y_n\}$とする.
  \[L \otimes_K \widehat{K}_\mathfrak{p} \ni a \otimes t = \left(\sum_{i=1}^n a^{(i)} y_i\right) \otimes t \mapsto \sum_{i=1}^n a^{(i)}t y_i \in  \widehat{K}_\mathfrak{p}{}^{\oplus n}\]
  によって$L \otimes_K \widehat{K}_\mathfrak{p}$は$\{y_1, \ldots, y_n\}$を基底とする階数$n$の自由$\widehat{K}_\mathfrak{p}$加群とみなせる(以下,この同型を顕には書かない).
  $L$の元に$x$をかける準同型
  \[\phi\colon L\ni a = \sum_{i=1}^n a^{(i)} y_i \mapsto ax = \sum_{i=1}^n b^{(i)} y_i \in L \quad (a^{(i)},b^{(i)}\in K)\]
  を考える.
  $\phi$は$L\otimes_K\widehat{K}_\mathfrak{p}$の元に$x\otimes1$をかける準同型
  \[\widehat{\phi} \colon L \otimes_K \widehat{K}_\mathfrak{p} \ni a \otimes t = \sum_{i=1}^n a^{(i)}t y_i \mapsto ax \otimes t = \sum_{i=1}^n b^{(i)}t y_i \in L\otimes_K\widehat{K}_\mathfrak{p}\]
  を誘導する.
  $\phi$を$K$基底$\{y_1, \ldots, y_n\}$で表現した行列と$\widehat{\phi}$を$\widehat{K}_\mathfrak{p}$基底$\{y_1, \ldots, y_n\}$で表した行列は同じなので,そのトレースは等しい:$\Tr_K(\phi)=\Tr_{\widehat{K}_\mathfrak{p}}(\widehat{\phi})$.
  補題1.10.6から$\Tr_{L/K}(x)=\Tr_K(\phi)$なので,
  \[\Tr_{L/K}(x)=\Tr_K(\phi)=\Tr_{\widehat{K}_\mathfrak{p}}(\widehat{\phi}).\]

  定理1.3.23 (2)から同型$\varphi \colon L \otimes_K \widehat{K}_\mathfrak{p} \to \widehat{L}_1 \times \cdots \times \widehat{L}_g$が存在するので,$\widehat{K}_\mathfrak{p}$準同型
  \[\phi_x=\varphi\widehat{\phi}\varphi^{-1} \colon \widehat{L}_1\times\cdots\times\widehat{L}_g\to \widehat{L}_1\times\cdots\times\widehat{L}_g\]
  が存在し,$\Tr_{\widehat{K}_\mathfrak{p}}(\widehat{\phi})=\Tr_{\widehat{K}_\mathfrak{p}}(\varphi\widehat{\phi}\varphi^{-1})=\Tr_{\widehat{K}_\mathfrak{p}}(\phi_x)$.
  \[
  \begin{CD}
    \widehat{L}_1\times\cdots\times\widehat{L}_g @>{\phi_x}>> \widehat{L}_1\times\cdots\times\widehat{L}_g\\
    @A{\varphi}AA  @A{\varphi}AA\\
    L\otimes_K\widehat{K}_\mathfrak{p} @>>{\widehat{\phi}}> L\otimes_K\widehat{K}_\mathfrak{p}\\
    @AAA @AAA\\
    \widehat{K}_\mathfrak{p} @= \widehat{K}_\mathfrak{p}
  \end{CD}
  \]
  $\varphi\widehat{\phi}=\phi_x\varphi$を$a\otimes t\in L\otimes_K\widehat{K}_\mathfrak{p}$に作用させて$\phi_x(\varphi(a\otimes t)) = \varphi(a\otimes t)\varphi(x\otimes1)$となる.
  $\varphi_i \colon B\hookrightarrow\widehat{B}_i$として,$\varphi(x\otimes1) = (\varphi_1(x), \ldots, \varphi_g(x))$である(定理1.3.23 (3))ので,$\phi_x$は$L_1\times\cdots\times L_g$の元に$(\varphi_1(x), \ldots, \varphi_g(x))$をかける写像である:
  \[\phi_x\colon\widehat{L}_1\times\cdots\times\widehat{L}_g\ni (c_1,\ldots,c_g)\mapsto(\varphi_1(x)c_1, \ldots, \varphi_g(x)c_g)\in \widehat{L}_1\times\cdots\times\widehat{L}_g.\]
  $\phi_x$を$\widehat{L}_i$に制限した$\widehat{K}_\mathfrak{p}$準同型
  \[\psi_{x,i}\colon\widehat{L}_i\ni c_i\mapsto \varphi_i(x)c_i\in\widehat{L}_i\]
  を考える.
  $[\widehat{L}_i:\widehat{K}_\mathfrak{p}]=n_i$とすれば,定理1.3.23 (4)(5)から$n_1+\cdots+n_g=n$.
  $\widehat{L}_i$の$\widehat{K}_\mathfrak{p}$基底を$\{z_i^{(1)},\ldots,z_i^{(n_i)}\}$とすれば$\widehat{L}_1\times\cdots\times\widehat{L}_g$の$\widehat{K}_\mathfrak{p}$基底として
  \[\left\{(z_1^{(1)},\ldots,0),\ldots,(z_1^{(n_1)},\ldots,0);(0,z_2^{(1)},\ldots,0),\ldots,(0,z_2^{(n_2)},\ldots,0);\ldots;(0,\ldots,z_g^{(1)}),\ldots,(0,\ldots,z_g^{(n_g)})\right\}\]
  を取ることができる.

  $\psi_{x,i}$を$\{z_i^{(1)},\ldots,z_i^{(n_i)}\}$で表現した行列を$M_i$とすれば,$\phi_x$を$\{(z_1^{(1)},\ldots,0),\ldots,(0,\ldots,z_g^{(n_g)})\}$で表現した行列は
  \[
  \begin{pmatrix}
    M_1 &        & \\
    & \ddots & \\
    &        & M_g
  \end{pmatrix}
  \]
  となる.よって,$\Tr_{\widehat{K}_\mathfrak{p}}(\phi_x)=\sum_i\Tr_{\widehat{K}_\mathfrak{p}}(\psi_{x,i})$.補題1.10.6から$\Tr_{\widehat{K}_\mathfrak{p}}(\psi_{x,i})=\Tr_{\widehat{L}_i/\widehat{K}_\mathfrak{p}}(x)$(包含写像$\varphi_i$は省略した).
  以上から,$\Tr_{L/K}(x)=\sum_i\Tr_{\widehat{L}_i/\widehat{K}_\mathfrak{p}}(x)$となる.
\end{proof}

\paragraph{補題1.10.12}~
\begin{screen}
  完備化と加法的付値
\end{screen}
\begin{proof}
  $x\in K^\times$に対し$A$での加法的付値は$xA_\mathfrak{p}=\mathfrak{p}^{\ord_\mathfrak{p}(x)}A_\mathfrak{p}$で定義される.
  また,$\mathfrak{p}$進付値は$\lvert x\rvert _\mathfrak{p}=\lvert A/\mathfrak{p}\rvert^{-\ord_\mathfrak{p}(x)}$で定義される.
  更に,$\mathfrak{p}$距離$d\colon K\times K\ni(x,y)\mapsto\lvert x-y\rvert _\mathfrak{p}\in\lvert A/\mathfrak{p}\rvert^a\ (a\in\mathbb{Z})$を定義する.
  距離空間$(K,d)$を完備化して距離空間$(\widehat{K}_\mathfrak{p},\widehat{d})$が得られる.
  補題1.2.6 (1)から$\widehat{d}\colon\widehat{K}_\mathfrak{p}\times\widehat{K}_\mathfrak{p}\ni(\widehat{x},\widehat{y})\mapsto\lvert A/\mathfrak{p}\rvert^a\ (a\in\mathbb{Z})$.
  完備化の定義から$x\in K$と$\vartheta\colon K\hookrightarrow\widehat{K}_\mathfrak{p}$によって
  \[\widehat{d}(\vartheta(x),\vartheta(0))=d(x,0)=\lvert A/\mathfrak{p}\rvert^{-\ord_\mathfrak{p}(x)}=\lvert x\rvert _\mathfrak{p}.\]

  ところで,定理1.2.8 (7)から$\widehat{A}_\mathfrak{p}$は$\widehat{\mathfrak{p}}=\mathfrak{p}\widehat{A}_\mathfrak{p}$を極大イデアルとする離散付値環である.
  ここで,$\widehat{x}\in\widehat{K}_\mathfrak{p}$に対し
  \[\widehat{d}(\widehat{x},\vartheta(0))=\lvert\widehat{x}\rvert _{\widehat{\mathfrak{p}}}=\lvert\widehat{A}_\mathfrak{p}/\widehat{\mathfrak{p}}\rvert^{-\ord_{\widehat{\mathfrak{p}}}(\widehat{x})}\]
  によって$\lvert\bullet\rvert _{\widehat{\mathfrak{p}}}$と$\ord_{\widehat{\mathfrak{p}}}(\bullet)$を定義する.
  命題1.2.13 (1)から$\widehat{A}_\mathfrak{p}/\widehat{\mathfrak{p}}\simeq A/\mathfrak{p}$なので
  \[\widehat{d}(\widehat{x},\vartheta(0))=\lvert\widehat{x}\rvert _{\widehat{\mathfrak{p}}}=\lvert\widehat{A}_\mathfrak{p}/\widehat{\mathfrak{p}}\rvert^{-\ord_{\widehat{\mathfrak{p}}}(\widehat{x})}=\lvert A/\mathfrak{p}\rvert^{-\ord_{\widehat{\mathfrak{p}}}(\widehat{x})}.\]
  $\vartheta(x)$に対しては,$\lvert\vartheta(x)\rvert _{\widehat{\mathfrak{p}}}=\lvert x\rvert _\mathfrak{p}$, $\ord_{\widehat{\mathfrak{p}}}(\vartheta(x))=\ord_\mathfrak{p}(x)$.
  更に,定理1.2.8 (6)から$\widehat{x}\in\widehat{K}_\mathfrak{p}$として,
  \[\lvert\widehat{A}_\mathfrak{p}/\widehat{\mathfrak{p}}\rvert^{-\ord_{\widehat{\mathfrak{p}}}(\widehat{x})}\leq\lvert\widehat{A}_\mathfrak{p}/\widehat{\mathfrak{p}}\rvert^{-n}\Leftrightarrow n\leq\ord_{\widehat{\mathfrak{p}}}(\widehat{x})\Leftrightarrow\widehat{x}\in\mathfrak{p}^n\widehat{A}_\mathfrak{p}=\widehat{\mathfrak{p}}^n\widehat{A}_\mathfrak{p}.\]
  よって,$\widehat{x}\in\widehat{\mathfrak{p}}^{\ord_{\widehat{\mathfrak{p}}}(\widehat{x})}\widehat{A}_\mathfrak{p}$, $\widehat{x}\notin\widehat{\mathfrak{p}}^{\ord_{\widehat{\mathfrak{p}}}(\widehat{x})+1}\widehat{A}_\mathfrak{p}$となるので,$\widehat{x}\widehat{A}_\mathfrak{p}=\widehat{\mathfrak{p}}^{\ord_{\widehat{\mathfrak{p}}}(\widehat{x})}\widehat{A}_\mathfrak{p}$.
  これは(離散付値環での)加法的付値の定義と一致している.

  以上から,$\widehat{x}\in\widehat{K}_\mathfrak{p}$の加法的付値$\ord_{\widehat{\mathfrak{p}}}$と$\widehat{\mathfrak{p}}$進付値$\lvert\bullet\rvert _{\widehat{\mathfrak{p}}}$は
  \[\widehat{x}\widehat{A}_\mathfrak{p}=\widehat{\mathfrak{p}}^{\ord_{\widehat{\mathfrak{p}}}(\widehat{x})}\widehat{A}_\mathfrak{p},\quad\lvert\widehat{x}\rvert _{\widehat{\mathfrak{p}}}=\lvert\widehat{A}_\mathfrak{p}/\widehat{\mathfrak{p}}\rvert^{-\ord_{\widehat{\mathfrak{p}}}(\widehat{x})}\]
  で定まり,$x\in K$に対しては(ほとんどの場合省略されるが)$\vartheta\colon K\hookrightarrow\widehat{K}_\mathfrak{p}$によって
  \[\ord_\mathfrak{p}(x)=\ord_{\widehat{\mathfrak{p}}}(\vartheta(x)),\quad \lvert x\rvert _\mathfrak{p}=\lvert\vartheta(x)\rvert _{\widehat{\mathfrak{p}}}.\]
\end{proof}

\begin{screen}
    $\pi\in B$に対し$\ord_{P_i}(\pi)=0\ (i=2,\ldots,g)$ならば,$\ord_\mathfrak{p}\left(\N_{L/K}(\pi)\right)=\ord_{\widehat{\mathfrak{p}}}\left(\N_{\widehat{L}_1/\widehat{K}_\mathfrak{p}}(\pi)\right)$
\end{screen}
\begin{proof}
  補題1.10.10から$\N_{L/K}(\pi)=\prod_i\N_{\widehat{L}_i/\widehat{K}_\mathfrak{p}}(\pi)$となる.命題1.1.13を使えば,
  \[\ord_{\widehat{\mathfrak{p}}}(\vartheta(\N_{L/K}(\pi)))=\ord_\mathfrak{p}(\N_{L/K}(\pi))=\sum_i\ord_{\widehat{\mathfrak{p}}}\left(\N_{\widehat{L}_i/\widehat{K}_\mathfrak{p}}(\pi)\right).\]
  $i=2,\ldots,g$に対し,$\ord_{P_i\widehat{B}_i}(\pi)=0$なので,$\pi\in\widehat{B}_i^\times$.
  よって命題I-8.5.2から$\N_{\widehat{L}_i/\widehat{K}_\mathfrak{p}}(\pi)\in\widehat{A}_\mathfrak{p}^\times$となるので$i=2,\ldots,g$に対し$\ord_{\widehat{\mathfrak{p}}}\left(\N_{\widehat{L}_i/\widehat{K}_\mathfrak{p}}(\pi)\right)=0$.
  以上から,$\ord_\mathfrak{p}\left(\N_{L/K}(\pi)\right)=\ord_{\widehat{\mathfrak{p}}}\left(\N_{\widehat{L}_1/\widehat{K}_\mathfrak{p}}(\pi)\right)$.
\end{proof}

\begin{screen}
    Dedekind環と完備化
\end{screen}
\begin{proof}
  $A$をDedekind環,$K$を$A$の商体,$L$を$K$の有限次拡大,$B$を$L$の$A$における整閉包,$\mathfrak{p}$を$A$の素イデアルとする.この時,$B$はDedekind環で,$L$は$B$の商体となる(命題1.3.2).
  $\mathfrak{p}$の上にある$B$の素イデアルを$P_1,\ldots,P_g$とする.
  \[
  \begin{CD}
    P_1            @<\text{PI}<< B @>\text{QF}>> L\\
    @AA\text{aPI}A @AA\text{IC}A   @AA\text{FE}A \\
    \mathfrak{p}   @<\text{PI}<< A @>\text{QF}>> K
  \end{CD}
  \]
  (PI:素イデアル,QF:商体,aPI:上にある素イデアル,IC:整閉包,FE:有限次拡大)

  $A$, $K$を$\mathfrak{p}$進距離で完備化すると位相環$\widehat{A}_\mathfrak{p}$,位相体$\widehat{K}_\mathfrak{p}$が得られる(系1.2.9).
  $\widehat{K}_\mathfrak{p}$は$\widehat{A}_\mathfrak{p}$の商体であり,$\widehat{A}_\mathfrak{p}$は$\mathfrak{p}\widehat{A}_\mathfrak{p}$を極大イデアルとする離散付値環である(定理1.2.8 (7)).更に,$B$, $L$を$P_1$進距離で完備化すると位相環$\widehat{B}_1$,位相体$\widehat{L}_1$が得られ,$\widehat{L}_1$は$\widehat{B}_1$の商体であり,$\widehat{B}_1$は$P_1\widehat{B}_1$を極大イデアルとする離散付値環である.
  $P_1\widehat{B}_1$は$\mathfrak{p}\widehat{A}_\mathfrak{p}$の上にある素イデアルである(補題1.3.3).この時,$\widehat{L}_1$は$\widehat{K}_\mathfrak{p}$の有限次拡大で$\widehat{B}_1$は$\widehat{L}_1$における$\widehat{A}_\mathfrak{p}$の整閉包である(定理1.3.23 (3)).
  \[
  \begin{CD}
    P_1\widehat{B}_1                     @<\text{MI1}<< \widehat{B}_1            @>\text{QF}>> \widehat{L}_1\\
    @AA\text{aPI}A                   @AA\text{IC}A                      @AA\text{FE}A \\
    \mathfrak{p}\widehat{A}_\mathfrak{p} @<\text{MI1}<< \widehat{A}_\mathfrak{p} @>\text{QF}>> \widehat{K}_\mathfrak{p}
  \end{CD}
  \]
  (MI1:唯一の極大イデアル)

  $M$を$\widehat{L}_1/\widehat{K}_\mathfrak{p}$の中間体とする.
  $M$と$\widehat{B}_1$は環なので$M\cap\widehat{B}_1$は環であり,$M\cap\widehat{B}_1$は$M$の$\widehat{A}_\mathfrak{p}$における整閉包である(命題I-8.1.4 (3)).
  $x\in\widehat{L}_1$が$M\cap\widehat{B}_1$上整なら$\widehat{B}_1$上整となる.
  $\widehat{B}_1$は整閉整域なので$x\in\widehat{B}_1$.
  $x\in\widehat{B}_1$は$\widehat{A}_\mathfrak{p}$上整なので$M\cap\widehat{B}_1$上整である(命題I-8.1.4 (2)).以上から,$\widehat{L}_1$の$M\cap\widehat{B}_1$における整閉包は$\widehat{B}_1$である.
  $M\cap\widehat{B}_1$はDedekind環で,$M\cap\widehat{B}_1$の商体は$M$である(命題1.3.2).

  $M\cap P_1\widehat{B}_1$が$M\cap\widehat{B}_1$の素イデアルであることは容易に示せる.

  $M\supset\mathfrak{p}\widehat{A}_\mathfrak{p}$, $P_1\widehat{B}_1\supset\mathfrak{p}\widehat{A}_\mathfrak{p}$なので$M\cap P_1\widehat{B}_1$は$\mathfrak{p}A_\mathfrak{p}$の上にある素イデアルである(補題1.3.3).
  $P_1\widehat{B}_1\supset M\cap P_1\widehat{B}_1$なので$P_1\widehat{B}_1$は$M\cap P_1\widehat{B}_1$の上にある素イデアルである(補題1.3.3).
  $M\cap\widehat{B}_1$は$M\cap P_1\widehat{B}_1$を極大イデアルとする完備離散付値環である(命題1.5.3).
  \[
  \begin{CD}
    P_1\widehat{B}_1                     @<\text{MI}<< \widehat{B}_1            @>\text{QF}>> \widehat{L}_1\\
    @AA\text{aPI}A                   @AA\text{IC}A                      @AA\text{FE}A \\
    M\cap P_1\widehat{B}_1               @<\text{MI}<< M\cap\widehat{B}_1       @>\text{QF}>> M\\
    @AA\text{aPI}A                   @AA\text{IC}A                      @AA\text{FE}A \\
    \mathfrak{p}\widehat{A}_\mathfrak{p} @<\text{MI}<< \widehat{A}_\mathfrak{p} @>\text{QF}>> \widehat{K}_\mathfrak{p}
  \end{CD}
  \]

  $A$が仮定1.1.2を満たせば,$\widehat{A}_\mathfrak{p}$も仮定1.1.2を満たす(命題1.2.13 (4)).
  $M\cap\widehat{B}_1$と$\widehat{B}_1$も仮定1.1.2を満たす(命題1.3.2).
\end{proof}

\begin{screen}
  \begin{thm}\label{CDVR_unr}
    上の状況で,剰余体$k=\widehat{A}_\mathfrak{p}/\mathfrak{p}\widehat{A}_\mathfrak{p}$, $l=\widehat{B}_1/P_1\widehat{B}_1$を考える.
    $a\in l$に対し,$[k(a):k]=[\widehat{K}_\mathfrak{p}(\alpha):\widehat{K}_\mathfrak{p}]$となる$\alpha\in\widehat{B}_1$が存在し,$\widehat{K}_\mathfrak{p}(\alpha)/\widehat{K}_\mathfrak{p}$は不分岐拡大である
  \end{thm}
\end{screen}
\[
\begin{CD}
  P_1\widehat{B}_1 @<\text{MI}<< \widehat{B}_1 @>\text{QF}>> \widehat{L}_1 @. l \\
  @AA\text{aPI}A @AA\text{IC}A @AA\text{FE}A @AA\text{FE}A \\
  \widehat{K}_\mathfrak{p}(\alpha)\cap P_1\widehat{B}_1 @<\text{MI}<< \widehat{K}_\mathfrak{p}(\alpha)\cap\widehat{B}_1 @>\text{QF}>> \widehat{K}_\mathfrak{p}(\alpha)\qquad @. \qquad k(a) \\
  @AA\text{aPI}A @AA\text{IC}A @AA{\text{FE}, \Phi(x)}A @AA{\text{FE}, \phi(x)}A \\
  \mathfrak{p}\widehat{A}_\mathfrak{p} @<\text{MI}<< \widehat{A}_\mathfrak{p} @>\text{QF}>> \widehat{K}_\mathfrak{p} @. k
\end{CD}
\]
\begin{proof}
  $\widehat{A}_\mathfrak{p}$は仮定1.1.2を満たすので,剰余体は有限体.
  $a$の$k$上最小多項式を$\phi(x)\in k[x]$とする.
  $k$は有限体なので完全体(系I-7.3.6)となり,$k(a)/k$は分離拡大である(定義I-7.3.1 (5)).
  よって$\phi(x)$は分離多項式で,$\phi(a)=0$, $\phi'(a)\neq0$.
  $\Phi(x)\equiv\phi(x)\bmod\mathfrak{p}\widehat{A}_\mathfrak{p}$となるモニック$\Phi(x)\in\widehat{A}_\mathfrak{p}[x]$及び$\alpha_0\equiv a\bmod P_1\widehat{B}_1$となる$\alpha_0\in\widehat{B}_1$を考える($\alpha_0$は$a$の代表元).
  $\Phi(\alpha_0)\equiv 0\bmod P_1\widehat{B}_1$, $\Phi'(\alpha_0)\not\equiv 0\bmod P_1\widehat{B}_1$であるので,Henselの補題から$\Phi(\alpha)=0$となる$\alpha\in\widehat{B}_1\ (\alpha\equiv\alpha_0\bmod P_1\widehat{B}_1)$が存在する.
  $\phi(x)$は$k$($\widehat{A}_\mathfrak{p}/\mathfrak{p}\widehat{A}_\mathfrak{p}$の商体)上既約なので,命題I-8.2.1とp.I-233の注から$\Phi(x)$は$\widehat{K}_\mathfrak{p}$上既約.
  よって,$\Phi(x)$は$\widehat{K}_\mathfrak{p}(\alpha)$の$\widehat{K}_\mathfrak{p}$上最小多項式となり,$[k(\alpha):k]=\deg (\phi(x))=\deg (\Phi(x))=[\widehat{K}_\mathfrak{p}(\alpha):\widehat{K}_\mathfrak{p}]$.

  $\widehat{K}_\mathfrak{p}(\alpha)\cap\widehat{B}_1$の剰余体$m=\widehat{K}_\mathfrak{p}(\alpha)\cap\widehat{B}_1/\widehat{K}_\mathfrak{p}(\alpha)\cap P_1\widehat{B}_1$を考える.
  $\alpha\in\widehat{K}_\mathfrak{p}(\alpha)\cap\widehat{B}_1$なので,$a\in m$である($\vartheta\colon m \hookrightarrow l$について$a \in \Im \vartheta$).
  $k(a)$は$a$を含む最小の体なので$[m:k(a)]\geq 1$.先程の結果と定理1.3.23 (4)から
  \[[k(a):k]=[\widehat{K}_\mathfrak{p}(\alpha):\widehat{K}_\mathfrak{p}]\geq f(\widehat{K}_\mathfrak{p}(\alpha)\cap P_1\widehat{B}_1:\mathfrak{p}\widehat{A}_\mathfrak{p})=[m:k]\]
  となるので,$[k(a):m]\geq 1$.以上から$[k(a):m]=1$,
  \[[\widehat{K}_\mathfrak{p}(\alpha):\widehat{K}_\mathfrak{p}]=f(\widehat{K}_\mathfrak{p}(\alpha)\cap P_1\widehat{B}_1:\mathfrak{p}\widehat{A}_\mathfrak{p})=[m:k]=[k(a):k]\]
  となるので,再び定理1.3.23 (4)から分岐指数は$e(\widehat{K}_\mathfrak{p}(\alpha)\cap P_1\widehat{B}_1:\mathfrak{p}\widehat{A}_\mathfrak{p})=1$.つまり,$\widehat{K}_\mathfrak{p}(\alpha)/\widehat{K}_\mathfrak{p}$は不分岐拡大.
\end{proof}

\begin{screen}
  $\pi$の$M$上最小多項式$f(x)$は$\widehat{B}_M = M\cap\widehat{B}_1$上のEisenstein多項式で,$\deg (f(x))=[\widehat{L}_1:M]$である
\end{screen}
\begin{proof}
  $\ord_{P_1}(\pi)=1$より$\pi\widehat{B}_1=\widehat{P}_1\widehat{B}_1$となるので,$\pi$は$\widehat{B}_1$の素元.
  $\widehat{L}_1/M$は完全分岐なので,命題1.10.7から$\pi$の$M$上最小多項式はEisenstein多項式で,その次数は$[\widehat{L}_1:M]$に等しい
  (命題の主張には書かれていないが,証明されている:p.62).
  また,Eisenstein多項式の定義から$f(x)\in\widehat{B}_M[x]$($M$に対応する離散付値環は$\widehat{B}_M$).
\end{proof}

\paragraph{命題1.10.13}~
\begin{screen}
  $Q_1,\ldots,Q_g$が$P_1,\ldots,P_t$と互いに素
\end{screen}
\begin{proof}
  $Q_i\cap A=\mathfrak{p}$, $P_j\cap A=\mathfrak{p}_j$に注意すれば,$Q_i=P_j \Rightarrow \mathfrak{p}=\mathfrak{p}_j$.
  対偶をとって,$\mathfrak{p}\neq\mathfrak{p}_j \Rightarrow Q_i\neq P_j$.
  $\mathfrak{p}\neq\mathfrak{p}_j$なので,$Q_i$と$P_j$は互いに素.
\end{proof}

\begin{screen}
  $I$の相対イデアル$\N_{L/K}(I)$も$\mathfrak{p}$と互いに素
\end{screen}
\begin{proof}
  相対イデアル$\N_{L/K}(I)$は$\Set{\N_{L/K}(b) | b\in I}$で生成される$A$のイデアル(定義1.10.1)なので,$A$の適当なイデアル$J$によって,$\N_{L/K}(I)=\left(\N_{L/K}(x)\right)+J$と書くことができる.
  $\left(\N_{L/K}(x)\right)$は$\mathfrak{p}$と互いに素なので,素イデアル分解には$\mathfrak{p}$は現れない:$\left(\N_{L/K}(x)\right)=\mathfrak{p}^0\cdots$.
  よって,定理I-8.3.17 (3)から$\N_{L/K}(I)=\left(\N_{L/K}(x)\right)+J$の素イデアル分解は$\mathfrak{p}^0\cdots$となる,つまり$\N_{L/K}(I)$は$\mathfrak{p}$と互いに素.
\end{proof}

\begin{screen}
  $\N_{L/K}(I)\supset\mathfrak{p}_1{}^{f_1}\cdots\mathfrak{p}_t{}^{f_t}$
\end{screen}
\begin{proof}
  $\N_{L/K}(I)\supset\mathfrak{p}_1{}^{f_1}\cdots\mathfrak{p}_t{}^{f_t}J$なので,定理I-8.3.17 (2)から$A$のイデアル$J'$があり,$J'\N_{L/K}(I)=\mathfrak{p}_1{}^{f_1}\cdots\mathfrak{p}_t{}^{f_t}J$.
  $J$の素イデアル分解を$\tilde{\mathfrak{p}}_1{}^{a_1}\cdots\tilde{\mathfrak{p}}_s{}^{a_s}$とすれば,$J'\N_{L/K}(I)=\mathfrak{p}_1{}^{f_1}\cdots\mathfrak{p}_t{}^{f_t}\tilde{\mathfrak{p}}_1{}^{a_1}\cdots\tilde{\mathfrak{p}}_s{}^{a_s}$.
  $\N_{L/K}(I)$が$J$と互いに素であることから,$\N_{L/K}(I)$の素イデアル分解には$\tilde{\mathfrak{p}}_1,\ldots,\tilde{\mathfrak{p}}_s$は現れない.
  よって,$J'=\tilde{\mathfrak{p}}_1{}^{a_1}\cdots\tilde{\mathfrak{p}}_s{}^{a_s}J''$となるので,$\tilde{\mathfrak{p}}_1{}^{a_1}\cdots\tilde{\mathfrak{p}}_s{}^{a_s}J''\N_{L/K}(I)=\mathfrak{p}_1{}^{f_1}\cdots\mathfrak{p}_t{}^{f_t}\tilde{\mathfrak{p}}_1{}^{a_1}\cdots\tilde{\mathfrak{p}}_s{}^{a_s}$.
  よって,$J''\N_{L/K}(I)=\mathfrak{p}_1{}^{f_1}\cdots\mathfrak{p}_t{}^{f_t}$.
  定理I-8.3.17 (2)から$\N_{L/K}(I)\supset\mathfrak{p}_1{}^{f_1}\cdots\mathfrak{p}_t{}^{f_t}$.
\end{proof}

\begin{screen}
  $I\widehat{B}_1=Q_{1,1}{}^{m_1}\widehat{B}_1=x\widehat{B}_1$
\end{screen}
\begin{proof}
  $x_1,\ldots,x_{m_1}$の$Q_{1,1}=P_1=\cdots=P_{m_1}$に関する加法的付値が$1$,$x_{m_1+1},\ldots,x_t$の$P_1$に関する加法的付値が$0$となる.よって,$\ord_{Q_{1,1}}(x_1)=1$なので,$x_1\widehat{B}_1=Q_{1,1}\widehat{B}_1$.また,$\ord_{Q_{1,1}}(x_t)=0$なので$x_t\widehat{B}_1=\widehat{B}_1$,つまり$x_t\in\widehat{B}_1^\times$.以上から,$x_1,\ldots,x_{m_1}$は$Q_{1,1}\widehat{B}_1$を生成し,$x_{m_1+1},\ldots,x_t$は$\widehat{B}_1^\times$の元である.
  $i\geq m_1+1$に対し$x_i \in P_i$,$x_i\in\widehat{B}_1^\times$なので$P_iP_1\widehat{B}_1=P_1\widehat{B}_1$.よって,$I=P_1\cdots P_t=Q_{1,1}{}^{m_1}P_{m_1+1}\cdots P_t$なので$I\widehat{B}_1=Q_{1,1}{}^{m_1}\widehat{B}_1$.
  $x=x_1\cdots x_t$で$x_1\widehat{B}_1=\cdots=x_{m_1}\widehat{B}_1=Q_{1,1}\widehat{B}_1$,$x_{m_1+1},\ldots,x_t\in\widehat{B}_1^\times$なので$x\widehat{B}_1=Q_{1,1}{}^{m_1}\widehat{B}_1$.
\end{proof}

\begin{screen}
  $I\widehat{B}_1=x\widehat{B}_1$,$y\in I$なら$z\in\widehat{B}_1$が存在して$y=zx$となる
\end{screen}
\begin{proof}
  $x\widehat{B}_1=I\widehat{B}_1\supset y\widehat{B}_1$なので,定理I-8.3.17 (2)から単項イデアル整域$\widehat{B}_1$のイデアル$z'\widehat{B}_1$が存在し$xz'\widehat{B}_1=y\widehat{B}_1$となる.
  $y\in y\widehat{B}_1$なので$y\in xz'\widehat{B}_1$,つまり$b\in\widehat{B}_1$があり$y=xz'b$.
  $z=z'b\in\widehat{B}_1$とすれば主張が従う.
\end{proof}

\paragraph{命題1.10.19}~
\begin{screen}
  $IA_\mathfrak{p}$が$\varDelta_{B/A,\mathfrak{p}}$を割り切る
\end{screen}
\begin{proof}
  $I$の定義から$s^{2n}\varDelta_{L/K}(u_1,\ldots,u_n)=\varDelta_{L/K}(su_1,\ldots,su_n)A=Ia$となる$a\in A$が存在する.
  よって,$aIA_\mathfrak{p}=s^{2n}\varDelta_{L/K}(u_1,\ldots,u_n)A_\mathfrak{p}$.
  $s^{2n}\in A_\mathfrak{p}^\times$なので,$=\varDelta_{L/K}(u_1,\ldots,u_n)A_\mathfrak{p}=\varDelta_{B/A,\mathfrak{p}}A_\mathfrak{p}$となるので従う.
\end{proof}

\section{完備化とDedekindの判別定理}

\begin{screen}
  $B$の$A$基底は$B_\mathfrak{p}$の$A_\mathfrak{p}$基底であり,$L$の$K$基底である
\end{screen}
\begin{proof}
  $B$の$A$基底を$\{w_1,\ldots,w_n\}$とする.$\forall b\in B$は$\sum y_iw_i\ (y_i\in A)$と表すことができる.
  $B_\mathfrak{p}$の任意の元は$b\in B$と$s\in A\setminus\mathfrak{p}$で$b/s$と表せる.上の式を代入して$b/s=\sum_{i=1}^n(y_i/s)w_i,\quad (y_i/s\in A_\mathfrak{p})$.
  $B_\mathfrak{p}$の任意の元は$\{w_1,\ldots,w_n\}$の$A_\mathfrak{p}$係数の線形結合で表すことができる.
  $\sum (y_i/s_i)w_i=0\ (y_i\in A,\quad s_i\in A\setminus\mathfrak{p})$とする.
  $\sum_{i=1}^n (y_is_1\cdots s_n/s_i)w_i=0$
  となり,$w_i$の係数は$A$の元.
  $\{w_1,\ldots,w_n\}$の$A$基底としての一時独立性から,$y_i=0$.よって$A_\mathfrak{p}$基底として一次独立.

  $L/K$も同様.
  $S=A\setminus\{0\}$として$L=S^{-1}B$と表せることを使う.
\end{proof}

\paragraph{命題1.11.1}~
\begin{screen}
  $\varDelta_{B/A,\mathfrak{p}}=\prod_{i=1}^g\varDelta_{\widehat{B}_i/\widehat{A}_\mathfrak{p}}$
\end{screen}
\begin{proof}
  $\{w_1, \ldots, w_n\}$を$B$の$A$基底とする.
  $m(w_i)$を$B$の元に$w_i \in B$をかける写像とする:
  \[m(w_i) \colon B \ni a = \sum_{i=1}^n a^{(i)}w_i \mapsto aw_i = \sum_{i=1}^n b^{(i)}w_i \in B.\]
  これは$B \otimes_A \widehat{A}_\mathfrak{p}$の元に$w_i \otimes 1 \in B \otimes_A \widehat{A}_\mathfrak{p}$をかける写像$\tilde{m}(w_i)$を誘導する:
  \[\tilde{m}(w_i) \colon B \otimes_A \widehat{A}_\mathfrak{p} \ni a \otimes t = \sum_{i=1}^n a^{(i)}w_i \otimes t \mapsto aw_i \otimes t = \sum_{i=1}^n b^{(i)}w_i \otimes t \in B \otimes_A \widehat{A}_\mathfrak{p}.\]
  定理1.3.23 (2)から,$\phi_i\colon B\hookrightarrow\widehat{B}_i$として同型
  \[\phi\colon B \otimes_A \widehat{A}_\mathfrak{p} \ni a\otimes t\rightarrow(\phi_1(a)t, \ldots, \phi_g(a)t) \in \widehat{B}_1 \times \cdots \times \widehat{B}_g\]
  が得られる.$\phi$によって$\{w_i\}_{1 \leq i \leq n}$は$\{(\phi_1(w_i), \ldots, \phi_g(w_i))\}_{1 \leq i \leq g}$に写るので,$\{(\phi_1(w_i), \ldots, \phi_g(w_i))\}_{1 \leq i \leq g}$は$\widehat{B}_1 \times \cdots \times \widehat{B}_g$の$\widehat{A}_\mathfrak{p}$基底である.

  $\widehat{m}=\phi \tilde{m} \phi^{-1}$として,
  \[\widehat{m} \colon \widehat{B}_1\times\cdots\times\widehat{B}_g \ni \phi(a \otimes t) \mapsto \phi(aw_i \otimes t) \in \widehat{B}_1\times\cdots\times\widehat{B}_g\]
  を構成する.
  \[
  \begin{CD}
    \widehat{B}_1\times\cdots\times\widehat{B}_g @>{\widehat{m}}>> \widehat{B}_1\times\cdots\times\widehat{B}_g\\
    @A{\phi}AA  @A{\phi}AA\\
    B_\mathfrak{p}\otimes_{A_\mathfrak{p}}\widehat{A}_\mathfrak{p} @>>{\tilde{m}}> B_\mathfrak{p}\otimes_{A_\mathfrak{p}}\widehat{A}_\mathfrak{p}\\
  \end{CD}
  \]
  $\phi(a \otimes t)=(\phi_1(a)t, \ldots, \phi_g(a)t)$,$\phi(aw_i \otimes t)=(\phi(w_i)\phi_1(a)t, \ldots, \phi(w_i)\phi_g(a)t)$であるので,$\widehat{m}(w_i)$は$\widehat{B}_1\times\cdots\times\widehat{B}_g$の元に$(\phi_1(w_i), \ldots, \phi_g(w_i))$をかける写像である.
  構成から(適当な基底を取ることによって)$\Tr(m)=\Tr(\tilde{m})=\Tr(\widehat{m})$であることが分かる.

  $\{v_{i,1},\ldots,v_{i,N_i}\}$を$\widehat{B}_i$の$\widehat{A}_\mathfrak{p}$基底とする($N_i=e_if_i$:定理1.3.23 (4)).
  \begin{align*}
    \overline{v}_1=(v_{1,1},0,\ldots,0),&\ldots,\overline{v}_{N_1}=(v_{1,N_1},0,\ldots,0)\\
    \overline{v}_{N_1+1}=(0,v_{2,1},0,\ldots,0),&\ldots,\overline{v}_{N_1+N_2}=(0,v_{2,N_2},0,\ldots,0)\\
    &\vdots\\
    \overline{v}_{n-N_g+1}=(0,\ldots,0,v_{g,1}),&\ldots,\overline{v}_{n}=(0,\ldots,0,v_{g,N_g})
  \end{align*}
  とすれば,$\{\overline{v}_1,\ldots,\overline{v}_n\}$は$\widehat{B}_1\times\cdots\times\widehat{B}_g$の$\widehat{A}_\mathfrak{p}$基底となる.
  $\{(\phi_1(w_i), \ldots, \phi_g(w_i))\}_{1 \leq i \leq g}$,$\{\overline{v}_1, \ldots, \overline{v}_n\}$は共に$\widehat{B}_1\times\cdots\times\widehat{B}_g$の$\widehat{A}_\mathfrak{p}$基底であるので,$A \in \GL_n(\widehat{A}_\mathfrak{p})$が存在し,
  \[(\phi_1(w_i), \ldots, \phi_g(w_i)) = \sum_{j=1}^n A_{ij} \overline{v}_j.\]
  補題1.10.6を使えば,
  \begin{align*}
    \Tr_{L/K}(w_iw_j) &= \Tr(m(w_i w_j)) = \Tr(\widehat{m}(w_i w_j)) = \Tr\left(\widehat{m}\left(\sum_{k=1}^n \sum_{l=1}^n A_{ik} A_{jl} \overline{v}_k \overline{v}_{l} \right) \right) \\
                      & = \sum_{k=1}^n \sum_{l=1}^n A_{ik} A_{jl} \Tr(\widehat{m}(\overline{v}_k \overline{v}_{l})).
  \end{align*}
  $\Tr_{L/K}(w_iw_j)$を$(i,j)$成分とする$n\times n$行列を$W$,$\Tr(\widehat{m}(\overline{v}_i \overline{v}_{j}))$を$(i,j)$成分とする$n\times n$行列を$M$とすれば,これは$W=AM{}^tA$と書ける.

  $M$の$(N_1+\cdots+N_{l-1}+i,N_1+\cdots+N_{l-1}+j)\ (1\leq i,j\leq N_l)$成分は$\widehat{B}_1\times\cdots\times\widehat{B}_g$に対して,
  $\overline{v}_{N_1+\cdots+N_{l-1}+i}\overline{v}_{N_1+\cdots+N_{l-1}+j}=(0,\ldots,0,v_{l,i}v_{l,j},0,\ldots,0)$をかける線形写像のトレースであり,
  これは$\widehat{B}_l$の元に$v_{l,i}v_{l,j}$をかける線形写像のトレース.
  よって補題1.10.6からこれは$\Tr_{\widehat{L}_l/\widehat{K}_\mathfrak{p}}(v_{l,i}v_{l,j})$に等しい.
  $i\neq j$であれば$v_{ik}v_{jl}=0$なので,$\Tr_{\widehat{L}_i/\widehat{K}_\mathfrak{p}}(v_{ik}v_{il})$を$(k,l)$成分とする$N_i\times N_i$行列を$M_i$とすれば,
  \[
  M=
  \begin{pmatrix}
    M_1 &        & \\
    & \ddots & \\
    &        & M_g
  \end{pmatrix}
  ,\quad \det M_i = \varDelta_{\widehat{L}_i/\widehat{K}_\mathfrak{p}}(v_{i,1},\ldots,v_{i,N_i}).\]
  よって,
  \[\varDelta_{L/K}(w_1,\ldots,w_n) = \det W = (\det A)^2 \det M = (\det A)^2 \prod_{i=1}^g \det M_i = (\det A)^2 \prod_{i=1}^g\varDelta_{\widehat{L}_i/\widehat{K}_\mathfrak{p}}(v_{i,1},\ldots,v_{i,N_i})\]
  となり,系I-6.7.9 (1)から$\det A \in \widehat{A}_\mathfrak{p}^\times$なので
  \begin{align*}
    \ord_{\mathfrak{p}}(\varDelta_{L/K}(w_1,\ldots,w_n)) &= \ord_{\mathfrak{p}\widehat{A}_\mathfrak{p}}\left(\prod_{i=1}^g\varDelta_{\widehat{L}_i/\widehat{K}_\mathfrak{p}}(v_{i,1},\ldots,v_{i,N_i})\right) \\
    &= \sum_{i = 1}^g\ord_{\mathfrak{p}\widehat{A}_\mathfrak{p}}\left(\varDelta_{\widehat{L}_i/\widehat{K}_\mathfrak{p}}(v_{i,1},\ldots,v_{i,N_i})\right).
  \end{align*}
  $\widehat{A}_\mathfrak{p}$は離散付値環で$\mathfrak{p}\widehat{A}_\mathfrak{p}$が唯一の素イデアルなので
  \[\varDelta_{B/A,\mathfrak{p}} = \mathfrak{p}^{\ord_{\mathfrak{p}}(\varDelta_{L/K}(w_1,\ldots,w_n))}\widehat{A}_\mathfrak{p} = \prod_{i = 1}^g\mathfrak{p}^{\ord_{\mathfrak{p}\widehat{A}_\mathfrak{p}}\left(\varDelta_{\widehat{L}_i/\widehat{K}_\mathfrak{p}}(v_{i,1},\ldots,v_{i,N_i})\right)}\widehat{A}_\mathfrak{p} = \prod_{i=1}^g\varDelta_{\widehat{B}_i/\widehat{A}_\mathfrak{p}}.\]
\end{proof}

\paragraph{命題1.11.4}~
\begin{screen}
  $y=s_1\cdots s_mx\in\delta(B/A)^{-1}$
\end{screen}
\begin{proof}
  $a\in B$として,$\Tr_{L/K}(as_1\cdots s_mx)=s_1\cdots s_m\Tr_{L/K}(ax)$.
  $a=b_1w_1+\cdots+b_mw_m\ (b_i\in A)$とすれば,
  \[=s_1\cdots s_m\Tr_{L/K}\left(\sum_ib_iw_ix\right)=s_1\cdots s_m\sum_ib_i\Tr_{L/K}\left(w_ix\right)=s_1\cdots s_m\sum_ib_ia_i/s_i\]
  なので$\Tr_{L/K}(as_1\cdots s_mx)\in A$.つまり,$s_1\cdots s_mx\in\delta(B/A)^{-1}$.
\end{proof}

\paragraph{命題1.11.6}~
\begin{screen}
  $x\in\delta(B/B_M)^{-1}(\delta(B_M/A)B)^{-1}$
\end{screen}
\begin{proof}
  $x\delta(B_M/A)\subset\delta(B/B_M)^{-1}$は本文から容易に分かる.
  $b\in\delta(B/B_M)^{-1}$であれば$\forall y\in B$に対し,定義から$by\in\delta(B/B_M)^{-1}$となるので,$bB\subset\delta(B/B_M)^{-1}$.
  以上から$x\delta(B_M/A)B\subset\delta(B/B_M)^{-1}$.
  $\delta(B_M/A)B$は$B$のイデアルなので有限生成(命題I-6.8.34)で,分数イデアル(Dedekind環のイデアルは分数イデアル.命題I-8.3.24からも分かる).
  よって,命題I-8.3.21から
  \[xB=x\delta(B_M/A)B(\delta(B_M/A)B)^{-1}\subset\delta(B/B_M)^{-1}(\delta(B_M/A)B)^{-1}.\]
  よって,$x\in\delta(B/B_M)^{-1}(\delta(B_M/A)B)^{-1}$.
\end{proof}

\paragraph{系1.11.11}~
\begin{screen}
  $\overline{g}(x)$の根が$A/\mathfrak{p}$上$M\cap B/M\cap P$を生成し,$g(x)$の根が$K$上$M$を生成する様な$g(x)$が存在し,$g(x)$は既約,$\overline{g}(x)$が既約・分離である
\end{screen}
\begin{proof}
  $k=A/\mathfrak{p}$,$l=B/P$,$m=C/P_M$とする.
  $M/K$は不分岐拡大なので$[m:k]=[M:K]=f$.
  $k$は有限体なので完全体(系I-7.3.6)となり$m/k$は分離拡大(よって,$g$の存在を示せば$\overline{g}(x)$は分離多項式と分かる)で,$m=k(a)$となる$a\in l$が存在する.
  これに対して追加定理\ref{CDVR_unr}(p.\pageref{CDVR_unr})の証明を適用すれば,$f$次の不分岐拡大$K(\alpha)/K\ (\alpha\in B)$が存在する.
  $g$は$\Phi$,$\overline{g}$は$\phi$に対応するので,$g(x)$, $\overline{g}(x)$は既約多項式.
  $M/K$は$f$次の最大不分岐拡大であったので$M=K(\alpha)$となる.
\end{proof}

\begin{screen}
  $R(g,g')\in A^\times$,$\varDelta_{C/A}=A$
\end{screen}
\begin{proof}
  $R(g,g')$を$\bmod\mathfrak{p}$で考えた$R(\overline{g},\overline{g}')$は$\overline{g}(x)$の判別式になる(系1.9.7).
  $\overline{g}(x)$は分離多項式なので$\overline{g}$と$\overline{g}'$は共通根を持たない.
  よって系1.9.6から$R(\overline{g},\overline{g}')\neq 0$.
  % 自然な写像$A\to A/\mathfrak{p}$の$\ker$は$\mathfrak{p}$なので,
  従って$R(g,g')\in A\setminus\mathfrak{p}$で,命題I-6.5.8から$A\setminus\mathfrak{p}=A^\times$.
  上で示したように,$g(x)\in A[x]$は$\alpha\in C$の$K$上最小多項式で,$\varDelta_{M/K}(1,\alpha,\cdots,\alpha^{f-1}) = R(g,g')$(命題1.9.9)なので$\varDelta_{M/K}(1,\alpha,\cdots,\alpha^{f-1})\in A^\times$.
  すなわち$\ord_\mathfrak{p}(\varDelta_{M/K}(1,\alpha,\cdots,\alpha^{f-1}))=0$.
  よって補題1.8.3から$\{1,\alpha,\ldots,\alpha^{f-1}\}$は$C$の$A$基底となる(p.49の序盤参照)ので,相対判別式の計算に$\{1,\alpha,\ldots,\alpha^{f-1}\}$を使用できる:
  \[\varDelta_{C/A}=\varDelta_{C/A,\mathfrak{p}}=\mathfrak{p}^{\ord_\mathfrak{p}(\varDelta_{M/K}(1,\alpha,\ldots,\alpha^{f-1}))}=\mathfrak{p}^{0}=A.\]
\end{proof}

\paragraph{命題1.11.14}~
\begin{screen}
  $A$を離散付値環,$\{x_1,\ldots,x_l\}\ (l=[N:K])$を$B_N$の$A$基底,$\varDelta_{N/K}(x_1,\ldots,x_l)\in A^\times$とすれば,$\{x_1,\ldots,x_l\}$は$L$の$M$基底となる
\end{screen}
\begin{proof}
  命題I-8.1.24から$B_N$は階数$l$の自由$A$加群(だから基底が$l$個).
  $\varDelta_{N/K}(x_1,\ldots,x_l)\neq 0$なので命題1.7.3 (2)から$\{x_1,\ldots,x_l\}\subset N$は$K$上一次独立.
  よって定理I-8.11.9 (3)から$M$上一次独立.命題I-8.11.9 (2)から$[L:M]=l$なので$\{x_1,\ldots,x_l\}$は$L$の$M$基底となる.
\end{proof}

\paragraph{定理1.11.16}~
\begin{screen}
  $A$が離散付値環なら,$B_M$の$A$基底を$\{v_1,\ldots,v_m\}$,$B_N$の$A$基底を$\{w_1,\ldots,w_n\}$とすれば,$\{v_1,\ldots,v_m\}$は$N$上一次独立で,$\{w_1,\ldots,w_n\}$は$K$上一次独立である
\end{screen}
\begin{proof}
  I-p.241のはじめの話から,$\{v_1,\ldots,v_m\}$と$\{w_1,\ldots,w_n\}$は$K$上一次独立.さらに,定理I-8.11.9から$\{v_1,\ldots,v_m\}$は$N$上一次独立となる.
\end{proof}

\begin{screen}
  上の状況で,$\Tr_{L/K}(v_iw_kv_jw_l)=\Tr_{M/K}(v_iv_j)\Tr_{N/K}(w_kw_l)$
\end{screen}
\begin{proof}
  まず,
  \[\Tr_{L/K}(v_iw_kv_jw_l)=\Tr_{L/K}(v_iv_jw_kw_l).\]
  命題I-8.1.18 (5)から
  \[=\Tr_{M/K}\left(\Tr_{L/M}(v_iv_jw_kw_l)\right).\]
  $v_iv_j\in M$なので,
  \[=\Tr_{M/K}\left(v_iv_j\Tr_{L/M}(w_kw_l)\right).\]
  (1.11.15)と同様の考察から$\Tr_{L/M}(w_kw_l)=\Tr_{N/K}(w_kw_l)$なので
  \[=\Tr_{M/K}\left(v_iv_j\Tr_{N/K}(w_kw_l)\right).\]
  補題I-8.1.17から$\Tr_{N/K}(w_kw_l)\in K$なので,
  \[=\Tr_{M/K}(v_iv_j)\Tr_{N/K}(w_kw_l).\]
\end{proof}

\begin{screen}
  上の状況で$\varDelta_L=\varDelta_M{}^n\varDelta_N{}^m$
\end{screen}
\begin{proof}
  $B$の$A$基底として
  \[\{v_1w_1,\ldots,v_1w_n;v_2w_1,\ldots,v_2w_n;\cdots;v_mw_1,\ldots,v_mw_n\}\]
  をとれる(p.81の上の方).
  $G$を$(i,j)$成分を$g_{ij}=\Tr_{L/K}(v_iv_j)$とする$m$次行列,$H$を$(k,l)$成分を$\Tr_{L/K}(w_kw_l)$とする$n$次行列とする.
  $((i-1)n+k,(j-1)n+l)$成分が$\Tr_{L/K}(v_iw_kv_jw_l)$の行列を$X$とする.
  $\Tr_{L/K}(v_iw_kv_jw_l)=\Tr_{L/K}(v_iv_j)\Tr_{L/K}(w_kw_l)$なので,
  \begin{align*}
    X=
    \begin{pmatrix}
      g_{11}H & \cdots & g_{1m}H \\
      g_{21}H & \cdots & g_{2m}H \\
      & \vdots & \\
      g_{m1}H & \cdots & g_{mm}H
    \end{pmatrix}
    =G\otimes H
  \end{align*}
  (行列のKronecker積)となる.線形代数で知られているように,$\det(G\otimes H)=(\det G)^n(\det H)^m$なので,
  \[\varDelta_{L/K}(v_1w_1,\ldots)=\det X=\det(G\otimes H)=(\det G)^n(\det H)^m=\varDelta_{M/K}(v_1,\ldots,v_m){}^n\varDelta_{N/K}(w_1,\ldots,w_n){}^m.\]
  $A=\mathbb{Z}$, $K=\mathbb{Q}$とすれば,主張が従う.
\end{proof}

% ここから未確認

\section{積公式}
\paragraph{補題1.12.1}
$\Tr_{L/K}$などは$\Tr_{K/\mathbb{Q}}$の間違い

\paragraph{定理1.12.2}~
\begin{screen}
  $x\in\mathcal{O}_K$に対し積公式$\prod_{v\in\mathfrak{M}}\lvert x\rvert _v=1$が成立すれば$x\in K^\times$に対しても成立する
\end{screen}
\begin{proof}
  まず$v\in\mathfrak{p}$については$\lvert x\rvert _\mathfrak{p}\lvert 1/x\rvert _\mathfrak{p}=1$.
  $v\in\mathfrak{M}_\mathbb{R}$に対しては$\sigma_v\in\Hom^\text{al}_\mathbb{Q}(K,\mathbb{C})$について$\lvert\sigma_v(x)\rvert\lvert\sigma_v(1/x)\rvert=\lvert\sigma_v(1)\rvert=1$.
  $v\in\mathfrak{M}_\mathbb{C}$に対しても同様に$\lvert\sigma_v(x)\rvert^2\lvert\sigma_v(1/x)\rvert^2=\lvert\sigma_v(1)\rvert^2=1$.
  以上から,$x\in\mathcal{O}_K$に対し,$\lvert x\rvert _v=\lvert 1/x\rvert _v$なので,$1/x\in K\ (x\in\mathcal{O}_K)$に対しても積公式が成立.
  $K$は$\mathcal{O}_K$の商体なので,$\forall x\in K$は$a/b\ (a,b\in\mathcal{O}_K)$と表すことができる.
  $a$, $1/b$に対し積公式が成立するので,上の議論と同様にして$x=a/b$に対しても成立.
\end{proof}

\section{Krasnerの補題}

\paragraph{定理1.13.1}~
\begin{screen}
  $A$, $B$を完備離散付値環,$L/K$をGalois拡大,$\tau\in\Gal(L/K(\beta))\ (\beta\in L)$として,$\lvert\tau(x)\rvert=\lvert x\rvert$
\end{screen}
\begin{proof}
  $B$の極大イデアルを$P$とする.$\ord_P(x)=n$とすれば,$xB=P^nB$となり,$\tau(x) \tau(B) = \tau(P)^n$.
  $\tau(B)=B$(追加補題\ref{sigmaB}, p.\pageref{sigmaB}),$\tau(P)=P$なので$\tau(x)B=P^n$.つまり,$\ord_P(\tau(x))=n$.
\end{proof}

\paragraph{系1.13.3}~
\begin{screen}
  局所体($\mathbb{Q}_p$の有限次拡大)が代数体($\mathbb{Q}$の有限次拡大)の完備化に同型である
\end{screen}
\begin{proof}
  本文の証明の前半から,局所体$L=\mathbb{Q}_p(\beta)$,代数体$K=\mathbb{Q}(\beta)$と表すことができる.
  $\mathcal{O}_L$を$L$の整数環,$P$を$\mathcal{O}_L$の素イデアルとする.この時,$\mathfrak{p}=P\cap\mathcal{O}_K$とする.
  $\mathbb{Z} \subset \mathbb{Z}_p$,$K \subset L$なので$\mathcal{O}_K \subset \mathcal{O}_L$である.
  $a,b\in\mathfrak{p}$,$x\in\mathcal{O}_K$とする.
  $a,b\in\mathcal{O}_K,P$なので$a+b\in\mathcal{O}_K,P$,つまり$a+b\in\mathfrak{p}$.
  $a,x\in\mathcal{O}_K$なので$ax\in\mathcal{O}_K$.
  $a\in P$,$x\in\mathcal{O}_L$なので$ax\in P$.よって,$ax\in\mathfrak{p}$.以上から,$\mathfrak{p}$は$\mathcal{O}_K$のイデアル.
  $a,b\in\mathcal{O}_K$で$ab\in \mathfrak{p}$とする.
  $ab\in P$で$P$は$\mathcal{O}_L$の素イデアルなので$a\in P$(若しくは$b\in P$)となり,$a\in\mathfrak{p}$.よって,$\mathfrak{p}$は$\mathcal{O}_K$の素イデアル.
  $\mathcal{O}_L$は完備離散付値環(命題I-9.1.31 (4),命題1.5.3 (2))なので$\mathfrak{p}\mathcal{O}_L=P^e\mathcal{O}_L$となる$e$が存在する(命題I-8.3.15).
  $\mathcal{O}_L/P$は$\mathcal{O}_K/\mathfrak{p}$の有限次拡大.
  $\mathcal{O}_L$は単項イデアル整域なので$P$は単項イデアル.
  $\mathcal{O}_L$は仮定1.1.2を満たす(命題1.5.3)ので,$\mathcal{O}_L/P$は有限体.
  よって$\mathcal{O}_K/\mathfrak{p}$も有限体.よって,定理1.3.23 (1)の証明と同様にして$K$(を$L$への包含)上では$P$進距離は$\mathfrak{p}$進距離の冪乗となる.
  $P\supset p\mathbb{Z}_p$,$\mathcal{O}_K\supset\mathbb{Z}$なので$\mathfrak{p}\supset p\mathbb{Z}$.
  よって補題1.3.3より$\mathfrak{p}$は$p\mathbb{Z}$の上にある素イデアル($p\mathbb{Z}=\mathfrak{p}\cap\mathbb{Z}$).
  \[
  \begin{CD}
    \mathfrak{p} @<<< \mathcal{O}_K @>>> K \\
    @AAA         @AAA               @AAA \\
    p\mathbb{Z}  @<<< \mathbb{Z}    @>>> \mathbb{Q}
  \end{CD}
  \]
  完備化して
  \[
  \begin{CD}
    \mathfrak{p}\mathcal{O}_{\widehat{K}}   @<<< \mathcal{O}_{\widehat{K}} @>>> \widehat{K}\\
    @AAA                              @AAA                     @AAA \\
    p\mathbb{Z}_p                     @<<< \mathbb{Z}_p        @>>> \mathbb{Q}_p
  \end{CD}
  \]
  $\beta\in K$なので$\beta\in\widehat{K}$.よって,$\widehat{K}\supset\mathbb{Q}_p(\beta)=L$.
  $f,g$共に$\mathbb{Q}_p$上既約なので$[L:\mathbb{Q}_p]=\deg f=\deg g=[K:\mathbb{Q}]$.
  定理1.3.23 (3)(4)から$[\widehat{K}:\mathbb{Q}_p]\leq[K:\mathbb{Q}]$なので$[L:\mathbb{Q}_p]\geq[\widehat{K}:\mathbb{Q}_p]$.
  以上から$\widehat{K}=L$.
\end{proof}

\paragraph{命題1.13.5}~
\begin{screen}
  $L/K$を$\mathbb{Q}_p$の有限次拡大,$\mathfrak{p}$を$\mathcal{O}_K$の素イデアル,$P$を$\mathfrak{p}$の上にある$\mathcal{O}_L$の素イデアルとする.
  $\mathcal{O}_L/P\simeq\mathcal{O}_K/\mathfrak{p}$ならば$u\in\mathcal{O}_L$に対し$u_0\equiv u\bmod P$となる$u_0\in\mathcal{O}_K$が存在する
\end{screen}
\begin{proof}
  $\mathcal{O}_K/\mathfrak{p}$の完全代表系を$\{a_1,a_2,\ldots\}$とする.
  $a_1$と$a_2$は別の同値類に属するので$a_1-a_2\not\in\mathfrak{p}$,つまり$a_1-a_2\in\mathcal{O}_K\setminus\mathfrak{p}$.
  よって$a_1-a_2\not\in P$.
  $a_1, a_2$は$\mathcal{O}_L$の元としても異なる同値類に属する.
  $\mathcal{O}_L/P$と$\mathcal{O}_K/\mathfrak{p}$の位数は等しいので,他の元についても同様にして$\mathcal{O}_L/P$の完全代表系として$\{a_1,a_2,\ldots\}$が取れる.
  よって,$u\in\mathcal{O}_L$が含まれる同値類の代表元を$u_0$とすればよい.
\end{proof}

\begin{screen}
  上の状況で,$L/K$が完全分岐で馴分岐,$F/L$の整数環$\mathcal{O}_F$の極大イデアルを$\mathcal{P}$,$\mathcal{P}$進距離を$\lvert\bullet\rvert$とする.
  この時,$L/K$の分岐指数$e$について$\lvert e\rvert=1$
\end{screen}
\begin{proof}
  $F$は$\mathbb{Q}_p$の有限次拡大でもあるので,$\lvert\mathcal{O}_F/\mathcal{P}\rvert$は$\lvert\mathbb{Z}_p/p\mathbb{Z}_p\rvert$の倍数.
  命題I-9.1.31 (1)から$\lvert\mathbb{Z}_p/p\mathbb{Z}_p\rvert=\lvert\mathbb{Z}/p\mathbb{Z}\rvert=p$なので,$\lvert\mathcal{O}_F/\mathcal{P}\rvert$は$p$の倍数.
  従って,$\mathcal{O}_F/\mathcal{P}$の標数は$p$(p.I-216など参照).
  $L/K$が馴分岐なので$p \nmid e$となり,$\mathcal{O}_F$の元として$e$と$0$は異なる同値類に属する.よって$e\not\in\mathcal{P}$であり,$\lvert e\rvert=1$.
\end{proof}

\section{$2$次の暴分岐}
\paragraph{命題1.14.1}~
\begin{screen}
  \begin{lem}\label{basis_Q2_prime_element}
    $K$を$\mathbb{Q}_2$の有限次拡大,$\mathfrak{p}$を$\mathcal{O}_K$の素イデアル,$F=K(\sqrt{\pi})$とする.
    $\pi$が$\mathcal{O}_K$の素元であれば$\mathcal{O}_F$の$\mathcal{O}_K$上基底が$\{1,\sqrt{\pi}\}$である(命題1.7.3使った方が楽)
  \end{lem}
\end{screen}
\begin{proof}
  $2=\pi^md\ (\ord_\mathfrak{p}(d)=0)$とする.
  $\ord_\mathfrak{p}(\pi)=1$である.
  $\alpha=a+b\sqrt{\pi}\in F\ (a,b\in K)$が$\mathcal{O}_K$上整であるとする.
  \[A=\Tr_{F/K}(\alpha)=(a+b\sqrt{\pi})+(a-b\sqrt{\pi})=2a,\quad B=\N_{F/K}(\alpha)=(a+b\sqrt{\pi})(a-b\sqrt{\pi})=a^2-\pi b^2.\]
  命題I-8.1.19から$A, B\in\mathcal{O}_K$.$4B=A^2-4\pi b^2\in 4\mathcal{O}_K$なので$4\pi b^2\in\mathcal{O}_K$,つまり$\ord_\mathfrak{p}(4\pi b^2)\geq 0$.
  よって$2m+1+2\ord_\mathfrak{p}(b)\geq 0$(命題1.1.3 (1))なので$\ord_\mathfrak{p}(b)\geq -m$.
  よって$b=c\pi^{-m}\ (c\in\mathcal{O}_K)$と表すことができる.
  $\ord_\mathfrak{p}(b)< 0$とする.つまり$0\leq\ord_\mathfrak{p}(c)\leq m-1$.
  $A^2-4\pi b^2=A^2-\pi c^2d^2\in 4\mathcal{O}_K$なので$A^2-\pi c^2d^2\equiv 0\bmod\mathfrak{p}^{2m}$.
  $A=\pi^kt\ (\ord_\mathfrak{p}(t)=0)$,$c=\pi^{s-1}u\ (\ord_\mathfrak{p}(u)=0, s\leq m)$とおけば,この式は$\pi^{2k}t^2-\pi^{2s-1}u^2d^2=\pi^{2m}w\ (\ord_\mathfrak{p}(w)\geq0)$となる.

  $2k>2s-1$であれば$\pi^{2s-1}(\pi^{2k-2s+1}t^2-u^2d^2)=\pi^{2m}w$.命題1.1.3 (3)から$\ord_\mathfrak{p}(\pi^{2k-2s+1}t^2-u^2d^2)=0$なので$2s-1\geq 2m$.
  これは$s\leq m$に矛盾.

  $2k< 2s-1$であれば$\pi^{2k}(t^2-\pi^{2s-2k-1}u^2d^2)=\pi^{2m}w$.
  命題1.1.3 (3)から$\ord_\mathfrak{p}(t^2-\pi^{2s-2k-1}u^2d^2)=0$なので$k\geq m$となり,$2s-1>2k\geq2m$となり$s\leq m$に矛盾.
  以上から$\ord_\mathfrak{p}(b)\geq 0$,つまり$b\in\mathcal{O}_K$.
  $a^2-\pi b^2\in\mathcal{O}_K$なので$a^2\in\mathcal{O}_K$,よって$a\in\mathcal{O}_K$.
  以上から$\mathcal{O}_F$は$\mathcal{O}_K+\mathcal{O}_K\sqrt{\pi}$.

  逆に$a+b\sqrt{\pi}\ (a,b\in\mathcal{O}_K)$は$x^2-2ax+a^2-\pi b^2$の根なので$\mathcal{O}_K$上整.
\end{proof}

\begin{screen}
  $K$を$\mathbb{Q}_2$の有限次拡大,$F=K(\sqrt{\pi'})$($\pi'$は$\mathcal{O}_K$の素元),$\mathfrak{p}$を$\mathcal{O}_K$の素イデアルとするとき,$\varDelta_{F/K}=4\mathfrak{p}$
\end{screen}
\begin{proof}
  $\sqrt{\pi'}$の$K$上最小多項式$x^2-\pi'$の判別式は$4\pi'$.命題1.9.9から$\varDelta_{F/K}(1,\sqrt{\pi'})=4\pi'$.
  $\{1,\sqrt{\pi'}\}$が$\mathcal{O}_F$の$\mathcal{O}_K$基底(上で示した).
  $K$は局所体なので命題1.5.3から$\mathcal{O}_K$は完備離散付値環.
  $\varDelta_{F/K}=\mathfrak{p}^{\ord_\mathfrak{p}(4\pi')}=4\pi'\mathcal{O}_K=4\mathfrak{p}$.
\end{proof}

\begin{screen}
  $K$を$\mathbb{Q}_2$の有限次拡大,$F=K(\sqrt{\mu})\ (\mu\in\mathcal{O}_K^\times)$,$\mathcal{O}_K/\mathfrak{p}=\mathbb{F}$とする.
  $n=\lvert\mathbb{F}\rvert$が偶数であれば$\mathbb{F}^\times\ni x\mapsto x^2\in\mathbb{F}^\times$が全単射である
\end{screen}
\begin{proof}
  定理I-7.4.10から$\mathbb{F}^\times$は位数$n-1$の巡回群:$\mathbb{F}^\times=\set{g^i | 0\leq i\leq n-2}$.
  $g^{n-1}=1$である.$g^i\neq g^j$とする.
  $1\leq\lvert i-j\rvert\leq n-2$なので$2\leq\lvert 2i-2j\rvert\leq 2n-4$.
  $n-1$は奇数なので$2i-2j\neq n-1$.よって$g^{2i}\neq g^{2j}$となり,単射.
  $g^1,g^2,\ldots,g^{n/2-1}$は$g^2,g^4,\ldots,g^{n-2}$に写る.$g^{n/2},g^{n/2+1},\ldots,g^{n-2}$は$g^{n-1+1},g^{n-1+3},\ldots,g^{n-1+n-3}$,つまり$g^1,g^3,\ldots,g^{n-3}$に写るので全射.
\end{proof}

\begin{screen}
  $\bmod\mathfrak{p}^l \to \bmod\mathfrak{p}^{l+1}, \bmod\mathfrak{p}^{l+2}, \ldots$の議論が成立していることの検証(p.91の中央付近)
\end{screen}
\begin{proof}
  本文から,$\ord_\mathfrak{p}(\mu-1)=l\ (\mu\in\mathcal{O}_K^\times\setminus(\mathcal{O}_K^\times)^2)$に対し$l$が$2m$で以下の偶数であれば$\mu'=(1+\pi^{l_1}c)^{-2}\mu$があり,
  $(1+\pi^{l_1}c)^{-2}\equiv 1+\pi^lu\bmod\mathfrak{p}^{l+1}$,$\mu'\equiv 1\bmod\mathfrak{p}^{l+1}$.
  $(1+\pi^{l_1}c)^{-2}\in(\mathcal{O}_K^\times)^2$となる(背理法で容易に示せる)ので$\mu'\in\mathcal{O}_K^\times\setminus(\mathcal{O}_K^\times)^2$(対偶を考えれば明らか).
  $\ord_\mathfrak{p}(\mu'-1)\geq l+1$となり,これによって$l=2m$となるか$l$が奇数になるまで$l$を大きくすることができる.
\end{proof}

\begin{screen}
  $K$を$\mathbb{Q}_2$の有限次拡大,$\mathfrak{p}$を$\mathcal{O}_K$の素イデアル,$F=K(\sqrt{\mu})$とする.
  $\mu\in\mathcal{O}_K^\times\setminus(\mathcal{O}_K^\times)^2$,$\mu=1+4u\ (u\in\mathcal{O}_K^\times)$であれば$F/K$が不分岐であることの証明.
  (上の検証で$l$が$2m$になった場合.$l$が奇数になった場合も同様)
\end{screen}
\begin{proof}
  $\alpha=(1+\sqrt{\mu})/2\in F$とすれば,$\alpha$は$g(x)=x^2-x+(1-\mu)/4$の根.
  $(1-\mu)/4\in\mathcal{O}_K$なので$g(x)\in\mathcal{O}_K[x]$であることから,$\alpha$は$\mathcal{O}_K$上モニックの根となる.
  つまり,$\alpha\in\mathcal{O}_F$.また,命題1.9.9から$\varDelta_{F/K}(1,\alpha)=\mu\neq 0$.
  よって,命題1.7.3 (2)から,$\{1,\alpha\}$は$F$の$K$基底となり$\mathcal{O}_K$上一次独立.
  $\{1,\alpha\}\in\mathcal{O}_F$なので,これは$\mathcal{O}_F$の$\mathcal{O}_K$基底になる.
  よって,$\mathcal{O}_F=\mathcal{O}_K[\alpha]$で,$\varDelta_{F/K}$の計算に$\varDelta_{F/K}(1,\alpha)=\mu$を使える.
  $K$は局所体なので命題1.5.3から$\mathcal{O}_K$は完備離散付値環.
  $\varDelta_{F/K}=\mathfrak{p}^{\ord_\mathfrak{p}(\mu)}=\mu\mathcal{O}_K$.
  $\mu\in\mathcal{O}_K^\times = \mathcal{O}_K \setminus \mathfrak{p}$(命題I-6.5.8)なので,$\varDelta_{F/K}$は$\mathfrak{p}$で割り切れない.よってDedekindの判別定理(定理1.11.12)から$F/K$は不分岐.
\end{proof}

\begin{screen}
  $F$を局所体,$\pi$を$\mathcal{O}_F$の素元,$F=K(\sqrt{\pi})$とすれば,$\sqrt{\pi}$が$\mathcal{O}_F$の素元である
\end{screen}
\begin{proof}
  追加補題\ref{basis_Q2_prime_element}(p.\pageref{basis_Q2_prime_element})から$\mathcal{O}_F=\mathcal{O}_K[\sqrt{\pi}]$,
  つまり$\mathcal{O}_F=\set{a+b\sqrt{\pi} | a,b\in\mathcal{O}_K}$なので,
  $\sqrt{\pi}\mathcal{O}_F=\{\pi b+a\sqrt{\pi}\mid a,b\in\mathcal{O}_K\}$.
  $c,d\in\mathcal{O}_F\setminus\sqrt{\pi}\mathcal{O}_F$とする.
  $c=c_1+c_2\sqrt{\pi}$,$d=d_1+d_2\sqrt{\pi}$とすれば,$c_1,d_1\not\in\pi\mathcal{O}_K$,つまり$\ord_\mathfrak{p}(c_1)=\ord_\mathfrak{p}(d_1)=0$.
  $cd=(c_1d_1+c_2d_2\pi)+(c_1d_2+c_2d_1)\sqrt{\pi}$で,命題1.1.3 (3)から$\ord_\mathfrak{p}(c_1d_1+c_2d_2\pi)=0$なので$cd\in\mathcal{O}_F\setminus\sqrt{\pi}\mathcal{O}_F$.
  つまり,$\sqrt{\pi}\mathcal{O}_F$は素イデアル.
\end{proof}

\begin{screen}
  $F$を局所体,$\pi$を$\mathcal{O}_F$の素元,$\mu=1+\pi^{2k+1}u\ (u\in\mathcal{O}_K^\times)$,$F=K(\sqrt{\mu})$とすれば$q=(1+\sqrt{\mu})/\pi^k$が$\mathcal{O}_F$の素元である
\end{screen}
\begin{proof}
  $F=K(\sqrt{\mu})=K(q)$.
  $q$はEisenstein多項式$g(x)=x^2-\pi^{m-k}x-\pi u$の根なので,$\mathcal{O}_K$上整となり$q\in\mathcal{O}_F$.
  $g(x)$の判別式は$\pi^{2(m-k)}+4\pi u\neq 0$なので命題1.9.9,命題1.7.3から$\{1,q\}$は$F$の$K$基底で$\mathcal{O}_F$の$\mathcal{O}_K$基底:$\mathcal{O}_F=\mathcal{O}_K[q]$,つまり$\mathcal{O}_F=\set{a+bq | a,b\in\mathcal{O}_K}$.
  $g(q)=0$なので$aq+bq^2=bu\pi+(a+b\pi^{m-k})q$.よって$q\mathcal{O}_F=\set{bu\pi+(a+b\pi^{m-k})q | a,b\in\mathcal{O}_K}$.
  $c,d\in\mathcal{O}_F\setminus q\mathcal{O}_F$とする.
  $c=c_1+c_2q$,$d=d_1+d_2q$とすれば$c_1,c_2\not\in\pi\mathcal{O}_K$,つまり$\ord_\mathfrak{p}(c_1)=\ord_\mathfrak{p}(d_1)=0$.
  \[cd=c_1d_1+c_2d_2q^2+(c_1d_2+c_2d_1)q=(c_1d_1+c_2d_2\pi u)+(c_1d_2+c_2d_1+c_2d_2\pi^{m-k})q\]
  で$\ord_\mathfrak{p}(c_1d_1+c_2d_2\pi u)=0$なので,$cd\in\mathcal{O}_F\setminus q\mathcal{O}_F$.
  よって$q\mathcal{O}_F$は$\mathcal{O}_F$の素イデアル.
\end{proof}
