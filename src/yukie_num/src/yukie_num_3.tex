\section{イデアルの相対ノルム}
\paragraph{補題1.10.6}~
\begin{screen}
    $m(\alpha)$が基底$\{\alpha^iv_j\}$に関して$M_\alpha$が対角に$n/l$個並んだブロック行列となる
\end{screen}
\begin{proof}
  $L$の元に$\alpha\ (\alpha^l+a_1\alpha^{l-1}+\cdots+a_l=0)$をかける写像は次のようになる:
  \begin{align*}
    m(\alpha) \colon l = \sum_{j=1}^{n/l}\sum_{i=0}^{l-1}b_{ji}\alpha^iv_j \mapsto \sum_{j=1}^{n/l}\sum_{i=0}^{l-1}b_{ji}\alpha^{i+1}v_j = \sum_{j=1}^{n/l}\left(\sum_{i=0}^{l-2}b_{ji}\alpha^{i+1}v_j+b_{j,l-1}\alpha^lv_j\right) \\
    = \sum_{j=1}^{n/l}\left(\sum_{i=1}^{l-1}b_{j, i-1}\alpha^iv_j-b_{j, l-1}\sum_{i=0}^{l-1}a_{l-i}\alpha^iv_j\right).
  \end{align*}
  ある$j$に対し,$m(\alpha)(l)$の$\alpha^iv_j$成分は$i=0$に対し$-b_{j,l-1}a_l$であり,$i=1,\ldots,l-i$に対し$-b_{j, l-1}a_{l-i}+b_{j, i-1}$となり,これを行列で表せば
  \[
  \left(
  \begin{array}{cccc}
    & & & -a_l\\
    1& & & -a_{l-1}\\
    & \ddots & & \vdots\\
    & & 1 & -a_1
  \end{array}
  \right)\left(
  \begin{array}{c}
    b_{j,0}\\
    b_{j,1}\\
    \vdots\\
    b_{j, l-1}
  \end{array}
  \right)
  \]
  となる.よって,この行列をあらわに書けば
  \[
  \left(
  \begin{array}{cccc|cccc|c|cccc}
    & & & -a_l & & & & & & & &\\
    1& & & -a_{l-1} & & & & & & & &\\
    & \ddots & & \vdots & & & & & & & &\\
    & & 1 & -a_1 & & & & & & & &\\
    \hline
    & & & & & & & -a_l & & & & &\\
    & & & & 1& & & -a_{l-1} & & & & &\\
    & & & & & \ddots & & \vdots & & & & &\\
    & & & & & & 1 & -a_1 & & & & &\\
    \hline
    & & & & & &  &  & \ddots & & & &\\
    \hline
    & & & & & & & & & & & & -a_l\\
    & & & & & & & & & 1 & & & -a_{l-1}\\
    & & & & & & & & & & \ddots & & \vdots\\
    & & & & & & & & & & & 1 & -a_1
  \end{array}
  \right).
  \]
\end{proof}

\paragraph{命題1.10.7}~
\begin{screen}
    $\pi_B\in B$が$A$上整なら$\pi_B$の$K$上最小多項式$f(x)\in K[x]$は$A[x]$の元である
\end{screen}
\begin{proof}
  $\pi_B$は$A$上整なので,$h(\pi_B)=0$なるモニック$h(x)\in A[x]$が存在し,さらに命題I-7.1.11(2)から$h(x)=g(x)f(x)$なる$g(x)\in K[x]$が存在する.
  $\pi_B=\alpha_1,\alpha_2,\ldots,\alpha_n\in\overline{K}$を$f(x)$の根とすると,これらは$h(x)$の根なので$A$上整である.
  $f(x)=\sum_{i=1}^na_ix^i$とおくと,係数$a_i$は$\alpha_i$の基本対称式となるので,やはり$A$上整である($\overline{K}$は$A$の拡大環なので命題I-8.1.4(4)から従う).
  $f(x)\in K[x]$なので,$a_i\in K$.よって,$a_i\in K$は$A$上整となるが,$A$が整閉整域であることから,$a_i\in A$.よって$f(x)\in A[x]$.
\end{proof}

\begin{screen}
    $\pi_B\in P$, $\sigma\in\Hom^\text{al}_{K}(L,\overline{K})$に対し$\sigma(\pi_B)\in\mathcal{P}$
\end{screen}
\begin{proof}
  $\tau\in\Hom^\text{al}_K(\tilde{L},\overline{K})=\Gal(\tilde{L}/K)$について,$\tau C=C$(追加補題\ref{sigmaB}, p.\pageref{sigmaB})となる.
  $\mathcal{P}$は$C$の素イデアルなので,$a,b\in\mathcal{P}$ならば$a+b\in\mathcal{P}$となる.
  $a\in\mathcal{P}, \tau^{-1}(c)\in C\ (c\in C)$に対し$a\tau^{-1}(c)\in\mathcal{P}$.
  よって,$\tau(a),\tau(b)\in\tau(\mathcal{P})$に対し$\tau(a)+\tau(b)=\tau(a+b)\in\tau(\mathcal{P})$, $\tau(a)\in\tau(\mathcal{P}), c\in C$に対し$\tau(a)c=\tau(a)\tau(\tau^{-1}(c))=\tau(a\tau^{-1}(c))\in\tau(\mathcal{P})$となるので,$\tau(\mathcal{P})$は$C$のイデアル.
  $\mathcal{P}$は$C$の素イデアルなので$a,b\in C$について,$\tau^{-1}(a)\tau^{-1}(b)\in\mathcal{P}$ならば$\tau^{-1}(a)\in\mathcal{P}$もしくは$\tau^{-1}(b)\in\mathcal{P}$.
  よって,$\tau(\tau^{-1}(a)\tau^{-1}(b))=ab\in\tau(\mathcal{P})$ならば$a\in\tau(\mathcal{P})$もしくは$b\in\tau(\mathcal{P})$となり,$\tau(\mathcal{P})$は$C$の素イデアル.
  $C$の素イデアルは唯一なので,$\tau(\mathcal{P})=\mathcal{P}$.
  命題I-8.1.4(2)から$C$は$B$上整.命題I-8.1.15から$P$の上にある$C$の素イデアルが存在し,これは$\mathcal{P}$.
  よって$\pi_B\in P\subset\mathcal{P}$.以上から,$\tau(\pi_B)\in\mathcal{P}$.
  $\tau$を$L$に制限すれば$\sigma\in\Hom^\text{al}_K(L,\overline{K})$になるので,$\sigma(\pi_B)\in\mathcal{P}$.
\end{proof}

\begin{screen}
  $A, B$が離散付値環で$P$の$\mathfrak{p}$上の分岐指数を$e$とすれば.
  $a\in K^\times$に対し,$\ord_P(a)=e\ord_\mathfrak{p}(a)$である
\end{screen}
\begin{proof}
  $\mathfrak{p}=\pi_AA$とすれば,$c,d\in A$によって$a=c/d=(\pi_A{}^ic')/(\pi_A{}^jd')=\pi_A{}^{i-j}c'/d'\ (c',d'\notin A\setminus\mathfrak{p})$なので$a=\pi_A{}^ns$なる$s\in\set{c/d | c,d\in A\setminus\mathfrak{p}}=A_\mathfrak{p}^\times$が存在する.
  $s^{-1}\in A_\mathfrak{p}$なので,$aA_\mathfrak{p}=\pi_A{}^nsA_\mathfrak{p}=\pi_A{}^nA_\mathfrak{p}$となり,$\ord_\mathfrak{p}(a)=n$.
  また,分岐指数が$e$なので,$\pi_AB_P=\pi_B{}^eB_P$,つまり$\ord_P(\pi_A)=e$.
  以上から$\ord_P(a)=\ord_P(\pi_A{}^ns)=n\ord(\pi_A)+\ord_P(s)$.
  ここで,$P$が$\mathfrak{p}$の上にあることに注意して$s \in \set{c/d | c,d\in B\setminus P}$なので$\ord_P(s)=0$.
  よって,$=n\ord(\pi_A)=en$.以上から,$\ord_P(a)=e\ord_\mathfrak{p}(a)$.
\end{proof}

\paragraph{補題1.10.10}~
\begin{screen}
  $\Tr_{L/K}(x)=\sum_i\Tr_{\widehat{L}_i/\widehat{K}_\mathfrak{p}}(x)$
\end{screen}
\begin{proof}
  $L$の$K$基底を$\{y_1, \ldots, y_n\}$とする.
  \[L \otimes_K \widehat{K}_\mathfrak{p} \ni a \otimes t = \left(\sum_{i=1}^n a^{(i)} y_i\right) \otimes t \mapsto \sum_{i=1}^n a^{(i)}t y_i \in  \widehat{K}_\mathfrak{p}{}^{\oplus n}\]
  によって$L \otimes_K \widehat{K}_\mathfrak{p}$は$\{y_1, \ldots, y_n\}$を基底とする階数$n$の自由$\widehat{K}_\mathfrak{p}$加群とみなせる(以下,この同型を顕には書かない).
  $L$の元に$x$をかける準同型
  \[\phi\colon L\ni a = \sum_{i=1}^n a^{(i)} y_i \mapsto ax = \sum_{i=1}^n b^{(i)} y_i \in L \quad (a^{(i)},b^{(i)}\in K)\]
  を考える.
  $\phi$は$L\otimes_K\widehat{K}_\mathfrak{p}$の元に$x\otimes1$をかける準同型
  \[\widehat{\phi} \colon L \otimes_K \widehat{K}_\mathfrak{p} \ni a \otimes t = \sum_{i=1}^n a^{(i)}t y_i \mapsto ax \otimes t = \sum_{i=1}^n b^{(i)}t y_i \in L\otimes_K\widehat{K}_\mathfrak{p}\]
  を誘導する.
  $\phi$を$K$基底$\{y_1, \ldots, y_n\}$で表現した行列と$\widehat{\phi}$を$\widehat{K}_\mathfrak{p}$基底$\{y_1, \ldots, y_n\}$で表した行列は同じなので,そのトレースは等しい:$\Tr_K(\phi)=\Tr_{\widehat{K}_\mathfrak{p}}(\widehat{\phi})$.
  補題1.10.6から$\Tr_{L/K}(x)=\Tr_K(\phi)$なので,
  \[\Tr_{L/K}(x)=\Tr_K(\phi)=\Tr_{\widehat{K}_\mathfrak{p}}(\widehat{\phi}).\]

  定理1.3.23(2)から同型$\varphi \colon L \otimes_K \widehat{K}_\mathfrak{p} \to \widehat{L}_1 \times \cdots \times \widehat{L}_g$が存在するので,$\widehat{K}_\mathfrak{p}$準同型
  \[\phi_x=\varphi\widehat{\phi}\varphi^{-1} \colon \widehat{L}_1\times\cdots\times\widehat{L}_g\to \widehat{L}_1\times\cdots\times\widehat{L}_g\]
  が存在し,$\Tr_{\widehat{K}_\mathfrak{p}}(\widehat{\phi})=\Tr_{\widehat{K}_\mathfrak{p}}(\varphi\widehat{\phi}\varphi^{-1})=\Tr_{\widehat{K}_\mathfrak{p}}(\phi_x)$.
  \[
  \begin{CD}
    \widehat{L}_1\times\cdots\times\widehat{L}_g @>{\phi_x}>> \widehat{L}_1\times\cdots\times\widehat{L}_g\\
    @A{\varphi}AA  @A{\varphi}AA\\
    L\otimes_K\widehat{K}_\mathfrak{p} @>>{\widehat{\phi}}> L\otimes_K\widehat{K}_\mathfrak{p}\\
    @AAA @AAA\\
    \widehat{K}_\mathfrak{p} @= \widehat{K}_\mathfrak{p}
  \end{CD}
  \]
  $\varphi\widehat{\phi}=\phi_x\varphi$を$a\otimes t\in L\otimes_K\widehat{K}_\mathfrak{p}$に作用させて$\phi_x(\varphi(a\otimes t)) = \varphi(a\otimes t)\varphi(x\otimes1)$となる.
  $\varphi_i \colon B\hookrightarrow\widehat{B}_i$として,$\varphi(x\otimes1) = (\varphi_1(x), \ldots, \varphi_g(x))$である(定理1.3.23(3))ので,$\phi_x$は$L_1\times\cdots\times L_g$の元に$(\varphi_1(x), \ldots, \varphi_g(x))$をかける写像である:
  \[\phi_x\colon\widehat{L}_1\times\cdots\times\widehat{L}_g\ni (c_1,\ldots,c_g)\mapsto(\varphi_1(x)c_1, \ldots, \varphi_g(x)c_g)\in \widehat{L}_1\times\cdots\times\widehat{L}_g.\]
  $\phi_x$を$\widehat{L}_i$に制限した$\widehat{K}_\mathfrak{p}$準同型
  \[\psi_{x,i}\colon\widehat{L}_i\ni c_i\mapsto \varphi_i(x)c_i\in\widehat{L}_i\]
  を考える.
  $[\widehat{L}_i:\widehat{K}_\mathfrak{p}]=n_i$とすれば,定理1.3.23(4)(5)から$n_1+\cdots+n_g=n$.
  $\widehat{L}_i$の$\widehat{K}_\mathfrak{p}$基底を$\{z_i^{(1)},\ldots,z_i^{(n_i)}\}$とすれば$\widehat{L}_1\times\cdots\times\widehat{L}_g$の$\widehat{K}_\mathfrak{p}$基底として
  \[\left\{(z_1^{(1)},\ldots,0),\ldots,(z_1^{(n_1)},\ldots,0);(0,z_2^{(1)},\ldots,0),\ldots,(0,z_2^{(n_2)},\ldots,0);\ldots;(0,\ldots,z_g^{(1)}),\ldots,(0,\ldots,z_g^{(n_g)})\right\}\]
  を取ることができる.

  $\psi_{x,i}$を$\{z_i^{(1)},\ldots,z_i^{(n_i)}\}$で表現した行列を$M_i$とすれば,$\phi_x$を$\{(z_1^{(1)},\ldots,0),\ldots,(0,\ldots,z_g^{(n_g)})\}$で表現した行列は
  \[
  \begin{pmatrix}
    M_1 &        & \\
    & \ddots & \\
    &        & M_g
  \end{pmatrix}
  \]
  となる.よって,$\Tr_{\widehat{K}_\mathfrak{p}}(\phi_x)=\sum_i\Tr_{\widehat{K}_\mathfrak{p}}(\psi_{x,i})$.補題1.10.6から$\Tr_{\widehat{K}_\mathfrak{p}}(\psi_{x,i})=\Tr_{\widehat{L}_i/\widehat{K}_\mathfrak{p}}(x)$(包含写像$\varphi_i$は省略した).
  以上から,$\Tr_{L/K}(x)=\sum_i\Tr_{\widehat{L}_i/\widehat{K}_\mathfrak{p}}(x)$となる.
\end{proof}

\paragraph{補題1.10.12}~
\begin{screen}
  完備化と加法的付値
\end{screen}
\begin{proof}
  $x\in K^\times$に対し$A$での加法的付値は$xA_\mathfrak{p}=\mathfrak{p}^{\ord_\mathfrak{p}(x)}A_\mathfrak{p}$で定義される.
  また,$\mathfrak{p}$進付値は$\lvert x\rvert_\mathfrak{p}=\lvert A/\mathfrak{p}\rvert^{-\ord_\mathfrak{p}(x)}$で定義される.
  更に,$\mathfrak{p}$距離$d\colon K\times K\ni(x,y)\mapsto\lvert x-y\rvert_\mathfrak{p}\in\lvert A/\mathfrak{p}\rvert^a\ (a\in\mathbb{Z})$を定義する.
  距離空間$(K,d)$を完備化して距離空間$(\widehat{K}_\mathfrak{p},\widehat{d})$が得られる.
  補題1.2.6(1)から$\widehat{d}\colon\widehat{K}_\mathfrak{p}\times\widehat{K}_\mathfrak{p}\ni(\widehat{x},\widehat{y})\mapsto\lvert A/\mathfrak{p}\rvert^a\ (a\in\mathbb{Z})$.
  完備化の定義から$x\in K$と$\vartheta\colon K\hookrightarrow\widehat{K}_\mathfrak{p}$によって
  \[\widehat{d}(\vartheta(x),\vartheta(0))=d(x,0)=\lvert A/\mathfrak{p}\rvert^{-\ord_\mathfrak{p}(x)}=\lvert x\rvert_\mathfrak{p}.\]

  ところで,定理1.2.8(7)から$\widehat{A}_\mathfrak{p}$は$\widehat{\mathfrak{p}}=\mathfrak{p}\widehat{A}_\mathfrak{p}$を極大イデアルとする離散付値環である.
  ここで,$\widehat{x}\in\widehat{K}_\mathfrak{p}$に対し
  \[\widehat{d}(\widehat{x},\vartheta(0))=\lvert\widehat{x}\rvert_{\widehat{\mathfrak{p}}}=\lvert\widehat{A}_\mathfrak{p}/\widehat{\mathfrak{p}}\rvert^{-\ord_{\widehat{\mathfrak{p}}}(\widehat{x})}\]
  によって$\lvert\bullet\rvert_{\widehat{\mathfrak{p}}}$と$\ord_{\widehat{\mathfrak{p}}}(\bullet)$を定義する.
  命題1.2.13(1)から$\widehat{A}_\mathfrak{p}/\widehat{\mathfrak{p}}\simeq A/\mathfrak{p}$なので
  \[\widehat{d}(\widehat{x},\vartheta(0))=\lvert\widehat{x}\rvert_{\widehat{\mathfrak{p}}}=\lvert\widehat{A}_\mathfrak{p}/\widehat{\mathfrak{p}}\rvert^{-\ord_{\widehat{\mathfrak{p}}}(\widehat{x})}=\lvert A/\mathfrak{p}\rvert^{-\ord_{\widehat{\mathfrak{p}}}(\widehat{x})}.\]
  $\vartheta(x)$に対しては,$\lvert\vartheta(x)\rvert_{\widehat{\mathfrak{p}}}=\lvert x\rvert_\mathfrak{p}$, $\ord_{\widehat{\mathfrak{p}}}(\vartheta(x))=\ord_\mathfrak{p}(x)$.
  更に,定理1.2.8(6)から$\widehat{x}\in\widehat{K}_\mathfrak{p}$として,
  \[\lvert\widehat{A}_\mathfrak{p}/\widehat{\mathfrak{p}}\rvert^{-\ord_{\widehat{\mathfrak{p}}}(\widehat{x})}\leq\lvert\widehat{A}_\mathfrak{p}/\widehat{\mathfrak{p}}\rvert^{-n}\Leftrightarrow n\leq\ord_{\widehat{\mathfrak{p}}}(\widehat{x})\Leftrightarrow\widehat{x}\in\mathfrak{p}^n\widehat{A}_\mathfrak{p}=\widehat{\mathfrak{p}}^n\widehat{A}_\mathfrak{p}.\]
  よって,$\widehat{x}\in\widehat{\mathfrak{p}}^{\ord_{\widehat{\mathfrak{p}}}(\widehat{x})}\widehat{A}_\mathfrak{p}$, $\widehat{x}\notin\widehat{\mathfrak{p}}^{\ord_{\widehat{\mathfrak{p}}}(\widehat{x})+1}\widehat{A}_\mathfrak{p}$となるので,$\widehat{x}\widehat{A}_\mathfrak{p}=\widehat{\mathfrak{p}}^{\ord_{\widehat{\mathfrak{p}}}(\widehat{x})}\widehat{A}_\mathfrak{p}$.
  これは(離散付値環での)加法的付値の定義と一致している.

  以上から,$\widehat{x}\in\widehat{K}_\mathfrak{p}$の加法的付値$\ord_{\widehat{\mathfrak{p}}}$と$\widehat{\mathfrak{p}}$進付値$\lvert\bullet\rvert_{\widehat{\mathfrak{p}}}$は
  \[\widehat{x}\widehat{A}_\mathfrak{p}=\widehat{\mathfrak{p}}^{\ord_{\widehat{\mathfrak{p}}}(\widehat{x})}\widehat{A}_\mathfrak{p},\quad\lvert\widehat{x}\rvert_{\widehat{\mathfrak{p}}}=\lvert\widehat{A}_\mathfrak{p}/\widehat{\mathfrak{p}}\rvert^{-\ord_{\widehat{\mathfrak{p}}}(\widehat{x})}\]
  で定まり,$x\in K$に対しては(ほとんどの場合省略されるが)$\vartheta\colon K\hookrightarrow\widehat{K}_\mathfrak{p}$によって
  \[\ord_\mathfrak{p}(x)=\ord_{\widehat{\mathfrak{p}}}(\vartheta(x)),\quad \lvert x\rvert_\mathfrak{p}=\lvert\vartheta(x)\rvert_{\widehat{\mathfrak{p}}}.\]
\end{proof}

\begin{screen}
    $\pi\in B$に対し$\ord_{P_i}(\pi)=0\ (i=2,\ldots,g)$ならば,$\ord_\mathfrak{p}\left(\N_{L/K}(\pi)\right)=\ord_{\widehat{\mathfrak{p}}}\left(\N_{\widehat{L}_1/\widehat{K}_\mathfrak{p}}(\pi)\right)$
\end{screen}
\begin{proof}
  補題1.10.10から$\N_{L/K}(\pi)=\prod_i\N_{\widehat{L}_i/\widehat{K}_\mathfrak{p}}(\pi)$となる.命題1.1.13を使えば,
  \[\ord_{\widehat{\mathfrak{p}}}(\vartheta(\N_{L/K}(\pi)))=\ord_\mathfrak{p}(\N_{L/K}(\pi))=\sum_i\ord_{\widehat{\mathfrak{p}}}\left(\N_{\widehat{L}_i/\widehat{K}_\mathfrak{p}}(\pi)\right).\]
  $i=2,\ldots,g$に対し,$\ord_{P_i\widehat{B}_i}(\pi)=0$なので,$\pi\in\widehat{B}_i^\times$.
  よって命題I-8.5.2から$\N_{\widehat{L}_i/\widehat{K}_\mathfrak{p}}(\pi)\in\widehat{A}_\mathfrak{p}^\times$となるので$i=2,\ldots,g$に対し$\ord_{\widehat{\mathfrak{p}}}\left(\N_{\widehat{L}_i/\widehat{K}_\mathfrak{p}}(\pi)\right)=0$.
  以上から,$\ord_\mathfrak{p}\left(\N_{L/K}(\pi)\right)=\ord_{\widehat{\mathfrak{p}}}\left(\N_{\widehat{L}_1/\widehat{K}_\mathfrak{p}}(\pi)\right)$.
\end{proof}

\begin{screen}
    Dedekind環と完備化
\end{screen}
\begin{proof}
  $A$をDedekind環,$K$を$A$の商体,$L$を$K$の有限次拡大,$B$を$L$の$A$における整閉包,$\mathfrak{p}$を$A$の素イデアルとする.この時,$B$はDedekind環で,$L$は$B$の商体となる(命題1.3.2).
  $\mathfrak{p}$の上にある$B$の素イデアルを$P_1,\ldots,P_g$とする.
  \[
  \begin{CD}
    P_1            @<\text{PI}<< B @>\text{QF}>> L\\
    @AA\text{aPI}A @AA\text{IC}A   @AA\text{FE}A \\
    \mathfrak{p}   @<\text{PI}<< A @>\text{QF}>> K
  \end{CD}
  \]
  (PI:素イデアル,QF:商体,aPI:上にある素イデアル,IC:整閉包,FE:有限次拡大)

  $A$, $K$を$\mathfrak{p}$進距離で完備化すると位相環$\widehat{A}_\mathfrak{p}$,位相体$\widehat{K}_\mathfrak{p}$が得られる(系1.2.9).
  $\widehat{K}_\mathfrak{p}$は$\widehat{A}_\mathfrak{p}$の商体であり,$\widehat{A}_\mathfrak{p}$は$\mathfrak{p}\widehat{A}_\mathfrak{p}$を極大イデアルとする離散付値環である(定理1.2.8(7)).更に,$B$, $L$を$P_1$進距離で完備化すると位相環$\widehat{B}_1$,位相体$\widehat{L}_1$が得られ,$\widehat{L}_1$は$\widehat{B}_1$の商体であり,$\widehat{B}_1$は$P_1\widehat{B}_1$を極大イデアルとする離散付値環である.
  $P_1\widehat{B}_1$は$\mathfrak{p}\widehat{A}_\mathfrak{p}$の上にある素イデアルである(補題1.3.3).この時,$\widehat{L}_1$は$\widehat{K}_\mathfrak{p}$の有限次拡大で$\widehat{B}_1$は$\widehat{L}_1$における$\widehat{A}_\mathfrak{p}$の整閉包である(定理1.3.23(3)).
  \[
  \begin{CD}
    P_1\widehat{B}_1                     @<\text{MI1}<< \widehat{B}_1            @>\text{QF}>> \widehat{L}_1\\
    @AA\text{aPI}A                   @AA\text{IC}A                      @AA\text{FE}A \\
    \mathfrak{p}\widehat{A}_\mathfrak{p} @<\text{MI1}<< \widehat{A}_\mathfrak{p} @>\text{QF}>> \widehat{K}_\mathfrak{p}
  \end{CD}
  \]
  (MI1:唯一の極大イデアル)

  $M$を$\widehat{L}_1/\widehat{K}_\mathfrak{p}$の中間体とする.
  $M$と$\widehat{B}_1$は環なので$M\cap\widehat{B}_1$は環であり,$M\cap\widehat{B}_1$は$M$の$\widehat{A}_\mathfrak{p}$における整閉包である(命題I-8.1.4(3)).
  $x\in\widehat{L}_1$が$M\cap\widehat{B}_1$上整なら$\widehat{B}_1$上整となる.
  $\widehat{B}_1$は整閉整域なので$x\in\widehat{B}_1$.
  $x\in\widehat{B}_1$は$\widehat{A}_\mathfrak{p}$上整なので$M\cap\widehat{B}_1$上整である(命題I-8.1.4(2)).以上から,$\widehat{L}_1$の$M\cap\widehat{B}_1$における整閉包は$\widehat{B}_1$である.
  $M\cap\widehat{B}_1$はDedekind環で,$M\cap\widehat{B}_1$の商体は$M$である(命題1.3.2).

  $M\cap P_1\widehat{B}_1$が$M\cap\widehat{B}_1$の素イデアルであることは容易に示せる.

  $M\supset\mathfrak{p}\widehat{A}_\mathfrak{p}$, $P_1\widehat{B}_1\supset\mathfrak{p}\widehat{A}_\mathfrak{p}$なので$M\cap P_1\widehat{B}_1$は$\mathfrak{p}A_\mathfrak{p}$の上にある素イデアルである(補題1.3.3).
  $P_1\widehat{B}_1\supset M\cap P_1\widehat{B}_1$なので$P_1\widehat{B}_1$は$M\cap P_1\widehat{B}_1$の上にある素イデアルである(補題1.3.3).
  $M\cap\widehat{B}_1$は$M\cap P_1\widehat{B}_1$を極大イデアルとする完備離散付値環である(命題1.5.3).
  \[
  \begin{CD}
    P_1\widehat{B}_1                     @<\text{MI}<< \widehat{B}_1            @>\text{QF}>> \widehat{L}_1\\
    @AA\text{aPI}A                   @AA\text{IC}A                      @AA\text{FE}A \\
    M\cap P_1\widehat{B}_1               @<\text{MI}<< M\cap\widehat{B}_1       @>\text{QF}>> M\\
    @AA\text{aPI}A                   @AA\text{IC}A                      @AA\text{FE}A \\
    \mathfrak{p}\widehat{A}_\mathfrak{p} @<\text{MI}<< \widehat{A}_\mathfrak{p} @>\text{QF}>> \widehat{K}_\mathfrak{p}
  \end{CD}
  \]

  $A$が仮定1.1.2を満たせば,$\widehat{A}_\mathfrak{p}$も仮定1.1.2を満たす(命題1.2.13(4)).
  $M\cap\widehat{B}_1$と$\widehat{B}_1$も仮定1.1.2を満たす(命題1.3.2).
\end{proof}

\begin{screen}
  \begin{thm}\label{CDVR_unr}
    上の状況で,剰余体$k=\widehat{A}_\mathfrak{p}/\mathfrak{p}\widehat{A}_\mathfrak{p}$, $l=\widehat{B}_1/P_1\widehat{B}_1$を考える.
    $a\in l$に対し,$[k(a):k]=[\widehat{K}_\mathfrak{p}(\alpha):\widehat{K}_\mathfrak{p}]$となる$\alpha\in\widehat{B}_1$が存在し,$\widehat{K}_\mathfrak{p}(\alpha)/\widehat{K}_\mathfrak{p}$は不分岐拡大である
  \end{thm}
\end{screen}
\[
\begin{CD}
  P_1\widehat{B}_1 @<\text{MI}<< \widehat{B}_1 @>\text{QF}>> \widehat{L}_1 @. l \\
  @AA\text{aPI}A @AA\text{IC}A @AA\text{FE}A @AA\text{FE}A \\
  \widehat{K}_\mathfrak{p}(\alpha)\cap P_1\widehat{B}_1 @<\text{MI}<< \widehat{K}_\mathfrak{p}(\alpha)\cap\widehat{B}_1 @>\text{QF}>> \widehat{K}_\mathfrak{p}(\alpha)\qquad @. \qquad k(a) \\
  @AA\text{aPI}A @AA\text{IC}A @AA{\text{FE}, \Phi(x)}A @AA{\text{FE}, \phi(x)}A \\
  \mathfrak{p}\widehat{A}_\mathfrak{p} @<\text{MI}<< \widehat{A}_\mathfrak{p} @>\text{QF}>> \widehat{K}_\mathfrak{p} @. k
\end{CD}
\]
\begin{proof}
  $\widehat{A}_\mathfrak{p}$は仮定1.1.2を満たすので,剰余体は有限体.
  $a$の$k$上最小多項式を$\phi(x)\in k[x]$とする.
  $k$は有限体なので完全体(系I-7.3.6)となり,$k(a)/k$は分離拡大である(定義I-7.3.1(5)).
  よって$\phi(x)$は分離多項式で,$\phi(a)=0$, $\phi'(a)\neq0$.
  $\Phi(x)\equiv\phi(x)\bmod\mathfrak{p}\widehat{A}_\mathfrak{p}$となるモニック$\Phi(x)\in\widehat{A}_\mathfrak{p}[x]$及び$\alpha_0\equiv a\bmod P_1\widehat{B}_1$となる$\alpha_0\in\widehat{B}_1$を考える($\alpha_0$は$a$の代表元).
  $\Phi(\alpha_0)\equiv 0\bmod P_1\widehat{B}_1$, $\Phi'(\alpha_0)\not\equiv 0\bmod P_1\widehat{B}_1$であるので,Henselの補題から$\Phi(\alpha)=0$となる$\alpha\in\widehat{B}_1\ (\alpha\equiv\alpha_0\bmod P_1\widehat{B}_1)$が存在する.
  $\phi(x)$は$k$($\widehat{A}_\mathfrak{p}/\mathfrak{p}\widehat{A}_\mathfrak{p}$の商体)上既約なので,命題I-8.2.1とp.I-233の注から$\Phi(x)$は$\widehat{K}_\mathfrak{p}$上既約.
  よって,$\Phi(x)$は$\widehat{K}_\mathfrak{p}(\alpha)$の$\widehat{K}_\mathfrak{p}$上最小多項式となり,$[k(\alpha):k]=\deg (\phi(x))=\deg (\Phi(x))=[\widehat{K}_\mathfrak{p}(\alpha):\widehat{K}_\mathfrak{p}]$.

  $\widehat{K}_\mathfrak{p}(\alpha)\cap\widehat{B}_1$の剰余体$m=\widehat{K}_\mathfrak{p}(\alpha)\cap\widehat{B}_1/\widehat{K}_\mathfrak{p}(\alpha)\cap P_1\widehat{B}_1$を考える.
  $\alpha\in\widehat{K}_\mathfrak{p}(\alpha)\cap\widehat{B}_1$なので,$a\in m$である($\vartheta\colon m \hookrightarrow l$について$a \in \Im \vartheta$).
  $k(a)$は$a$を含む最小の体なので$[m:k(a)]\geq 1$.先程の結果と定理1.3.23(4)から
  \[[k(a):k]=[\widehat{K}_\mathfrak{p}(\alpha):\widehat{K}_\mathfrak{p}]\geq f(\widehat{K}_\mathfrak{p}(\alpha)\cap P_1\widehat{B}_1:\mathfrak{p}\widehat{A}_\mathfrak{p})=[m:k]\]
  となるので,$[k(a):m]\geq 1$.以上から$[k(a):m]=1$,
  \[[\widehat{K}_\mathfrak{p}(\alpha):\widehat{K}_\mathfrak{p}]=f(\widehat{K}_\mathfrak{p}(\alpha)\cap P_1\widehat{B}_1:\mathfrak{p}\widehat{A}_\mathfrak{p})=[m:k]=[k(a):k]\]
  となるので,再び定理1.3.23(4)から分岐指数は$e(\widehat{K}_\mathfrak{p}(\alpha)\cap P_1\widehat{B}_1:\mathfrak{p}\widehat{A}_\mathfrak{p})=1$.つまり,$\widehat{K}_\mathfrak{p}(\alpha)/\widehat{K}_\mathfrak{p}$は不分岐拡大.
\end{proof}

\begin{screen}
  $\pi$の$M$上最小多項式$f(x)$は$\widehat{B}_M = M\cap\widehat{B}_1$上のEisenstein多項式で,$\deg (f(x))=[\widehat{L}_1:M]$である
\end{screen}
\begin{proof}
  $\ord_{P_1}(\pi)=1$より$\pi\widehat{B}_1=\widehat{P}_1\widehat{B}_1$となるので,$\pi$は$\widehat{B}_1$の素元.
  $\widehat{L}_1/M$は完全分岐なので,命題1.10.7から$\pi$の$M$上最小多項式はEisenstein多項式で,その次数は$[\widehat{L}_1:M]$に等しい
  (命題の主張には書かれていないが,証明されている:p.62).
  また,Eisenstein多項式の定義から$f(x)\in\widehat{B}_M[x]$($M$に対応する離散付値環は$\widehat{B}_M$).
\end{proof}

\paragraph{命題1.10.13}~
\begin{screen}
  $Q_1,\ldots,Q_g$が$P_1,\ldots,P_t$と互いに素
\end{screen}
\begin{proof}
  $Q_i\cap A=\mathfrak{p}$, $P_j\cap A=\mathfrak{p}_j$に注意すれば,$Q_i=P_j \Rightarrow \mathfrak{p}=\mathfrak{p}_j$.
  対偶をとって,$\mathfrak{p}\neq\mathfrak{p}_j \Rightarrow Q_i\neq P_j$.
  $\mathfrak{p}\neq\mathfrak{p}_j$なので,$Q_i$と$P_j$は互いに素.
\end{proof}

\begin{screen}
  $I$の相対イデアル$\N_{L/K}(I)$も$\mathfrak{p}$と互いに素
\end{screen}
\begin{proof}
  相対イデアル$\N_{L/K}(I)$は$\Set{\N_{L/K}(b) | b\in I}$で生成される$A$のイデアル(定義1.10.1)なので,$A$の適当なイデアル$J$によって,$\N_{L/K}(I)=\left(\N_{L/K}(x)\right)+J$と書くことができる.
  $\left(\N_{L/K}(x)\right)$は$\mathfrak{p}$と互いに素なので,素イデアル分解には$\mathfrak{p}$は現れない:$\left(\N_{L/K}(x)\right)=\mathfrak{p}^0\cdots$.
  よって,定理I-8.3.17(3)から$\N_{L/K}(I)=\left(\N_{L/K}(x)\right)+J$の素イデアル分解は$\mathfrak{p}^0\cdots$となる,つまり$\N_{L/K}(I)$は$\mathfrak{p}$と互いに素.
\end{proof}

\begin{screen}
  $\N_{L/K}(I)\supset\mathfrak{p}_1{}^{f_1}\cdots\mathfrak{p}_t{}^{f_t}$
\end{screen}
\begin{proof}
  $\N_{L/K}(I)\supset\mathfrak{p}_1{}^{f_1}\cdots\mathfrak{p}_t{}^{f_t}J$なので,定理I-8.3.17(2)から$A$のイデアル$J'$があり,$J'\N_{L/K}(I)=\mathfrak{p}_1{}^{f_1}\cdots\mathfrak{p}_t{}^{f_t}J$.
  $J$の素イデアル分解を$\tilde{\mathfrak{p}}_1{}^{a_1}\cdots\tilde{\mathfrak{p}}_s{}^{a_s}$とすれば,$J'\N_{L/K}(I)=\mathfrak{p}_1{}^{f_1}\cdots\mathfrak{p}_t{}^{f_t}\tilde{\mathfrak{p}}_1{}^{a_1}\cdots\tilde{\mathfrak{p}}_s{}^{a_s}$.
  $\N_{L/K}(I)$が$J$と互いに素であることから,$\N_{L/K}(I)$の素イデアル分解には$\tilde{\mathfrak{p}}_1,\ldots,\tilde{\mathfrak{p}}_s$は現れない.
  よって,$J'=\tilde{\mathfrak{p}}_1{}^{a_1}\cdots\tilde{\mathfrak{p}}_s{}^{a_s}J''$となるので,$\tilde{\mathfrak{p}}_1{}^{a_1}\cdots\tilde{\mathfrak{p}}_s{}^{a_s}J''\N_{L/K}(I)=\mathfrak{p}_1{}^{f_1}\cdots\mathfrak{p}_t{}^{f_t}\tilde{\mathfrak{p}}_1{}^{a_1}\cdots\tilde{\mathfrak{p}}_s{}^{a_s}$.
  よって,$J''\N_{L/K}(I)=\mathfrak{p}_1{}^{f_1}\cdots\mathfrak{p}_t{}^{f_t}$.
  定理I-8.3.17(2)から$\N_{L/K}(I)\supset\mathfrak{p}_1{}^{f_1}\cdots\mathfrak{p}_t{}^{f_t}$.
\end{proof}

\begin{screen}
  $I\widehat{B}_1=Q_{1,1}{}^{m_1}\widehat{B}_1=x\widehat{B}_1$
\end{screen}
\begin{proof}
  $x_1,\ldots,x_{m_1}$の$Q_{1,1}=P_1=\cdots=P_{m_1}$に関する加法的付値が$1$,$x_{m_1+1},\ldots,x_t$の$P_1$に関する加法的付値が$0$となる.よって,$\ord_{Q_{1,1}}(x_1)=1$なので,$x_1\widehat{B}_1=Q_{1,1}\widehat{B}_1$.また,$\ord_{Q_{1,1}}(x_t)=0$なので$x_t\widehat{B}_1=\widehat{B}_1$,つまり$x_t\in\widehat{B}_1^\times$.以上から,$x_1,\ldots,x_{m_1}$は$Q_{1,1}\widehat{B}_1$を生成し,$x_{m_1+1},\ldots,x_t$は$\widehat{B}_1^\times$の元である.
  $i\geq m_1+1$に対し$x_i \in P_i$,$x_i\in\widehat{B}_1^\times$なので$P_iP_1\widehat{B}_1=P_1\widehat{B}_1$.よって,$I=P_1\cdots P_t=Q_{1,1}{}^{m_1}P_{m_1+1}\cdots P_t$なので$I\widehat{B}_1=Q_{1,1}{}^{m_1}\widehat{B}_1$.
  $x=x_1\cdots x_t$で$x_1\widehat{B}_1=\cdots=x_{m_1}\widehat{B}_1=Q_{1,1}\widehat{B}_1$,$x_{m_1+1},\ldots,x_t\in\widehat{B}_1^\times$なので$x\widehat{B}_1=Q_{1,1}{}^{m_1}\widehat{B}_1$.
\end{proof}

\begin{screen}
  $I\widehat{B}_1=x\widehat{B}_1$,$y\in I$なら$z\in\widehat{B}_1$が存在して$y=zx$となる
\end{screen}
\begin{proof}
  $x\widehat{B}_1=I\widehat{B}_1\supset y\widehat{B}_1$なので,定理I-8.3.17(2)から単項イデアル整域$\widehat{B}_1$のイデアル$z'\widehat{B}_1$が存在し$xz'\widehat{B}_1=y\widehat{B}_1$となる.
  $y\in y\widehat{B}_1$なので$y\in xz'\widehat{B}_1$,つまり$b\in\widehat{B}_1$があり$y=xz'b$.
  $z=z'b\in\widehat{B}_1$とすれば主張が従う.
\end{proof}

\paragraph{命題1.10.19}~
\begin{screen}
  $IA_\mathfrak{p}$が$\varDelta_{B/A,\mathfrak{p}}$を割り切る
\end{screen}
\begin{proof}
  $I$の定義から$s^{2n}\varDelta_{L/K}(u_1,\ldots,u_n)=\varDelta_{L/K}(su_1,\ldots,su_n)A=Ia$となる$a\in A$が存在する.
  よって,$aIA_\mathfrak{p}=s^{2n}\varDelta_{L/K}(u_1,\ldots,u_n)A_\mathfrak{p}$.
  $s^{2n}\in A_\mathfrak{p}^\times$なので,$=\varDelta_{L/K}(u_1,\ldots,u_n)A_\mathfrak{p}=\varDelta_{B/A,\mathfrak{p}}A_\mathfrak{p}$となるので従う.
\end{proof}

\section{完備化とDedekindの判別定理}

\begin{screen}
  $B$の$A$基底は$B_\mathfrak{p}$の$A_\mathfrak{p}$基底であり,$L$の$K$基底である
\end{screen}
\begin{proof}
  $B$の$A$基底を$\{w_1,\ldots,w_n\}$とする.$\forall b\in B$は$\sum y_iw_i\ (y_i\in A)$と表すことができる.
  $B_\mathfrak{p}$の任意の元は$b\in B$と$s\in A\setminus\mathfrak{p}$で$b/s$と表せる.上の式を代入して$b/s=\sum_{i=1}^n(y_i/s)w_i,\quad (y_i/s\in A_\mathfrak{p})$.
  $B_\mathfrak{p}$の任意の元は$\{w_1,\ldots,w_n\}$の$A_\mathfrak{p}$係数の線形結合で表すことができる.
  $\sum (y_i/s_i)w_i=0\ (y_i\in A,\quad s_i\in A\setminus\mathfrak{p})$とする.
  $\sum_{i=1}^n (y_is_1\cdots s_n/s_i)w_i=0$
  となり,$w_i$の係数は$A$の元.
  $\{w_1,\ldots,w_n\}$の$A$基底としての一時独立性から,$y_i=0$.よって$A_\mathfrak{p}$基底として一次独立.

  $L/K$も同様.
  $S=A\setminus\{0\}$として$L=S^{-1}B$と表せることを使う.
\end{proof}

\paragraph{命題1.11.1}~
\begin{screen}
  $\varDelta_{B/A,\mathfrak{p}}=\prod_{i=1}^g\varDelta_{\widehat{B}_i/\widehat{A}_\mathfrak{p}}$
\end{screen}
\begin{proof}
  $\{w_1, \ldots, w_n\}$を$B$の$A$基底とする.
  $m(w_i)$を$B$の元に$w_i \in B$をかける写像とする:
  \[m(w_i) \colon B \ni a = \sum_{i=1}^n a^{(i)}w_i \mapsto aw_i = \sum_{i=1}^n b^{(i)}w_i \in B.\]
  これは$B \otimes_A \widehat{A}_\mathfrak{p}$の元に$w_i \otimes 1 \in B \otimes_A \widehat{A}_\mathfrak{p}$をかける写像$\tilde{m}(w_i)$を誘導する:
  \[\tilde{m}(w_i) \colon B \otimes_A \widehat{A}_\mathfrak{p} \ni a \otimes t = \sum_{i=1}^n a^{(i)}w_i \otimes t \mapsto aw_i \otimes t = \sum_{i=1}^n b^{(i)}w_i \otimes t \in B \otimes_A \widehat{A}_\mathfrak{p}.\]
  定理1.3.23(2)から,$\phi_i\colon B\hookrightarrow\widehat{B}_i$として同型
  \[\phi\colon B \otimes_A \widehat{A}_\mathfrak{p} \ni a\otimes t\rightarrow(\phi_1(a)t, \ldots, \phi_g(a)t) \in \widehat{B}_1 \times \cdots \times \widehat{B}_g\]
  が得られる.$\phi$によって$\{w_i\}_{1 \leq i \leq n}$は$\{(\phi_1(w_i), \ldots, \phi_g(w_i))\}_{1 \leq i \leq g}$に写るので,$\{(\phi_1(w_i), \ldots, \phi_g(w_i))\}_{1 \leq i \leq g}$は$\widehat{B}_1 \times \cdots \times \widehat{B}_g$の$\widehat{A}_\mathfrak{p}$基底である.

  $\widehat{m}=\phi \tilde{m} \phi^{-1}$として,
  \[\widehat{m} \colon \widehat{B}_1\times\cdots\times\widehat{B}_g \ni \phi(a \otimes t) \mapsto \phi(aw_i \otimes t) \in \widehat{B}_1\times\cdots\times\widehat{B}_g\]
  を構成する.
  \[
  \begin{CD}
    \widehat{B}_1\times\cdots\times\widehat{B}_g @>{\widehat{m}}>> \widehat{B}_1\times\cdots\times\widehat{B}_g\\
    @A{\phi}AA  @A{\phi}AA\\
    B_\mathfrak{p}\otimes_{A_\mathfrak{p}}\widehat{A}_\mathfrak{p} @>>{\tilde{m}}> B_\mathfrak{p}\otimes_{A_\mathfrak{p}}\widehat{A}_\mathfrak{p}\\
  \end{CD}
  \]
  $\phi(a \otimes t)=(\phi_1(a)t, \ldots, \phi_g(a)t)$,$\phi(aw_i \otimes t)=(\phi(w_i)\phi_1(a)t, \ldots, \phi(w_i)\phi_g(a)t)$であるので,$\widehat{m}(w_i)$は$\widehat{B}_1\times\cdots\times\widehat{B}_g$の元に$(\phi_1(w_i), \ldots, \phi_g(w_i))$をかける写像である.
  構成から(適当な基底を取ることによって)$\Tr(m)=\Tr(\tilde{m})=\Tr(\widehat{m})$であることが分かる.

  $\{v_{i,1},\ldots,v_{i,N_i}\}$を$\widehat{B}_i$の$\widehat{A}_\mathfrak{p}$基底とする($N_i=e_if_i$:定理1.3.23(4)).
  \begin{align*}
    \overline{v}_1=(v_{1,1},0,\ldots,0),&\ldots,\overline{v}_{N_1}=(v_{1,N_1},0,\ldots,0)\\
    \overline{v}_{N_1+1}=(0,v_{2,1},0,\ldots,0),&\ldots,\overline{v}_{N_1+N_2}=(0,v_{2,N_2},0,\ldots,0)\\
    &\vdots\\
    \overline{v}_{n-N_g+1}=(0,\ldots,0,v_{g,1}),&\ldots,\overline{v}_{n}=(0,\ldots,0,v_{g,N_g})
  \end{align*}
  とすれば,$\{\overline{v}_1,\ldots,\overline{v}_n\}$は$\widehat{B}_1\times\cdots\times\widehat{B}_g$の$\widehat{A}_\mathfrak{p}$基底となる.
  $\{(\phi_1(w_i), \ldots, \phi_g(w_i))\}_{1 \leq i \leq g}$,$\{\overline{v}_1, \ldots, \overline{v}_n\}$は共に$\widehat{B}_1\times\cdots\times\widehat{B}_g$の$\widehat{A}_\mathfrak{p}$基底であるので,$A \in \GL_n(\widehat{A}_\mathfrak{p})$が存在し,
  \[(\phi_1(w_i), \ldots, \phi_g(w_i)) = \sum_{j=1}^n A_{ij} \overline{v}_j.\]
  補題1.10.6を使えば,
  \begin{align*}
    \Tr_{L/K}(w_iw_j) &= \Tr(m(w_i w_j)) = \Tr(\widehat{m}(w_i w_j)) = \Tr\left(\widehat{m}\left(\sum_{k=1}^n \sum_{l=1}^n A_{ik} A_{jl} \overline{v}_k \overline{v}_{l} \right) \right) \\
                      & = \sum_{k=1}^n \sum_{l=1}^n A_{ik} A_{jl} \Tr(\widehat{m}(\overline{v}_k \overline{v}_{l})).
  \end{align*}
  $\Tr_{L/K}(w_iw_j)$を$(i,j)$成分とする$n\times n$行列を$W$,$\Tr(\widehat{m}(\overline{v}_i \overline{v}_{j}))$を$(i,j)$成分とする$n\times n$行列を$M$とすれば,これは$W=AM{}^tA$と書ける.

  $M$の$(N_1+\cdots+N_{l-1}+i,N_1+\cdots+N_{l-1}+j)\ (1\leq i,j\leq N_l)$成分は$\widehat{B}_1\times\cdots\times\widehat{B}_g$に対して,
  $\overline{v}_{N_1+\cdots+N_{l-1}+i}\overline{v}_{N_1+\cdots+N_{l-1}+j}=(0,\ldots,0,v_{l,i}v_{l,j},0,\ldots,0)$をかける線形写像のトレースであり,
  これは$\widehat{B}_l$の元に$v_{l,i}v_{l,j}$をかける線形写像のトレース.
  よって補題1.10.6からこれは$\Tr_{\widehat{L}_l/\widehat{K}_\mathfrak{p}}(v_{l,i}v_{l,j})$に等しい.
  $i\neq j$であれば$v_{ik}v_{jl}=0$なので,$\Tr_{\widehat{L}_i/\widehat{K}_\mathfrak{p}}(v_{ik}v_{il})$を$(k,l)$成分とする$N_i\times N_i$行列を$M_i$とすれば,
  \[
  M=
  \begin{pmatrix}
    M_1 &        & \\
    & \ddots & \\
    &        & M_g
  \end{pmatrix}
  ,\quad \det M_i = \varDelta_{\widehat{L}_i/\widehat{K}_\mathfrak{p}}(v_{i,1},\ldots,v_{i,N_i}).\]
  よって,
  \[\varDelta_{L/K}(w_1,\ldots,w_n) = \det W = (\det A)^2 \det M = (\det A)^2 \prod_{i=1}^g \det M_i = (\det A)^2 \prod_{i=1}^g\varDelta_{\widehat{L}_i/\widehat{K}_\mathfrak{p}}(v_{i,1},\ldots,v_{i,N_i})\]
  となり,系I-6.7.9(1)から$\det A \in \widehat{A}_\mathfrak{p}^\times$なので
  \begin{align*}
    \ord_{\mathfrak{p}}(\varDelta_{L/K}(w_1,\ldots,w_n)) &= \ord_{\mathfrak{p}\widehat{A}_\mathfrak{p}}\left(\prod_{i=1}^g\varDelta_{\widehat{L}_i/\widehat{K}_\mathfrak{p}}(v_{i,1},\ldots,v_{i,N_i})\right) \\
    &= \sum_{i = 1}^g\ord_{\mathfrak{p}\widehat{A}_\mathfrak{p}}\left(\varDelta_{\widehat{L}_i/\widehat{K}_\mathfrak{p}}(v_{i,1},\ldots,v_{i,N_i})\right).
  \end{align*}
  $\widehat{A}_\mathfrak{p}$は離散付値環で$\mathfrak{p}\widehat{A}_\mathfrak{p}$が唯一の素イデアルなので
  \[\varDelta_{B/A,\mathfrak{p}} = \mathfrak{p}^{\ord_{\mathfrak{p}}(\varDelta_{L/K}(w_1,\ldots,w_n))}\widehat{A}_\mathfrak{p} = \prod_{i = 1}^g\mathfrak{p}^{\ord_{\mathfrak{p}\widehat{A}_\mathfrak{p}}\left(\varDelta_{\widehat{L}_i/\widehat{K}_\mathfrak{p}}(v_{i,1},\ldots,v_{i,N_i})\right)}\widehat{A}_\mathfrak{p} = \prod_{i=1}^g\varDelta_{\widehat{B}_i/\widehat{A}_\mathfrak{p}}.\]
\end{proof}

\paragraph{命題1.11.4}~
\begin{screen}
  $y=s_1\cdots s_mx\in\delta(B/A)^{-1}$
\end{screen}
\begin{proof}
  $a\in B$として,$\Tr_{L/K}(as_1\cdots s_mx)=s_1\cdots s_m\Tr_{L/K}(ax)$.
  $a=b_1w_1+\cdots+b_mw_m\ (b_i\in A)$とすれば,
  \[=s_1\cdots s_m\Tr_{L/K}\left(\sum_ib_iw_ix\right)=s_1\cdots s_m\sum_ib_i\Tr_{L/K}\left(w_ix\right)=s_1\cdots s_m\sum_ib_ia_i/s_i\]
  なので$\Tr_{L/K}(as_1\cdots s_mx)\in A$.つまり,$s_1\cdots s_mx\in\delta(B/A)^{-1}$.
\end{proof}

\paragraph{命題1.11.6}~
\begin{screen}
  $x\in\delta(B/B_M)^{-1}(\delta(B_M/A)B)^{-1}$
\end{screen}
\begin{proof}
  $x\delta(B_M/A)\subset\delta(B/B_M)^{-1}$は本文から容易に分かる.
  $b\in\delta(B/B_M)^{-1}$であれば$\forall y\in B$に対し,定義から$by\in\delta(B/B_M)^{-1}$となるので,$bB\subset\delta(B/B_M)^{-1}$.
  以上から$x\delta(B_M/A)B\subset\delta(B/B_M)^{-1}$.
  $\delta(B_M/A)B$は$B$のイデアルなので有限生成(命題I-6.8.34)で,分数イデアル(Dedekind環のイデアルは分数イデアル.命題I-8.3.24からも分かる).
  よって,命題I-8.3.21から
  \[xB=x\delta(B_M/A)B(\delta(B_M/A)B)^{-1}\subset\delta(B/B_M)^{-1}(\delta(B_M/A)B)^{-1}.\]
  よって,$x\in\delta(B/B_M)^{-1}(\delta(B_M/A)B)^{-1}$.
\end{proof}

\paragraph{系1.11.11}~
\begin{screen}
  $\overline{g}(x)$の根が$A/\mathfrak{p}$上$M\cap B/M\cap P$を生成し,$g(x)$の根が$K$上$M$を生成する様な$g(x)$が存在し,$g(x)$は既約,$\overline{g}(x)$が既約・分離である
\end{screen}
\begin{proof}
  $k=A/\mathfrak{p}$,$l=B/P$,$m=C/P_M$とする.
  $M/K$は不分岐拡大なので$[m:k]=[M:K]=f$.
  $k$は有限体なので完全体(系I-7.3.6)となり$m/k$は分離拡大(よって,$g$の存在を示せば$\overline{g}(x)$は分離多項式と分かる)で,$m=k(a)$となる$a\in l$が存在する.
  これに対して追加定理\ref{CDVR_unr}(p.\pageref{CDVR_unr})の証明を適用すれば,$f$次の不分岐拡大$K(\alpha)/K\ (\alpha\in B)$が存在する.
  $g$は$\Phi$,$\overline{g}$は$\phi$に対応するので,$g(x)$, $\overline{g}(x)$は既約多項式.
  $M/K$は$f$次の最大不分岐拡大であったので$M=K(\alpha)$となる.
\end{proof}

\begin{screen}
  $R(g,g')\in A^\times$,$\varDelta_{C/A}=A$
\end{screen}
\begin{proof}
  $R(g,g')$を$\bmod\mathfrak{p}$で考えた$R(\overline{g},\overline{g}')$は$\overline{g}(x)$の判別式になる(系1.9.7).
  $\overline{g}(x)$は分離多項式なので$\overline{g}$と$\overline{g}'$は共通根を持たない.
  よって系1.9.6から$R(\overline{g},\overline{g}')\neq 0$.
  % 自然な写像$A\to A/\mathfrak{p}$の$\ker$は$\mathfrak{p}$なので,
  従って$R(g,g')\in A\setminus\mathfrak{p}$で,命題I-6.5.8から$A\setminus\mathfrak{p}=A^\times$.
  上で示したように,$g(x)\in A[x]$は$\alpha\in C$の$K$上最小多項式で,$\varDelta_{M/K}(1,\alpha,\cdots,\alpha^{f-1}) = R(g,g')$(命題1.9.9)なので$\varDelta_{M/K}(1,\alpha,\cdots,\alpha^{f-1})\in A^\times$.
  すなわち$\ord_\mathfrak{p}(\varDelta_{M/K}(1,\alpha,\cdots,\alpha^{f-1}))=0$.
  よって補題1.8.3から$\{1,\alpha,\ldots,\alpha^{f-1}\}$は$C$の$A$基底となる(補題の主張はアレ;p.49の上の方参照)ので,相対判別式の計算に$\{1,\alpha,\ldots,\alpha^{f-1}\}$を使用できる:
  \[\varDelta_{C/A}=\varDelta_{C/A,\mathfrak{p}}=\mathfrak{p}^{\ord_\mathfrak{p}(\varDelta_{M/K}(1,\alpha,\ldots,\alpha^{f-1}))}=\mathfrak{p}^{0}=A.\]
\end{proof}

\paragraph{命題1.11.14}~
\begin{screen}
  $A$を離散付値環,$\{x_1,\ldots,x_l\}\ (l=[N:K])$を$B_N$の$A$基底,$\varDelta_{N/K}(x_1,\ldots,x_l)\in A^\times$とすれば,$\{x_1,\ldots,x_l\}$は$L$の$M$基底となる
\end{screen}
\begin{proof}
  命題I-8.1.24から$B_N$は階数$l$の自由$A$加群(だから基底が$l$個).
  $\varDelta_{N/K}(x_1,\ldots,x_l)\neq 0$なので命題1.7.3(2)から$\{x_1,\ldots,x_l\}\subset N$は$K$上一次独立.
  よって定理I-8.11.9(3)から$M$上一次独立.命題I-8.11.9(2)から$[L:M]=l$なので$\{x_1,\ldots,x_l\}$は$L$の$M$基底となる.
\end{proof}

\paragraph{定理1.11.16}~
\begin{screen}
  $A$が離散付値環なら,$B_M$の$A$基底を$\{v_1,\ldots,v_m\}$,$B_N$の$A$基底を$\{w_1,\ldots,w_n\}$とすれば,$\{v_1,\ldots,v_m\}$は$N$上一次独立で,$\{w_1,\ldots,w_n\}$は$K$上一次独立である
\end{screen}
\begin{proof}
  I-p.241のはじめの話から,$\{v_1,\ldots,v_m\}$と$\{w_1,\ldots,w_n\}$は$K$上一次独立.さらに,定理I-8.11.9から$\{v_1,\ldots,v_m\}$は$N$上一次独立となる.
\end{proof}

\begin{screen}
  上の状況で,$\Tr_{L/K}(v_iw_kv_jw_l)=\Tr_{M/K}(v_iv_j)\Tr_{N/K}(w_kw_l)$
\end{screen}
\begin{proof}
  まず,
  \[\Tr_{L/K}(v_iw_kv_jw_l)=\Tr_{L/K}(v_iv_jw_kw_l).\]
  命題I-8.1.18(5)から
  \[=\Tr_{M/K}\left(\Tr_{L/M}(v_iv_jw_kw_l)\right).\]
  $v_iv_j\in M$なので,
  \[=\Tr_{M/K}\left(v_iv_j\Tr_{L/M}(w_kw_l)\right).\]
  (1.11.15)と同様の考察から$\Tr_{L/M}(w_kw_l)=\Tr_{N/K}(w_kw_l)$なので
  \[=\Tr_{M/K}\left(v_iv_j\Tr_{N/K}(w_kw_l)\right).\]
  補題I-8.1.17から$\Tr_{N/K}(w_kw_l)\in K$なので,
  \[=\Tr_{M/K}(v_iv_j)\Tr_{N/K}(w_kw_l).\]
\end{proof}

\begin{screen}
  上の状況で$\varDelta_L=\varDelta_M{}^n\varDelta_N{}^m$
\end{screen}
\begin{proof}
  $B$の$A$基底として
  \[\{v_1w_1,\ldots,v_1w_n;v_2w_1,\ldots,v_2w_n;\cdots;v_mw_1,\ldots,v_mw_n\}\]
  をとれる(p.81の上の方).
  $G$を$(i,j)$成分を$g_{ij}=\Tr_{L/K}(v_iv_j)$とする$m$次行列,$H$を$(k,l)$成分を$\Tr_{L/K}(w_kw_l)$とする$n$次行列とする.
  $((i-1)n+k,(j-1)n+l)$成分が$\Tr_{L/K}(v_iw_kv_jw_l)$の行列を$X$とする.
  $\Tr_{L/K}(v_iw_kv_jw_l)=\Tr_{L/K}(v_iv_j)\Tr_{L/K}(w_kw_l)$なので,
  \begin{align*}
    X=
    \begin{pmatrix}
      g_{11}H & \cdots & g_{1m}H \\
      g_{21}H & \cdots & g_{2m}H \\
      & \vdots & \\
      g_{m1}H & \cdots & g_{mm}H
    \end{pmatrix}
    =G\otimes H
  \end{align*}
  (行列のKronecker積)となる.線形代数で知られているように,$\det(G\otimes H)=(\det G)^n(\det H)^m$なので,
  \[\varDelta_{L/K}(v_1w_1,\ldots)=\det X=\det(G\otimes H)=(\det G)^n(\det H)^m=\varDelta_{M/K}(v_1,\ldots,v_m){}^n\varDelta_{N/K}(w_1,\ldots,w_n){}^m.\]
  $A=\mathbb{Z}$, $K=\mathbb{Q}$とすれば,主張が従う.
\end{proof}

% ここから未確認

\section{積公式}
\paragraph{補題1.12.1}
$\Tr_{L/K}$じゃなくて$\Tr_{K/\mathbb{Q}}$じゃね?

\paragraph{定理1.12.2}~
\begin{screen}
  $x\in\mathcal{O}_K$に対し積公式$\prod_{v\in\mathfrak{M}}\lvert x\rvert_v=1$が成立すれば$x\in K^\times$に対しても成立する
\end{screen}
\begin{proof}
  まず$v\in\mathfrak{p}$については$\lvert x\rvert_\mathfrak{p}\lvert 1/x\rvert_\mathfrak{p}=1$.
  $v\in\mathfrak{M}_\mathbb{R}$に対しては$\sigma_v\in\Hom^\text{al}_\mathbb{Q}(K,\mathbb{C})$について$\lvert\sigma_v(x)\rvert\lvert\sigma_v(1/x)\rvert=\lvert\sigma_v(1)\rvert=1$.
  $v\in\mathfrak{M}_\mathbb{C}$に対しても同様に$\lvert\sigma_v(x)\rvert^2\lvert\sigma_v(1/x)\rvert^2=\lvert\sigma_v(1)\rvert^2=1$.
  以上から,$x\in\mathcal{O}_K$に対し,$\lvert x\rvert_v=\lvert 1/x\rvert_v$なので,$1/x\in K\ (x\in\mathcal{O}_K)$に対しても積公式が成立.
  $K$は$\mathcal{O}_K$の商体なので,$\forall x\in K$は$a/b\ (a,b\in\mathcal{O}_K)$と表すことができる.
  $a$, $1/b$に対し積公式が成立するので,上の議論と同様にして$x=a/b$に対しても成立.
\end{proof}

\section{Krasnerの補題}

\paragraph{定理1.13.1}~
\begin{screen}
  $A$, $B$を完備離散付値環,$L/K$をGalois拡大,$\tau\in\Gal(L/K(\beta))\ (\beta\in L)$として,$\lvert\tau(x)\rvert=\lvert x\rvert$
\end{screen}
\begin{proof}
  $B$の極大イデアルを$P$とする.$\ord_P(x)=n$とすれば,$xB=P^nB$となり,$\tau(x) \tau(B) = \tau(P)^n$.
  $\tau(B)=B$(追加補題\ref{sigmaB}, p.\pageref{sigmaB}),$\tau(P)=P$なので$\tau(x)B=P^n$.つまり,$\ord_P(\tau(x))=n$.
\end{proof}

\paragraph{系1.13.3}~
\begin{screen}
  局所体($\mathbb{Q}_p$の有限次拡大)が代数体($\mathbb{Q}$の有限次拡大)の完備化に同型である
\end{screen}
\begin{proof}
  本文の証明の前半から,局所体$L=\mathbb{Q}_p(\beta)$,代数体$K=\mathbb{Q}(\beta)$と表すことができる.
  $\mathcal{O}_L$を$L$の整数環,$P$を$\mathcal{O}_L$の素イデアルとする.この時,$\mathfrak{p}=P\cap\mathcal{O}_K$とする.
  $\mathbb{Z} \subset \mathbb{Z}_p$,$K \subset L$なので$\mathcal{O}_K \subset \mathcal{O}_L$である.
  $a,b\in\mathfrak{p}$,$x\in\mathcal{O}_K$とする.
  $a,b\in\mathcal{O}_K,P$なので$a+b\in\mathcal{O}_K,P$,つまり$a+b\in\mathfrak{p}$.
  $a,x\in\mathcal{O}_K$なので$ax\in\mathcal{O}_K$.
  $a\in P$,$x\in\mathcal{O}_L$なので$ax\in P$.よって,$ax\in\mathfrak{p}$.以上から,$\mathfrak{p}$は$\mathcal{O}_K$のイデアル.
  $a,b\in\mathcal{O}_K$で$ab\in \mathfrak{p}$とする.
  $ab\in P$で$P$は$\mathcal{O}_L$の素イデアルなので$a\in P$(若しくは$b\in P$)となり,$a\in\mathfrak{p}$.よって,$\mathfrak{p}$は$\mathcal{O}_K$の素イデアル.
  $\mathcal{O}_L$は完備離散付値環(命題I-9.1.31(4),命題1.5.3(2))なので$\mathfrak{p}\mathcal{O}_L=P^e\mathcal{O}_L$となる$e$が存在する(命題I-8.3.15).
  $\mathcal{O}_L/P$は$\mathcal{O}_K/\mathfrak{p}$の有限次拡大.
  $\mathcal{O}_L$は単項イデアル整域なので$P$は単項イデアル.
  $\mathcal{O}_L$は仮定1.1.2を満たす(命題I-1.5.3)ので,$\mathcal{O}_L/P$は有限体.
  よって$\mathcal{O}_K/\mathfrak{p}$も有限体.よって,定理1.3.23(1)の証明と同様にして$K$(を$L$への包含)上では$P$進距離は$\mathfrak{p}$進距離の冪乗となる.
  $P\supset p\mathbb{Z}_p$,$\mathcal{O}_K\supset\mathbb{Z}$なので$\mathfrak{p}\supset p\mathbb{Z}$.
  よって補題1.3.3より$\mathfrak{p}$は$p\mathbb{Z}$の上にある素イデアル($p\mathbb{Z}=\mathfrak{p}\cap\mathbb{Z}$).
  \[
  \begin{CD}
    \mathfrak{p} @<<< \mathcal{O}_K @>>> K \\
    @AAA         @AAA               @AAA \\
    p\mathbb{Z}  @<<< \mathbb{Z}    @>>> \mathbb{Q}
  \end{CD}
  \]
  完備化して
  \[
  \begin{CD}
    \mathfrak{p}\mathcal{O}_{\widehat{K}}   @<<< \mathcal{O}_{\widehat{K}} @>>> \widehat{K}\\
    @AAA                              @AAA                     @AAA \\
    p\mathbb{Z}_p                     @<<< \mathbb{Z}_p        @>>> \mathbb{Q}_p
  \end{CD}
  \]
  $\beta\in K$なので$\beta\in\widehat{K}$.よって,$\widehat{K}\supset\mathbb{Q}_p(\beta)=L$.
  $f,g$共に$\mathbb{Q}_p$上既約なので$[L:\mathbb{Q}_p]=\deg f=\deg g=[K:\mathbb{Q}]$.
  定理1.3.23(3)(4)から$[\widehat{K}:\mathbb{Q}_p]\leq[K:\mathbb{Q}]$なので$[L:\mathbb{Q}_p]\geq[\widehat{K}:\mathbb{Q}_p]$.
  以上から$\widehat{K}=L$.
\end{proof}

\paragraph{命題1.13.5}~
\begin{screen}
  $L/K$を$\mathbb{Q}_p$の有限次拡大,$\mathfrak{p}$を$\mathcal{O}_K$の素イデアル,$P$を$\mathfrak{p}$の上にある$\mathcal{O}_L$の素イデアルとする.
  $\mathcal{O}_L/P\simeq\mathcal{O}_K/\mathfrak{p}$ならば$u\in\mathcal{O}_L$に対し$u_0\equiv u\bmod P$となる$u_0\in\mathcal{O}_K$が存在する
\end{screen}
\begin{proof}
  $\mathcal{O}_K/\mathfrak{p}$の完全代表系を$\{a_1,a_2,\ldots\}$とする.
  $a_1$と$a_2$は別の同値類に属するので$a_1-a_2\not\in\mathfrak{p}$,つまり$a_1-a_2\in\mathcal{O}_K\setminus\mathfrak{p}$.
  よって$a_1-a_2\not\in P$.
  $a_1, a_2$は$\mathcal{O}_L$の元としても異なる同値類に属する.
  $\mathcal{O}_L/P$と$\mathcal{O}_K/\mathfrak{p}$の位数は等しいので,他の元についても同様にして$\mathcal{O}_L/P$の完全代表系として$\{a_1,a_2,\ldots\}$が取れる.
  よって,$u\in\mathcal{O}_L$が含まれる同値類の代表元を$u_0$とすればよい.
\end{proof}

\begin{screen}
  上の状況で,$L/K$が完全分岐で馴分岐,$F/L$の整数環$\mathcal{O}_F$の極大イデアルを$\mathcal{P}$,$\mathcal{P}$進距離を$\lvert\bullet\rvert$とする.
  この時,$L/K$の分岐指数$e$について$\lvert e\rvert=1$
\end{screen}
\begin{proof}
  $F$は$\mathbb{Q}_p$の有限次拡大でもあるので,$\lvert\mathcal{O}_F/\mathcal{P}\rvert$は$\lvert\mathbb{Z}_p/p\mathbb{Z}_p\rvert$の倍数.
  命題I-9.1.31(1)から$\lvert\mathbb{Z}_p/p\mathbb{Z}_p\rvert=\lvert\mathbb{Z}/p\mathbb{Z}\rvert=p$なので,$\lvert\mathcal{O}_F/\mathcal{P}\rvert$は$p$の倍数.
  従って,$\mathcal{O}_F/\mathcal{P}$の標数は$p$(p.I-216とか参照).
  $L/K$が馴分岐なので$p \nmid e$となり,$\mathcal{O}_F$の元として$e$と$0$は異なる同値類に属する.よって$e\not\in\mathcal{P}$であり,$\lvert e\rvert=1$.
\end{proof}

\section{$2$次の暴分岐}
\paragraph{命題1.14.1}~
\begin{screen}
  \begin{lem}\label{basis_Q2_prime_element}
    $K$を$\mathbb{Q}_2$の有限次拡大,$\mathfrak{p}$を$\mathcal{O}_K$の素イデアル,$F=K(\sqrt{\pi})$とする.
    $\pi$が$\mathcal{O}_K$の素元であれば$\mathcal{O}_F$の$\mathcal{O}_K$上基底が$\{1,\sqrt{\pi}\}$である(命題1.7.3使った方が楽)
  \end{lem}
\end{screen}
\begin{proof}
  $2=\pi^md\ (\ord_\mathfrak{p}(d)=0)$とする.
  $\ord_\mathfrak{p}(\pi)=1$である.
  $\alpha=a+b\sqrt{\pi}\in F\ (a,b\in K)$が$\mathcal{O}_K$上整であるとする.
  \[A=\Tr_{F/K}(\alpha)=(a+b\sqrt{\pi})+(a-b\sqrt{\pi})=2a,\quad B=\N_{F/K}(\alpha)=(a+b\sqrt{\pi})(a-b\sqrt{\pi})=a^2-\pi b^2.\]
  命題I-8.1.19から$A, B\in\mathcal{O}_K$.$4B=A^2-4\pi b^2\in 4\mathcal{O}_K$なので$4\pi b^2\in\mathcal{O}_K$,つまり$\ord_\mathfrak{p}(4\pi b^2)\geq 0$.
  よって$2m+1+2\ord_\mathfrak{p}(b)\geq 0$(命題1.1.3(1))なので$\ord_\mathfrak{p}(b)\geq -m$.
  よって$b=c\pi^{-m}\ (c\in\mathcal{O}_K)$と表すことができる.
  $\ord_\mathfrak{p}(b)< 0$とする.つまり$0\leq\ord_\mathfrak{p}(c)\leq m-1$.
  $A^2-4\pi b^2=A^2-\pi c^2d^2\in 4\mathcal{O}_K$なので$A^2-\pi c^2d^2\equiv 0\bmod\mathfrak{p}^{2m}$.
  $A=\pi^kt\ (\ord_\mathfrak{p}(t)=0)$,$c=\pi^{s-1}u\ (\ord_\mathfrak{p}(u)=0, s\leq m)$とおけば,この式は$\pi^{2k}t^2-\pi^{2s-1}u^2d^2=\pi^{2m}w\ (\ord_\mathfrak{p}(w)\geq0)$となる.

  $2k>2s-1$であれば$\pi^{2s-1}(\pi^{2k-2s+1}t^2-u^2d^2)=\pi^{2m}w$.命題1.1.3(3)から$\ord_\mathfrak{p}(\pi^{2k-2s+1}t^2-u^2d^2)=0$なので$2s-1\geq 2m$.
  これは$s\leq m$に矛盾.

  $2k< 2s-1$であれば$\pi^{2k}(t^2-\pi^{2s-2k-1}u^2d^2)=\pi^{2m}w$.
  命題1.1.3(3)から$\ord_\mathfrak{p}(t^2-\pi^{2s-2k-1}u^2d^2)=0$なので$k\geq m$となり,$2s-1>2k\geq2m$となり$s\leq m$に矛盾.
  以上から$\ord_\mathfrak{p}(b)\geq 0$,つまり$b\in\mathcal{O}_K$.
  $a^2-\pi b^2\in\mathcal{O}_K$なので$a^2\in\mathcal{O}_K$,よって$a\in\mathcal{O}_K$.
  以上から$\mathcal{O}_F$は$\mathcal{O}_K+\mathcal{O}_K\sqrt{\pi}$.

  逆に$a+b\sqrt{\pi}\ (a,b\in\mathcal{O}_K)$は$x^2-2ax+a^2-\pi b^2$の根なので$\mathcal{O}_K$上整.
\end{proof}

\begin{screen}
  $K$を$\mathbb{Q}_2$の有限次拡大,$F=K(\sqrt{\pi'})$($\pi'$は$\mathcal{O}_K$の素元),$\mathfrak{p}$を$\mathcal{O}_K$の素イデアルとするとき,$\varDelta_{F/K}=4\mathfrak{p}$
\end{screen}
\begin{proof}
  $\sqrt{\pi'}$の$K$上最小多項式は$x^2-\pi'$で,その判󠄁󠄁別式は$4\pi'$.命題1.9.9から$\varDelta_{F/K}(1,\sqrt{\pi'})=4\pi'$.
  $\{1,\sqrt{\pi'}\}$が$\mathcal{O}_F$の$\mathcal{O}_K$基底(上で示した).
  $K$は局所体なので命題1.5.3から$\mathcal{O}_K$は完備離散付値環.
  $\varDelta_{F/K}=\mathfrak{p}^{\ord_\mathfrak{p}(4\pi')}=4\pi'\mathcal{O}_K=4\mathfrak{p}$.
\end{proof}

\begin{screen}
  $K$を$\mathbb{Q}_2$の有限次拡大,$F=K(\sqrt{\mu})\ (\mu\in\mathcal{O}_K^\times)$,$\mathcal{O}_K/\mathfrak{p}=\mathbb{F}$とする.
  $n=\lvert\mathbb{F}\rvert$が偶数であれば$\mathbb{F}^\times\ni x\mapsto x^2\in\mathbb{F}^\times$が全単射である
\end{screen}
\begin{proof}
  定理I-7.4.10から$\mathbb{F}^\times$は位数$n-1$の巡回群:$\mathbb{F}^\times=\set{g^i | 0\leq i\leq n-2}$.
  $g^{n-1}=1$である.$g^i\neq g^j$とする.
  $1\leq\lvert i-j\rvert\leq n-2$なので$2\leq\lvert 2i-2j\rvert\leq 2n-4$.
  $n-1$は奇数なので$2i-2j\neq n-1$.よって$g^{2i}\neq g^{2j}$となり,単射.
  $g^1,g^2,\ldots,g^{n/2-1}$は$g^2,g^4,\ldots,g^{n-2}$に写る.$g^{n/2},g^{n/2+1},\ldots,g^{n-2}$は$g^{n-1+1},g^{n-1+3},\ldots,g^{n-1+n-3}$,つまり$g^1,g^3,\ldots,g^{n-3}$に写るので全射.
\end{proof}

\begin{screen}
  $\bmod\mathfrak{p}^l \to \bmod\mathfrak{p}^{l+1}, \bmod\mathfrak{p}^{l+2}, \ldots$の議論が成立していることの検証(p.91の真ん中らへん)
\end{screen}
\begin{proof}
  本文から,$\ord_\mathfrak{p}(\mu-1)=l\ (\mu\in\mathcal{O}_K^\times\setminus(\mathcal{O}_K^\times)^2)$に対し$l$が$2m$で以下の偶数であれば$\mu'=(1+\pi^{l_1}c)^{-2}\mu$があり,
  $(1+\pi^{l_1}c)^{-2}\equiv 1+\pi^lu\bmod\mathfrak{p}^{l+1}$,$\mu'\equiv 1\bmod\mathfrak{p}^{l+1}$.
  $(1+\pi^{l_1}c)^{-2}\in(\mathcal{O}_K^\times)^2$となる(背理法で容易に示せる)ので$\mu'\in\mathcal{O}_K^\times\setminus(\mathcal{O}_K^\times)^2$(対偶を考えれば明らか).
  $\ord_\mathfrak{p}(\mu'-1)\geq l+1$となり,これによって$l=2m$となるか$l$が奇数になるまで$l$を大きくすることができる.
\end{proof}

\begin{screen}
  $K$を$\mathbb{Q}_2$の有限次拡大,$\mathfrak{p}$を$\mathcal{O}_K$の素イデアル,$F=K(\sqrt{\mu})$とする.
  $\mu\in\mathcal{O}_K^\times\setminus(\mathcal{O}_K^\times)^2$,$\mu=1+4u\ (u\in\mathcal{O}_K^\times)$であれば$F/K$が不分岐であることの証明.
  (上の検証で$l$が$2m$になった場合.$l$が奇数になった場合も同様)
\end{screen}
\begin{proof}
  $\alpha=(1+\sqrt{\mu})/2\in F$とすれば,$\alpha$は$g(x)=x^2-x+(1-\mu)/4$の根.
  $(1-\mu)/4\in\mathcal{O}_K$なので$g(x)\in\mathcal{O}_K[x]$であることから,$\alpha$は$\mathcal{O}_K$上モニックの根となる.
  つまり,$\alpha\in\mathcal{O}_F$.また,命題1.9.9から$\varDelta_{F/K}(1,\alpha)=\mu\neq 0$.
  よって,命題1.7.3(2)から,$\{1,\alpha\}$は$F$の$K$基底となり$\mathcal{O}_K$上一次独立.
  $\{1,\alpha\}\in\mathcal{O}_F$なので,これは$\mathcal{O}_F$の$\mathcal{O}_K$基底になる.
  よって,$\mathcal{O}_F=\mathcal{O}_K[\alpha]$で,$\varDelta_{F/K}$の計算に$\varDelta_{F/K}(1,\alpha)=\mu$を使える.
  $K$は局所体なので命題1.5.3から$\mathcal{O}_K$は完備離散付値環.
  $\varDelta_{F/K}=\mathfrak{p}^{\ord_\mathfrak{p}(\mu)}=\mu\mathcal{O}_K$.
  $\mu\in\mathcal{O}_K^\times = \mathcal{O}_K \setminus \mathfrak{p}$(命題I-6.5.8)なので,$\varDelta_{F/K}$は$\mathfrak{p}$で割り切れない.よってDedekindの判別定理(定理1.11.12)から$F/K$は不分岐.
\end{proof}

\begin{screen}
  $F$を局所体,$\pi$を$\mathcal{O}_F$の素元,$F=K(\sqrt{\pi})$とすれば,$\sqrt{\pi}$が$\mathcal{O}_F$の素元である
\end{screen}
\begin{proof}
  追加補題\ref{basis_Q2_prime_element}(p.\pageref{basis_Q2_prime_element})から$\mathcal{O}_F=\mathcal{O}_K[\sqrt{\pi}]$,
  つまり$\mathcal{O}_F=\set{a+b\sqrt{\pi} | a,b\in\mathcal{O}_K}$なので,
  $\sqrt{\pi}\mathcal{O}_F=\{\pi b+a\sqrt{\pi}\mid a,b\in\mathcal{O}_K\}$.
  $c,d\in\mathcal{O}_F\setminus\sqrt{\pi}\mathcal{O}_F$とする.
  $c=c_1+c_2\sqrt{\pi}$,$d=d_1+d_2\sqrt{\pi}$とすれば,$c_1,d_1\not\in\pi\mathcal{O}_K$,つまり$\ord_\mathfrak{p}(c_1)=\ord_\mathfrak{p}(d_1)=0$.
  $cd=(c_1d_1+c_2d_2\pi)+(c_1d_2+c_2d_1)\sqrt{\pi}$で,命題1.1.3(3)から$\ord_\mathfrak{p}(c_1d_1+c_2d_2\pi)=0$なので$cd\in\mathcal{O}_F\setminus\sqrt{\pi}\mathcal{O}_F$.
  つまり,$\sqrt{\pi}\mathcal{O}_F$は素イデアル.
\end{proof}

\begin{screen}
  $F$を局所体,$\pi$を$\mathcal{O}_F$の素元,$\mu=1+\pi^{2k+1}u\ (u\in\mathcal{O}_K^\times)$,$F=K(\sqrt{\mu})$とすれば$q=(1+\sqrt{\mu})/\pi^k$が$\mathcal{O}_F$の素元である
\end{screen}
\begin{proof}
  $F=K(\sqrt{\mu})=K(q)$.
  $q$はEisenstein多項式$g(x)=x^2-\pi^{m-k}x-\pi u$の根なので,$\mathcal{O}_K$上整となり$q\in\mathcal{O}_F$.
  $g(x)$の判別式は$\pi^{2(m-k)}+4\pi u\neq 0$なので命題1.9.9,命題1.7.3から$\{1,q\}$は$F$の$K$基底で$\mathcal{O}_F$の$\mathcal{O}_K$基底:$\mathcal{O}_F=\mathcal{O}_K[q]$,つまり$\mathcal{O}_F=\set{a+bq | a,b\in\mathcal{O}_K}$.
  $g(q)=0$なので$aq+bq^2=bu\pi+(a+b\pi^{m-k})q$.よって$q\mathcal{O}_F=\set{bu\pi+(a+b\pi^{m-k})q | a,b\in\mathcal{O}_K}$.
  $c,d\in\mathcal{O}_F\setminus q\mathcal{O}_F$とする.
  $c=c_1+c_2q$,$d=d_1+d_2q$とすれば$c_1,c_2\not\in\pi\mathcal{O}_K$,つまり$\ord_\mathfrak{p}(c_1)=\ord_\mathfrak{p}(d_1)=0$.
  \[cd=c_1d_1+c_2d_2q^2+(c_1d_2+c_2d_1)q=(c_1d_1+c_2d_2\pi u)+(c_1d_2+c_2d_1+c_2d_2\pi^{m-k})q\]
  で$\ord_\mathfrak{p}(c_1d_1+c_2d_2\pi u)=0$なので,$cd\in\mathcal{O}_F\setminus q\mathcal{O}_F$.
  よって$q\mathcal{O}_F$は$\mathcal{O}_F$の素イデアル.
\end{proof}
