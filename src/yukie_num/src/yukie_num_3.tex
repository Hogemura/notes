\chapter{整数環と判別式の例}
\setcounter{section}{1}
\section{Kummer理論}
\paragraph{命題2.2.5}~
\begin{screen}
  $G, H$の間のperfect pairingを$\Phi$,$\sigma_g\colon H\ni h\mapsto\Phi(g, h)\in\mathbb{C}^1$とすれば,写像$\chi\colon G\ni g\mapsto\sigma_g\in H^\ast$は準同型
\end{screen}
\begin{proof}
  $G$は加群,$\mathbb{C}^1$は乗法群.$\forall h\in H$に対し,
  \[\left[\chi(g_1)\chi(g_2)\right](h)=\sigma_{g_1}(h)\sigma_{g_2}(h)=\Phi(g_1, h)\Phi(g_2, h)=\Phi(g_1+g_2, h)=\sigma_{g_1+g_2}(h)=\chi(g_1+g_2)(h).\]
  よって,$\chi(g_1)\chi(g_2)=\chi(g_1+g_2)$となり$\chi$は準同型.
\end{proof}

\paragraph{定理2.2.8}~
\begin{screen}
  $R/(K^\times)^n$の完全代表系を$\{a_1, \ldots, a_t\}$とすれば,$K(\sqrt[n]{R})=K(\sqrt[n]{a_1}, \ldots, \sqrt[n]{a_t})$
\end{screen}
\begin{proof}
  $\sqrt[n]{a_1}, \ldots, \sqrt[n]{a_t}\in\sqrt[n]{R}$なので$K(\sqrt[n]{a_1}, \ldots, \sqrt[n]{a_t})\subset K(\sqrt[n]{R})$.
  $\forall a\in R$は$b\in K^\times$によって$a=a_ib^n$となるので$\sqrt[n]{a}=\sqrt[n]{a_i}b$,つまり$\sqrt[n]{R}=\set{\sqrt[n]{a_i}b | b\in K^\times}$.
  よって,$\sqrt[n]{R}$の有限部分集合が生成する体は$K(\sqrt[n]{a_i}, \sqrt[n]{a_j}, \ldots)$という形になるので,$K(\sqrt[n]{R})\subset K(\sqrt[n]{a_1}, \ldots, \sqrt[n]{a_t})$.
  以上から$K(\sqrt[n]{R})=K(\sqrt[n]{a_1}, \ldots, \sqrt[n]{a_t})$.
\end{proof}

\paragraph{例2.2.9}
$a, b, c, d\in\mathbb{Z}$に対して$R=2^a3^b5^c7^d(\mathbb{Q}^\times)^2$とすれば,$\mathbb{Q}^\times\supset R\supset(\mathbb{Q}^\times)^2$が成立する.
$K=\mathbb{Q}(\sqrt{R})=\mathbb{Q}(\sqrt{2}, \sqrt{3}, \sqrt{5}, \sqrt{7})$とする.
$R/(\mathbb{Q}^\times)^2$の完全代表系として$2^i3^j5^k7^l\ (i, j, k, l=0, 1)$が取れる.
準同型
\begin{align*}
  \phi\colon\mathbb{Z}^4\ni(a, b, c, d) &\mapsto 2^a3^b5^c7^d\in\mathbb{Q}^\times\\
  &\mapsto 2^i3^j5^k7^l\in R/(\mathbb{Q}^\times)^2
\end{align*}
を考える.明らかに$\phi$は全射なので$\Im\phi=R/(\mathbb{Q}^\times)^2$.
$R/(\mathbb{Q}^\times)^2$の単位元は$2^03^05^07^0=1$.
$\phi(a, b, c, d)=1$となるのは$a, b, c, d$が偶数の時なので,$\ker\phi=(2\mathbb{Z})^4$.
よって準同型定理から$R/(\mathbb{Q}^\times)^2\simeq\mathbb{Z}^4/(2\mathbb{Z})^4\simeq(\mathbb{Z}/2\mathbb{Z})^4$.
命題2.2.3,定理2.2.8から$\Gal(K/\mathbb{Q})\simeq(R/(\mathbb{Q}^\times)^2)^\ast\simeq R/(\mathbb{Q}^\times)^2\simeq(\mathbb{Z}/2\mathbb{Z})^4$.

\paragraph{例2.2.10}
$\mu\in\mathbb{Z}$,$l$を素数,$\zeta$を$1$の原始$l$乗根,$F=\mathbb{Q}(\zeta)$とする.
定理I-8.11.7から$\Gal(F/\mathbb{Q})\simeq(\mathbb{Z}/l\mathbb{Z})^\times$.
命題I-7.4.3 (2)から$\lvert\Gal(F/\mathbb{Q})\rvert=l-1=[F:\mathbb{Q}]$.
よって,$\Hom_\mathbb{Q}^\text{al}(F, \overline{\mathbb{Q}})=\{\sigma_1, \ldots, \sigma_{l-1}\}$.

$\mu=a^l\ (a\in F^\times)$と仮定する.命題I-8.1.18 (2)から
\[\N_{F/\mathbb{Q}}(\mu)=\N_{F/\mathbb{Q}}(a^l)=[\N_{F/\mathbb{Q}}(a)]^l=\prod_{i=1}^{l-1}\sigma_i(\mu)=\prod_{i=1}^{l-1}\mu=\mu^{l-1}=a^{l(l-1)}.\]
補題I-8.1.17から$\N_{F/\mathbb{Q}}(a)\in\mathbb{Q}$なので互いに素な$b, c\in\mathbb{Z}$によって$\N_{F/\mathbb{Q}}(a)=c/b$.
上の式に代入して$c^l=(ba^{l-1})^l$,よって$c=ba^{l-1}$となり矛盾.以上から$\mu\not\in(F^\times)^l$.

$\mu$が属する$F^\times/(F^\times)^l$の代表元を$g$とする.
$\mu^l\in(F^\times)^l$であり,$l$は素数なので$g$の位数は$l$.
$1\leq i< j\leq l$に対し$g^i=g^j$とする.
$g^{j-i}=1$となるが$1\leq j-i\leq l-1$なので矛盾.よって$\{1, g, \ldots, g^{l-1}\}$は位数$l$の$F^\times/(F^\times)^l$部分群.
$F^\times\supset R=\{\, \mu^i(F^\times)^l\mid1\leq i\leq l\, \}\supset(F^\times)^l$とすれば$F(\sqrt[n]{R})=F(\sqrt[n]{\mu})$.命題2.2.3と定理2.2.8から
\[\Gal(F(\sqrt[n]{\mu})/F)\simeq(R/(F^\times)^l)^\ast\simeq(R/(F^\times)^l)\simeq\{1, \mu, \ldots, \mu^{l-1}\}\]
なので$[F(\sqrt[n]{\mu}):F]=l$.

\paragraph{命題2.2.11}~
\begin{screen}
  $A$を仮定1.1.2を満たすDedekind環,$\mathfrak{p}$を$A$の素イデアル,$f(x)=x^e-\mu$,$e$は$A/\mathfrak{p}$の標数$p$で割り切れない,$\mu\in A\setminus\mathfrak{p}$とすれば,$f(x)$の判別式$\varDelta(f)$は$\mathfrak{p}$に含まれない
\end{screen}
\begin{proof}
  系1.9.7から$\varDelta(f)=\pm\mu^{e-1}e^e$.仮定1.1.2から$A$の商体$K$の標数は$0$なので$\mathbb{Q}$を含み,$a/1\ (a\in\mathbb{Z})$という形の元を含む.
  よって$\mathbb{Z}\subset A$.
  $\mathfrak{p}$は$p\mathbb{Z}$の上にあり,$e \not\in p\mathbb{Z} = \mathfrak{p} \cap \mathbb{Z}$なので$e\not\in\mathfrak{p}$である.
  よって,$\mu^{e-1}e^e\not\in\mathfrak{p}$となる.
\end{proof}

\begin{screen}
  $L=K(\sqrt[e]{\mu})$,$g(x)\in A[x]$を$\sqrt[e]{\mu}$の$K$上最小多項式,$\varDelta(g)\in A\setminus\mathfrak{p}$とすれば$\mathfrak{p}$は$L$で不分岐
\end{screen}
\begin{proof}
  $g(x)\in A[x]$なので$\sqrt[e]{\mu}\in L$は$A$上整,すなわち$B$の元となり$\varDelta_{L/K}(1, \sqrt[e]{\mu}, \ldots, \sqrt[e]{\mu}^{n-1})=\varDelta(g)\in A\setminus\mathfrak{p}\subset A_\mathfrak{p}^\times$(命題1.9.9).
  補題1.7.3 (2)から$\{ 1, \sqrt[e]{\mu}, \ldots, \sqrt[e]{\mu}^{n-1} \}$は$B$に含まれる$L$の$K$基底となり,補題1.8.3から$\varDelta_{L/K, \mathfrak{p}}=A$.
  $\varDelta_{L/K}$は$\mathfrak{p}$で割り切れないのでDedekindの判別定理から$\mathfrak{p}$は$L$で不分岐.
\end{proof}

\begin{screen}
  $B\otimes_AA_\mathfrak{p}$の$A_\mathfrak{p}$基底が取れる
\end{screen}
\begin{proof}
  命題I-6.5.9,補題I-8.3.3,補題I-8.3.13から$A_\mathfrak{p}$は単項イデアル整域.
  命題I-8.1.14から$L$における$A_\mathfrak{p}$の整閉包は$B_\mathfrak{p}$.
  命題I-8.1.24から$B_\mathfrak{p}$は階数$[L:K]=n$の自由$A_\mathfrak{p}$加群なので$n$個の$A_\mathfrak{p}$基底を取れる.
  補題1.3.22から$B_\mathfrak{p}\simeq A_\mathfrak{p}\otimes_AB\simeq B\otimes_AA_\mathfrak{p}$
  よって,$B_\mathfrak{p}$の$A_\mathfrak{p}$を上の同型によって写せば,$B\otimes_AA_\mathfrak{p}$の$A_\mathfrak{p}$基底となる.
\end{proof}

\section{$3$次体}
よく使う命題を列挙しておく.
\begin{itemize}
  \item 系I-8.1.25.$\mathcal{O}_K$は階数$[K:\mathbb{Q}]$の自由$\mathbb{Z}$加群
  \item 命題1.2.14.$\forall\mathfrak{p} \in \Spec A$に対し$a \in \widehat{A}_\mathfrak{p}$なら$a \in A$
  \item 系1.7.5.$\varDelta_K(v_1, \ldots, v_n) = (\mathcal{O}_K:V)^2 \varDelta_K$
  \item 命題1.8.9.$\displaystyle(B:M) = \prod_{\mathfrak{p} \in \Spec A} (B \otimes_A \widehat{A}_\mathfrak{p}:M \otimes_A \widehat{A}_\mathfrak{p})$
  \item 命題1.9.9.$\varDelta(f) = \varDelta_{L/K}(1, \alpha, \ldots)$
  \item 命題1.10.7.CDVRでのEisensteinと完全分岐
  \item 命題1.11.1.$\varDelta_{B/A, \mathfrak{p}} = \displaystyle \prod_{i=1}^g \varDelta_{\widehat{B}_i/\widehat{A}_\mathfrak{p}}$
\end{itemize}

\paragraph{例2.3.3}~
\begin{screen}
  $t$の$\mathbb{Q}$上最小多項式$f(x)$の判別式が$49$ならば$(\mathcal{O}_K:\mathbb{Z}[t])=1, 7$
\end{screen}
\begin{proof}
  命題1.9.9から$\varDelta_{K/\mathbb{Q}}(1, t, t^2)=\varDelta(f)=49$.
  $\{1, t, t^2\}$は$K=\mathbb{Q}(t)$の$\mathbb{Q}$基底で,$\mathcal{O}_K$に含まれる.
  よって系1.7.5 (2)から$\varDelta_{K/\mathbb{Q}}(1, t, t^2)=(\mathcal{O}_K:\mathbb{Z}[t])^2\varDelta_K$となり従う.
\end{proof}

\begin{screen}
  $(B_\mathfrak{p}:M\otimes_AA_\mathfrak{p})=1$,$\mathcal{O}_K=\mathbb{Z}[t]$,$\varDelta_K=49$
\end{screen}
\begin{proof}
  $f(x+2) = g(x)$とする.$g(x)$はEisenstein多項式で,$\mathbb{Q}_7$上既約である.$\mathbb{Q}_7$は平坦$\mathbb{Q}$加群なので
  \[ K \otimes_\mathbb{Q} \mathbb{Q}_7 \simeq \left( \frac{\mathbb{Q}[x]}{(g(x))} \right) \otimes_\mathbb{Q} \mathbb{Q}_7 \simeq \frac{\mathbb{Q}[x] \otimes_\mathbb{Q} \mathbb{Q}_7}{(g(x)) \otimes_\mathbb{Q} \mathbb{Q}_7} \simeq \frac{\mathbb{Q}_7[x]}{(g(x))} \simeq \mathbb{Q}_7(t). \]
  従って,定理1.3.23から,$7\mathbb{Z}$の上にある$\mathcal{O}_K$の素イデアルは$1$個で,$K$の完備化は$\mathbb{Q}_7(t)$.
  $\mathbb{Q}_7[t]$の整数環は$\mathbb{Z}_7[t]$なので,再び定理1.3.23から$\mathcal{O}_K\otimes_\mathbb{Z}\mathbb{Z}_7\simeq\mathbb{Z}_7[t]$.
  さらに,$\mathbb{Z}[t]\otimes_\mathbb{Z}\mathbb{Z}_7 \simeq \mathbb{Z}_7[t]$で,これらの同型は同じ写像である($a \otimes b \mapsto ab$).
  従って,命題1.8.9 (3)から$(B_\mathfrak{p}:M\otimes_AA_\mathfrak{p}) = (B\otimes_A\widehat{A}_\mathfrak{p}:M\otimes_A\widehat{A}_\mathfrak{p}) = (\mathcal{O}_K\otimes_\mathbb{Z}\mathbb{Z}_7:\mathbb{Z}[t]\otimes_\mathbb{Z}\mathbb{Z}_7) = 1$.

  $\mathfrak{p}=p\mathbb{Z}\neq7\mathbb{Z}$に対しては$(B_\mathfrak{p}:M\otimes_AA_\mathfrak{p})$は命題1.8.9 (2)から$p$の冪なので$7$の倍数では無い.よって命題1.8.9 (1)から$(B:M)=(\mathcal{O}_K:\mathbb{Z}[t])=1$.上の話と合わせて$\varDelta_K=\varDelta_{K/\mathbb{Q}}(1, t, t^2)=49$.
\end{proof}

\begin{screen}
  $t=2\cos2\pi/7$として$\mathbb{Q}(t)/\mathbb{Q}$はGalois拡大である
\end{screen}
\begin{proof}
  例I-8.11.8から$t$の$\mathbb{Q}$上最小多項式は$x^3+x^2-2x+1$で,$t$の共軛は$2\cos4\pi/7$と$2\cos6\pi/7$.
  $2\cos4\pi/7=4t^2-2$,$2\cos6\pi/7=-4t^2-t+1$なので$t$の共軛は全て$\mathbb{Q}(t)$に含まれる.
  よって系I-7.3.10から$\mathbb{Q}(t)/\mathbb{Q}$は正規拡大.
  $\mathbb{Q}$は完全体(系I-7.3.6)なので$\mathbb{Q}(t)/\mathbb{Q}$は分離拡大.
\end{proof}

\paragraph{命題2.3.4}~
\begin{screen}
  同型$\phi\colon\mathcal{O}_K\otimes_\mathbb{Z}\mathbb{Z}_p\to\mathbb{Z}_p{}^n$が存在すれば,
  $\mathcal{O}_K$の$\mathbb{Z}$基底を$\{1, \alpha, \ldots, \alpha^{n-1}\}$,
  $\beta=\phi(\alpha)$として$\{1, \beta, \ldots, \beta^{n-1}\}$が$\mathbb{Z}_p{}^n$の$\mathbb{Z}_p$基底となる
\end{screen}
\begin{proof}
  $\mathcal{O}_K \otimes_\mathbb{Z} \mathbb{Z}_p$の$\mathbb{Z} \otimes_\mathbb{Z} \mathbb{Z}_p$基底として,$\{1 \otimes 1, \alpha \otimes 1, \ldots, \alpha^{n-1} \otimes 1\}$が取れることを示す.

  先ず,$\mathcal{O}_K \otimes_\mathbb{Z} \mathbb{Z}_p$の任意の元は
  \[\xi = (a_0 \otimes b_0)(1 \otimes 1) + (a_1\otimes b_1)(\alpha \otimes 1) + \cdots + (a_{n-1} \otimes b_{n-1})(\alpha^{n-1} \otimes 1)\]
  と表すことができるので,$\{1 \otimes 1, \alpha \otimes 1, \ldots, \alpha^{n-1} \otimes 1\}$は$\mathbb{Z} \otimes_\mathbb{Z} \mathbb{Z}_p$上$\mathcal{O}_K \otimes_\mathbb{Z} \mathbb{Z}_p$を生成する.

  $\{1, \alpha, \ldots, \alpha^{n-1}\}$は$\mathcal{O}_K$の$\mathbb{Z}$基底なので
  \[ \mathbb{Z}^n \ni (a_0, \ldots, a_{n-1}) \mapsto a_0+a_1\alpha+\cdots+a_{n-1}\alpha^{n-1} \in \mathcal{O}_K \]
  は全単射.$\mathbb{Z}_p$は平坦$\mathbb{Z}$加群なので,$(\mathbb{Z} \otimes_\mathbb{Z} \mathbb{Z}_p)^n \simeq \mathbb{Z}^n \otimes_\mathbb{Z} \mathbb{Z}_p \to \mathcal{O}_K \otimes_\mathbb{Z} \mathbb{Z}_p$も全単射である.
  従って,同型
  \begin{align*}
    (\mathbb{Z} \otimes_\mathbb{Z} \mathbb{Z}_p)^n \ni \begin{pmatrix} a_0 \otimes b_0 \\ a_1 \otimes b_1 \\ \vdots \\ a_{n-1} \otimes b_{n-1} \end{pmatrix} & \mapsto a_0\begin{pmatrix} 1 \\ 0 \\ \vdots \\ 0 \end{pmatrix} \otimes b_0 + a_1\begin{pmatrix} 0 \\ 1 \\ \vdots \\ 0 \end{pmatrix} \otimes b_1 + \cdots + a_{n-1}\begin{pmatrix} 0 \\ \vdots \\ 0 \\ 1 \end{pmatrix} \otimes b_{n-1} \in \mathbb{Z}^n \otimes_\mathbb{Z} \mathbb{Z}_p \\[10pt]
    & \mapsto a_0 \otimes b_0 + \cdots + a_{n-1}\alpha^{n-1} \otimes b_{n-1} \in \mathcal{O}_K \otimes_\mathbb{Z} \mathbb{Z}_p
  \end{align*}
  が得られる.よって,$\xi = 0$なら$a_i \otimes b_i = 0$となるので,$\{1 \otimes 1, \alpha \otimes 1, \ldots, \alpha^{n-1} \otimes 1\}$は一次独立.
  \begin{align*}
    \phi(\xi) &= \phi((a_0 \otimes b_0)(1 \otimes 1)+(a_1 \otimes b_1)(\alpha \otimes 1)+\cdots+(a_{n-1} \otimes b_{n-1})(\alpha^{n-1} \otimes 1))\\
    &= \phi((a_0 \otimes b_0)(1 \otimes 1))+\phi((a_1 \otimes b_1)(\alpha \otimes 1))+\cdots+\phi((a_{n-1} \otimes b_{n-1})(\alpha^{n-1} \otimes 1))\\
    &= \phi(a_0 \otimes b_0)\phi(1 \otimes 1)+\phi(a_1 \otimes b_1)\phi(\alpha \otimes 1)+\cdots+\phi(a_{n-1} \otimes b_{n-1})\phi(\alpha^{n-1} \otimes 1)\\
    &= \phi(a_0 \otimes b_0)\phi(1 \otimes 1)+\phi(a_1 \otimes b_1)\phi(\alpha \otimes 1)+\cdots+\phi(a_{n-1} \otimes b_{n-1})\phi(\alpha \otimes 1)^{n-1}.
  \end{align*}
  なので,
  \[\phi(\xi) = 0 \Leftrightarrow \xi = 0 \Leftrightarrow a_i \otimes b_i = 0 \Leftrightarrow \phi(a_i \otimes b_i)=0\]
  となるので,$\beta=\phi(\alpha \otimes 1)$として$\{1, \beta, \ldots, \beta^{n-1}\}$が$\mathbb{Z}_p{}^n$の$\mathbb{Z}_p$基底となる.
\end{proof}

\begin{screen}
  $\beta = (\beta_1, \ldots, \beta_n)\in\mathbb{Z}_p{}^n$として,$\mathbb{Z}_p{}^n$の$\mathbb{Z}_p$基底が$\{1, \beta, \ldots, \beta^{n-1}\}$であれば$\det(1, \beta, \ldots, \beta^{n-1})\in\mathbb{Z}_p^\times$
\end{screen}
\begin{proof}
  $\{1, \beta, \ldots, \beta^{n-1}\}$は$\mathbb{Z}_p{}^n$の$\mathbb{Z}_p$基底なので
  \[(1, 0, \ldots, 0)=(a_0^{(1)}, \ldots, a_{n-1}^{(1)})
  \begin{pmatrix}
    1               & \cdots & 1\\
    \beta_1         & \cdots & \beta_n\\
    & \vdots & \\
    \beta_1{}^{n-1} & \cdots & \beta_n{}^{n-1}
  \end{pmatrix}
  \]
  となる$a_0^{(1)}, \ldots, a_{n-1}^{(1)}\in\mathbb{Z}_p$が存在する.同様にして
  \[
  \begin{pmatrix}
    1 &   &        & \\
    & 1 &        & \\
    &   & \ddots & \\
    &   &        & 1
  \end{pmatrix}
  =
  \begin{pmatrix}
    a_0^{(1)} & \cdots & a_{n-1}^{(1)} \\
    & \vdots & \\
    a_0^{(n)} & \cdots & a_{n-1}^{(n)} \\
  \end{pmatrix}
  \begin{pmatrix}
    1               & \cdots & 1\\
    \beta_1         & \cdots & \beta_n\\
    & \vdots & \\
    \beta_1{}^{n-1} & \cdots & \beta_n{}^{n-1}
  \end{pmatrix}
  =:AX
  \]
  となるので,行列式をとって$1=\det A\det X$.よって$\det X\in\mathbb{Z}_p^\times$.
\end{proof}

\begin{screen}
  $\beta_1, \ldots, \beta_n\in\mathbb{Z}_p\ (p< n)$とすれば$i\neq j$で$\beta_i\equiv\beta_j\bmod p\mathbb{Z}_p$
\end{screen}
\begin{proof}
  命題I-9.1.31 (1)から$\lvert\mathbb{Z}_p/p\mathbb{Z}_p\rvert=p$.
  $\beta_1-\beta_2, \ldots, \beta_1-\beta_n$のうち少なくとも1つが$\equiv0$,もしくは1組が$\equiv$である.
  後者の場合はその二つの差が$\equiv0$となり主張を満たす.
\end{proof}

\begin{screen}
  $\alpha\in\mathcal{O}_K$,$K=\mathbb{Q}(\alpha)$で,$\mathcal{O}_K$が冪基底を持たなければ,$(\mathcal{O}_K\otimes_\mathbb{Z}\mathbb{Z}_p:\mathbb{Z}_p[\alpha])\neq1$で,これは$p$の冪であり,$(\mathcal{O}_K:\mathbb{Z}[\alpha])$の約数になる
\end{screen}
\begin{proof}
  $(\mathcal{O}_K\otimes_\mathbb{Z}\mathbb{Z}_p:\mathbb{Z}[\alpha] \otimes_\mathbb{Z} \mathbb{Z}_p) = 1$とする.
  完全系列
  \[
  \begin{tikzcd}
    0 \arrow[r] & \mathbb{Z}[\alpha] \arrow[r] & \mathcal{O}_K \arrow[r] & \mathcal{O}_K/\mathbb{Z}[\alpha] \arrow[r] & 0
  \end{tikzcd}
  \]
  に対し,$\mathbb{Z}_p$は平坦$\mathbb{Z}$加群なので,
  \[
  \begin{tikzcd}
    0 \arrow[r] & \mathbb{Z}[\alpha] \otimes_\mathbb{Z} \mathbb{Z}_p \arrow[r] & \mathcal{O}_K \otimes_\mathbb{Z} \mathbb{Z}_p \arrow[r] & (\mathcal{O}_K/\mathbb{Z}[\alpha]) \otimes_\mathbb{Z} \mathbb{Z}_p \arrow[r] & 0
  \end{tikzcd}
  \]
  も完全.従って,
  \begin{align*}
    0 &= \mathcal{O}_K \otimes_\mathbb{Z} \mathbb{Z}_p / \mathbb{Z}[\alpha] \otimes_\mathbb{Z} \mathbb{Z}_p \simeq (\mathcal{O}_K/\mathbb{Z}[\alpha]) \otimes_\mathbb{Z} \mathbb{Z}_p \\
    &\simeq (\mathcal{O}_K/\mathbb{Z}[\alpha]) \otimes_{\mathbb{Z}_{(p)}} \mathbb{Z}_{(p)} \otimes_\mathbb{Z} \mathbb{Z}_p \simeq (\mathcal{O}_K/\mathbb{Z}[\alpha]) \otimes_{\mathbb{Z}_{(p)}} \mathbb{Z}_p
  \end{align*}
  となる.
  $\mathbb{Z}_p \simeq \varprojlim (\mathbb{Z}/p^n\mathbb{Z}) \simeq \varprojlim (\mathbb{Z}_{(p)}/p^n\mathbb{Z}_{(p)})$なので,$\mathbb{Z}_p$は局所Noether環$\mathbb{Z}_{(p)}$の完備化であり,$\mathbb{Z}_{(p)}$上忠実平坦.
  従って,$\mathcal{O}_K/\mathbb{Z}[\alpha] = 0$,すなわち$\mathcal{O}_K = \mathbb{Z}[\alpha]$となり矛盾.

  例2.3.3と同様に,$A=\mathbb{Z}$,$B=\mathcal{O}_K$,$M=\mathbb{Z}[\alpha]$,$\mathfrak{p}=p\mathbb{Z}$とすれば
  $(B_\mathfrak{p}:M\otimes_AA_\mathfrak{p})=(\mathcal{O}_K\otimes_\mathbb{Z}\mathbb{Z}_p:\mathbb{Z}_p[\alpha])$.
  命題1.8.9 (2)からこれは$\mathcal{N}(\mathfrak{p})=p$の冪.主張の後半は命題1.8.9 (1)から従う.
\end{proof}

\paragraph{例2.3.5}~
\begin{screen}
  $K=\mathbb{Q}(\alpha)$,$\alpha$の$\mathbb{Q}$上最小多項式を$f(x)\in\mathbb{Z}[x]$,$\mathbb{Q}_2$での$f(x)$の根を$\beta_1, \beta_2, \beta_3\in\mathbb{Z}_2$とする.
  この時,$K\otimes_\mathbb{Q}\mathbb{Q}_2\simeq\mathbb{Q}_2{}^3$,$\mathcal{O}_K\otimes_\mathbb{Z}\mathbb{Z}_2\simeq\mathbb{Z}_2{}^3$で,$\alpha\mapsto\beta_1, \beta_2, \beta_3$と対応する
\end{screen}
\begin{proof}
  $\mathbb{Q}_2$は平坦$\mathbb{Q}$加群なので,
  \begin{align*}
    K \otimes_\mathbb{Q} \mathbb{Q}_2 &= \mathbb{Q}(\alpha) \otimes_\mathbb{Q} \mathbb{Q}_2 \simeq \left( \frac{\mathbb{Q}[x]}{(f(x))} \right) \otimes_\mathbb{Q} \mathbb{Q}_2 \simeq \frac{\mathbb{Q}[x] \otimes_\mathbb{Q} \mathbb{Q}_2}{(f(x)) \otimes_\mathbb{Q} \mathbb{Q}_2} \simeq \mathbb{Q}_2[x]/(f(x)) \\
    &= \mathbb{Q}_2[x]/((x - \beta_1)(x - \beta_2)(x - \beta_3)) \simeq \mathbb{Q}_2[x]/(x - \beta_1) \times \mathbb{Q}_2[x]/(x - \beta_2) \times \mathbb{Q}_2[x]/(x - \beta_3) \\
    &\simeq \mathbb{Q}_2 \times \mathbb{Q}_2 \times \mathbb{Q}_2
  \end{align*}
  である.この同型によって,$\alpha \otimes 1 \mapsto [x] \otimes 1 \mapsto [x] \mapsto ([x], [x], [x]) \mapsto (\beta_1, \beta_2, \beta_3)$となる.

  定理1.3.23から$2\mathbb{Z}$の上にある$\mathcal{O}_K$の素イデアルは$3$つであり,それらによる$K$の完備化を$\widehat{K}_1$,その整数環を$\widehat{\mathcal{O}}_1$などと表す.
  この時,
  \[
  \begin{tikzcd}
    K \otimes_\mathbb{Q} \mathbb{Q}_2 \arrow[r, "\simeq"]\arrow[d, phantom, "\ni" sloped] & \widehat{K}_1 \times \widehat{K}_2 \times \widehat{K}_3 \arrow[r, "\simeq"]\arrow[d, phantom, "\ni" sloped] & \mathbb{Q}_2{}^3\arrow[d, phantom, "\ni" sloped] \\
    \alpha \otimes 1 \arrow[r, mapsto] & (\phi_1(\alpha), \phi_2(\alpha), \phi_3(\alpha)) \arrow[r, mapsto] & (\beta_1, \beta_2, \beta_3)
  \end{tikzcd}
  \]
  が得られる.追加定理\ref{Thm_1_3_23_2}(p.\pageref{Thm_1_3_23_2})から,これを整数環に制限して
  \[
  \begin{tikzcd}
    \mathcal{O}_K \otimes_\mathbb{Z} \mathbb{Z}_2 \arrow[r, "\simeq"]\arrow[d, phantom, "\ni" sloped] & \widehat{\mathcal{O}}_1 \times \widehat{\mathcal{O}}_2 \times \widehat{\mathcal{O}}_3 \arrow[r, "\simeq"]\arrow[d, phantom, "\ni" sloped] & \mathbb{Z}_2{}^3\arrow[d, phantom, "\ni" sloped] \\
    \alpha \otimes 1 \arrow[r, mapsto] & (\phi_1(\alpha), \phi_2(\alpha), \phi_3(\alpha)) \arrow[r, mapsto] & (\beta_1, \beta_2, \beta_3)
  \end{tikzcd}
  \]
  となる.
\end{proof}
% \begin{proof}
%   \label{decomposition_of_K_Q2}
%   まず,$\mathbb{Q}(\alpha)\otimes_\mathbb{Q}\mathbb{Q}_2\simeq\mathbb{Q}_2{}^3$を証明する.環準同型:
%   \begin{align*}
%     \mathbb{Q}_2[x]/(f(x)) &\ni b_0x^2+\cdots+b_2+(f(x))\\
%     &\mapsto x^2\otimes_\mathbb{Q}b_0+\cdots+1\otimes_\mathbb{Q}b_2+(f(x))\otimes_\mathbb{Q}\mathbb{Q}_2 &\in \mathbb{Q}[x]\otimes_\mathbb{Q}\mathbb{Q}_2/(f(x))\otimes_\mathbb{Q}\mathbb{Q}_2\\
%     &\mapsto (x^2/(f(x)))\otimes_\mathbb{Q}b_0+\cdots+(1/(f(x)))\otimes_\mathbb{Q}b_2 &\in (\mathbb{Q}[x]/(f(x)))\otimes_\mathbb{Q}\mathbb{Q}_2\\
%     &\mapsto \alpha^2\otimes_\mathbb{Q}b_0+\cdots+1\otimes_\mathbb{Q}b_2 &\in \mathbb{Q}(\alpha)\otimes_\mathbb{Q}\mathbb{Q}_2
%   \end{align*}
%   と
%   \begin{align*}
%     \mathbb{Q}_2[x]/(f(x)) &\ni b_0x^2+\cdots+b_2+(f(x))\\
%     &\mapsto b_0x^2+\cdots+b_2+((x-\beta_1)(x-\beta_2)(x-\beta_3)) &\in \mathbb{Q}_2[x]/((x-\beta_1)(x-\beta_2)(x-\beta_3))\\
%     &\mapsto \bigoplus_{i=1}^3(b_0x^2+\cdots+b_2+(x-\beta_i)) &\in \bigoplus_{i=1}^3\mathbb{Q}_2[x]/((x-\beta_i))\\
%     &\mapsto \bigoplus_{i=1}^3(b_0{\beta_i}^2+\cdots+b_2) &\in \bigoplus_{i=1}^3\mathbb{Q}_2
%   \end{align*}
%   から環準同型
%   \[\phi\colon\mathbb{Q}(\alpha)\otimes_\mathbb{Q}\mathbb{Q}_2\ni\alpha^2\otimes_\mathbb{Q}b_0+\cdots+1\otimes_\mathbb{Q}b_2\mapsto \bigoplus_{i=1}^3(b_0{\beta_i}^2+\cdots+b_2)\in\bigoplus_{i=1}^3\mathbb{Q}_2\]
%   が得られる.$\forall(c_1, c_2, c_3)\in\mathbb{Q}_2{}^3$に対し,$b_0, b_1, b_2$を
%   \[
%   \begin{pmatrix}
%     \beta_1{}^2 & \beta_1 & 1\\
%     \beta_2{}^2 & \beta_2 & 1\\
%     \beta_3{}^2 & \beta_3 & 1
%   \end{pmatrix}^{-1}
%   \begin{pmatrix}
%     c_1\\
%     c_2\\
%     c_3
%   \end{pmatrix}
%   =
%   \begin{pmatrix}
%     b_0\\
%     b_1\\
%     b_2
%   \end{pmatrix}
%   \]
%   と選ぶことができるので,$\phi$は全射.
%   上の式から,$(c_1, c_2, c_3)=(0, 0, 0)$なら$(b_0, b_1, b_2)=(0, 0, 0)$なので$\ker\phi=\{0\}$.
%   よって$\phi$は全単射となり(環として)同型:$K\otimes_\mathbb{Q}\mathbb{Q}_2\simeq\mathbb{Q}_2{}^3$.
%   特に,$\phi$によって$\alpha\otimes_\mathbb{Q}1$は$(\beta_1, \beta_2, \beta_3)$に写る.
%
%   $2\mathbb{Z}$の上にある$\mathcal{O}_K$の素イデアルを$\mathfrak{p}_1, \ldots, \mathfrak{p}_g$とする.
%   $\mathfrak{p}_i$によよる$K, \mathcal{O}_K$の完備化を$\widehat{K}_i, \widehat{B}_i$とする.
%   定理1.3.23 (2)から$K\otimes_\mathbb{Q}\mathbb{Q}_2\simeq\widehat{K}_1\times\cdots\times\widehat{K}_g$.
%   上で示したように,$K\otimes_\mathbb{Q}\mathbb{Q}_2\simeq\mathbb{Q}_2{}^3$なので,$\widehat{K}_1\times\cdots\times\widehat{K}_g\simeq\mathbb{Q}_2{}^3$となる.
%   体の直積は異なる数の体の直積と同型にはなり得ないので$g=3$.
%   $\mathbb{Q}_2$の整数環は$\mathbb{Z}_2$なので,$\phi_i$の$\widehat{B}_i$への制限を$\overline{\phi}_i$とすれば補題</a>から$\overline{\phi}_i\colon\widehat{B}_i\simeq\mathbb{Z}_2$となる.
%   定理1.3.23 (2)から$K\otimes_\mathbb{Q}\mathbb{Q}_2\simeq\widehat{K}_1\times\widehat{K}_2\times\widehat{K}_3$.
%   さらに,$\phi_1\times\phi_2\times\phi_3$によって$\widehat{K}_1\times\widehat{K}_2\times\widehat{K}_3\simeq\mathbb{Q}_2{}^3$.
%   これを合成すれば,上で先程構成した$\phi$になり,$\alpha\otimes_\mathbb{Q}1\mapsto(\beta_1, \beta_2, \beta_3)$.
%   Thm1.3.23から,上の同型を$\mathcal{O}_K\otimes_\mathbb{Z}\mathbb{Z}_2$に制限すれば,$\mathcal{O}_K\otimes_\mathbb{Z}\mathbb{Z}_2\simeq\mathbb{Z}_2{}^3$が得られ,やはり$\alpha \otimes 1\mapsto(\beta_1, \beta_2, \beta_3)$.
%   (定理1.3.23 (2)の同型と$\overline{\phi}_1\times\cdots\times\overline{\phi}_3$の合成)
%
%   $g=3$なので定理1.3.23 (4)(5)から$[\widehat{K}_i:\mathbb{Q}_2]=1$となり,補題</a>から$\widehat{K}_i=\mathbb{Q}_2$.
%   さらにその整数環は$\mathbb{Z}_2$.つまり上の$\phi$と$\overline{\phi}$は自己同型.
% \end{proof}

\begin{screen}
  $\mathcal{O}_K\supset\mathbb{Z}[\alpha]$なら,奇素数$p$に対して$(\alpha^2-\alpha)/2\in\mathcal{O}_K\otimes_\mathbb{Z}\mathbb{Z}_p$
\end{screen}
\begin{proof}
  $\alpha^2-\alpha\in\mathcal{O}_K$である.
  $\ord_p(1/2)=0$なので$\lvert1/2\rvert=1$,すなわち$1/2\in\mathbb{Z}_p$(定理I-9.1.26 (5)).
  よって$(\alpha^2-\alpha) \otimes 1/2\in\mathcal{O}_K\otimes_\mathbb{Z}\mathbb{Z}_p$.
  % (定理1.3.23 (2)により,これを$\widehat{\mathcal{O}}_1\times\cdots$にうつせば,$(\phi_1(\alpha^2-2)/2, \ldots)$となる)
\end{proof}

\begin{screen}
  全ての素数$p$に対し$\gamma \otimes 1\in\mathcal{O}_K\otimes_\mathbb{Z}\mathbb{Z}_p$であれば$\gamma\in\mathcal{O}_K$
\end{screen}
\begin{proof}
  $p\mathbb{Z}$の上にある$\mathcal{O}_K$の素イデアルを$P_1, \ldots$,$P_i$進距離による$\mathcal{O}_K$の完備化を$\widehat{\mathcal{O}}_i$とする.
  定理1.3.23 (2)から$\phi_i\colon\mathcal{O}_K\hookrightarrow\widehat{\mathcal{O}}_i$として$\phi\colon\mathcal{O}_K\otimes_\mathbb{Z}\mathbb{Z}_p\ni x\otimes y\mapsto(\phi_1(x)y, \ldots)\in\widehat{\mathcal{O}}_1\times\cdots$なので,$\phi_i(\gamma)\in\widehat{\mathcal{O}}_i$.
  よって全ての$p$を考えれば,$\mathcal{O}_K$の全ての素イデアル$P$に対し,$P$進距離による完備化を$\widehat{\mathcal{O}}$とすれば$\gamma \in \widehat{\mathcal{O}}$.
  よって命題1.2.14から$\gamma\in\mathcal{O}_K$.
\end{proof}

\begin{screen}
  $K=\mathbb{Q}(\alpha)$,$\varDelta_K=-503$,$\alpha^3+6\alpha^2-\alpha+2=0$,$\mathcal{O}_K$の$\mathbb{Z}$基底を$\{1, \alpha, (\alpha^2-\alpha)/2\}$とすれば,$x\in\mathcal{O}_K$,$K=\mathbb{Q}(x)$として$(\mathcal{O}_K:\mathbb{Z}[x])$が$2$の倍数である
\end{screen}
\begin{proof}
  % 任意の$x\in\mathcal{O}_K$に対し,$\mathbb{Q}(x)=\mathbb{Q}(\alpha)$となる.まず,$\mathbb{Q}(x)\subset\mathbb{Q}(\alpha)$は明らか.
  % $a_0x^n+\cdots+a_{n-1}x+a_n\in\mathbb{Q}(x)$とする.
  % $\mathcal{O}_K$の元は$a, b, c\in\mathbb{Z}$を使って,$x=a+b\alpha+c(\alpha^2-\alpha)/2$と表すことができる.
  % $x^2$以上の次数の項に対し,$x=a+b\alpha+c(\alpha^2-\alpha)/2$を代入して,$\alpha^3=-6\alpha^2+\alpha-2$を使えば$a_0x^n+\cdots+a_{n-2}x^2=b_0\alpha^2+b_1\alpha+b_2$となる.よって
  % \begin{align*}
  %   a_0x^n+\cdots+a_{n-1}x+a_n &= b_0\alpha^2+b_1\alpha+b_2+a_{n-1}\left(a+b\alpha+c\frac{\alpha^2-\alpha}{2}\right)+a_n\\
  %   &= \left(b_0+\frac{c}{2}a_{n-1}\right)\alpha^2+\left(b_1+ba_{n-1}-\frac{c}{2}a_{n-1}\right)+(b_2+aa_{n-1}+a_n)
  % \end{align*}
  % となる.ここで,適当に$a_{n-1}$と$a_n$を選べば$a_0x^n+\cdots+a_{n-1}x+a_n=d\alpha$となる.よって,$\alpha=a_0/dx^n+\cdots+a_n/d$となるので,$\mathbb{Q}(\alpha)\subset\mathbb{Q}(x)$.

  p.106の真ん中の方法と同様に,$x$の最小多項式を求める.
  \begin{lstlisting}
    ? m = [a, b - c / 2, c / 2; -c, a + c / 2, b - 7 * c / 2; 7 * c - 2 * b, b - 9 * c / 2, a - 6 * b + 43 * c / 2]
    ? charpoly(m)
    ? poldisc(%2)
  \end{lstlisting}
  特性多項式$g(x)$は
  \begin{align*}
    x^3 &+ (-3a + (6b - 22c))x^2 + (3a^2 + (-12b + 44c)a + (-b^2 + 7cb - 9c^2))x \\
        &+ (-a^3 + (6b - 22c)a^2 + (b^2 - 7cb + 9c^2)a + (2b^3 - 9cb^2 + 7c^2b - 2c^3))
  \end{align*}
  で,判別式は
  \[\varDelta(g)=-2012b^6 + 30180cb^5 - 175547c^2b^4 + 487910c^3b^3 - 634283c^4b^2 + 311860c^5b - 50300c^6.\]
  系1.7.5 (2),命題1.9.9から
  \begin{align*}
    (\mathcal{O}_K:\mathbb{Z}[x])^2 &= \frac{\varDelta(c)}{\varDelta_K} = 4b^6+60cb^5+349c^2b^4-970c^3b^3+1261c^4b^2-620c^5b+100c^6\\
    &\equiv c^2b^2(349b^2+1261c^2)\equiv0\bmod4
  \end{align*}
  となる.
\end{proof}

\begin{screen}
  $f(x)=x^3+6x^2-x+2$,$\alpha$を$f(x)$の根,$K=\mathbb{Q}(\alpha)$,$A=\mathbb{Z}[x]/(f(x))\simeq\mathbb{Z}[\alpha]$とした時,$A/503A\simeq\mathbb{F}_{503}[x]/((x-39)^2)\times\mathbb{F}_{503}[x]/(x+84)$
\end{screen}
\begin{proof}
  $f(x)\equiv(x-39)^2(x+84)\bmod503$なので,$g(x)\in\mathbb{Z}[x]\ (\deg g\leq2)$として自然な全射準同型
  \[A=\mathbb{Z}[x]/(f(x))\ni g(x)+(f(x))\mapsto \overline{g}(x)+((x-39)^2(x+84))\in\mathbb{F}_{503}[x]/((x-39)^2(x+84))\]
  の$\ker$は$g(x)$のうち係数が$503$の倍数の物で代表される:$503A$.よって準同型定理から$A/503A\simeq\mathbb{F}_{503}[x]/((x-39)^2(x+84))$.
  $-116(x-39)^2+116(x-6)(x+84)=-234900\equiv1$なので,$((x-39)^2)+(x+84)=\mathbb{F}_{503}[x]$.よって中国式剰余定理から
  \[A/503A\simeq\mathbb{F}_{503}[x]/((x-39)^2(x+84))\simeq\mathbb{F}_{503}[x]/((x-39)^2)\times\mathbb{F}_{503}[x]/(x+84).\]
  ($503$は$(\mathcal{O}_K:\mathbb{Z}[\alpha])$を割らないので,素イデアル分解に関する定理を使った方が早い)
\end{proof}

\paragraph{例2.3.6}
\subparagraph{$n = 4$の場合}~

\begin{screen}
  $g(x+1)\in\mathbb{Z}_2[x]$がモニックなEisenstein多項式であれば$g(x)=x^2+cx+d\ (c\equiv4, \; d\equiv1\bmod8)$の根を$\gamma$,$F=\mathbb{Q}_2(\gamma)$とすれば$\mathcal{O}_F=\mathbb{Z}_2[\gamma]$
\end{screen}
\begin{proof}
  $h(x)=g(x+1)$とすれば$h(x)$はEisenstein多項式で,その根は$\gamma-1$.
  % $a_0\gamma^n+\cdots+a_n\in\mathbb{Q}_2[\gamma]$について$=a_0(\gamma-1)^n+(a_1-na_0)(\gamma-1)^{n-1}+\cdots$と変形できるので,$\in\mathbb{Q}_2[\gamma-1]$となり$\mathbb{Q}_2[\gamma]\subset\mathbb{Q}_2[\gamma-1]$.
  % $a_0(\gamma-1)^n+\cdots+a_n\in\mathbb{Q}_2[\gamma-1]$について$=a_0\gamma^n+(a_1-na_0)\gamma^{n-1}$と変形できるので,$\in\mathbb{Q}_2[\gamma]$となり$\mathbb{Q}_2[\gamma]\supset\mathbb{Q}_2[\gamma-1]$.以上から
  % $\mathbb{Q}_2[\gamma]=\mathbb{Q}_2[\gamma-1]$.命題I-7.1.11から
  $\mathbb{Q}_2(\gamma)=\mathbb{Q}_2(\gamma-1)$は容易に分かる.
  $\mathbb{Z}_2$, $\mathcal{O}_F$はともに完備離散付値環である.
  従って命題1.10.7から$F/\mathbb{Q}_2$は完全分岐で,分岐指数は$2$.
  $\mathcal{O}_F$の素イデアルを$P$とする.
  $\ord_P(c+2)=2\ord_{2\mathbb{Z}_2}(c+2)=2$,同様に$\ord_P(c+d+1)=2$となる.命題1.1.3と$h(\gamma-1)=0$から
  \[2=\ord_P((\gamma-1)^2+(c+2)(\gamma-1))\geq\min\{2\ord_P(\gamma-1), \ord_P(c+2)+\ord_P(\gamma-1)=2+\ord_P(\gamma-1)\}, \]
  ただし$2\ord_P(\gamma-1)\neq2+\ord_P(\gamma-1)$すなわち$\ord_P(\gamma-1)\neq2$なら等号が成立.
  $\gamma-1$は$\mathbb{Z}_2$上整(実際$h(x)$が存在する)なので,$\ord_P(\gamma-1)\geq0$.
  $\ord_P(\gamma-1)=0$なら$2=0$で矛盾.
  $\ord_P(\gamma-1)=1$は式を満たす.
  $\ord_P(\gamma-1)\geq2$では右辺が$4$以上なので不適.よって$\ord_P(\gamma-1)=1$,つまり$\gamma-1$は$\mathcal{O}_F$の素元となり,再び命題1.10.7から$\mathcal{O}_F=\mathbb{Z}_2[\gamma-1]=\mathbb{Z}_2[\gamma]$となる.
\end{proof}

\begin{screen}
  上の状況で$\varDelta_{F/\mathbb{Q}_2}=(4)$
\end{screen}
\begin{proof}
  $h(x)=g(x+1)$は$\mathbb{Q}_2$上既約なので$g(x)$も$\mathbb{Q}_2$上既約である(対偶取る).
  よって,$F=\mathbb{Q}_2(\gamma)$で$\gamma$の$\mathbb{Q}_2$上最小多項式は$g(x)\in\mathbb{Z}_2[x]$で,$\mathcal{O}_F=\mathbb{Z}_2[\gamma]$.
  $\varDelta(g)\equiv4\bmod8$,命題1.9.9から$\varDelta_{F/\mathbb{Q}_2}(1, \gamma)=\varDelta(g)$.
  よって$\ord_P(\varDelta_{F/\mathbb{Q}_2}(1, \gamma)) = 2\ord_{2\mathbb{Z}_2}(\varDelta_{F/\mathbb{Q}_2}(1, \gamma)) = 4$.
  % (補題I-8.1.17から$\ord_{2\mathbb{Z}_2}$が正当化される).
  $(2)=P^2$(分岐指数が$2$)なので,$\varDelta_{F/\mathbb{Q}_2}=\varDelta_{F/\mathbb{Z}_Q, P}=P^4=(2^2)=(4)$.
\end{proof}

\begin{screen}
  $K=\mathbb{Q}(\alpha)$,$\alpha$の$\mathbb{Q}$上最小多項式を$f(x)\in\mathbb{Z}[x]$とする.
  $f(x)$が$\mathbb{Q}_2$上$(x-\beta)(x^2+cx+d)$($\beta\in\mathbb{Z}_2$,$x^2+cx+d$は$\mathbb{Q}_2$上既約)となり,
  $x^2+cx+d$の根を$\gamma$,$F=\mathbb{Q}_2(\gamma)$とすれば,
  $\mathbb{Q}_2$同型として$K\otimes_\mathbb{Q}\mathbb{Q}_2\simeq\mathbb{Q}_2\times F$で,$\varDelta_{\widehat{K}_1/\mathbb{Q}_2}=\varDelta_{\mathbb{Q}_2/\mathbb{Q}_2}$, $\varDelta_{\widehat{K}_2/\mathbb{Q}_2}=\varDelta_{F/\mathbb{Q}_2}$
\end{screen}
\begin{proof}
  例2.3.5と同様に.
  $\mathbb{Q}_2$同型
  \[\mathbb{Q}(\alpha) \otimes_\mathbb{Q} \mathbb{Q}_2 \simeq \mathbb{Q}[x]/(x-\beta) \times \mathbb{Q}[x]/(x^2 + cx + d) \simeq \mathbb{Q}_2\times F\]
  を得る.
  従って,$2\mathbb{Z}$の上にある$\mathcal{O}_K$の素イデアルは$2$つ($\mathfrak{p}_1$, $\mathfrak{p}_2$とする)で,それらによる$K$の完備化を$\widehat{K}_1, \widehat{K}_2$とする.
  $\mathbb{Q}_2$同型$\widehat{K}_1\times\widehat{K}_2\simeq\mathbb{Q}_2\times F$から,$\mathbb{Q}_2$同型$\widehat{K}_1\simeq\mathbb{Q}_2$,$\widehat{K}_2\simeq F$が得られる.

  以下の補題を使えば$\varDelta_{\widehat{K}_1/\mathbb{Q}_2}=\varDelta_{\mathbb{Q}_2/\mathbb{Q}_2}$, $\varDelta_{\widehat{K}_2/\mathbb{Q}_2}=\varDelta_{F/\mathbb{Q}_2}$が分かる.
\end{proof}

\begin{screen}
  \begin{lem}
    \label{LF_iso_relative_disc}
    $L/K$, $M/K$を局所体とする.
    $K$同型$\phi\colon L\simeq M$があれば,$\varDelta_{L/K} = \varDelta_{M/K}$.
  \end{lem}
\end{screen}
\begin{proof}
  $L, M$における$\mathcal{O}_K =: A$の整閉包を$B, C$とする.
  命題1.5.3 (2)から$A$, $B$, $C$は完備離散付値環で,$\phi(B)=C$となる(追加補題\ref{iso_field_integral_closure}, p.\pageref{iso_field_integral_closure}).
  それぞれの素イデアルを$\mathfrak{p}, P, \mathcal{P}$とすると,$\phi(P)=\mathcal{P}$となることは容易に分かる.
  命題I-6.5.8 (2)から$S=A\setminus\mathfrak{p}$は$A^\times$に含まれ,$S \subset (B \setminus P) \subset B^\times$,$S\subset C^\times$が成立する.
  よって$A_\mathfrak{p} = S^{-1}A = A$,$B_\mathfrak{p} = B$,$C_\mathfrak{p} = C$である.

  $\{v_1, \ldots, v_n\}$を$B$の$A$基底とする(命題I-8.1.24 (3)).$\phi$は$K$の元を不変にするので$A$の元も変えない.
  $a_1v_1+\cdots+a_nv_n=0$ならば$a_1=\cdots=a_n=0$.
  よって,$a_1\phi(v_1)+\cdots+a_n\phi(v_n)=0$ならば$a_1=\cdots=a_n=0$なので$\{\phi(v_1), \ldots, \phi(v_n)\}$は$A$上一次独立.
  $\forall c\in C$に対し$\phi^{-1}(c)=a_1v_1+\cdots+a_nv_n$となる$a_1, \ldots\in A$が存在する.
  これを$\phi$で写せば$c=a_1\phi(v_1)+\cdots+a_n\phi(v_n)$となるので,$\{\phi(v_1), \ldots, \phi(v_n)\}$は$C$の$A$基底となる.

  $\Hom_K^\text{al}(M, \overline{K})$と$\Hom_K^\text{al}(L, \overline{K})$は$\phi$の合成によって1対1に対応する.
  $\sigma_i \in \Hom_K^\text{al}(M, \overline{K})$とすれば,
  \[\Tr_{M/K}(\phi(v_i)\phi(v_j))=\sum_i\sigma_i(\phi(v_i)\phi(v_j))=\sum_i\sigma_i\circ\phi(v_i)\sigma_i\circ\phi(v_j)=\Tr_{L/K}(v_iv_j).\]
  よって$\varDelta_{L/K}(v_1, \ldots, v_n)=\varDelta_{M/K}(\phi(v_1), \ldots, \phi(v_n))$となり,$\varDelta_{L/K, \mathfrak{p}}=\varDelta_{M/K, \mathfrak{p}}$.
  以上から,相対判別式は等しい:$\varDelta_{L/K}=\varDelta_{M/K}$.
\end{proof}

\begin{screen}
  % $K=\mathbb{Q}(\alpha)$,$\alpha$の$\mathbb{Q}$上最小多項式を$f(x)\in\mathbb{Z}[x]$として
  $\varDelta_{K/\mathbb{Q}, 2}=(4)$なら$\varDelta_K=\varDelta(f)=-436$で$\mathcal{O}_K = \mathbb{Z}[\alpha]$
\end{screen}
\begin{proof}
  命題1.9.9から$\varDelta_{K/\mathbb{Q}}(1, \alpha, \alpha^2) = \varDelta(f) = -436$.
  系1.7.5 (2)から$\varDelta_{K/\mathbb{Q}}(1, \alpha, \alpha^2) = (\mathcal{O}_K:\mathbb{Z}[\alpha])^2\varDelta_K$.
  命題1.8.5から$\varDelta_K$は$4$で割り切れるので,$\varDelta_K = -436$.
  $(\mathcal{O}_K:\mathbb{Z}[\alpha]) = 1$となるので,$\mathcal{O}_K = \mathbb{Z}[\alpha]$.
\end{proof}

\subparagraph{$n = 6$の場合}
前半は$n = 4$の場合と同様.
$g(2y+1)$の根を$\gamma$と置いたので,$f(x)=(x-\beta)g(x)$の根は$\beta, 2\gamma+1$.

写像$\phi$は$K\hookrightarrow K\otimes_\mathbb{Q}\mathbb{Q}_2\simeq\mathbb{Q}_2\times F$.
この話から,これを整数環に制限すれば$\mathcal{O}_K\hookrightarrow\mathcal{O}_K\otimes_\mathbb{Z}\mathbb{Z}_2\simeq\mathbb{Z}_2\times\mathcal{O}_F$.
したがって,$a\in K$を$\phi$で写した物が$\mathbb{Z}_2\times\mathcal{O}_F$の元ならば$a\in\mathcal{O}_K$と言える.

\subparagraph{$n = 14$の場合}
$g(4y+1)/16 = y^2+(c+2)/4y+(c+d+1)/16$を$2\mathbb{Z}_2$(係数の$\ord_2$は$0$)を法として考えれば$x^2+x+1$となり,$\pm1$は根にならないので,命題I-8.1.10から$\mathbb{Z}_2/2\mathbb{Z}_2$上既約.
命題I-8.2.1からこれは$\mathbb{Z}_2$上既約.

$\mathcal{O}_F$を求める.
$\alpha=a+b\gamma\in F~ (a, b\in\mathbb{Q}_2)$が$\mathbb{Z}_2$上整,すなわち$\alpha \in \mathcal{O}_F$とする.
$\gamma$の$\mathbb{Q}_2$上共軛を$\overline{\gamma}$とすれば
\[\Tr_{F/\mathbb{Q}_2}(\gamma)=\gamma+\overline{\gamma}=-(c+2)/4, \quad\N_{F/\mathbb{Q}_2}(\gamma)=\gamma\overline{\gamma}=(c+d+1)/16\]
で$\ord_2(\Tr_{F/\mathbb{Q}_2})=0$,$\ord_2(\N_{F/\mathbb{Q}_2})=0$.また,
\begin{align*}
  A &:= \Tr_{F/\mathbb{Q}_2}(\alpha) = (a+b\gamma) + (a+b\overline{\gamma}) = 2a + b\Tr_{F/\mathbb{Q}_2}(\gamma), \\
  B &:= \N_{F/\mathbb{Q}_2}(\alpha) = (a+b\gamma)(a+b\overline{\gamma}) = a^2 + ab\Tr_{F/\mathbb{Q}_2}(\gamma) + b^2\N_{F/\mathbb{Q}_2}(\gamma)
\end{align*}
とする.命題I-8.1.19から$A, B\in\mathbb{Z}_2$.
$4B=A^2+b^2[4\N_{F/\mathbb{Q}_2}(\alpha)-\Tr_{F/\mathbb{Q}_2}(\alpha)^2]\in4\mathbb{Z}_2$なので$b^2[4\N_{F/\mathbb{Q}_2}(\alpha)-\Tr_{F/\mathbb{Q}_2}(\alpha)^2]\in\mathbb{Z}_2$.
$\ord_2(4\N_{F/\mathbb{Q}_2}(\alpha)-\Tr_{F/\mathbb{Q}_2}(\alpha)^2)=0$なので$\ord_2(b)\geq0$,つまり$b\in\mathbb{Z}_2$(命題1.1.3,定理I-9.1.26 (5)).
$B$の定義式で$\ord_2$を考えて$a\in\mathbb{Z}_2$.
従って,$\mathcal{O}_F \subset \mathbb{Z}_2[\gamma]$であり,$\mathcal{O}_F = \mathbb{Z}_2[\gamma]$.

$y^2+(c+2)/4y+(c+d+1)/16$の判別式は$\equiv1\bmod4$.
よって命題1.9.9から$\varDelta_{F/\mathbb{Q}_2}(1, \gamma)$は$2$で割り切れないので,補題1.8.3から$\varDelta_{F/\mathbb{Q}_2, 2}$は$2\mathbb{Z}_2$で割り切れず,$\varDelta_{F/\mathbb{Q}_2}$も$2\mathbb{Z}_2$で割り切れない.
よって,Dedekindの判別定理から$2\mathbb{Z}_2$は$F/\mathbb{Q}_2$で不分岐.
定理I-9.1.26 (7)から$2\mathbb{Z}_2$は$\mathbb{Z}_2$の唯一の素イデアルなので$F/\mathbb{Q}_2$は不分岐拡大.

$\varDelta_{K/\mathbb{Q}, 2}=\varDelta_{\mathbb{Q}_2/\mathbb{Q}_2}\varDelta_{F/\mathbb{Q}_2}$は$2$で割り切れないので,$\varDelta_{K}$は$2$で割り切れない(命題1.8.5).
命題1.9.9から系1.7.5 (2)から$\varDelta(f)=-5296=(\mathcal{O}_K:\mathbb{Z}[\alpha])^2\varDelta_K$なので$\varDelta_K=-331$,$(\mathcal{O}_K:\mathbb{Z}[\alpha])=4$.

\paragraph{命題2.3.9}
まず,$p\mid a$もしくは$p\mid b$について$\mathcal{O}_K\otimes_\mathbb{Z}\mathbb{Z}_p= V\otimes_\mathbb{Z}\mathbb{Z}_p$を証明し,$(\mathcal{O}_K\otimes_\mathbb{Z}\mathbb{Z}_p:V\otimes_\mathbb{Z}\mathbb{Z}_p)=1$.
次に,$p\nmid 3, a, b$に対しても同じことが証明できる.

\begin{itemize}
  \item $a$が$3$の倍数なら素数$p$は上のいずれかに属する.よって命題1.8.9 (1)(3)から$(\mathcal{O}_K:V)=1$
  \item $a, b$が$3$の倍数でなければ,命題1.8.9 (3)から,$p\neq3$については,命題1.8.9 (1)の右辺の因子は$1$.よって命題1.8.9 (1)から$(\mathcal{O}_K:V)=(\mathcal{O}_K\otimes_\mathbb{Z}\mathbb{Z}_3:V\otimes_\mathbb{Z}\mathbb{Z}_3)$
  \begin{itemize}[label=•]
    \item $a^2b^4\not\equiv1\bmod9$なら$(\mathcal{O}_K\otimes_\mathbb{Z}\mathbb{Z}_3:V\otimes_\mathbb{Z}\mathbb{Z}_3)=1$なので$(\mathcal{O}_K:V)=1$
    \item $a^2b^4\equiv1\bmod9$なら,命題1.8.9 (2)から$(\mathcal{O}_K\otimes_\mathbb{Z}\mathbb{Z}_3:V\otimes_\mathbb{Z}\mathbb{Z}_3)$は$3$の冪.
    $\varDelta_{K/\mathbb{Q}, 3}=(3)$なので$\varDelta_K$の$\ord_3$は$1$.よって,$(\mathcal{O}_K:V)=3$
  \end{itemize}
\end{itemize}

\begin{screen}
  $a, b$を平方因子を持たない互いに素な整数,$\beta_1=\sqrt[3]{ab^2}$,$\beta_2=\sqrt[3]{a^2b}$,$K=\mathbb{Q}(\beta_1)$,
  $V=\mathbb{Z}+\mathbb{Z}\beta_1+\mathbb{Z}\beta_2$として素数$p$が$a$を割り切れば,
  $p$は$K/\mathbb{Q}$で完全分岐で$\mathcal{O}_K\otimes_\mathbb{Z}\mathbb{Z}_p = V\otimes_\mathbb{Z}\mathbb{Z}_p$
  % の証明.(p.111の上の方)
\end{screen}
\begin{proof}
  $x^3-ab^2$はEisenstein多項式なので$\mathbb{Q}$上既約であり,$\beta_1$の$\mathbb{Q}$上最小多項式となる.
  これは$\mathbb{Q}_p$上でもEisenstein多項式なので既約であり,$[\mathbb{Q}_p(\beta_1):\mathbb{Q}_p]=3$.
  $p$の上にある$\mathcal{O}_K$の素イデアルによる$K$の完備化を$\widehat{K}_1, \ldots, \widehat{K}_g$とする.
  $\beta_1 \in K \subset \widehat{K}_i$なので$\mathbb{Q}_p(\beta_1) \subset \widehat{K}_i$となり$[\widehat{K}_i:\mathbb{Q}_p(\beta_1)]\geq1$.
  よって$[\widehat{K}_i:\mathbb{Q}_p]\geq3$.
  定理1.3.23 (4)(5)から$[\widehat{K}_1:\mathbb{Q}_p] + \cdots + [\widehat{K}_g:\mathbb{Q}_p] = 3$なので$g=1$.
  すなわち,$p\mathbb{Z}$の上にある$\mathcal{O}_K$の素イデアルは1つ.
  $\widehat{K}_1$を$\widehat{K}$と書けば,$[\widehat{K}:\mathbb{Q}_p]=3$で$[\widehat{K}:\mathbb{Q}_p(\beta_1)]=1$.
  従って,$\widehat{K} = \mathbb{Q}_p(\beta_1)$.$\widehat{K}$の整数環を$\widehat{\mathcal{O}}_K$とする.
  $\widehat{\mathcal{O}}_K$の極大イデアルを$P$とすると命題1.10.7から$\widehat{K}/\mathbb{Q}_p$は完全分岐で分岐指数は$e(P/p\mathbb{Z}_p)=3$.
  % <a href="./3.htmlord_pP_e" target="_blank">
  % この話</a>から,

  よって$\ord_p(a)=1$,$\ord_p(b)=0$に注意して
  \[3\ord_P(\beta_1)=\ord_P(\beta_1{}^3)=\ord_P(ab^2)=3\ord_p(ab^2)=3.\]
  よって$\ord_P(\beta_1)=1$となり$\beta_1$は$\widehat{\mathcal{O}}_K$の素元.
  再び命題1.10.7から$\widehat{\mathcal{O}}_K=\mathbb{Z}_p[\beta_1]$.
  定理1.3.23 (6)から$p\mathbb{Z}$の分岐指数は$3$なので,$K/\mathbb{Q}$で完全分岐.

  $V \subset \mathcal{O}_K$なので,$V \otimes_\mathbb{Z} \mathbb{Z}_p \subset \mathcal{O}_K \otimes_\mathbb{Z} \mathbb{Z}_p$.
  $b \in \mathbb{Z}_p^\times$,$\beta_2 = \beta_1{}^2/b$に注意して,テンソル積の普遍性から加群準同型$\phi \colon V \otimes_\mathbb{Z} \mathbb{Z}_p \to \mathbb{Z}_p[\beta_1]$を得る($\sigma \colon \mathcal{O}_K \hookrightarrow \widehat{\mathcal{O}}_K$).
  % $V = \mathbb{Z} + \mathbb{Z}\beta_1 + \mathbb{Z} \beta_2$なので,
  \[
  \begin{tikzcd}
    V \times \mathbb{Z}_p \arrow[r]\arrow[d] & \mathbb{Z}_p[\beta_1] \\
    V \otimes_\mathbb{Z} \mathbb{Z}_p \arrow[ur, "\phi"', dashed]
  \end{tikzcd}
  \quad
  \begin{tikzcd}
    (v, c) \arrow[r, mapsto]\arrow[d, mapsto] & \sigma(v)c \\
    v \otimes c \arrow[ur, mapsto]
  \end{tikzcd}
  % \begin{tikzcd}
  %   (a_0 + a_1 \beta_1 + a_2 \beta_2, c) \arrow[r, mapsto]\arrow[d, mapsto] & a_0c + a_1c \beta_1 + \dfrac{a_2c}{b} \beta_1{}^2 \\
  %   (a_0 + a_1 \beta_1 + a_2 \beta_2) \otimes c \arrow[ur, mapsto]
  % \end{tikzcd}
  \]
  % 明示的に書けば,
  % \[\phi\colon V\otimes_\mathbb{Z}\mathbb{Z}_p\ni1\otimes_\mathbb{Z}b_0+\beta_1\otimes_\mathbb{Z}b_1+\beta_2\otimes_\mathbb{Z}b_2\mapsto b_0+\beta_1b_1+\beta_1{}^2b_2/b\in\mathbb{Z}_p[\beta_1]\]
  % となる.
  % \begin{align*}
  %   \phi(1\otimes_\mathbb{Z}b_0+\beta_1\otimes_\mathbb{Z}b_1+\beta_2\otimes_\mathbb{Z}b_2) &= b_0+\beta_1b_1+\beta_1{}^2b_2/b\\
  %   \phi(1\otimes_\mathbb{Z}b_0'+\beta_1\otimes_\mathbb{Z}b_1'+\beta_2\otimes_\mathbb{Z}b_2') &= b_0'+\beta_1b_1'+\beta_1{}^2b_2'/b
  % \end{align*}
  % とする.
  % $\beta_1{}^2=b\beta_2$,$\beta_2{}^2=a\beta_1$,$\beta_1\beta_2=ab$から
  % \begin{align*}
  %   & (1\otimes_\mathbb{Z}b_0+\beta_1\otimes_\mathbb{Z}b_1+\beta_2\otimes_\mathbb{Z}b_2)(1\otimes_\mathbb{Z}b_0'+\beta_1\otimes_\mathbb{Z}b_1'+\beta_2\otimes_\mathbb{Z}b_2')\\
  %   & = 1\otimes_\mathbb{Z}(b_0b_0'+abb_1b_2'+abb_2b_1')+\beta_1\otimes_\mathbb{Z}(b_0b_1'+b_1b_0'+ab_2b_2')+\beta_2\otimes_\mathbb{Z}(b_0b_2'+b_2b_0'+bb_1b_1')
  % \end{align*}
  % なのでこれを$\phi$で写せば
  % \[(b_0b_0'+abb_1b_2'+abb_2b_1')+\beta_1(b_0b_1'+b_1b_0'+bb_2b_2')+\beta_1{}^2(b_0b_2'/b+b_2b_0'/b+b_1b_1').\]
  % $\beta_1{}^3=ab^2$から
  % \begin{align*}
  %   &\phi(1\otimes_\mathbb{Z}b_0+\beta_1\otimes_\mathbb{Z}b_1+\beta_2\otimes_\mathbb{Z}b_2) \phi(1\otimes_\mathbb{Z}b_0'+\beta_1\otimes_\mathbb{Z}b_1'+\beta_2\otimes_\mathbb{Z}b_2')\\
  %   &= (b_0+\beta_1b_1+\beta_1{}^2b_2/b)(b_0'+\beta_1b_1'+\beta_1{}^2b_2'/b)\\
  %   &= (b_0b_0'+abb_1b_2'+abb_2b_1')+\beta_1(b_0b_1'+b_1b_0'+ab_2b_2')+\beta_1{}^2(b_0b_2'/b+b_2b_0'/b+b_1b_1')
  % \end{align*}
  % なので$\phi$が積を保つことも分かる.よって,$\phi$は環準同型.$\phi$は明らかに全射.$\ker\phi=\{0\}$なので単射.よって$\phi\colon V\otimes_\mathbb{Z}\mathbb{Z}_p\simeq\mathbb{Z}_p[\beta_1]$.
  $\phi$が$\mathbb{Z}_p$代数の同型であることは容易に分かる.
  定理1.3.23 (2)から同型
  \[ \mathcal{O}_k \otimes_\mathbb{Z} \mathbb{Z}_p \ni v \otimes c \mapsto \sigma(v)c \in \widehat{\mathcal{O}}_K = \mathbb{Z}_p[\beta_1] \]
  の存在が分かるので,$\mathcal{O}_K \otimes_\mathbb{Z} \mathbb{Z}_p = V \otimes_\mathbb{Z} \mathbb{Z}_p$.
  % 定理1.3.23 (2)から同型
  % \[\mathbb{Z}_p[\beta_1]\ni b_0+\beta_1b_1+\beta_1{}^2b_2/b\mapsto1\otimes_\mathbb{Z}b_0+\beta_1\otimes_\mathbb{Z}b_1+\beta_1{}^2\otimes_\mathbb{Z}b_2/b\in\mathcal{O}_K\otimes_\mathbb{Z}\mathbb{Z}_p\]
  % が存在する.上の同型と合わせて,同型
  % \[V\otimes_\mathbb{Z}\mathbb{Z}_p\ni1\otimes_\mathbb{Z}b_0+\beta_1\otimes_\mathbb{Z}b_1+\beta_2\otimes_\mathbb{Z}b_2\mapsto1\otimes_\mathbb{Z}b_0+\beta_1\otimes_\mathbb{Z}b_1+\beta_1{}^2\otimes_\mathbb{Z}b_2/b\in\mathcal{O}_K\otimes_\mathbb{Z}\mathbb{Z}_p\]
  % が得られる.$\beta_2\otimes_\mathbb{Z}b_2=\beta_1{}^2/b\otimes_\mathbb{Z}b_2/b\cdot b=\beta_1{}^2\otimes_\mathbb{Z}b_2$なので,これは$V\otimes_\mathbb{Z}\mathbb{Z}_p$と$\mathcal{O}_K\otimes_\mathbb{Z}\mathbb{Z}_p$の間の恒等写像.よって,$V\otimes_\mathbb{Z}\mathbb{Z}_p=\mathcal{O}_K\otimes_\mathbb{Z}\mathbb{Z}_p$.
\end{proof}

\begin{screen}
  $a, b$を平方因子を持たない互いに素な整数,$\beta_1=\sqrt[3]{ab^2}$,$\beta_2=\sqrt[3]{a^2b}$,$K=\mathbb{Q}(\beta_1)$,
  $V=\mathbb{Z}+\mathbb{Z}\beta_1+\mathbb{Z}\beta_2$,$p\nmid3, a, b$なら$p$は$K$で不分岐で$\mathcal{O}_K\otimes_\mathbb{Z}\mathbb{Z}_p = V\otimes_\mathbb{Z}\mathbb{Z}_p$
  % の証明.(p.111上の方)
\end{screen}
\begin{proof}
  命題2.2.11から$\mathbb{Z}_{(p)}[\beta_1]=\mathcal{O}_{K_{(p)}}\simeq\mathcal{O}_K\otimes_\mathbb{Z}\mathbb{Z}_{(p)}$.
  よって,
  \[\mathcal{O}_K\otimes_\mathbb{Z}\mathbb{Z}_p\simeq\mathcal{O}_K\otimes_\mathbb{Z}\mathbb{Z}_{(p)}\otimes_{\mathbb{Z}_{(p)}}\mathbb{Z}_p\simeq\mathbb{Z}_{(p)}[\beta_1]\otimes_{\mathbb{Z}_{(p)}}\mathbb{Z}_p\simeq\mathbb{Z}_p[\beta_1].\]
  これは$v \otimes c \mapsto \sigma(v)c$で与えられるので,先程と同様に$\mathcal{O}_K\otimes_\mathbb{Z}\mathbb{Z}_p = V\otimes_\mathbb{Z}\mathbb{Z}_p$.
  % $b\in\mathbb{Z}_p^\times$なので,上と同様にして$\mathbb{Z}_p[\beta_1]\simeq V\otimes_\mathbb{Z}\mathbb{Z}_p$となり示された.
  % (上と同様に恒等写像による同型で$\mathcal{O}_K\otimes_\mathbb{Z}\mathbb{Z}_p=V\otimes_\mathbb{Z}\mathbb{Z}_p$)
\end{proof}

\begin{screen}
  $\alpha_1{}^2 \equiv 1 \bmod 9$の場合.$\lambda$の行き先
\end{screen}
\begin{proof}
  $f(x) = (x - \gamma)(x + \gamma x + \gamma^2)$である.例2.3.5と同様にすれば,次の写像が得られる.
  \[
  \begin{tikzcd}
    K \arrow[r, hookrightarrow]\arrow[d, phantom, "\ni" sloped] & K \otimes_\mathbb{Q} \mathbb{Q}_3 \arrow[r, "\simeq"]\arrow[d, phantom, "\ni" sloped] & \mathbb{Q}_3 \times \mathbb{Q}_3(\omega\gamma) \arrow[d, phantom, "\ni" sloped] \\
    a_0 + a_1 \beta_1 + a_2 \beta_1{}^2 \arrow[r, mapsto] & (a_0 + a_1 \beta_1 + a_2 \beta_1{}^2) \otimes 1 \arrow[r, mapsto] & (a_0 + a_1 \gamma + a_2 \gamma^2, a_0 + a_1 \omega\gamma + a_2 \omega^2\gamma^2)
  \end{tikzcd}
  \]
  これを整数環に制限すれば,$\phi \colon \mathcal{O}_K \hookrightarrow \mathcal{O}_K \otimes_\mathbb{Z} \mathbb{Z}_3 \simeq \mathbb{Z}_3 \times \mathbb{Z}_3[\omega\gamma]$である.
\end{proof}

\section{$\mathbb{Q}_p$の$2$次拡大}
\paragraph{命題2.4.2}~
\begin{screen}
  $K=\mathbb{Q}_2(\sqrt{3})$,$\alpha=\sqrt{3}-1$として,$\mathcal{O}_K=\mathbb{Z}_2[\alpha]$
\end{screen}
\begin{proof}
  $\mathcal{O}_K$の素イデアルを$P$とする.
  $\alpha$の$\mathbb{Q}_2$上最小多項式はEisenstein多項式なので,$\alpha$が$\mathcal{O}_K$の素元であることを証明すれば良い(命題1.10.7).
  $K/\mathbb{Q}_2$は完全分岐で$e(P/2\mathbb{Z}_2)=2$(命題1.10.7)なので$\ord_P(2)=2\ord_2(2)=2$.
  $\ord_P(\alpha)=0$とする.命題1.1.3から$\ord_P(\alpha+2)=0$なので$\ord_P(\alpha(\alpha+2))=0$.
  しかし,$\alpha(\alpha+2)=2$なので,$\ord_P(\alpha(\alpha+2))=2$.よって矛盾.
  $\ord_P(\alpha)\geq2$とする.$\ord_P(\alpha+2)\geq2$なので$\ord_P(\alpha(\alpha+2))\geq4$.
  先程と同様に矛盾.よって,$\ord_P(\alpha)=1$となり示された.
\end{proof}

\section{$\mathbb{Q}$の双$2$次拡大}
\paragraph{命題2.5.2}~
\begin{screen}
  $K=\mathbb{Q}(\sqrt{a}, \sqrt{b})$を双$2$次拡大,$M=\mathbb{Q}(\sqrt{a})$,$N=\mathbb{Q}(\sqrt{b})$,$p$が$N/\mathbb{Q}$で不分岐とする.
  $\mathcal{O}_K \otimes_\mathbb{Z} \mathbb{Z}_p$は$\mathbb{Z}_p$上$Q(\sqrt{a})$, $Q(\sqrt{b})$の整数環で生成され,
  $\varDelta_{K/\mathbb{Q}, p}= \varDelta_{\mathbb{Q}(\sqrt{a})/\mathbb{Q}, p}{}^2 \varDelta_{\mathbb{Q}(\sqrt{b})/\mathbb{Q}, p}{}^2$
\end{screen}
\begin{proof}
  $S=\mathbb{Z}\setminus p\mathbb{Z}\subset\mathbb{Z}_p^\times$とする.命題1.3.7から$p$は$S^{-1}\mathcal{O}_N/S^{-1}\mathbb{Z}$でも不分岐.
  \[
  \begin{tikzcd}
    &[-40pt] K=\mathbb{Q}(\sqrt{a}, \sqrt{b}) \arrow[ld, dash]\arrow[rd, dash] \\
    M=\mathbb{Q}(\sqrt{a}) &[-40pt] &[-40pt] N=\mathbb{Q}(\sqrt{b})\\
    &[-40pt] \mathbb{Q} \arrow[ul, dash]\arrow[ur, dash, "\text{unr}" description]
  \end{tikzcd}
  \]

  $\mathbb{Z}_{(p)} = S^{-1}\mathbb{Z}$に対して命題1.11.14の適用を考えよう.
  命題1.11.14 (2)から$S^{-1}\mathcal{O}_K$は$S^{-1}\mathcal{O}_N, S^{-1}\mathcal{O}_M$で生成される.
  定理1.11.16 (2)の証明と同様に,$S^{-1}\mathcal{O}_N$の$S^{-1}\mathbb{Z}$基底を$\{v_1, v_2\}$,
  $S^{-1}\mathcal{O}_M$の$S^{-1}\mathbb{Z}$基底を$\{w_1, w_2\}$とすれば,$S^{-1}\mathcal{O}_K$の$S^{-1}\mathbb{Z}$基底は$\{v_1w_1, \ldots, v_2w_2\}$.
  同定理(4)の証明と同様に$\varDelta_{K/\mathbb{Q}, p}= \varDelta_{N/\mathbb{Q}, p}{}^2\varDelta_{M/\mathbb{Q}, p}{}^2$となる.

  $\forall x\in S^{-1}\mathcal{O}_K$に対し,$x_{ij}\in S^{-1}\mathbb{Z}$が存在し,$x=\sum_{ij}x_{ij}v_iw_j$と表すことができる.
  $y\in\mathbb{Z}_p$として,
  \[S^{-1}\mathcal{O}_K\otimes_\mathbb{Z}\mathbb{Z}_p\ni x \otimes y = \sum_{ij}(x_{ij} \otimes y)(v_iw_j \otimes 1)\]
  と表すことができる.
  $x=a/s\ (a\in\mathcal{O}_K, s\in S)$,
  $v_i=\tilde{v}_i/s_i\ (\tilde{v}_i\in\mathcal{O}_M, s_i\in S)$,
  $w_j=\tilde{w}_j/t_j\ (\tilde{w}_j\in\mathcal{O}_N, t_j\in S)$として,
  \[\mathcal{O}_K\otimes_\mathbb{Z}\mathbb{Z}_p\ni a \otimes y/s=\sum_{ij}(1 \otimes x_{ij}y)(\tilde{v}_i\tilde{w}_j \otimes 1/s_it_j)\]
  となる.これは,$\mathcal{O}_K\otimes_\mathbb{Z}\mathbb{Z}_p$の$\mathbb{Z}\otimes_\mathbb{Z}\mathbb{Z}_p$基底である.
  同様に,$\mathcal{O}_M\otimes_\mathbb{Z}\mathbb{Z}_p$の$\mathbb{Z}\otimes_\mathbb{Z}\mathbb{Z}_p$基底として$\{\tilde{v}_i \otimes 1/s_i\}$,
  $\mathcal{O}_N\otimes_\mathbb{Z}\mathbb{Z}_p$の$\mathbb{Z}\otimes_\mathbb{Z}\mathbb{Z}_p$基底として$\{\tilde{w}_j \otimes 1/t_j\}$が取れる.

  $\varDelta_{M/\mathbb{Q}}(1, \sqrt{a})=4a$なので,$\{1, \sqrt{a}\}$は$S^{-1}\mathcal{O}_M$の$S^{-1}\mathbb{Z}$基底(補題1.8.3).
  $S^{-1}\mathcal{O}_N$の$S^{-1}\mathbb{Z}$基底は$\{1, \sqrt{b}\}$となる.
  この場合は,上での$s_i, t_j$は$1$として良い.

  命題2.5.4 (1)なら$\{1, (1+\sqrt{a})/2\}$が$S^{-1}\mathcal{O}_M$の$S^{-1}\mathbb{Z}$基底になる.
\end{proof}

\section{$4$次巡回体}
\paragraph{命題2.6.4}
$\{1, \alpha, \sqrt{d}, \beta\}$で生成される$\mathbb{Z}$加群を$V$とする.
$\varDelta_{K/\mathbb{Q}}(1, \alpha, \sqrt{d}, \beta)=(\mathcal{O}_K:V)^2\varDelta_K=2^8d^3$が成立する(系1.7.5 (2)).
$\forall p\mid d$に対し,$\varDelta{K/\mathbb{Q}, p}=(p^3)$なので$p^3\mid\varDelta_K$.
よって$(\mathcal{O}_K:V)^2$は$2$の冪乗.
命題1.8.9から$(\mathcal{O}_K:V)=\prod_p(\mathcal{O}_K\otimes_\mathbb{Z}\mathbb{Z}_p:V\otimes_\mathbb{Z}\mathbb{Z}_p)$となるが,
これが$2$の冪乗なので奇素数$p$に対しては$(\mathcal{O}_K\otimes_\mathbb{Z}\mathbb{Z}_p:V\otimes_\mathbb{Z}\mathbb{Z}_p)=1$.
