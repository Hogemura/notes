\chapter{Minkowskiの定理とその応用}
\setcounter{section}{1}
\section{判別式の評価と類数の有限性}
\paragraph{例3.2.13}
素数$p$が$p\mathcal{O}_K=P_1\cdots P_g$と素イデアル分解できたとする.
命題1.10.15から$\mathcal{N}((p))=\mathcal{N}(P_1\cdots P_g)=\mathcal{N}(P_1)\cdots\mathcal{N}(P_g)$.
命題1.10.17(2)から$\mathcal{N}((p))=\N_{K/\mathbb{Q}}(p)=p^4$なので,$\mathcal{N}(P_i)$は$p$の冪乗.
$\mathcal{O}_K$の素イデアル$P$の下にある素数$p$は一意的に定まるので,今の議論から,$\mathcal{N}(P)$は素数$p$の冪乗であり,$p$の上にある.
よって,$\mathcal{O}_K$の素イデアル$P$が$\mathcal{N}(P)=2$を満たすならば$2\mathbb{Z}$の上にある.

\paragraph{例3.2.14}
素数$p$について,命題1.10.17(2)から$\mathcal{N}((p))=\lvert\N_{K/\mathbb{Q}}(p)\rvert=p^2$.
$(p)=\mathfrak{p}_1\cdots\mathfrak{p_t}$と素イデアル分解されるとする.
命題1.10.15から$\mathcal{N}((p))=\mathcal{N}(\mathfrak{p}_1)\cdots\mathcal{N}(\mathfrak{p}_t)$.
よって,$t=2$で$\mathcal{N}(\mathfrak{p}_1)=\mathcal{N}(\mathfrak{p}_2)=p$.
よって,ノルム$2$の素イデアルは$(2)$の素イデアル分解に現れる物だけ.

$\mathcal{O}_K=\mathbb{Z}[\alpha]$のイデアル$(2)$を素イデアル分解する.準同型
\[ \mathcal{O}_K\ni a+b\alpha \mapsto \overline{a}+\overline{b}\alpha+(2)\in\mathcal{O}_K/(2) \]
は全射.$\overline{a},\overline{b}$を$a,b$を$2$で割った余りとする.準同型
\begin{align*}
  \mathcal{O}_K/(2)\ni\overline{a}+\overline{b}\alpha+(2) &\mapsto \overline{a}+\overline{b}x+(x(x+1))\in\mathbb{F}_2[x]/(x(x+1))\\
  &\mapsto(\overline{a}+\overline{b}x+(x),\overline{a}+\overline{b}x+(x+1))\in\mathbb{F}_2[x]/(x)\times\mathbb{F}_2[x]/(x+1)\\
  &\mapsto(\overline{a},\overline{a}-\overline{b})\in\mathbb{F}_2\times\mathbb{F}_2
\end{align*}
は同型.
従って,$(2)=\mathfrak{p}_1\mathfrak{p}_2$となる.
特に,$\mathfrak{p}_1,\mathfrak{p}_2$は$2$の上にある.中国式剰余定理から,
\[\mathcal{O}_K/(2)\ni\overline{a} + \overline{b}\alpha+(2) \mapsto (\overline{a} + \overline{b}\alpha + \mathfrak{p}_1, \overline{a} + \overline{b}\alpha + \mathfrak{p}_2) \in\mathcal{O}_K/\mathfrak{p}_1 \times \mathcal{O}_K/\mathfrak{p}_2\]
は同型.以上をまとめて,
\[
\begin{tikzcd}
  \mathcal{O}_K \arrow[r, twoheadrightarrow]\arrow[d, phantom, "\ni" sloped]
  & \mathcal{O}_K/\mathfrak{p}_1\times\mathcal{O}_K/\mathfrak{p}_2 \arrow[r, equal]\arrow[d, phantom, "\ni" sloped]
  & \mathcal{O}_K/\mathfrak{p}_1\times\mathcal{O}_K/\mathfrak{p}_2 \arrow[r, "\simeq"]\arrow[d, phantom, "\ni" sloped]
  & \mathbb{F}_2\times\mathbb{F}_2 \arrow[d, phantom, "\ni" sloped]\\
  a+b\alpha \arrow[r, mapsto] & (a+b\alpha+\mathfrak{p}_1,a+b\alpha+\mathfrak{p}_2) \arrow[r, equal]
  & (\overline{a}+\overline{b}\alpha+\mathfrak{p}_1,\overline{a}+\overline{b}\alpha+\mathfrak{p}_2) \arrow[r, mapsto]
  & (\overline{a},\overline{a}-\overline{b})
\end{tikzcd}
\]
となる.最後の同型を$\phi\colon\mathcal{O}_K/\mathfrak{p}_1\times\mathcal{O}_K/\mathfrak{p}_2\to\mathbb{F}_2\times\mathbb{F}_2$とする.
$\phi$が同型なことに注意して,
\[a+b\alpha\in\mathfrak{p}_1\Leftrightarrow(0,\bullet)\in\mathcal{O}_K/\mathfrak{p}_1\times\mathcal{O}_K/\mathfrak{p}_2\Leftrightarrow(0,\bullet)\in\mathbb{F}_2\times\mathbb{F}_2\Leftrightarrow\overline{a}=0\Leftrightarrow a+b\alpha\in(2,\alpha)\]
なので$\mathfrak{p}_1=(2,\alpha)$.同様に,
\[a+b\alpha\in\mathfrak{p}_2\Leftrightarrow(\bullet,0)\in\mathcal{O}_K/\mathfrak{p}_1\times\mathcal{O}_K/\mathfrak{p}_2\Leftrightarrow(\bullet,0)\in\mathbb{F}_2\times\mathbb{F}_2\Leftrightarrow\overline{a}-\overline{b}=0\Leftrightarrow a+b\alpha\in(2,1+\alpha)\]
なので$\mathfrak{p}_2=(2,1+\alpha)$.

\paragraph{例3.2.22}
% <p style="margin-left: 1em; text-indent: 1em;}
$f(\mathfrak{p}_2/p\mathbb{Z})=1$なので,
$[\mathcal{O}_K/\mathfrak{p}_2:\mathbb{F}_p]=1$.つまり,$\mathcal{O}_K/\mathfrak{p}_2$の完全代表系として$\{0,\ldots,p-1\}$が取れる.
$\alpha\equiv k\bmod\mathfrak{p}_2\ (k\in\{0,\ldots,p-1\})$とする.
$\alpha$が$\mathfrak{p}_2$を法として平方非剰余なので$\alpha\not\equiv l^2\bmod\mathfrak{p}_2 ~ (\forall l\in\{0,\ldots,p-1\})$.
よって$k-l^2\not\in\mathfrak{p}_2$.
$\mathfrak{p}_2$は$p\mathbb{Z}$の上にあり,$k,l\in\mathbb{Z}$なので$k-l^2\not\in p\mathbb{Z}$.つまり$p$を法として$k$は平方非剰余.
$-\alpha$が$\mathfrak{p}_2$を法として平方非剰余なので,同様にして,$p$を法として$-k$は平方非剰余.系I-1.11.5から$-1$は$p$を法として$-1$は平方剰余.

\section*{離散部分群}
Dirichletの単数定理の証明に使う.\footnote{\verb|https://mathematics-pdf.com/pdf/dirichlet_unit_theorem.pdf|から必要な部分を取ってきた}

\begin{dfn}
  $m\in\mathbb{Z}_{>0}$とする.
  $\mathbb{R}^m$の部分集合$S$が離散的とは,任意の$C\in\mathbb{R}_{>0}$に対し
  \[\#\Set{(x_1,\ldots,x_m)\in S\mid\max_{1\leq j\leq m}\lvert x_j\rvert\leq C}<\infty\]
  が成立することである.定義から明らかに,離散的集合の有界部分集合は有限集合となる.
\end{dfn}

\begin{screen}
  \begin{lem}
    \label{discrete_subgroup_lemma_1}
    $\mathbb{R}$の任意の離散的部分群$\Gamma$は$\exists\gamma\in\Gamma$によって$\Gamma=\mathbb{Z}\gamma$となる.
  \end{lem}
\end{screen}
\begin{proof}
  $\Gamma=\{\boldsymbol{0}\}$なら$\gamma=\mathbb{0}$とすればよい.
  $\Gamma\neq\{\boldsymbol{0}\}$とする.$\lvert\gamma_1\rvert>0$となる$\gamma_1\in\Gamma$を取る.
  $\Gamma$は離散的なので,$\lvert x\rvert\leq C$となる$x\in\Gamma$は有限個しかない.
  $\gamma=\min\{\,\lvert x\rvert\mid x\in\Gamma,\quad 0<\lvert x\rvert\leq C\,\}$とおく.
  $\mathbb{Z}\gamma$が$\Gamma$の部分群であることは明らかなので$\mathbb{Z}\gamma\subset\Gamma$.

  $\forall x\in\Gamma$に対し,$q_0=\min\{\,q\in\mathbb{Z}\mid q>0,\quad\lvert x\rvert\leq q\lvert\gamma\rvert\}$が存在する.
  $0\leq q_0\lvert\gamma\rvert-\lvert x\rvert<\lvert\gamma\rvert$は明らか.
  $x,\gamma\in\Gamma$で$-x,-\gamma\in\Gamma$なので$\lvert x\rvert,\lvert\gamma\rvert\in\Gamma$.
  よって$q_0\lvert\gamma\rvert-\lvert x\rvert\in\Gamma$.
  $\gamma$の最小性から,$q_0\lvert\gamma\rvert-\lvert x\rvert=0$.
  よって,$\lvert x\rvert\in\mathbb{Z}\gamma$なので$x=\pm\lvert x\rvert\in\mathbb{Z}\gamma$となり$\Gamma\subset\mathbb{Z}\gamma$.
  以上から$\Gamma=\mathbb{Z}\gamma$.
\end{proof}

\begin{screen}
  \begin{lem}
    \label{discrete_subgroup_lemma_2}
    $\Gamma\neq\{\boldsymbol{0}\}$を$\mathbb{R}^m$の離散的部分群とすれば,$\exists\gamma\in\Gamma\setminus\{\boldsymbol{0}\}$によって$\mathbb{R}\gamma\cap\Gamma=\mathbb{Z}\gamma$となる.
  \end{lem}
\end{screen}
\begin{proof}
  $\gamma_0\in\Gamma\setminus\{\boldsymbol{0}\}$として,$M=\{u\gamma_0\in\mathbb{R}^m\mid0< u\leq1\}$とおくと,$M$は有界集合.
  $\gamma_0\in\Gamma\cap M$なので,$\Gamma\cap M\neq\varnothing$.
  よって,$\{\,u\in\mathbb{R}\mid u\gamma_0\in\Gamma,\quad 0< u\leq1\,\}$も空でない有限集合となり,最小元$u_0$が存在する.
  $\gamma=u_0\gamma_0$とおく.

  $\alpha\in\Gamma\cap\mathbb{R}\gamma$とおくと,$\alpha=u\gamma\ (u\in\mathbb{R})$と表される.
  $\alpha-\lfloor u\rfloor\gamma\in\Gamma$は$(u-\lfloor u\rfloor)u_0\gamma_0$となるが,$0\leq u-\lfloor u\rfloor< 1$なので$0\leq(u-\lfloor u\rfloor)u_0\gamma_0< u_0$.
  $u_0$の最小性から,$u-\lfloor u\rfloor=0$なので,$\alpha=\lfloor u\rfloor\gamma\in\mathbb{Z}\gamma$となり,$\Gamma\cap\mathbb{R}\gamma\subset\mathbb{Z}\gamma$.
  $\Gamma\cap\mathbb{R}\gamma\supset\mathbb{Z}\gamma$は明らかなので,$\Gamma\cap\mathbb{R}\gamma=\mathbb{Z}\gamma$.
\end{proof}


$\Gamma$を$\mathbb{R}^m$の離散的部分群で$\gamma\in\Gamma\setminus\{\boldsymbol{0}\}$とする.
$\boldsymbol{b}_1,\ldots,\boldsymbol{b}_m\in\mathbb{R}^m$を,$\{\gamma,\boldsymbol{b}_2,\ldots,\boldsymbol{b}_m\}$が$\mathbb{R}^m$の$\mathbb{R}$基底になるように取る.
$m\geq2$に対し,$\mathbb{R}$線形写像$L_\gamma$を
\[L_\gamma\colon\mathbb{R}^m\ni x_1\gamma+x_2\boldsymbol{b}_2+\cdots+x_m\boldsymbol{b}_m\mapsto(x_2,\ldots,x_m)\in\mathbb{R}^{m-1}\]
とおく.

\begin{screen}
  \begin{lem}
    \label{discrete_subgroup_lemma_3}
    $\gamma\in\Gamma\setminus\{\boldsymbol{0}\}$に対し,$\ker L_\gamma=\mathbb{R}\gamma$.
  \end{lem}
\end{screen}
\begin{proof}
  $\alpha\in\ker L_\gamma$とし,$x_1,\ldots,x_m\in\mathbb{R}$によって$\alpha=x_1\gamma+x_2\boldsymbol{b}_2+\cdots+x_m\boldsymbol{b}_m$とすれば,$(x_2,\ldots,x_m)=\boldsymbol{0}$.
  つまり,$\alpha=x_1\gamma\in\mathbb{R}\gamma$なので,$\ker L_\gamma\subset\mathbb{R}\gamma$.
  $\ker L_\gamma\supset\mathbb{R}\gamma$は明らかなので,$\ker L_\gamma=\mathbb{R}\gamma$.
\end{proof}

\begin{screen}
  \begin{lem}
    \label{discrete_subgroup_lemma_4}
    $\forall\gamma\in\Gamma\setminus\{\boldsymbol{0}\}$に対し,$\Gamma'=L_\gamma(\Gamma)$は$\mathbb{R}^{m-1}$の離散的部分群.
  \end{lem}
\end{screen}
\begin{proof}
  $L_\gamma$は線形写像なので,$\mathbb{R}^m$から$\mathbb{R}^{m-1}$への(加群の)準同型.よって$\Gamma'$は$\mathbb{R}^{m-1}$の部分群.
  $C\in\mathbb{R}_{>0}$とする.$(x_2,\ldots,x_m)\in\Gamma'$とし,
  \begin{align}
    \max_{2\leq j\leq m}\lvert x_j\rvert\leq C\label{lem4eq1}
  \end{align}
  であるとする.$\Gamma'=L_\gamma(\Gamma)$なので,$\exists x_1\in\mathbb{R}$によって$x_1\gamma+x_2\boldsymbol{b}_2+\cdots+x_m\boldsymbol{b}_m\in\Gamma$となる.
  $\gamma\in\Gamma$なので
  \[(x_1-\lfloor x_1\rfloor)\gamma+x_2\boldsymbol{b}_2+\cdots+x_m\boldsymbol{b}_m=(x_1\gamma+x_2\boldsymbol{b}_2+\cdots+x_m\boldsymbol{b}_m)-\lfloor x_1\rfloor\gamma\in\Gamma.\]
  $0\leq x_1-\lfloor x_1\rfloor< 1$で,$(x_1-\lfloor x_1\rfloor)\gamma+x_2\boldsymbol{b}_2+\cdots+x_m\boldsymbol{b}_m$の$L_\gamma$による像は$(x_2,\ldots,x_m)$.
  よって,\eqref{lem4eq1}を満たす$(x_2,\ldots,x_m)$の個数は,
  \begin{align}
    x_1\gamma+x_2\boldsymbol{b}_2+\cdots+x_m\boldsymbol{b}_m\ (0\leq x_1< 1)\label{lem4eq2}
  \end{align}
  となる$\Gamma$の元の個数以下.

  一方,$\gamma,\boldsymbol{b}_2,\ldots,\boldsymbol{b}_m\in\mathbb{R}^m$の各成分の絶対値の最大値を$C'$とすれば,\eqref{lem4eq2}の形の元を$(y_1,\ldots,y_m)$で表すとき,$\lvert y_j\rvert\ (1\leq j\leq m)$の最大値は$C'+(m-1)CC'$以下.
  $\Gamma$は離散的なので,\eqref{lem4eq2}の形の元は有限個しかない.
  したがって,\eqref{lem4eq1}を満たす$(x_2,\ldots,x_m)\in\Gamma'$も有限個,つまり$\Gamma'$は離散的.
\end{proof}

\begin{screen}
  \begin{thm}
    \label{discrete_subgroup_theorem}
    $\mathbb{R}^m$の任意の離散的部分群$\Gamma$に対し,$\Gamma\simeq\mathbb{Z}^{\oplus s}\ (s\leq m).$
  \end{thm}
\end{screen}
\begin{proof}
  $\Gamma=\{\boldsymbol{0}\}$なら明らかなので,$\Gamma\neq\{\boldsymbol{0}\}$の場合について,$\Gamma$の$\mathbb{Z}$基底$\{\gamma_1,\ldots,\gamma_s\}$の存在を$m$に関する数学的帰納法によって証明する.

  $m=1$の時は追加補題\ref{discrete_subgroup_lemma_1}から従う.

  $m\geq2$とし,$\mathbb{R}^{m-1}$の任意の離散的部分群に対し,定理の主張が成立すると仮定する.
  追加補題\ref{discrete_subgroup_lemma_2}から,$\mathbb{R}\gamma_1\cap\Gamma=\mathbb{Z}\gamma_1$を満たす$\gamma_1\in\Gamma\setminus\{\boldsymbol{0}\}$が存在する.
  $L=L_{\gamma_1}$,$\Gamma'=L(\Gamma)$とおく.追加補題\ref{discrete_subgroup_lemma_4}から,$\Gamma'$は$\mathbb{R}^{m-1}$の離散的部分群.
  よって,仮定から,$\Gamma'$の$\mathbb{Z}$基底$\{\gamma_2',\ldots,\gamma_s'\}~ (s\leq m)$が存在する.
  $\Gamma'=L(\Gamma)$なので$L(\gamma_j)=\gamma_j'$となる$\gamma_1,\ldots,\gamma_s\in\Gamma$が存在する.
  $n_1,\ldots,n_s\in\mathbb{Z}$とし,$n_1\gamma_1+\cdots+n_s\gamma_s=\boldsymbol{0}$と仮定する.
  この時,$n_2\gamma_2'+\cdots+n_s\gamma_s'=L(n_1\gamma_1+\cdots+n_s\gamma_s)=\boldsymbol{0}$.
  $\{\gamma_2',\ldots,\gamma_s'\}$は$\Gamma'$の$\mathbb{Z}$基底なので,$n_2=\cdots=n_s=0$.
  さらに,$\gamma_1\neq\boldsymbol{0}$なので,$n_1=0$.
  よって,$\{\gamma_1,\ldots,\gamma_s\}$は$\mathbb{Z}$上一次独立.

  $\alpha\in\Gamma$とすると,$L(\alpha)\in\Gamma'$なので$\exists n_2,\ldots,n_s\in\mathbb{Z}$によって$L=n_2\gamma_2'+\cdots+n_s\gamma_s'=L(n_2\gamma_2+\cdots+n_s\gamma_s)$.
  よって,$L(\alpha-(n_2\gamma_2+\cdots+n_s\gamma_s))=\boldsymbol{0}$,つまり$\alpha-(n_2\gamma_2+\cdots+n_s\gamma_s)\in\ker L\cap\Gamma$.
  追加補題\ref{discrete_subgroup_lemma_3}から,$\ker L=\mathbb{R}\gamma_1$なので,$\ker L\cap\Gamma=\mathbb{R}\gamma_1\cap\Gamma=\mathbb{Z}\gamma_1$(追加補題\ref{discrete_subgroup_lemma_3}).
  よって,$\exists n_1\in\mathbb{Z}$によって$\alpha-(n_2\gamma_2+\cdots+n_s\gamma_s)=n_1\gamma_1$,つまり$\alpha=n_1\gamma_1+\cdots+n_s\gamma_s$.
  よって,$\{\gamma_1,\ldots,\gamma_s\}$は$\Gamma$の$\mathbb{Z}$基底.
\end{proof}

\begin{screen}
  \begin{lem}
    \label{place_restricted_is_finite}
    $\forall i=1,\ldots,n$に対し$\lvert\sigma_i(\alpha)\rvert< C$となるような$\alpha\in\mathcal{O}_K$は有限個しかない
  \end{lem}
\end{screen}
\begin{proof}
  $\{w_1,\ldots,w_n\}$を$\mathcal{O}_K$の整基底とする.
  $\forall\alpha\in\mathcal{O}_K$は$x_1,\ldots,x_n\in\mathbb{Z}$によって$\alpha=x_1w_1+\cdots+x_nw_n$と表すことができる.
  \[P=
  \begin{pmatrix}
    \sigma_1(w_1) & \cdots & \sigma_1(w_n)\\
    \vdots & \ddots & \vdots \\
    \sigma_n(w_1) & \cdots & \sigma_n(w_n)\\
  \end{pmatrix}
  ,\quad\boldsymbol\alpha=
  \begin{pmatrix}
    \sigma_1(\alpha)\\
    \vdots\\
    \sigma_1(\alpha)\\
  \end{pmatrix}
  ,\quad\boldsymbol x=
  \begin{pmatrix}
    x_1\\
    \vdots\\
    x_n\\
  \end{pmatrix}
  \]
  とすれば,$\boldsymbol\alpha=P\boldsymbol x$.
  $P$は正則なので,$\boldsymbol x=P^{-1}\boldsymbol\alpha$.
  $P^{-1}=(p_{ij})$とおき,$C_1$を$p_{ij}$のうち最大のものとする.
  $i=1,\ldots,n$に対して
  \[\lvert x_i\rvert=\lvert p_{i1}\sigma_1(\alpha)+\cdots+p_{in}\sigma_n(\alpha)\rvert\leq\lvert p_{i1}\rvert\lvert\sigma_1(\alpha)\rvert+\cdots+\lvert p_{in}\rvert\lvert\sigma_n(\alpha)\rvert\leq nc_1C\]
  なので,このような$x_i$は有限個しか存在しない.
\end{proof}

\section{Dirichletの単数定理}
\paragraph{定理3.3.1}
$\phi$の構成.
$\phi\colon\mathcal{O}_K^\times\to H$は積を和にする:$\phi(\varepsilon_1\varepsilon_2)=\phi(\varepsilon_1)+\phi(\varepsilon_2)$.
$\mathcal{O}_K^\times$は乗法群で$H$は$\mathbb{R}$加群.

p.163の真ん中らへんの段落から.
$r=r_1+r_2-1\leq1$とする.
$\phi(\varepsilon_1),\ldots,\phi(\varepsilon_r)$が一次独立な条件の元で$\phi(\mathcal{O}_K^\times)$が$\mathbb{R}^r$の離散的部分群であることを証明する.
$\phi(\mathcal{O}_K^\times)$が$\mathbb{R}^r$の部分加群なのは明らか.
$C\in\mathbb{R}_{>0}$を考える.$\varepsilon\in\mathcal{O}_K$を$\lvert\log\sigma_i(\varepsilon)\rvert\ (1\leq j\leq r)$の最大値が$C$以下になるように取る.
この条件は$i=1,\ldots,r$に対し$e^{-C}\leq\lvert\sigma_i(\varepsilon)\leq e^C$が成立することと同値.
$r_2>0$の時,$j=1,\ldots,r_2$に対し$\lvert\sigma_{r_1+r_2+j}(\varepsilon)\rvert=\lvert\sigma_{r_1+j}(\varepsilon)\rvert$なので,$i\neq r_1+r_2,n$に対し$e^{-C}\leq\lvert\sigma_i(\varepsilon)\leq e^C$が成立する.
$1=\left\lvert\N_{K/\mathbb{Q}}(\varepsilon)\right\rvert=\prod_{i=1}^n\lvert\sigma_i(\varepsilon)\rvert$なので,
\[\lvert\sigma_{r_1+r_2}(\varepsilon)\rvert\lvert\sigma_n(\varepsilon)\rvert=\prod_{\substack{1\leq j\leq n\\ i\neq r_1+r_2,n}}\frac{1}{\lvert\sigma_i(\varepsilon)\rvert}< e^{(n-2)C}.\]
よって,$\lvert\sigma_{r_1+r_2}(\varepsilon)\rvert=\lvert\sigma_n(\varepsilon)\rvert< e^{(n-2)C/2}$.
したがって,$r_2>0$なら$i=1,\ldots,r$に対して$\lvert\sigma_i(\varepsilon)\rvert< e^{C}$が成立.

$r_2=0$なら$r=r_1-1=n-1$なので$i=1,\ldots,n-1$に対して$e^{-C}\leq\lvert\sigma_i(\varepsilon)\leq e^C$.
$1=\left\lvert\N_{K/\mathbb{Q}}(\varepsilon)\right\rvert=\prod_{i=1}^n\lvert\sigma_i(\varepsilon)\rvert$なので,
\[\lvert\sigma_n(\varepsilon)\rvert=\prod_{i=1}^{n-1}\frac{1}{\lvert\sigma_i(\varepsilon)\rvert}< e^{(n-1)C}.\]
よって$r_2=0$でも$i=1,\ldots,r$に対して$\lvert\sigma_i(\varepsilon)\rvert< e^{C}$が成立.

$\forall i=1,\ldots,n$に対し$\lvert\sigma_i(\alpha)\rvert< e^C$を満たす$\mathcal{O}_K$の元は有限個しかない(追加補題\ref{place_restricted_is_finite}).
よって,$\phi(\mathcal{O}_K^\times)$は離散的.
よって,離散部分群の追加定理\ref{discrete_subgroup_theorem}から,$\phi(\mathcal{O}_K^\times)\simeq\mathbb{Z}^{\oplus s}\ (s\leq r)$.
また,$\{\phi(\varepsilon_1),\ldots,\phi(\varepsilon_r)\}$は$\mathbb{R}$上一次独立なので,$\mathbb{Z}$上一次独立.
よって,$s=r$となり,$\phi(\mathcal{O}_K^\times)\simeq\mathbb{Z}^{\oplus r}$.

また,$K$に含まれる$1$の冪根を$R_K$とすれば,$\ker\phi=R_K$なので,$\mathcal{O}_K^\times=\mathbb{Z}^{\oplus r}\oplus R_K$.
つまり,$\mathcal{O}_K^\times$の元は$1$の冪根と基本単数$\varepsilon_1,\ldots,\varepsilon_r$の冪乗の積の形になる.

\chapter{円分体}
\section{円分体の整数環II}

\begin{screen}
  \begin{lem}
    \label{disc_diff}
    判別式について
  \end{lem}
\end{screen}
\begin{proof}
  $f(x)$の根を$\alpha_1,\ldots,\alpha_n$とする.
  $f(x)=(x-\alpha_1)\cdots(x-\alpha_n)$となる.
  \[f'(\alpha_i)=(\alpha_i-\alpha_1)\cdots(\alpha_i-\alpha_{i-1})(\alpha_i-\alpha_{i+1})\cdots(\alpha_i-\alpha_n)\]
  なので,
  \[\varDelta(f)=\prod_{i< j}(\alpha_i-\alpha_j)^2=\prod_{i\neq j}(-1)^{n(n-1)/2}(\alpha_i-\alpha_j)=\prod_{i=1}^nf'(\alpha_i).\]
\end{proof}

\paragraph{命題4.1.1}~
\begin{screen}
  $f_k(x)$の判別式は$\text{単数} \times \prod (\zeta_{p^k}{}^{i-j} - 1)$(ただし,$0 < i \neq j < p^k$, $p \nmid i, j$)
  % p.172下らへんの式変形
\end{screen}
\begin{proof}
  \[S=\Set{i | 0< i< p^k, \gcd(i,p)=1 } = \Set{i+jp | 1\leq i\leq p-1; 0\leq j\leq p^{k-1}-1}\]
  とおく.$\#S=(p-1)p^{k-1}$.
  $f_k(x)$の判別式は
  \begin{align*}
    \prod_{i< j\in S}\left(\zeta_{p^k}{}^i-\zeta_{p^k}{}^j\right)^2 &= (-1)^{\#S(\#S-1)/2}\prod_{i\neq j\in S}\zeta_{p^k}{}^j\left(\zeta_{p^k}{}^{i-j}-1\right) \\
    &= (-1)^{\#S(\#S-1)/2}\prod_{i\neq j\in S}\zeta_{p^k}{}^j\prod_{i\neq j\in S}\left(\zeta_{p^k}{}^{i-j}-1\right).
  \end{align*}
  まず,1つ目の積から考える.
  $\zeta_{p^k}{}^j\ (j\in S)$は$f_k(x)$の異なる根なので,これらの積は$f_k(x)$の定数項と(符号を除いて)等しい.よって,
  \[\zeta:=\prod_{j\in S}\zeta_{p^k}{}^j=(-1)^{\#S}.\]
  また,$\prod_{i\neq j\in S}\zeta_{p^k}{}^j$を計算する際は,$i,j\in S$の条件で積を求めてから,$i=j\in S$の項で割れば良い:
  \[\prod_{i\neq j\in S}\zeta_{p^k}{}^j=\left(\prod_{i=j\in S}\zeta_{p^k}{}^j\right)^{-1}\prod_{i,j\in S}\zeta_{p^k}{}^j=\zeta^{-1}\prod_{i\in S}\prod_{j\in S}\zeta_{p^k}{}^j=\zeta^{-1}\prod_{i\in S}\zeta=\zeta^{\#S-1}=(-1)^{\#S(\#S -1)}=1.\]
  よって,
  \[\varDelta(f_k)=(-1)^{\#S(\#S-1)/2}\prod_{i\neq j\in S}\left(\zeta_{p^k}{}^{i-j}-1\right).\]
\end{proof}

\begin{screen}
  $f_k$の判別式は$p$の冪
\end{screen}
\begin{proof}
  \[f_k(x)=x^{(p-1)p^{k-1}}+x^{(p-2)p^{k-1}}+\cdots+x^{p^{k-1}}+1=\frac{x^{p^k-1}}{x^{p^{k-1}}-1}\]
  なので,
  \[t_i := f'_k(\zeta_{p^k}{}^i)=p^k\frac{\zeta_{p^k}{}^{i(p^k-1)}}{\zeta_{p^k}{}^{ip^{k-1}}-1}=p^k\frac{\zeta_{p^k}{}^{i(p^k-1)}}{\zeta_p{}^i-1}.\]
  $f_k(x)$の判別式は$(-1)^{\#S(\#S-1)/2}\prod_{i\in S}t_i$に等しい(追加補題\ref{disc_diff}).
  まず,$\prod_{i\in S}p^k=p^{k(p-1)p^{k-1}}$.
  $\prod_{i\in S}\zeta_{p^k}{}^{i(p^k-1)}=\zeta^{p^k-1}=(-1)^{(p^k-1)\#S}=1$.
  $\zeta_p{}^i-1\ (i=1,\ldots,p-1)$は$x^{p-1}+px^{p-2}+\cdots+p$の異なる根なので,$\prod_{i=1}^{p-1}(\zeta_p{}^i-1)=(-1)^{p-1}p=p$.
  よって,$\prod_{i\in S}\zeta_p{}^i-1=\prod_{i=1}^{p-1}(\zeta_p{}^i-1)^{p^{k-1}}=p^{p^{k-1}}$.

  以上から,
  \[\varDelta(f_k)=(-1)^{\#S(\#S-1)/2}p^{(kp-k-1)p^{k-1}}.\]
  $\varDelta(f_k)>0$となるのは,$\#S=(p-1)p^{k-1}$が$4$の倍数のとき.
  これは,奇素数$p$に対しては,$p\equiv1\bmod4$のとき.$p=2$に対しては$2^k = 8, 16, \ldots$.
\end{proof}

\setcounter{section}{3}
\section{Kronecker-Weberの定理}
\paragraph{補題4.4.3}~
\begin{screen}
  $\mathbb{Q}(\zeta_{p^n})$の$p$の上にある素イデアル$\mathfrak{p}$による完備化は$\mathbb{Q}_p(\zeta_{p^n})$
\end{screen}
\begin{proof}
  % <!-- p.172のはじめらへんに書いてあった -->
  円分多項式$\Phi_p(x)$の$x^{p-1}$係数は$1$,定数項は$p$.
  $1\leq s\leq p-2$に対し,$x^s$の係数は
  \begin{align*}
    \sum_{k=0}^{p-1-s}\binom{s+k}{s} &= \sum_{k=0}^{p-1-s}\binom{s+k-1}{s-1}+\sum_{k=0}^{p-1-s}\binom{s+k-1}{s}\\
    &= -\binom{p-1}{s-1}+\sum_{k=0}^{p-1-(s-1)}\binom{(s-1)+k}{s-1}+\sum_{k=0}^{p-2-s}\binom{s+k}{s}
  \end{align*}
  であるので,
  \[\sum_{k=0}^{p-1-(s-1)}\binom{(s-1)+k}{s-1}=\binom{p-1}{s-1}+\binom{p-1}{s}=\binom{p}{s}.\]
  この式の左辺が$x^{s-1}$の係数であることに注意すれば,$x^s$の係数は$\binom{p}{s+1}$と分かる.
  また,$s=0,p-1$でもこの式は成立するので,
  \[\Phi_p(x+1)=\sum_{k=0}^{p-1}\binom{p}{s+1}x^s.\]
  特に,$\Phi_p(x+1)$はEisenstein多項式.I-p.282から
  \[\Phi_{p^n}(x) = \frac{x^{p^n}-1}{\Phi_1\cdots\Phi_{p^{n-1}}} = \frac{x^{p^{n-1}}-1}{\Phi_1(x)\cdots\Phi_{p^{n-2}}}\frac{x^{p^n}-1}{(x^{p^{n-1}}-1)\Phi_{p^{n-1}}}
  = \Phi_{p^{n-1}}\frac{\Phi_p(x^{p^{n-1}})}{\Phi_{p^{n-1}}(x)} = \Phi_p(x^{p^{n-1}}).\]
  上の結果と合わせて,$\Phi_{p^n}(x+1)$は$p$についてのEisenstein多項式.
  したがって,$\mathbb{Q}_p$上でも既約であり,$[\mathbb{Q}_p(\zeta_{p^n}):\mathbb{Q}_p]=p^n-p^{n-1}$.

  p.172の議論と同様に進める.
  $N=p^n-p^{n-1}=[\mathbb{Q}(\zeta_{p^n}):\mathbb{Q}]$,$\mathfrak{p}=(\zeta_{p^n}-1)$とおく.
  $p\mathcal{O}=\mathfrak{p}^N$なので,$\mathfrak{p}$は$p$の上にある素イデアルで,$e(\mathfrak{p}/p)=N$で$f(\mathfrak{p}/p)=1$.
  また,$\mathfrak{p}$は$p$の上にある唯一の素イデアル(定理1.3.23).
  $\mathbb{Q}(\zeta_{p^n})$の$\mathfrak{p}$による完備化を$\widehat{\mathbb{Q}}(\zeta_{p^n})$とおく.先程の議論から,
  \[[\widehat{\mathbb{Q}}(\zeta_{p^n}):\mathbb{Q}_p]=N,\quad f(\widehat{\mathbb{Q}}(\zeta_{p^n})/\mathbb{Q}_p)=1,\quad e(\widehat{\mathbb{Q}}(\zeta_{p^n})/\mathbb{Q}_p)=N.\]
  $\widehat{\mathbb{Q}}(\zeta_{p^n})/\mathbb{Q}_p$は$\mathbb{Q}_p$の上にあり,$\zeta_{p^n}$を含むので,$[\widehat{\mathbb{Q}}(\zeta_{p^n}):\mathbb{Q}_p(\zeta_{p^n})]\geq1$.
  上の結果と併せて,$[\widehat{\mathbb{Q}}(\zeta_{p^n}):\mathbb{Q}_p(\zeta_{p^n})]=1$.よって示せた.

  これを使えば,$\mathbb{Q}_p(\zeta_{p^{e_p}})$における$\mathbb{Q}_p$の最大不分岐拡大は$\mathbb{Q}_p$(p.181,2段落目)なので,
  $\mathbb{Q}(\zeta_{p^{e_p}})$における$\mathbb{Q}$の最大不分岐拡大は$\mathbb{Q}$となる.
\end{proof}

\begin{screen}
  $\mathbb{Q}_p(\zeta_{n})/\mathbb{Q}_p$で$p\nmid n$は不分岐
\end{screen}
\begin{proof}
  定理4.1.2(2)から$\mathbb{Q}(\zeta_{n})$の判別式は$p$で割り切れない.
  よって,Dedekindの判別定理から$p$は$\mathbb{Q}(\zeta_{n})/\mathbb{Q}$で不分岐.
  $p$の上にある$\mathbb{Z}[\zeta_n]$の素イデアルで$\mathbb{Q}(\zeta_n)$を完備化した体を$\widehat{K}$をとすれば,$\widehat{K}/\mathbb{Q}_p$で$p$は不分岐(定理1.3.23(6)).
  $\mathbb{Q}_p(\zeta_n)\subset\widehat{K}$なので,$\mathbb{Q}_p(\zeta_n)/\mathbb{Q}_p$でも$p$は不分岐(命題1.3.5(1)).
\end{proof}

\paragraph{定理4.4.2}~
\begin{screen}
  有限次Abel拡大は次数が素数冪の巡回拡大の合成である
\end{screen}
\begin{proof}
  $L/K$を有限次Abel拡大,有限Abel群の基本定理から位数が素数冪の巡回群$C_1,\ldots,C_r$があり,$\Gal(L/K)\simeq C_1\times\cdots\times C_r$.
  $G_i=C_1\times\cdots C_{i-1}\times\{1\}\times C_{i+1}\times\cdots\times C_r$,$G_i$の不変体を$L_i$とする.
  $L_1\cdots L_r$は$L_1,\ldots,L_r$を含む最小の$L$の部分体.
  Galoisの基本定理から,これに対応する$\Gal(L/K)$の部分群は$G_1,\ldots,G_r$に含まれる最大のもの.
  これは$G_1\cap\cdots\cap G_r=\{1_G\}$.従って,$L_1\cdots L_r=L$となる.
  $[L:L_1]=\# G_i=\# C_1\cdots \#C_{i-1}\# C_{i+1}\cdots\# C_r$なので,$[L_1:K]=\# C_1$.従って$L_i$は$K$の素数冪次の巡回拡大.
\end{proof}

\begin{screen}
  $\Gal(N/\mathbb{Q}_2)\simeq(\mathbb{Z}/2\mathbb{Z})^4$となる$N$は存在しない
\end{screen}
\begin{proof}
  $[N:\mathbb{Q}_2]=16$なので,$[N:M]=8$となる中間体$M$が存在する.
  $\Gal(N/M)$と同型な$(\mathbb{Z}/2\mathbb{Z})^4$の部分群は位数が$8$であり,
  このような部分群は次の$15$個存在する\footnote{\verb|https://github.com/Hogemura/python_calc/blob/master/Zmod2Z.py|}.
  したがって,$\mathbb{Q}_2$の$2$次拡大が15個存在することになり矛盾(命題2.4.2).
\end{proof}

\begin{screen}
  $H$と$H_1$の構成(Galois群が$(\mathbb{Z}/4\mathbb{Z})^3$と同型になる$\mathbb{Q}_2$の有限次拡大$N$が存在しないことを証明する部分)
\end{screen}
\begin{proof}
  Galois群$\Gal(N/\mathbb{Q}_2(\sqrt{-1}))$を同型で写した,$(\mathbb{Z}/4\mathbb{Z})^3$の部分群を$H$とする.
  $[N:\mathbb{Q}_2(\sqrt{-1})]=32$なので$\lvert H\rvert=32$.
  $H$は位数$4$の元2つと位数$2$の元1つで生成されることが容易に分かる.
  $\alpha$,$\beta$,$\gamma$を位数$4$の元として,$H$の生成系は$\{\alpha,\beta,2\gamma\}$と表すことができる.
  $H$の任意の元は自然数$a, b, c$によって$a\alpha+b\beta+2c\gamma$と書ける.

  $a\alpha+b\beta+2c\gamma=0$,$A=(\alpha,\beta,2\gamma)\in\M_3(\mathbb{Z})$とする.
  $\det A=0$なら$\{\alpha,\beta,2\gamma\}$は線形従属となり,$\langle\alpha,\beta,2\gamma\rangle$は位数が$32$未満となるので矛盾.
  従って$\det A\neq0$であり,$(a,b,c)=(0,0,0)$.
  つまり,$\{\alpha,\beta,2\gamma\}$は線形独立.
  従って$\forall h\in H$に対し,$h=a\alpha+b\beta+2c\gamma$となる$(a,b,c)$が一意に定まる.
  これを$(a,b,2c)$と表すことにする.
  $(a,b,4c)$で表される元は$a\alpha+b\beta+4c\gamma=a\alpha+b\beta$なので,$\langle\alpha,\beta\rangle\simeq(\mathbb{Z}/4\mathbb{Z})^2$.
  これを$H_1$と書けば,$(\mathbb{Z}/4\mathbb{Z})^4/H_1\simeq\mathbb{Z}/4\mathbb{Z}$.
\end{proof}

\chapter{Gauss和・Jacobi和と有限体上の方程式}
\setcounter{section}{1}
\section{Gauss和の応用}
\paragraph{補題5.2.6}~
\begin{screen}
  $N(x^n=a)=\lvert G/G_1\rvert$
\end{screen}
\begin{proof}
  定理I-7.4.10から$\mathbb{F}_p^\times$巡回群であるので,$g$で生成されるとする.
  $n\nmid p-1$なら$\mathbb{F}_p^\times=(\mathbb{F}_p^\times)^n$.
  実際,$i=g^\alpha$,$j=g^\beta$,$1\leq\alpha <\beta\leq p-1$,$i^n = j^n$とすれば,$g^{n\beta}-g^{n\alpha}=g^{n\alpha}(g^{n(\beta-\alpha)} - 1)=0$となるが,$p-1\nmid n(\beta-\alpha)$なので,$i= j$となる.
  従って,$N(x^n=1)=1$.

  他方,$n\mid p-1$,$p-1=nk$とする.$(\mathbb{F}_p^\times)^k=\{1,g^k,\ldots,g^{(n-1)k}\}$である.
  $i,j$を$x^n=a$の解,さらに,$i=g^\alpha$,$j=g^\beta$,$0\leq\alpha <\beta\leq p-1$とする.
  $a = i^n = j^n$なので,$g^{n(\beta-\alpha)} =1$,従って$(p-1)l = n(\beta-\alpha)$となる$l$が存在する.
  $kl=\beta-\alpha$なので,$j=i^{\beta/\alpha}=i^{1+kl/\alpha}=i\times g^{kl}\in i(\mathbb{F}_p^\times)^k$.
  従って,$x^n=a$の解は全て$\mathbb{F}_p^\times/(\mathbb{F}_p^\times)^k$の同値類に属する.

  逆に,$i^n=a$ならば,$i(\mathbb{F}_p^\times)^k$の全ての元は$x^n=a$を満たす.
  従って,$N(x^n=a)=\lvert(\mathbb{F}_p^\times)^k\rvert=n$.
\end{proof}

\setcounter{section}{4}
\section{不定方程式$3x^3 + 4y^3 + 5z^3 = 0$}
\paragraph{定理5.5.1}~
\begin{screen}
  $P\mid I_1, I_2$ならば$P=P_2$もしくは$P=P_3$
\end{screen}
\begin{proof}
  $\mathcal{O}_K$の類数は$1$なので,$P=(\beta)$となる$\beta\in\mathcal{O}_K$があり,$\beta\mid 3\alpha^2x^2, 12y^2$となる.
  $x^2, y^2$は互いに素なので,$ax^2+by^2=1$となる整数$a, b$が存在するので,$\beta\mid 12\alpha^2$であることが分かる.
  従って,$\beta=2-\alpha$,$\beta=3+2\alpha+\alpha^2$もしくは$\beta=\alpha$である.
  $(\alpha)=\mathfrak{p}_1{}\cdots\mathfrak{p}_t$と素イデアル分解されたとする.
  $p_i\mathbb{Z}=\mathfrak{p}_i\cap\mathbb{Z}$とおく.
  命題1.10.13,命題1.0.14から$\N_{K/\mathbb{Q}}((\alpha))=p_1^{f_1}\cdots p_t{}^{f_t}=(\N_{K/\mathbb{Q}}(\alpha))=(6)$.
  したがって,$p_1=2$, $p_2=3$となり,$2$の上にある唯一の素イデアル,$P_3$は$3$の上にある唯一の素イデアルなので,$(\alpha)=P_2P_3$.
\end{proof}

\chapter{$2$次体の整数論}
\section{$2$次体の基本単数}
\paragraph{定理6.1.4}
$D>1$(ただし$\equiv 0, 1 \bmod 4$)を与えると,Pell方程式$x^2-Dy^2=\pm4$及び$\theta$が決まる.
$b,c$を$\theta^2+b\theta+c=0$となるように定める.
$\theta$は簡約な$2$次実無理数なので(命題6.1.33),連分数展開が可能(命題6.1.29):$\theta=[k_0,k_1,\ldots,k_{n-1},\theta]$.
これによって最小整数解$\varepsilon = (x_1 + y_1\sqrt{D})/2$を定める(命題6.1.33).
$\varepsilon^l$はPell方程式の解であり(命題6.1.35),Pell方程式の解は$\varepsilon^l$の形である(系6.1.39).

\paragraph{系6.1.39}~
\begin{screen}
  $t=u$の場合
\end{screen}
\begin{proof}
  $(x, y)$がPell方程式の解で
  \[
  \begin{pmatrix}
    \frac{x-by}{2} & -cy \\
    y & \frac{x+by}{2}
  \end{pmatrix}
  =
  \begin{pmatrix}
    r & s \\
    t & u
  \end{pmatrix}
  =
  \begin{pmatrix}
    s+1 & s \\
    1 & 1
  \end{pmatrix}
  \]
  であるとする.$x=s+2$,$y=1$,$b=-s$なので$\theta=(-b+\sqrt{D})/2=(s+\sqrt{D})/2$となる.
  $\theta=[s,1,\theta]$なので,$q_0=0$,$q_1=1$,$q_2=1$である,従って,
  \[\frac{x_1+y_1\sqrt{D}}{2} = q_2\theta + q_1 = \theta + 1 = \frac{s+2+\sqrt{D}}{2} = \frac{x+y\sqrt{D}}{2}\]
  となり,主張が従う.
\end{proof}

\section{$2$次体の類数}
\paragraph{定理6.2.1}
虚$2$次体では狭義のイデアル類とイデアル類が一致するので,イデアル類と$\Sym_{\mathrm{prim}, D}^2(\mathbb{Z}^2)$の正の同値類は
\[\alpha=(\alpha_1, \alpha_2) \mapsto f_\alpha,\quad ax^2 + bxy + cy^2 \mapsto \left\langle a, \frac{-b + \sqrt{D}}{2} \right\rangle\]
によって1対1に対応する(定理6.2.14).
系6.2.29から$\Sym_{\mathrm{prim}, D}^2(\mathbb{Z}^2)$の異なる簡約$2$次形式は対等にならない.
また,証明から$\Sym_{\mathrm{prim}, D}^2(\mathbb{Z}^2)$の任意の元は簡約$2$次形式と正に対等であることが分かる.
従って,$\Sym_{\mathrm{prim}, D}^2(\mathbb{Z}^2)$の正の同値類の完全代表系として簡約$2$次形式を取ることができる.
よって,$K$の類数は簡約$2$次形式の係数の組み合わせの数に等しい.

\paragraph{定理6.2.2}
イデアル類群と判別式$D$の$2$次実無理数の対等による同値類$Y_D$と1対1に対応する(定理6.2.19).
判別式$D$の$2$次実無理数は$(-b+\sqrt{D})/2a$の形である.
全ての判別式$D$の$2$次実無理数は判別式$D$の簡約な$2$次実無理数と対等である(系)ので,$Y_D$の完全代表系を判別式$D$の簡約な$2$次実無理数から選ぶことができる.
あとは,判別式$D$の簡約な$2$次実無理数のうち,対等である(すなわち連分数展開に於いて循環節が一致する:命題6.2.31)ものを同一視すれば,$Y_D$の完全代表系が得られる.
また,判別式$D$の$2$次実無理数$(-b+\sqrt{D})/2a$が簡約であることは,これを第1根に持つ$2$次形式$ax^2+bxy+cy^2$が簡約形式であることと同値(命題6.2.21).
従って,類数を求める際は,判別式$D$の簡約$2$次形式の係数のみを考えれば良い.
